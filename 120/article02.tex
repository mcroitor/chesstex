%\diagnum[2.]{1}
\subsection*{Ладья против Ладьи, Слона и Пешки: ничейные мотивы}
\markright{}
\addcontentsline{toc}{subsection}{Ладья против Ладьи, Слона и Пешки: ничейные мотивы}
\dianamestyle{fullname}

Обычно, при таком соотношении сил, слабейшая сторона сдаётся практически сразу -- у чёрных подавляющее преимущество. Однако существует несколько возможностей игры на ничью у слабейшей стороны. Это: 
\begin{itemize}
\setlength\itemsep{-0.25em}
\item отыгрыш материала; 
\item связывание; 
\item вечный шах; 
\item пат. 
\end{itemize}

Таким образом, слабейшая сторона (пусть у нас это будут белые) обладает довольно широким набором защитных средств. Их можно использовать в тех случаях, когда взаимодействие чёрных фигур нарушено: стеснённая позиция фигур, отдалённость чёрного короля. Или же, в случае запатованности белого короля, игра на пат. Приведённые ниже позиции демонстрируют приведенные возможности.
\begin{multicols}{2}[]
\begin{center}
\begin{diagram}%
  \pieces[2+4]{sLd8, sBh6, wTf4, sTa3, sKc2, wKg2}%
  \stipulation{ничья}%
\end{diagram}%
\end{center}\index{Кройтор Михаил}

На диаграмме 9 представлена простая позиция, в которой белые объявляют вечный шах. В случае размена ладей -- известная позиционная ничья. 

\bold{1. \rook{}f2+ \king{}d1 2. \rook{}f1+ \king{}e2 3. \rook{}f2+} с ничьёй.

\begin{center}
\begin{diagram}%
  \pieces[2+4]{sLd8, sBh6, wTf5, sKa4, sTa3, wKg2}%
  \stipulation{ничья}%
\end{diagram}%
\end{center}\index{Кройтор Михаил}

Похожая идея на следующей диаграмме, только вместо вечного шаха -- преследование ладьи.

\bold{1. \rook{}f3 \rook{}a2+ 2. \rook{}f2 \rook{}a1 3. \rook{}f1} с ничьёй.

Следует обратить внимание, что в этих случаях на руку слабейшей стороне ладейная пешка и удачный угол для белого короля.

\begin{center}
\begin{diagram}%
  \pieces[2+4]{sBd7, sLd5, sKg5, sTb4, wKe2, wTa1}%
  \stipulation{ничья}%
\end{diagram}%
\end{center}\index{Кройтор Михаил}

На диаграмме 11, привязывая ладью к слону, белые добиваются ничьей. 

\bold{1. \rook{}a5 \rook{}d4 2. \king{}e3 \rook{}d1 3. \king{}e2 \rook{}d4 4. \king{}e3}. 

Попытка развязаться приводит к отыгрышу слона белыми: 

\bold{4... \rook{}e4+ 5. \king{}d3 \rook{}e5 6. \king{}d4 \king{}f6 7. \rook{}xd5 \rook{}xd5+ 8. \king{}xd5}.

С очевидной ничьёй.

Еще одна несложная позиция показана на диаграмме 12, демонстрирующая возможность отыгрыша пешки.

\begin{center}
\begin{diagram}%
  \pieces[2+4]{sTa8, sLd8, wTe8, wKf8, sKh8, sBd7}%
  \stipulation{ничья}%
\end{diagram}%
\end{center}\index{Кройтор Михаил}

\bold{1. \king{}f7+ \king{}h7 2. \rook{}g8!} 

У черных нет выбора. Слон и ладья связаны: 

\bold{2... d5 3. \rook{}g7+ \king{}h8 4. \rook{}g8+ \king{}h7 5. \rook{}g7+ \king{}h8}. 

мирясь с вечным шахом, или 

\bold{5... \king{}h6 6. \rook{}g6+ \king{}h5 7. \rook{}d6} 

и белые отыгрывают пешку.

Первые позиции прадставляли собой ничейные схемы. Теперь рассмотрим ряд этюдов.

\begin{center}
\begin{diagram}%
  \author{Погосянц, Эрнест}%
  \year{1980}%
  \pieces[2+4]{wTd8, sLe7, sBe6, wKc6, sTa5, sKa4}%
  \stipulation{ничья}%
\end{diagram}%
\end{center}\index{Погосянц Эрнест}

Решение позиции, представленной на диаграмме 13, следующее: 

\bold{1. \rook{}e8 \rook{}a7 2. \king{}b6 \rook{}d7 3. \king{}c6 \rook{}d6+ 4. \king{}c5}. 

Проигрывает \italic{4. \king{}c7? \king{}b5 5.\rook{}xe7 \king{}c5} с победой чёрных. 

\bold{4... \rook{}d5+ 5. \king{}c6 \rook{}d6+ 6. \king{}c5 \rook{}d7+ 7. \king{}c6 \rook{}a7 8. \king{}b6}. 

Чёрные не могут развязаться. Ничья. 

Позиция похожа на 11, только чёрной ладье предоставлена большая свобода.

\begin{center}
\begin{diagram}%
  \author{Погосянц, Эрнест}%
  \year{1978}%
  \pieces[2+4]{sKe5, sBh5, wKf3, wTa2, sLe1, sTg1}%
  \stipulation{ничья}%
\end{diagram}%
\end{center}\index{Погосянц Эрнест}

Этюды Погосянца не отличались большой сложностью в большинстве своём, они скорее походили на концовки этюдов. Многие его зарисовки были впоследствии развиты другими этюдистами.

Ещё один этюд Погосянца представлен на диаграмме 14. Решение сводится к известной позиции, где одинокий король благополучно борется с вражеским воинством. 

\bold{1. \rook{}h2 h4} 

\italic{1... \rook{}g5 2. \rook{}e2+} -- забирая слона.

\bold{2. \rook{}h1! \rook{}xh1 3. \king{}g2} забирая ладью. 

Данная ловля ладьи известна со времён Троицкого.

\begin{center}
\begin{diagram}%
  \author{Фриц, Индрих}%
  \year{1940}%
  \pieces[2+4]{wTf8, sKc3, sBe3, sTa1, sLb1, wKf1}%
  \stipulation{ничья}%
\end{diagram}%
\end{center}\index{Фриц Индрих}

Окончание на диаграмме 15 демонстрирует пат и способ его достижения, часто встречающийся в окончании ладья против ладьи и слона. 

\bold{1. \king{}e2 \king{}d4} 

В случае \italic{1... \bishop{}d3+} решает \italic{2. \king{}xe3 \rook{}e1+ 3. \king{}f2 \rook{}f1+ 4. \king{}e3 \rook{}xf8} пат. 

\bold{2. \rook{}f4+!} 

Но не \italic{2. \rook{}d8+? \king{}e4 3. \rook{}e8+ \king{}f4 4. \rook{}f8+ (4. \rook{}xe3 \rook{}a2+) 4... \bishop{}f5} с победой чёрных. 

\bold{2... \bishop{}e4 3. \rook{}xe4+! \king{}xe4} Пат.

\begin{center}
\begin{diagram}%
  \author{Надареишвили, Гиа}%
  \year{1973}%
  \pieces[2+4]{sTd8, sLh6, sBa4, wKb4, sKa2, wTc1}%
  \stipulation{ничья}%
\end{diagram}%
\end{center}\index{Надареишвили Гиа}

В позиции на диаграмме 16 от пешки белый король легко отбрасывается.

\bold{1. \rook{}c2+} 

Проигрывает \italic{1. \rook{}h1 \rook{}d4+ 2. \king{}c3 \rook{}d6 3. \king{}b4 a3}. 

\bold{1... \king{}b1 2. \rook{}h2 \rook{}d4+ 3. \king{}c3}. 

Можно и мат получить... \italic{3. \king{}a3?? \bishop{}c1+}. 

\bold{3... \rook{}d6 4. \king{}b4 \rook{}d4+}. 

Не проходило \italic{4... \rook{}a6 5. \king{}b5 \rook{}a8 6. \rook{}xh6 a3 7. \rook{}h1+ \king{}c2 8. \rook{}h2+ \king{}d3 9. \rook{}h3+ \king{}e4 10. \rook{}h4+ \king{}f3 11. \rook{}h1! a2 12. \rook{}a1 \king{}e2 13. \king{}c4!} (тут ошибочно \italic{13. \king{}b4? \king{}d3!} (не 13... \king{}d2? 14. \king{}b3 \rook{}b8+ 15. \king{}c4) \italic{14. \king{}b3 \rook{}b8+ 15. \king{}a3 \king{}c2! 16. \rook{}xa2+ \king{}c3} и чёрные проигрывают) \italic{13... \king{}d2 14. \king{}b3 \rook{}b8+ 15. \king{}c4 \rook{}b2 16. \rook{}h1} с ничьей. 

\bold{5. \king{}c3 \bishop{}g7}. 

В случае \italic{5... \rook{}d1 6. \rook{}xh6} ничья. Чёрные защитили все свои фигуры, но одновременно сильно стеснили белого короля. Это позволяет слабейшей стороне провести патовую комбинацию. 

\bold{6. \rook{}b2+ \king{}a1}. 

Или \italic{6... \king{}c1 7. \rook{}b1+ \king{}xb1}. 

\bold{7. \rook{}b1+! \king{}a2 8. \rook{}b2+ \king{}a3 9. \rook{}b3+! axb3}. Пат.

\begin{center}
\begin{diagram}%
  \author{Нестореску, Вирджил}%
  \year{1986}%
  \pieces[3+4]{sLb8, wKb6, wTa5, sBe4, sTe2, wBf2, sKc1}%
  \stipulation{ничья}%
\end{diagram}%
\end{center}\index{Нестореску Вирджил}

Этюд под номером 17 относится к числу моих любимых. 

\bold{1. \king{}c5}. 

Белые свою пешку защитить не в состоянии. Прямая попытка связать ладью опровергается жертвой слона: 
\italic{1. \rook{}a4? \bishop{}g3! 2. fxg3 e3 3. g4 \rook{}b2+ 4. \king{}c6 e2 5. \rook{}e4 \rook{}d2 6. g5 \king{}d1} и чёрные побеждают. 

\bold{1... \rook{}xf2 2. \king{}d4 \rook{}f4}. 

Или \italic{2... \rook{}e2 3. \king{}c3 \rook{}c2+ 4. \king{}d4 \rook{}e2 5. \king{}c3 \rook{}e3+ (5... \king{}b1 6. \rook{}b5+) 6. \king{}d4 \rook{}e2 (6... \rook{}e1 7. \rook{}a1+) 7. \king{}c3} с ничьей.

\bold{3. \king{}e3}. 

Куда идёт король?! Здесь белые могли увлечься другими продолжениями: \italic{3. \king{}c3? \king{}d1 -+; 3. \rook{}a2? \rook{}h4 4. \king{}e3 \bishop{}d6 5. \rook{}a5 \rook{}f4+ -+; 3. \rook{}a1+? \king{}d2 4. \rook{}a2+ \king{}d1 5. \king{}e3 \rook{}h4 6. \rook{}d2+ \king{}c1 7. \rook{}d7 \bishop{}h2 -+}. 

\bold{3... \rook{}h4 4. \rook{}c5+ \king{}d1 5. \rook{}d5+ \king{}e1 6. \rook{}d4 \bishop{}a7}. 

Замечательный пат в центре доски при связанной ладье! Игра может закончиться так: \italic{6... \bishop{}e5 7. \rook{}c4 (7. \rook{}xe4?? \rook{}h3\mate{}) 7... \rook{}h3+ 8. \king{}xe4} -– ничья. В своё время я тоже рассматривал финальную позицию, но такого удачного развития добиться не смог.

\begin{center}
\begin{diagram}%
  \author{Бенко, Пал}%
  \year{1996}%
  \pieces[2+4]{sBa7, wKa6, wTg3, sLh2, sTc1, sKe1}%
  \stipulation{ничья}%
\end{diagram}%
\end{center}\index{Бенко Пал}

В этюде на диаграмме 18 первые 3 хода белых легко находятся. 

\bold{1. \rook{}g2 \bishop{}b8 2. \king{}b7! \rook{}b1+ 3. \king{}a8!}

Белые добились многого: за три хода они смогли запатовать своего короля, привязали чёрные фигуры друг к другу. Теперь слабейшая сторона грозит ``бешенством'' ладьи. 

\bold{3... \rook{}b5}

Не помогает чёрным и \italic{3... \rook{}b4 4. \rook{}a2 \king{}f1 5. \rook{}a5 \rook{}b6 6. \rook{}a2 a6 7. \rook{}f2+ \king{}g1 8. \rook{}g2+ \king{}h1 9. \rook{}a2} -– белые добиваются ничьей. 

\bold{4. \rook{}e2+!}

Ошибочным был бы выпад \italic{4. \rook{}b2? a6! 5. \rook{}a2 a5 6. \rook{}b2 \rook{}b4!}

\bold{4... \king{}f1 5. \rook{}f2+ \king{}g1 6. \rook{}g2+ \king{}h1 7. \rook{}g1+!}

\italic{7. \rook{}g4? \king{}h2 8. \rook{}b4 \rook{}b6! 9. \rook{}g4 \king{}h3!}

\bold{7... \king{}h2}

Что делать теперь? 

\bold{8. \rook{}g4!} 

Предугадывая продвижение пешки. \italic{8. \rook{}b1? a6 9. \rook{}a1 a5 10. \rook{}b1 \rook{}b4} -- пешка постепенно подходит. \italic{11. \rook{}a1 \bishop{}c7} с победой чёрных. 

\bold{8... \king{}h3 9. \rook{}b4 \rook{}b6 10. \rook{}g4! \rook{}b5}. 

Тут целая группа ветвлений: \italic{10... \bishop{}g3 11. \rook{}a4! (11. \king{}xa7? \bishop{}f2); 10... \rook{}e6 11. \rook{}g8; 10... \rook{}a6 11. \rook{}g7;}

\bold{11. \rook{}b4 a6 12. \rook{}a4 a5}.

На \italic{12... \rook{}b6} следует играть \italic{13. \rook{}b4 \bishop{}c7 14. \rook{}a4 \rook{}g6 15. \king{}b7}. 

\bold{13. \rook{}b4 axb4}. Пат.

\begin{center}
\begin{diagram}%
  \author{Чеховер, Виталий}%
  \year{1957}%
  \pieces[3+4]{sBd7, sLc5, wBd5, wTf5, sTc3, wKe2, sKh1}%
  \stipulation{ничья}%
\end{diagram}%
\end{center}\index{Чеховер Виталий}

Ещё один пример позиционной ничьей приведён на диаграмме 19. Здесь уже, в отличие от позиций 11 и 13, может проявить активность и чёрный король. 

\bold{1.d6! \bishop{}xd6 2. \rook{}d5 \rook{}c6 3. \king{}f3 \rook{}a6}.

Не страшно для белых и \italic{3... \king{}h2 4. \rook{}h5+ \king{}g1 5. \rook{}d5}.

\bold{4. \rook{}d1+ \king{}h2 5. \rook{}d5 \rook{}b6 6. \rook{}h5+}.

Чёрного короля нельзя выпускать. Попытка горизонтального шаха приводила к поражению: \italic{6. \rook{}d2+? \king{}h3 7. \rook{}d5 \king{}h4}. 

\bold{6... \king{}g1 7. \rook{}d5 \rook{}b3+ 8. \king{}e2 \rook{}b2+ 9. \king{}e3}.

Плохо \italic{9. \king{}f3? \rook{}f2+ 10. \king{}e4 \rook{}f6} с поражением.

\bold{9... \rook{}b3+ 10. \king{}e2 \rook{}b6 11. \king{}f3}.

Чёрные не могут развязать свои фигуры, король же их отрезан. Ничья.

\begin{center}
\begin{diagram}%
  \author{Тимман, Ян}%
  \year{1993}%
  \pieces[3+4]{sBe6, sTg6, sLb5, wBf4, wTd3, wKh2, sKa1}%
  \stipulation{ничья}%
\end{diagram}%
\end{center}\index{Тимман Ян}

На диаграмме 20 показан этюд известного гроссмейстера - практика, претендента на мировую корону. 

\bold{1. \rook{}d1+.}

В случае \italic{1. \rook{}d6?} белые бесславно проигрывают, например, следующим образом: \italic{1... \bishop{}e8 2. \rook{}b6 \bishop{}f7 3. \king{}h3 \rook{}g1 4. \king{}h4 \king{}a2 5. \king{}h3 \rook{}e1 6. \king{}g4 \bishop{}g6 7. \king{}g5 \bishop{}f5} и белым не спастись. 

\bold{1... \king{}b2.} 

Или \italic{1... \king{}a2 2. f5! exf5 3. \rook{}d5 \rook{}g2+ 4. \king{}h3 \rook{}b2! 5. \king{}h4! f4 6. \rook{}f5 \rook{}b4 7. \king{}g5}, и ничья. 

\bold{2.f5 exf5.}

Ничья и в варианте \italic{2... \rook{}h6+ 3. \king{}g3 \king{}c2 4. \rook{}d6}. 

\bold{3. \rook{}d5 \rook{}g2+!}

Интересная контригра! Понятно, что ладью брать чревато поражением. 

\bold{4. \king{}h3.}

Отход в угол проигрывает из-за рентгена: \italic{4. \king{}h1? \bishop{}c6}. 

\bold{4... \bishop{}f1 5. \rook{}d1!}

Сразу отыграть пешку не удаётся: \italic{5. \rook{}xf5? \rook{}f2+ 6. \king{}g4 \bishop{}h3+}. 

\bold{5... \bishop{}e2.} 

Или \italic{5... \rook{}f2+ 6. \king{}g3}. 

\bold{6. \rook{}e1 \bishop{}f3 7. \rook{}f1 \bishop{}e4 8. \rook{}xf5.}

Ничья – белые отыграли пешку.

\begin{center}
\begin{diagram}%
  \author{Бенко, Пал}%
  \year{1996}%
  \pieces[2+4]{sKc8, wKe7, sBg7, wTd4, sTc2, sLh1}%
  \stipulation{ничья}%
\end{diagram}%
\end{center}\index{Бенко Пал}

Очень поучительная позиция представлена на диаграмме 21. Несмотря на то, что у чёрных большое материальное преимущество, бросаются в глаза недостатки их позиции: разобщённость фигур, отрезанность короля от пешки, очень неудачная позиция слона. Первым делом белые переводят ладью на активную позицию. Конечно же, лучшая позиция для ладьи -- позади пешки: 

\bold{1. \rook{}d8+} 

Хуже прямолинейное \italic{1. \rook{}g4? \rook{}g2 2. \rook{}c4+ \king{}b7 3. \rook{}c1 \rook{}h2 4. \rook{}g1 \bishop{}g2 5. \king{}e6 g6 6. \king{}e5 \rook{}h5+ 7. \king{}f4 \rook{}f5+ 8. \king{}e3 \rook{}e5+ 9. \king{}f4 \rook{}e2} и белым на спастись. 

\bold{1... \king{}c7 2. \rook{}g8!} 

В случае \italic{2. \rook{}d7+?} белые проигрывают: \italic{2... \king{}b6!! (2... \king{}c6? 3. \rook{}d6+) 3. \rook{}d6+ \rook{}c6} с ничьей. 

\bold{2... \rook{}g2 3. \king{}f7 g5}

\bold{4. \king{}e6!!} 

``Отталкивание плечом''. Если \italic{4. \king{}f6?}, то \italic{4... g4 5. \rook{}g7+ \king{}d6 6. \rook{}g6 g3 7. \king{}f5+ \king{}e7 8. \rook{}g7+ \king{}e8 9. \rook{}g5 \rook{}f2+ 10. \king{}g4 g2 11. \king{}h3 \rook{}f5!} и черные побеждают. 

\bold{4... g4 5. \rook{}g7+ \king{}c6 6. \rook{}g6! g3 7. \king{}f5+! \king{}d7 8. \rook{}g5!} 

Здесь надо проявлять аккуратность. Поспешный марш королём после \italic{8. \king{}f4? \rook{}f2+} приводит к поражению белых. 

\bold{8... \rook{}f2+ 9. \king{}g4 g2 10. \king{}h3 \rook{}f5 11. \rook{}g7+.} 

Теперь видна разница между ходами \italic{4. \king{}e6!} и \italic{4. \king{}f6?} -- в первом случае у белых есть спасительный шах. 

\bold{11... \king{}e6 12. king{}h2 \king{}f6 13. \rook{}g3}! 

Ещё одна тонкость \italic{13. \rook{}g8? \rook{}g5; 13. \rook{}g4? \rook{}g5; 13. \rook{}a7? \rook{}f1}. 

\bold{13... \rook{}g5 14. \rook{}f3+ \king{}e5 15. \king{}g1!}

Тут можно и мат получить: \italic{15. \rook{}e3+? \king{}f4 16. \rook{}e1 g1=\queen{}+ 17. \rook{}xg1 \rook{}h5\mate{}}.

Ничья! Например, следующим образом \italic{15... \rook{}f5 16. \rook{}g3 \king{}f4 17. \rook{}g8 \king{}f3 18. \king{}h2 \king{}f2 19. \rook{}xg2+}. Данную позицию полезно запомнить, так же, как и принцип игры в ней – белая ладья курсирует по линии ``g'', белый король -- по полям g1 и h1. В случае, если чёрный король становится на f2 -- идет жертва ладьи на g2.

\begin{center}
\begin{diagram}%
  \author{Брон, Владимир}%
  \year{1961}%
  \pieces[3+5]{sKc8, sTh7, wBe6, sBg6, wKc5, wTg4, sBg3, sLa1}%
  \stipulation{ничья}%
\end{diagram}%
\end{center}\index{Брон Владимир}

Интересен этюд Брона своей позиционной ничьёй (диаграмма 22):

\bold{1.Kd6!} 

Другие продолжения проигрывают: \italic{1. \rook{}xg3? \rook{}g7 2. \king{}d6 \bishop{}f6 3. \rook{}f3 \bishop{}e7+}; или \italic{1. \rook{}xg6 \rook{}g7} с поражением. 

\bold{1... \bishop{}f6!}%

В случае \italic{1... \bishop{}b2} можно, не стесняясь, брать пешку g3 -- \italic{2. \rook{}xg3! \rook{}h6 3. e7}; а при \italic{1... \rook{}a7} -- пешку g6 -- \italic{2. \rook{}xg6 \rook{}a6+ 3. \king{}d5 \rook{}a3 4. \rook{}g8+}. 

\bold{2. \rook{}xg6 \bishop{}h4}.

Здесь ``тонкое место''.

\bold{3. e7!}

Ошибочны следующие варианты: А) \italic{3. \king{}c6? \rook{}c7+ 4. \king{}b6 \rook{}c3 5. \rook{}g8+ \bishop{}d8+ 6. \king{}b5 \king{}c7 7. \rook{}g7+ \king{}d6 8. \rook{}d7+ \king{}xe6 9. \rook{}xd8 –+}. Б) \italic{3. \rook{}g4? \king{}b7 4. \rook{}b4+ \king{}a7! 5. \king{}a6 \bishop{}d8 6. \rook{}a4+ \king{}b8 7. \rook{}b4+ \king{}c8 8. \rook{}g4 \rook{}c7+ 9. \king{}d5 \rook{}c3 -+}.

\bold{3... \bishop{}xe7+}.

Если брать ладьёй, то ничья очевидна -- \italic{3... \rook{}xe7 4. \rook{}g8+ \king{}b7 5. \rook{}xg3}. 

\bold{4. \king{}c6! \rook{}h3}. 

Построена ``крепость''. Чёрная ладья привязана к защите пешки, в свою очередь слон не может покинуть короля. 

\bold{5. \rook{}g8+ \bishop{}d8 6. \rook{}g6!} 

Попытка пробежаться чёрным королём тоже ничего не даёт: \italic{6... \king{}b8 7. \rook{}g8 \king{}a7 8. \rook{}xd8 =}. 

\bold{6... \bishop{}e7 7. \rook{}g8+ \bishop{}d8 8. \rook{}g6.} 

Ничья.

\begin{center}
\begin{diagram}%
  \author{Фриц, Индрих}%
  \year{1984}%
  \pieces[3+4]{sKe7, wKf5, sBd4, sTc3, wBd3, wTa2, sLh1}%
  \stipulation{ничья}%
\end{diagram}%
\end{center}\index{Фриц Индрих}

В этюде 23 у сильнейшей стороны опять неудачно расположены фигуры: чёрный слон в углу доски, белые могут привязать чёрную ладью к защите пешки. Так и начнём: 

\bold{1. \king{}e5.}

В случае \italic{1. \rook{}h2? \rook{}c5+ 2. \king{}f4 \bishop{}c6 3. \rook{}h6 \king{}d7} чёрные подводят короля к пешкам и побеждают. 

\bold{1... \rook{}xd3 2. \rook{}a7+.} 

У этюда есть серьёзная дуаль: \italic{2. \rook{}a1! \bishop{}c6 (2... \bishop{}f3 3. \rook{}a4 \rook{}e3+ 4. \king{}f4) 3. \rook{}c1 \rook{}c3 (3... \bishop{}b5 4. \rook{}c5 \bishop{}a6 5. \rook{}d5) 4. \rook{}a1! \rook{}c4 (4... d3 5. \king{}d4 \rook{}b3 6. \king{}c4) 5. \rook{}a3!} - пешка отыгрывается. 

\bold{2... \king{}d8 3. \rook{}a1 \bishop{}f3.}

Ничья и в случае \italic{3... \bishop{}c6 4. \rook{}c1 \rook{}c3 5. \rook{}d1 \rook{}c5+ 6. \king{}d6 \rook{}c4 7. \king{}e5.} 

\bold{4. \rook{}a4.}

или \italic{4. \rook{}a6}, что тоже приводит к ничье. 

\bold{4... \rook{}e3+ 5. \king{}f4 \rook{}d3 6. \king{}e5.} 

Знакомая позиция по варианту \italic{3... \bishop{}c6}, но уже в горизонтальном исполнении.

\begin{center}
\begin{diagram}%
  \author{Фриц, Индрих}%
  \year{1979}%
  \pieces[2+4]{wTd8, wKc6, sBb4, sLb3, sTf3, sKf2}%
  \stipulation{ничья}%
\end{diagram}%
\end{center}\index{Фриц Индрих}

Этюд того же автора показан на следующей диаграмме.

\bold{1. \king{}c5.}

Заранее уходя из под шаха. Если играть на первом ходу ладьёй, то чёрные получают выигрывающий темп: \italic{1. \rook{}b8? \bishop{}a4+ 2. \king{}c5 b3 3. \king{}b4 \rook{}f4+ 4. \king{}c3 \king{}e2;} или \italic{1. \rook{}d4? \bishop{}a4+ 2. \king{}c5 b3 3. \rook{}xa4 b2 4. \rook{}b4 \king{}e1.}

\bold{1... \rook{}f4 2. \rook{}b8 \rook{}c4+ 3. \king{}b5.}

В случае \italic{3. \king{}b6?} белые быстро проигрывают \italic{3... \bishop{}c2 4. \king{}a5 (4. \king{}b5 \rook{}f4) 4...b3 -+}. 

\bold{3... \bishop{}a2!}

Или \italic{3... \king{}e2 4. \king{}a5! =}. 

\bold{4. \king{}a4!!}

Точный ход! Проигрывало \italic{4. \rook{}a8? b3 5. \king{}xc4 b2+}. Теперь, несмотря на то, что чёрные могут продвинуть свою пешку с шахом (к чему они вынуждаются), белые запирают слона! Если убрать с доски ладьи, то получается известная позиционная ничья. 

\bold{4... b3+ 5. \king{}a3 \rook{}c2 6. \rook{}e8.}

Замечательная позиция позиционной ничьей! Чёрный король отрезан от зоны действий, слон привязан к защите пешки, пешка -- к защите слона, а сама ладья сделать ничего не в состоянии. Белым достаточно играть ладьей по линии ``е'', а в случае \italic{6... \rook{}e2} (или просто бегают ладьей и королём -- \italic{6... \rook{}d2 7. \rook{}e7 \rook{}h2 8. \rook{}e8 \rook{}c2 9. \rook{}e7 \king{}f2;) 7. \rook{}xe2 \king{}xe2 8. \king{}b2}. Ничья.\\
\begin{center}
\begin{diagram}%
  \author{Фриц, Индрих}%
  \year{1979}%
  \pieces[2+4]{wTe8, wKc7, sBb5, sLb4, sTf2, sKc1}%
  \stipulation{ничья}%
\end{diagram}%
\end{center}\index{Фриц Индрих}

Некоторая вариация предыдущей позиции -- в смысле положения фигур -- показана на диаграмме 25. Вступление, потому, такое же. 

\bold{1. \king{}c6}

Опять же ходы ладьёй ни к чему не приводят: \italic{1. \rook{}e5? \bishop{}a5+ 2. \king{}c6 b4 3. \rook{}xa5 (3. \king{}b5 b3) 3... b3; 1. \rook{}b8? \bishop{}a5+ 2. \king{}c6 b4 3. \rook{}b5 (3. \king{}b5 b3) 3... \rook{}a2}.

\bold{1... \rook{}f5 2. \rook{}b8 \rook{}c5+ 3. \king{}b6 \bishop{}a3}

Или \italic{3... \king{}c2 4. \king{}a6}, и пешка отыгрывается. 

\bold{4. \rook{}a8}

Здесь уже прорыв королём не приносит желаемого результата – его чёрный оппонент слишком близок к пешке \italic{4. \king{}a5? b4+ 5. \king{}a4 \king{}c2}. 

\bold{4... \bishop{}b4}

Ещё варианты: \italic{4... b4 5. \king{}xc5 b3+ 6. \king{}c4 b2 7. \rook{}b8 =; 4... \king{}b2 5. \rook{}a5 b4 6. \rook{}xc5 =}.

\bold{5. \rook{}b8 \bishop{}a3 6. \rook{}a8} 

Чередование нападения и угрозы отыгрыша материала.

\begin{center}
\begin{diagram}%
  \author{Хлынка, Михал}%
  \year{1992}%
  \pieces[2+4]{sLe8, sKh8, sBg5, sTh4, wKf3, wTa1}%
  \stipulation{ничья}%
\end{diagram}%
\end{center}\index{Хлынка Михал}

Ещё одна поучительная позиция на диаграмме 26. 

\bold{1. \rook{}a5!}

Ход, связывающий чёрные силы. Быстро налаживают координацию сил они в случае \italic{1. \rook{}a8? \rook{}f4+ 2. \king{}g3 \rook{}f8 3. \rook{}a5 \rook{}g8 4. \king{}g4 \bishop{}d7+}. 

\bold{1... \bishop{}h5+}

Если \italic{1... \bishop{}c6+}, то \italic{2. \king{}g3 \rook{}h5 3. \king{}g4 \bishop{}f3+ 4. \king{}g3 (4. \king{}xf3? g4+) 4... \bishop{}d1 5. \rook{}d5 \bishop{}b3 6. \king{}g4}. 

\bold{2. \king{}g3 \rook{}g4+ 3. \king{}h3 \king{}g7} 

Критическая положение, представляющее позиционную ничью. Пока чёрный король находится на королевском фланге - белые шахуют. Как только он направляется на ферзевый фланг - белая ладья занимает 5-ю горизонталь и курсирует по полям a5-e5.

\bold{4. \rook{}a7+!}

Шаховать надо – чтобы отбросить черного короля от королевского фланга -- \italic{4. \rook{}b5? \rook{}h4+ 5. \king{}g3 \king{}h6 -+}. 

\bold{4... \king{}f6 5. \rook{}a6+ \king{}e5 6. \rook{}a5+ \king{}d6 7. \rook{}a6+!}

Опять плохо без шаха -- \italic{7. \rook{}b5? \rook{}g1! 8. \king{}h2 \bishop{}e8! 9. \rook{}b8 \rook{}e1 -+} или \italic{7. \rook{}f5? \rook{}g1 8. \king{}h2 \king{}e6! 9. \rook{}a5 \rook{}g4 10. \king{}h3 \rook{}h4+ 11. \king{}g3 \king{}f6! 12. \rook{}a6+ \king{}g7 13. \rook{}a7+ \bishop{}f7 -+}. 

\bold{7... \king{}c5 8. \rook{}a5+ \king{}b4 9. \rook{}e5!}

Опять надо играть точно, чтобы не попасть под бой слона: \italic{9. \rook{}d5? \rook{}g1 10. \king{}h2 \bishop{}f3! -+}. 

\bold{9... \rook{}g1 10. \king{}h2 \rook{}g4 11. \king{}h3}

Чёрные не могут усилить свою позицию.

Данная статья не претендует на полноценное исследование данного типа окончаний. Однако она демонстрирует ряд ресурсы защиты слабейшей стороны.
\end{multicols}