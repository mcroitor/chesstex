\subsection*{Мухи творчества}
\markright{}
\addcontentsline{toc}{subsection}{Мухи творчества}
\dianamestyle{shortname}

\begin{multicols}{2}[]
И кому нужна эта композиция?! Это прямо кошмар какой-то! Берешься делать одно, получается совсем другое, а потом ещё оказывается, что до тебя это давным-давно сделали... Но, по порядку: Задумал я составить красивую задачу. Помните, как это было у Вюрцбурга?

\begin{center}
\begin{diagram}%
  \author{Вюрцбург, Отто}%
  \source{Grand Rapids Herald}\year{1932}%
  \pieces[4+2]{wDa7, sKc6, sBd6, wKb3, wTd2, wLf1}%
  \stipulation{мат в 2 хода}%
  \themes{%
    2 flights giving key
  }%
\end{diagram}%
\end{center}\index{Вюрцбург Отто}

Изящная миниатюра решается перекрытием слона, с построением батареи:

\bold{1. \rook{}e2} цугцванг: \bold{1... \king{}b5 2. \rook{}c2\mate; 1... \king{}d5 2. \bishop{}g2\mate; 1... d5 2. \rook{}e6\mate}

Вот и мне захотелось составить что-то похожее. Накидал схему из четырёх фигур:

\begin{center}
\begin{diagram}%
  \pieces[3+1]{wTe8, sKd4, wKb2, wLf1}%
  \stipulation{схема}%
\end{diagram}%
\end{center}

В этой позиции должен решать ход \italic{1. \rook{}e2!} с вариантом \italic{1... \king{}c4 2. \rook{}e4\mate}. Очевидно, что надо отрезать черного короля сверху. Добавил ферзя на h5 -- ладью переставил на e5 -- чтобы не было мата в 1 ход. И что получилось?!

\begin{center}
\begin{diagram}%
  \pieces[4+1]{wTe5, wDh5, sKd4, wKb2, wLf1}%
  \stipulation{задача, \bold{1. \rook{}e2!}}%
\end{diagram}%
\end{center}

Получилась задача с двумя вариантами, изображенная на диаграмме 40, причем второй вылез сам, не пришлось ничего перестраивать. Главное, удачно добавить фигуру! Однако задачей я остался недоволен, и на это это нашлась следующая причина: матовать голого запатованного черного короля -- дело нехитрое! При взгляде на позицию сразу становится понятно, что ходить надо ладьей... Потому ладью я убрал с поля e5, а от мата попробовал защититься пешкой. Однако, ничего не получилось.

\begin{center}
\begin{diagram}%
  \pieces[4+2]{wTe7, sBd6, wDh5, sKd4, wKb2, wLf1}%
  \stipulation{схема, \bold{1. \rook{}e2?}}%
\end{diagram}%
\end{center}

Оказалось, что без добавления материала мата на ход пешкой не пристроишь. Кроме того, появиляются множественные побочные решения. Поэтому приходится крутить доску, переставлять фигуры... В данном случае доска перевёрнута вдоль главной диагонали A1--H8.

\begin{center}
\begin{diagram}%
  \pieces[4+2]{wDe7, wLa6, sBd6, wTh5, sKd4, wKb2}%
  \stipulation{схема}%
\end{diagram}%
\end{center}

В позиции на диаграмме 42 уже проходит задуманное решение. Ферзя специально поставил на e7 чтобы он подхватывал поле с6 после хода черной пешки: в этом случае появляется вариант \italic{1... d5 2. \rook{}b4\mate}. Но кроме нужного решения существует еще 10 побочных!!! Что делать теперь?! Попробовал заменить первый ход, однако перестановка ладьи не помогла. Оставил ее в покое, начал двигать ферзя. Тут картина резко изменилась.

\begin{center}
\begin{diagram}%
  \pieces[4+2]{wLa6, sBd6, wTb5, sKd4, wKb2, wDe2}%
  \stipulation{схема}%
\end{diagram}%
\end{center}

В следующей редакции, показанной на диаграмме 43, кроме самого решения есть еще 3 побочных: \italic{1. \bishop{}b7, \queen{}e6, \queen{}f3}. А это уже легко исправить стандартными средствами! Доска до конца влево -- и хода \queen{}f3 нет! (ну, ему эквивалентного). Потом, переставил слона на b8 -- еще одна дырка прикрыта... В полученной позиции от \queen{}h6 спасет только черная пешка на g7.

Добавил и получил приличную задачу с хорошими (хоть и не теми, что хотел) вариантами (диаграмма 44):

\begin{center}
\begin{diagram}%
  \pieces[4+3]{wLb8, sBg7, sBg6, wTe5, sKg4, wKe2, wDh2}%
  \stipulation{\#2}%
\end{diagram}%
\end{center}

\bold{1. \queen{}h7!}

Цугцванг.

\bold{1. ... g5 2. \rook{}e4\mate; 1. ... \king{}f4 2. \queen{}h4\mate 1. ... \king{}g3 2. \rook{}g5\mate}

Неплохо, не так ли? ...но что это?! Где-то это мы уже видели... Оказалось, что если поменять в задачке фланги, получается известная миниатюра (диаграмма 45) одного из гигантов композиции Леонида Ивановича Куббеля. Одно расстройство... Успокаивает только одно: что я смог сделать то же, что когда-то выполнил великий мастер.

\begin{center}
\begin{diagram}%
  \author{Куббель, Леонид}%
  \source{Шахматный листок}\year{1926}%
  \pieces[4+3]{wLg8, sBb7, sBb6, wTd5, sKb4, wDa2, wKd2}%
  \stipulation{мат в 2 хода}%
\end{diagram}%
\end{center}
\end{multicols}\index{Куббель Леонид}

Данная статья показывает процесс составления задачи: нахождение идеи, использование стандартных методов (перестановка фигуры, сдвиг доски, поворот доски, добавление материала), а также нахождение предшественников. Надеюсь, эти ``мухи'' окажутся полезными начинающим композиторам.
