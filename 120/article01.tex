%\diagnum[1.]{1}
\subsection*{Как научиться решать задачи}
\markright{}
\addcontentsline{toc}{subsection}{Как научиться решать задачи}
\dianamestyle{noname}

Стратегия решения шахматных задач описана в ряде книг: старенькая брошюра Гуляева А. П ``Как решать шахматные задачи'' \cite{Guleaev}, брошюра Владимирова Я. Г. ``Как решать задачи и этюды'' из серии ``Библиотечка Шахматиста'' \cite{Vladimirov} и других. Данная статья не претендует на изложение новых принципов, не перечисленных в указанных книгах, или же на более подробное и понятное описание. Однако советы, по очень плохой привычке, всё равно приводятся.

Приводимые советы относятся к решению в целом и не затрагивают стратегию решения задач на различных конкурсах.

\begin{multicols}{2}[]
Чаще всего решателю попадаются задачи так называемого популярного жанра. Этот жанр предполагает легкую позицию (обычно число фигур не превышает 16), неожиданный первый ход (без шаха; предоставляющий чёрному королю поле, два и более; открывающий белого короля для шахов; уводящий фигуру от театра действий и т.д.) и разнообразие вариантов. Впрочем, не бывает правил без исключений. Однако, исключения для того и существуют, чтобы подтверждать правила, и мы их рассматривать не будем. Поэтому перейдем непосредственно к советам и некоторым примерам.

\subsubsection*{Ищите мат}

Один из самых распространённых и действенных приёмов решения задачи -- поиск матовых картин. Обычно чёрный король находится в своеобразной клетке, изредка есть один-два свободных поля. Первым делом необходимо посмотреть, куда можно поставить какую-нибудь фигуру, чтобы достроить мат. Также надо проверить, можно ли сделать маты на ходы короля. Попробуем просмотреть ход мыслей человека, решающего задачу на диаграмме 1.

\begin{center}
\begin{diagram}%
\author{Кройтор, Михаил Васильевич}%
  \pieces[5+5]{wSc8, wDb6, sBg6, sKe5, wKg4, sLa3, wLb3, wBd3, sBe3, sSg2}%
  \stipulation{\#2}%
\end{diagram}%
\end{center}\index{Кройтор Михаил}

Чёрный король запатован. Не видно способа, позволяющего согнать чёрного слона с диагонали a3-f8. Понятно, что ни конем, ни слоном белые мат поставить не в состоянии: белого коня держит удачно чёрный слон, а белый слон не того цвета. Остается попробовать подвигать ферзем. Про пешку думать не будем, к ней можно будет вернуться, если с ферзем не повезет. На какие поля надо его можно поместить, чтобы получить мат? Сразу становится видно, что будь ферзь на большой диагонали (поля c3, g7, h8) то чёрному королю стоял бы мат. Однако прямой выпад \italic{1. \queen{}d8?} с угрозой \italic{2. \queen{}h8\mate{}} не помогает, так как чёрные перекрывают ферзя \italic{1... \bishop{}f8!} и мата нет. Попробуем предоставить чёрному королю поле.  Если чёрный король стоит на d4, то не видно, куда может приземлиться ферзь для мата. А если \italic{1... \king{}f6? 2. \queen{}d4\mate{}}! Появляется идея \bold{1. \queen{}a7!} с угрозой \bold{2. \queen{}g7\mate{}} и вариантом \bold{1... \king{}f6 2.\queen{}d4\mate{}}. А если \bold{1... \bishop{}e7(f8)}? Тогда открывается вертикаль `а' и становится возможным мат \bold{2. \queen{}а1\mate{}}! Больше защит у чёрных нет, задача решена. 

Хочется заметить, что мы не рассматривали вариант \italic{1. \queen{}с7+?}, и мотивировка этому двойная: во-первых, задачи редко начинаются с шаха, что сразу же позволяет нам отбросить целую группу ходов; во-вторых, предоставляется поле d4 (это более важный довод), с отходом на него чёрного короля мы не смогли найти мата.

\subsubsection*{Создайте угрозу}

\begin{center}
\begin{diagram}%
\author{Кройтор, Михаил Васильевич}%
  \pieces[5+7]{sLf7, sBd6, sTh6, sBd5, sKe5, wSd4, sBh4, wKb3, sBc3, wBf3, wSh3, wDd1}%
  \stipulation{\#2}%
\end{diagram}%
\end{center}\index{Кройтор Михаил}

Следующую задачу будет решить проще ввиду существования варианта \bold{1... \king{}f6}. Мат в этом случае будет при ферзе на g5. Иногда, конечно, белые подхватывают свободные поля, но в этом случае, обычно, предоставляются взамен другие. Пробуем -- \italic{1. \queen{}g1?} и сразу же выясняем, что на любой ход черными, например, слоном, мата нет. Поэтому следует обратиться ко второму совету: \italic{``создайте угрозу''}.

Еще одна возможность подхватить поле g5 ферзем: \bold{1. \queen{}с1!} создает угрозу \bold{2. \queen{}f4\mate{}}. Этим ходом предоставляется чёрному королю поле (жертвуется конь): \bold{1... \king{}xd4 2. \queen{}xc3\mate{}}. На другие защиты от угрозы также находятся маты: \bold{1... \king{}f6 2. \queen{}g5\mate{}, 1... \rook{}f6 2. \queen{}e3\mate{}}!

\begin{center}
\begin{diagram}%
\author{Кройтор, Михаил Васильевич}%
  \pieces[6+6]{sTe7, sSg6, wTg5, sTh5, wDa4, wBd4, sKf4, sBa3, wTg3, wSf2, wKa1, sLb1}%
  \stipulation{\#2}%
\end{diagram}%
\end{center}\index{Кройтор Михаил}

На диаграмме 3 чёрный король опять ограничен, и в этом приняли участие все фигуры белых за исключением ферзя (пешка -- не фигура, а король...). Потому поищем, где бы белый ферзь стоял удобнее всего. Кандидат номер один -- поле f3. Пробуем: \italic{1. \queen{}b3? -- 2. \queen{}f3\mate{}}, но \italic{1... \bishop{}d3!}, перекрывая белого ферзя слоном. \italic{1. \queen{}d1? -- 2. \queen{}f3\mate{}}, но \italic{1... \rook{}е2!}, перекрывая его же ладьей. Наконец, третья, последняя попытка -- \bold{1. \queen{}с6! -- 2. \queen{}f3\mate{}}, -- оказывается решением. Возникают варианты с занимательной мотивацией: перекрытие ферзя слоном приводит к перекрытию черной ладьи, чем белые и пользуются, и наоборот: \bold{1... \bishop{}e4 2. \queen{}c1\mate{}; 1... \rook{}e4 2. \queen{}f7\mate{}}. Подобное перекрытие двух разнородных фигур без жертвы на критическом поле называется \textit{перекрытием Гримшоу}, в честь ее изобретателя. Кстати, тут можно дать еще один совет, касательно самого решения.

\textit{Любому композитору приятно, когда указывают не просто первый ход решения, но и все варианты, ложные следы (это и есть те наши начальные попытки решения), иллюзорную игру. Желательно отделять главные варианты от дополнительных.}

Впрочем, этот совет не применим на конкурсах решения.

Главные варианты мы уже указали, осталось дело за дополнительными: \bold{1. ... \rook{}е3 / \knight{}e5 / \knight{}h4 2. \rook{}3g4 / \queen{}с1 / \knight{}h3\mate{}}.

Теперь перейдем к более сложным вещам -- к трехходовкам. Первую задачу попробуем решить сходу, без всяких советов.

\begin{center}
\begin{diagram}%
\author{Кройтор, Михаил Васильевич}%
  \pieces[7+2]{sBd7, wLg7, wKd6, wBd4, wBf4, sKe3, wBg3, wDc2, wBg2}%
  \stipulation{\#3}%
\end{diagram}%
\end{center}\index{Кройтор Михаил}

Первое, что бросается в глаза, это патовая ситуация. Кстати, если бы не было пешки d4 -- шел бы короткий мат (\italic{1. \king{}d5 d6 2. \bishop{}d4\mate{}}). Вообще, пешка эта какая-то лишняя -- надо будет попытаться ее отдать. Без пешки и мат становится виден -- слоном, например, с поля с5. Появляется идея: \italic{1. \king{}с7? d5 2. \bishop{}f8 \king{}d4 3. \bishop{}c5\mate{}}; но \italic{1... d6!} и мата не видно. Можно сразу распатовать короля -- \bold{1. \bishop{}f8! \king{}xd4}, но что делать теперь? Если король чёрных вернется на е3, то мат будет слоном... Значит, нужно слона открыть! \bold{2. \king{}xd7! \king{}e3 3. \bishop{}c5\mate{}}; -- основной вариант. И дополнительный мат -- \bold{2... \king{}d6 3. \queen{}d3\mate{}!} Задача решена!

\begin{center}
\begin{diagram}%
\author{Кройтор, Михаил Васильевич}%
  \pieces[4+3]{wDd8, wSg7, sKe5, sBc4, sBc3, wBc2, wKf2}%
  \stipulation{\#3}%
\end{diagram}%
\end{center}\index{Кройтор Михаил}

Следующая задача легко решается, если мы начинаем искать матовые позиции. Было бы просто замечательно сыграть \italic{1. \king{}е3?} с угрозой \bold{2. \queen{}d4\mate{}}, но выясняется, что черным пат! А куда может чёрный король идти? У него в распоряжении доступны поля е4 и f4. При короле на f4 находится мат ферзем с поля f5. А при короле на е4? Если приземлить ферзя на с5, то получается мат в 2 хода! Следующий вариант: \italic{1... \king{}е4 2. \queen{}с5! \king{}f4 3. \queen{}f4\mate{}}! Возникает только вопрос, с какого поля на с5 пришел ферзь? Сразу можно отбросить различные шахи, поэтому единственный ход, внушающий доверие, это \bold{1. \queen{}b6!} -- цугцванг! -- чёрному королю предоставляется поле d5, на которое легко находится мат \bold{1... \king{}d5 2. \king{}e3 \king{}e5 3. \queen{}e6\mate{}}. И последний вариант, на оставшееся отступление короля: \bold{1... \king{}f4 2. \queen{}e3+ \king{}g4 3. \queen{}g3\mate{}}! Как видно, задача во всех трех вариантах завершается одинаковыми матовыми картинами: такое однообразие называется \textit{эхоматами}.

\begin{center}
\begin{diagram}%
\author{Кройтор, Михаил Васильевич}%
  \pieces[5+4]{wKb8, wLb7, sBc7, wSe7, sBg5, sBc4, sKd4, wDg3, wBb2}%
  \stipulation{\#3}%
\end{diagram}%
\end{center}\index{Кройтор Михаил}

Попробуем решить еще одну задачу. На диаграмме 6 у чёрного короля есть свободное поле с5. При уходе на него мата не видно. Если попробовать отобрать его, например, ходом \italic{1. \queen{}а3?}, тогда, после \italic{1... \king{}е5} чёрный король уходит в большое плавание. Значит, необходимо построить мат на этот отход короля. Следует заметить, что уже есть один заготовленный мат на ход \bold{1... с3 2. \queen{}:с3\mate{}}. А вот ход \bold{1... с6} перекрывает белого слона, что дает чёрному королю 2 дополнительных поля и, соответственно, повышает шансы продлить агонию. Потому первый ход попробуем сделать слоном, и лучше всего на с6. В этом случае мы получаем множество выгод: сами перекрываем пешку с7; уводим из-под возможного перекрытия слона и, что важно в первую очередь, подхватываем поле b5. Теперь уже появляется мат на отход короля на свободное поле! \bold{1. \bishop{}с6!} Цугцванг! \bold{1... \king{}с5 2. \queen{}е5+ \king{}b6(b4) 3. \queen{}b5\mate{}}! А если черные сыграют \bold{1... g4}? Ведь у чёрных все равно есть свободное для короля поле... Однако и это им не поможет! \bold{1... g4 2. \king{}xc7} цугцванг \bold{2... \king{}с5 3. \queen{}d6\mate{}; 2... c3 3. \queen{}xс3\mate{}}. 

Задача решена! В ней была своя сложность, а именно, создание позиций цугцванга. Учитывая, что маты на ходы чёрных отсутствовали (кроме, разве что, на ход пешкой с4), то все осложнялось построением матовых картин. Вообще говоря, задачи на цугцванг, требующие построения матовых картин, являются очень сложными для решения. Особенно тяжело найти правильный ответ в случае, когда есть \italic{иллюзорная игра}, то есть существуют маты на ходы чёрных в начальной позиции, и после вступительного хода маты на те же ходы чёрных меняются.

\subsubsection*{Знайте врага в лицо}

Как уже было отмечено, при решении задач очень помогает знание задачных тем. Эти знания позволяют автоматически определять в задаче механизмы, которые реализуют ту или иную тему. Поэтому полезно знать такие магические слова (и их смысл) как \italic{перекрытие Гримшоу, тема Новотного, альбино, пикенини, сдвоение Тёртона, коневое колесо, аннигиляция} и много других. Уже была отмечена тема перекрытия Гримшоу, в котоой происходит взаимное перекрытие двух разноходящих фигур. Родственной этой теме является перекрытие Новотного, в которой перекрытие разноходящих фигур осуществляется за счёт жертвы.

\begin{center}
\begin{diagram}%
\author{Кройтор, Михаил Васильевич}%
  \pieces[7+6]{wKa8, wSc7, wSf7, sBc6, sBd6, wDa5, sLe5, sKc4, sBb3, sTd3, wLb2, wBd2, wBe2}%
  \stipulation{\#2}%
\end{diagram}%
\end{center}\index{Кройтор Михаил}

В задаче на диаграмме 7 сразу можно заметить, что чёрные слон и ладья могут перекрывать друг друга на поле d4. Поэтому пытаемся сразу же пожертвовать на этом поле фигуру: \bold{1. \bishop{}d4} с угрозой \bold{2. \queen{}a4\mate{}}. Других матов нет. Однако, на взятие слона они появляются: \bold{1... \bishop{}xd4 2. \knight{}xd6\mate{}, 1... \rook{}xd4 2. \queen{}c3\mate{}}. Кстати, маты эти стали возможны не только потому, что перекрылась одна фигура, но, также, и потому, что вторая фигура была отвлечена. И дополнительный вариант: \bold{1... \king{}xd4 2. \queen{}b4\mate{}}.

Вариации перекрытия Новотного бывают самые различные, например:

\begin{itemize}
\setlength\itemsep{-0.25em}
\item маты при принятии жертвы сохраняются; 
\item при жертве фигуры грозят одни маты, а при её принятии -- другие;
\item в начальной позиции жертвуемая фигура уже стоит на точке пересечения линий чёрных фигур. 
\end{itemize}

И зная их особенности, легко можно в рисунке задачи выделить механизмы образования этих перекрытий и, соответственно, решить задачу.

\subsubsection*{Проверьте роль каждой фигуры}

\begin{center}
\begin{diagram}%
\author{Кройтор, Михаил Васильевич}%
  \pieces[6+3]{sSb7, wLb6, sKe5, wSg5, sSb4, wTd4, wSc3, wTf3, wKd2}%
  \stipulation{\#2}%
\end{diagram}%
\end{center}\index{Кройтор Михаил}

Одним из принципов шахматной композиции является принцип экономичности, который, в частности, означает, что все фигуры должны принимать участие в игре. Конечно, существуют так называемые технические фигуры: фигуры, которые убирают побочные решения и дуали. Технические фигуры обычно легко определить, но роль некоторых не так очевидна, и если при решении определить её роль, то сразу же становится ясным её решение.

Например, на диаграмме 8 в задаче сразу видно, что чёрный король запатован. И в ограничении участвуют все фигуры, однако конь с3 и ладья d4 избыточно держат поля e4 и d5 (есть, конечно, вероятность, что задача на цугцванг, тогда могут ходить и слон, и вторая ладья). Если ходить конём, то угроза не образуется и на ход конём b4 мата не видно. Попытаемся ходить ладьёй, тем более, что у неё только одно безопасное место: \bold{1. \rook{}d7} Сразу же видна угроза \bold{2. \bishop{}d4\mate{}} Если чёрные пытаются подхватить поле d4, то образуются новые маты; \bold{1... \knight{}d6 2. \rook{}e7\mate{}; 1... \knight{}c5 2. \bishop{}c7\mate{}; 1... \knight{}c6 / \knight{}c2 2. \rook{}d5\mate{}}.\\

Мы не рассматривали задачи со сложными защитами чёрных -- ограничились очень маленькой областью композиции. Однако, принципы решения, указанные мной, легко применять для любой задачи, независимо от ее сложности. Повторю их: первым делом проанализируйте позицию, посмотрите возможные защиты чёрных. Попробуйте на эти защиты сконструировать маты. Если получилось -- посмотрите, как матующая фигура (или фигура, участвующая в построении мата) добирается до этого поля -- у вас сразу же появится определенная группа ходов, и только один из них не будет иметь опровержения. Наконец, оцените загруженность фигур -- это поможет выявить кандидата на первый ход (это делается на стадии анализа). Очень помогает в решении композиций знакомство с ее идеями, темами, формулировками (врага надо знать в лицо!). 

Успехов в познавании шахматного искусства!
\end{multicols}
