\subsection*{Кооперативные маты}
\markright{}
\addcontentsline{toc}{subsection}{Кооперативные маты}
\dianamestyle{noname}
\begin{diagram}%
  \author{Туревский, Дмитрий Евгеньевич; Кройтор, Михаил Васильевич}%
  \year{2012}%
  \award{2nd HM}\tournament{Ясиноватая 21}%
  \pieces[3+6]{sLc7, wBe7, sSe6, wSf6, sLc4, sKd4, sBd3, sSe3, wKb2}%
  \stipulation{h\mate{}2 \ 2 Sol.}%
  \solution{%
    \bold{1.\knight{}d8 exd8=\bishop{} 2.\bishop{}e5 \bishop{}b6\mate; \\
    1.\bishop{}d8 exd8=\knight{} 2.\knight{}c5 \knight{}c6\mate{}}
  }%
  \themes{%
    White underpromotion
  }%
  \comment{%
    Варианты задачи связаны переменой превращений пешки на жертвы чёрных. Составлена совместно с Туревским Дмитрием.
  }%
\end{diagram}%
\index{Туревский Дмитрий}\index{Кройтор Михаил}
\hfill
\begin{diagram}%
  \author{Кройтор, Михаил}%
  \year{2016}%
  \award{5th Сomm.}\tournament{Гравюра}%
  \pieces[3+6]{wTf8, wKh7, sSf5, sSf4, sKh4, sTg3, sTh3, wTf1, sDg1}%
  \stipulation{h\mate{}2 \ 2 Sol.}%
  \solution{%
    \bold{1.\knight{}h5 \rook{}f4+ 2.\king{}g5 \rook{}xf5\mate{}; \\
    1.\knight{}h6 \rook{}f5 2.\king{}g4 \rook{}xf4\mate{} }
  }%
  \themes{%
    Umnov, Check avoidance
  }%
  \comment{%
    В задаче мне хотелось выполнить чередование белых ходов. Этого удалось добиться предупреждением шаха белому королю.
  }%
\end{diagram}%
\index{Кройтор Михаил}
\hfill
\begin{diagram}%
  \author{Кройтор, Михаил}%
  \year{2016}%
  \award{3rd Place}\tournament{Cupa Bucovinei}%
  \pieces[4+5]{sBc6, sLd4, sKe4, wSd3, sLh3, wDb2, wKa1, wLb1, sTh1}%
  \stipulation{h\mate{}2 \ 4 Sol.}%
  \solution{%
    \bold{1.\bishop{}g1 \queen{}b4+ 2.\king{}d5 \bishop{}a2\mate{}; \\
    1.\bishop{}e3 \queen{}h2 2.\king{}f3 \knight{}e5\mate{}\\
    1.\king{}d5 \king{}a2 2.\king{}c4 \queen{}b3\mate{}; \\
    1.\king{}f3 \queen{}xd4 2.\king{}g2 \queen{}f2\mate{}}
  }%
\end{diagram}%
\index{Кройтор Михаил}
\hfill
\begin{diagram}%
  \author{Кройтор, Михаил; Гургуй, Дан-Константин}%
  \year{2016}%
  \award{3rd Place}\tournament{Trofeul 'e4 e5'}%
  \pieces[4+3]{wKe8, wSd6, sBf6, sKd5, wTc4, sSf4, wBe2}%
  \stipulation{h\mate{}2 \ 2 Sol.}%
  \solution{%
    \bold{1.\knight{}d3 \king{}e7 2.\knight{}e5 e4\mate{};\\
    1.\king{}e5 \king{}d7 2.\knight{}d5 Re4\mate{}}
  }%
  \comment{%
    Составлен совместно с Гургуй Дан-Константин, румынским композитором.
  }%
\end{diagram}%
\index{Кройтор Михаил}\index{Гургуй Дан-Константин}
\hfill
\begin{diagram}%
  \author{Кройтор, Михаил}%
  \source{Vratnica-64}\year{2018}%
  \tournament{Hadzi-Vaskov 70 JT}%
  \pieces[4+7]{sTe8, sLa7, sLa6, sBg6, wDh6, wBg5, wBe2, sBh2, wKc1, sKe1, sDh1}%
  \stipulation{h\mate{}2 \ 2 Sol.}%
  \solution{%
    \bold{1.\bishop{}xe2 \queen{}xh2 2.\bishop{}f1 \queen{}d2\mate{};\\
    1.\rook{}xe2 \queen{}h5 2.\rook{}f2 \queen{}d1\mate{}}
  }%
  \themes{%
    Ambush, Annihilation
  }%
  \comment{
    В обоих решениях чёрные уничтожают белую пешку, мешающую мату, а чёрные становятся в засаду.
  }
\end{diagram}%
\index{Кройтор Михаил}
\hfill
\begin{diagram}%
  \author{Кройтор, Михаил; Білокінь, Юрій Володимирович}%
  \source{Ideal-Mate Review}\year{2010}%
  \pieces[3+4]{wKh7, sDe5, wTf5, sLd4, sKe4, wBf2, sLb1}%
  \stipulation{h\mate{}3 \ 2 Sol.}%
  \solution{%
    \bold{1.\bishop{}d3 \king{}h6 2.\bishop{}e3+ \king{}g6 3.\queen{}d4 f3\mate{};\\ 
    1.\queen{}d5 \rook{}f6 2.\king{}e5+ \king{}g7 3.\bishop{}e4 f4\mate{}}
  }%
  \comment{%
    Составлено совместно с Білокінь Юріем.
  }%
\end{diagram}%
\index{Кройтор Михаил}\index{Білокінь Юрій}
\hfill
\begin{diagram}%
  \author{Кройтор, Михаил}%
  \year{2012}%
  \award{Special HM}\tournament{Ясиноватая-21}%
  \pieces[3+5]{sTd8, sTd7, sBf7, wBf6, sLc5, sKe4, wLf4, wKg4}%
  \stipulation{h\mate{}3 \ 2 Sol.}%
  \solution{%
    \bold{1.\bishop{}e7 fxe7 2.\rook{}d3 e8=\knight{} 3.\rook{}d4 \knight{}f6\mate{};\\
    1.\rook{}e7 fxe7 2.\rook{}d3 e8=\bishop{} 3.\bishop{}d4 \bishop{}c6\mate{}}
  }%
\end{diagram}%
\index{Кройтор Михаил}
\hfill
\begin{diagram}%
  \author{Кройтор, Михаил}%
  \source{Mat Plus}\year{2010}%
  \pieces[4+3]{wKh8, wBa3, sKc3, sSc2, wTa1, wLc1, sSf1}%
  \stipulation{h\mate{}4 \ 2 Sol.}%
  \solution{%
    \bold{1.\knight{}b4 \bishop{}d2+ 2.\king{}b2 \rook{}xf1 3.\knight{}a2 \bishop{}e1 4.\king{}a1 \bishop{}c3\mate{}; \\
    1.\knight{}xa3 \bishop{}e3 2.\knight{}b1 \rook{}a8 3.\king{}b2 \bishop{}a7 4.\king{}a1 \bishop{}d4\mate{}}
  }%
  \themes{%
    Miniature
  }%
\end{diagram}%
\index{Кройтор Михаил}
\hfill
\begin{diagram}%
  \author{Кройтор, Михаил; Жилко, Дмитрий Витальевич}%
  \year{2011}%
  \award{Сomm.}\tournament{``Белоконь-60''}%
  \pieces[3+4]{sKc6, sSe5, wKb4, wBd4, wBc3, sTe3, sLh2}%
  \stipulation{h\mate{}4 \ 2 Sol.}%
  \solution{%
    \bold{1.\knight{}d7 c4 2.\bishop{}c7 c5 3.\rook{}b3+ \king{}c4 4.\rook{}b7 d5\mate{} \\
    1.\king{}d5 \king{}b3 2.\knight{}c6 \king{}c2 3.\rook{}e6 \king{}d3 4.\bishop{}d6 c4\mate{}}
  }%
  \comment{%
    Составлен совместно с Жилко Дмитрием.
  }%
\end{diagram}%
\index{Кройтор Михаил}\index{Жилко Дмитрий}
\hfill
\begin{diagram}%
  \author{Кройтор, Михаил}%
  \year{2009}%
  \award{5th Prize}\tournament{SuperProblem}%
  \pieces[2+3]{wKh8, sDa3, sKb3, wDa1, sDc1}%
  \stipulation{h\mate{}4}%
  \solution{%
    \bold{1.\king{}c4 \king{}g7 2.\queen{}d3 \queen{}a5 3.\king{}d4 \king{}f6 4.\queen{}c4 \queen{}e5\mate{}}
  }%
\end{diagram}%
\index{Кройтор Михаил}%
\hfill
\begin{diagram}%
  \author{Кройтор, Михаил}%
  \year{2012}%
  \award{2nd HM}\tournament{8 ECSC}%
  \pieces[6+6]{sBb4, sBf4, sBa3, wBb3, wBf3, wBa2, wTb2, sTc2, sBd2, wBf2, wKa1, sKc1}%
  \stipulation{h\mate{}4 \ 2 Sol.}%
  \solution{%
    1... \rook{}b1\mate{} \\
    \bold{1.\king{}d1 \king{}b1 2.\king{}e2 \king{}:c2 3.\king{}e1 \king{}d3 4.\king{}d1 \rook{}b1\mate{} \\
    1.\rook{}c3 \rook{}b1+ 2.\king{}c2 \rook{}e1 3.\rook{}d3 \rook{}e4 4.\king{}c3 \rook{}c4\mate{}}
  }%
  \comment{%
    Тема конкурса: в начальной позиции белые могут поставить мат в 1 ход. В действительном решении нет выжидательного хода, поэтому мат меняется.
  }%
\end{diagram}%
\index{Кройтор Михаил}%
\hfill
\begin{diagram}%
  \author{Кройтор, Михаил}%
  \source{SuperProblem}\year{2012}%
  \award{7th Place}\tournament{TT-32}%
  \pieces[7+10]{sTa8, sSd8, sKe8, sBb6, sBg5, sBh5, sBd4, sBa3, wBb3, sBc3, wBd3, sBe3, wBa2, wBc2, wBe2, wTa1, wKe1}%
  \stipulation{h\mate{}5}%
  \solution{%
    \bold{1.\knight{}b7 \OOO 2.\OOO \rook{}f1 3.\king{}b8 \rook{}f6 4.\king{}a8 \rook{}xb6 5.\rook{}b8 \rook{}a6\mate{}}
  }%
  \themes{%
    Castling
  }%
  \index{Кройтор Михаил}
  \comment{%
    Тема конкурса: рокировка за обе стороны.
  }%
\end{diagram}%
\hfill
\begin{diagram}%
  \author{Кройтор, Михаил}%
  \source{Mat Plus}\year{2009}%
  \pieces[4+6]{sBe7, sBh7, wBe6, sBd5, wKh5, sBd4, sBf4, wBd3, wBf2, sKg1}%
  \stipulation{h\mate{}7}%
  \solution{%
    \bold{1.h6 \king{}g6 2.\king{}f1 \king{}f7 3.\king{}e2 \king{}xe7 4.\king{}xd3 \king{}f6 5.\king{}e4 e7 6.d3 e8=\bishop{} 7.d4 \bishop{}c6\mate{}}
  }%
  \themes{%
    Meredith
  }%
\end{diagram}%
\index{Кройтор Михаил}
\hfill
\begin{diagram}%
  \author{Кройтор, Михаил}%
  \source{Шахматная композиция}\year{2012}%
  \pieces[4+4]{wKf8, sBb5, sBb4, wBb3, sBc3, wLa2, wBc2, sKa1}%
  \stipulation{h\mate{}8}%
  \solution{%
    \bold{1.\king{}b2 \king{}e7 2.\king{}xc2 \king{}d6 3.\king{}b2 \king{}c5 4.c2 \king{}xb4 5.c1=\bishop{} \king{}c5 6.\king{}a3 b4 7.\king{}a4 \bishop{}b1 8.\bishop{}a3 Bc2\mate{}}
  }%
\end{diagram}%
\index{Кройтор Михаил}
\hfill
\begin{diagram}%
  \author{Кройтор, Михаил}%
  \source{Mat Plus}\year{2010}%
  \pieces[4+4]{sBb5, sBb4, sKe4, wBb3, sBd3, wBb2, wBd2, wKd1}%
  \stipulation{h\mate{}10}%
  \solution{%
    \bold{1.\king{}f3 \king{}c1 2.\king{}e2 \king{}b1 3.\king{}xd2 \king{}a1 4.\king{}c2 \king{}a2 5.d2 \king{}a1 6.d1=\queen{}+ \king{}a2 7.\queen{}d8 \king{}a1 8.\king{}xb3 \king{}b1 9.\king{}a4 \king{}a2 10.\queen{}a5 b3\mate{}}
  }%
  \themes{%
    Meredith, Kindergarten
  }%
\end{diagram}%
\index{Кройтор Михаил}
\hfill

\pagebreak
\subsection*{Решения}
\markright{}
\addcontentsline{toc}{subsection}{Решения}
\dianamestyle{fullname}
\putsol
