\subsection*{Мат в 2 хода}
\markright{}
\addcontentsline{toc}{subsection}{Мат в 2 хода}
\diagramxii
\dianamestyle{noname}
\begin{diagram}%
  \author{Кройтор, Михаил}%
  \source{Задачи и этюды}\year{2002}%
  \award{Special HM}%
  \pieces[5+4]{wSf7, wKc6, wLh6, sKd4, sBf3, wBg3, wDc2, sTd1, sLe1}%
  \stipulation{\mate2}%
  \solution{%
    \bold{1.\knight{}g5! {\textasciitilde} 2.\knight{}xf3\mate; 1... \king{}e3 2.\knight{}e6\mate; 1... \king{}e5 2.\bishop{}g7\mate; 1... \rook{}d3 2.\queen{}c5\mate; 1... \bishop{}c3 2.\queen{}e4\mate}
  }%
  \comment{%
    Первым ходом белые предоставляют чёрному королю два свободных поля. Первая опубликованная задача. На самом деле, чёрные ладья и слон могут быть сняты с доски. Эти фигуры были установлены, чтобы добавить ещё два варианта решения, от чего, на мой взгляд, задача только выиграла.
  }%
  \themes{%
    2 flights giving key
  }%
\end{diagram}%
\index{Кройтор Михаил}
\hfill
\begin{diagram}%
  \author{Кройтор, Михаил}%
  \source{Рыбинск 7 дней}\year{2006}%
  \award{2nd Сomm.}%
  \pieces[5+2]{wSf7, sBg6, wKh6, sKg4, wBg3, wSd2, wDh2}%
  \stipulation{\mate2}%
  \solution{%
    1.\knight{}g5?? {\textasciitilde} 2.\queen{}h3\mate; 1... \king{}f5!\\    
    1.\knight{}d6?? {\textasciitilde} 2.\queen{}h4\mate; 1... g5!\\
    1.\king{}h7?? Цугцванг 1...g5 2.\knight{}h6\mate; 1... \king{}f5!\\    
    \bold{1.\king{}g7! Цугцванг 1... \king{}f5 2.\queen{}h3\mate; 1... g5 2.\knight{}h6\mate}
  }%
  \comment{%
  Красивая популярная задача. Была составлена под воздействием знаменитых миниатюр Куббеля, Пустового.\\
  \chessboard[setfen=8/8/8/8/8/8/8/8 w - - 0 1]
  }%
\end{diagram}%
\index{Кройтор Михаил}
\hfill
\begin{diagram}%
  \author{Кройтор, Михаил}%
  \source{Mat Plus}\year{2007}%
  \pieces[4+4]{wSe8, wKf6, sBc5, sKd5, sBe5, wLa4, sBd4, wDd3}%
  \stipulation{\mate2}%
  \solution{%
    \bold{1.\queen{}g3 ! Цугцванг\\
       1... d3 2.\queen{}xd3\mate; 1... c4 2.\queen{}xe5\mate; 1... \king{}c4 2.\queen{}b3\mate; 1... \king{}e4 2.\bishop{}c6\mate; 1... e4 2.\queen{}g8\mate}
  }%
  \comment{%
  Первым ходом белые предоставляют чёрному королю 2 поля. Следует отметить маты на блокирование этих полей и игру чёрного короля на эти поля. 
  }%
\end{diagram}%
\index{Кройтор Михаил}
\hfill
\begin{diagram}%
  \author{Кройтор, Михаил}%
  \source{Mat Plus}\year{2009}%
  \pieces[8+6]{sLb8, wSb7, sSg6, sTd5, sKe5, sSb4, wSd4, sLe4, wTf4, wTf3, wLa2, wBd2, wKh2, wDg1}%
  \stipulation{\mate2}%
  \solution{%
    \bold{1.\rook{}e3! {\textasciitilde} 2.\rook{}exe4\mate; 1... \king{}xf4 2.\queen{}g3\mate; 1... \king{}xd4 2.\queen{}a1\mate; 1... \knight{}xf4 2.\queen{}g7\mate; 1... \rook{}xd4 2.\queen{}g5\mate}
  }%
  \themes{%
    2 flights giving key
  }%
  \comment{%
    Жертва двух фигур. Интересна роль чёрного слона b8, который в ряде попыток связывает ладью f4. Задача в таком виде была представлена питерским композитором Петром Забирохиным. Первоначальная редакция задачи следующая:\\
    \chessboard[setfen=8/8/8/8/8/8/8/8 w - - 0 1]
  }%
\end{diagram}%
\index{Кройтор Михаил}
\hfill
\begin{diagram}%
  \author{Кройтор, Михаил}%
  \source{ChessStar}\year{2012}%
  \award{1st-2nd Prize}%
  \pieces[4+3]{wKa7, sBc7, sBd7, sKc6, wBd6, wTg6, wDg5}%
  \stipulation{\mate2}%
  \twins{%
    b) Move g6 f6
  }%
  \solution{%
    a) \bold{1.\rook{}g8! 1... \king{}xd6 2.\rook{}g6\mate; 1... cxd6 2.\rook{}c8\mate} \\
    b) \bold{Move g6 \ra f6}: \bold{1.\rook{}f5! 1... \king{}xd6 2.\queen{}f6\mate; 1... cxd6 2.\queen{}c1\mate}
  }%
  \comment{%
    Перемена матов в форме близнецов. Задача учавствовала (и была отмечена) в конкурсе миниатюр.
  }%
\end{diagram}%
\index{Кройтор Михаил}
\hfill
\begin{diagram}%
  \author{Кройтор, Михаил}%
  \source{ChessStar}\year{2014}%
  \award{3rd Сomm., miniatures}%
  \pieces[4+2]{wDd7, sBc5, wKd5, wLh5, sKe3, wTc1}%
  \stipulation{\mate2}%
  \solution{%
    1. \queen{}g7! Цугцванг \\
    1... c4 2. \queen{}d4\mate; 1... \king{}d3 / d2 2. \queen{}c3\mate; 1... \king{}f2 2. \queen{}g1\mate; 1... \king{}f4 2. \queen{}e5\mate
  }%
  \comment{%
  }%
\end{diagram}%
\index{Кройтор Михаил}
\hfill
\begin{diagram}%
  \author{Кройтор, Михаил}%
  \source{E4 E5}\year{2016}%
  \award{3rd Place}\tournament{Cupa Bucovinei 2015}%
  \pieces[9+7]{sSb7, wKc7, wBd7, wSg7, sBc6, sKd5, wTa4, wBd4, wTf4, sBg4, wLc3, sDf3, wDd2, sLb1, sSg1, wLh1}%
  \stipulation{\mate2}%
  \solution{%
    1.\rook{}e4! {\textasciitilde} 2.\rook{}e5\mate; 1... \queen{}xe4 2.\queen{}g5\mate; 1... \bishop{}xe4 2.\queen{}a2\mate; 1... \king{}xe4 2.d5\mate
  }%
  \comment{%
    Задача была составлена еще школьником и опубликована через 15 лет. Тематическое развязывание чёрного ферзя.
  }%
\end{diagram}%
\index{Кройтор Михаил}
\hfill
\begin{diagram}%
  \author{Кройтор, Михаил}%
  \source{E4 E5}\year{2016}%
  \award{4th Place}\tournament{Cupa Bucovinei 2015}%
  \pieces[10+8]{sBe6, wSf6, sTa5, wTb5, wLc5, sKe5, wBg5, wTb4, wBg4, sBc3, sLe3, sBf3, wKh3, sLc2, wBd2, wSf2, sTa1, wDc1}%
  \stipulation{\mate2}%
  \solution{%
    1.d3? {\textasciitilde} 2.\rook{}e4\mate / \queen{}xe3\mate, но 1... \bishop{}d4! \\
    1.\rook{}d4? {\textasciitilde} 2.\knight{}d7\mate, но 1... \bishop{}xd4! \\
    1.\knight{}2e4! {\textasciitilde} 2.\bishop{}d6\mate; 1... \bishop{}xe4 2.\rook{}xe4\mate; 1... \king{}f4 2.\bishop{}xe3\mate; 1... \bishop{}xc5 2.d4\mate
  }%
  \comment{%
  Опять же, задача была составлена в школьные годы.
  }%
\end{diagram}%
\index{Кройтор Михаил}
\hfill
\begin{diagram}%
  \author{Кройтор, Михаил}%
  \year{2016}%
  \award{1st HM}\tournament{Гравюра-2016}%
  \pieces[5+3]{wTe8, wKh8, wSg6, sBd5, sKd4, sLe2, wSf2, wDc1}%
  \stipulation{\mate2}%
  \solution{%
    1... \bishop{}c4 2.\queen{}e3\mate\\    
    1.\rook{}e3? {\textasciitilde} 2.\queen{}c3\mate; 1... \bishop{}d3 2.\rook{}xd3\mate, но 1... \bishop{}c4!\\
    1.\knight{}e5? {\textasciitilde} 2.\knight{}c6\mate; 1... \bishop{}b5 2.\knight{}f3\mate, но 1... \bishop{}c4!\\
    1.\knight{}h4? {\textasciitilde} 2.\knight{}f5\mate; 1... \bishop{}c4 2.\queen{}e3\mate; 1... \bishop{}d3 2.\knight{}f3\mate, но 1... \bishop{}g4!\\      
    1.\knight{}e4! {\textasciitilde} 2.\queen{}c3\mate; 1... \bishop{}d3 2.\queen{}c5\mate; 1... dxe4 2.\rook{}d8\mate
  }%
  \comment{%
  Перемена матов.На протяжении 10 лет Игорь Агапов продвигал `'малый'' жанр: задачи с материалом от 8 до 10 фигур. Многие композиторы с большим удовольствием принимали участие в его конкурсах, потому что каждый составлял `'маленькие'' задачи, однако опасался конкурировать с масштабными полотнами. Конкурсы Гравюры позволили достать произведения из тетрадок. 
  }%
\end{diagram}%
\index{Кройтор Михаил}
\hfill
\begin{diagram}%
  \author{Кройтор, Михаил}%
  \year{2016}%
  \award{4th Сomm.}\tournament{Московский конкурс - 2016}%
  \pieces[7+6]{wLa8, wKb8, sBe7, wSa5, sTd5, sBe5, wTa4, sLb4, sKe4, wBb3, sSa2, wBc2, wDf2}%
  \stipulation{\mate2}%
  \solution{%
    1... \knight{}c1 2.\rook{}xb4\mate; 1... \knight{}c3 2.\rook{}xb4\mate \\
    1.\knight{}b7? {\textasciitilde} 2.\knight{}c5\mate; 1... \rook{}d8+ 2.\knight{}xd8\mate; но 1... \rook{}b5! \\
    1.\knight{}c4! {\textasciitilde} 2.\knight{}d2\mate; 1... \bishop{}d6+ 2.\knight{}xd6\mate
  }%
  \comment{%
    Идея задачи в выборе развязывания чёрных фигур одной фигурой. Следующим шагом ожидается увеличение количества связок до трёх или даже четырёх.
  }%
\end{diagram}%
\index{Кройтор Михаил}
\hfill
\begin{diagram}%
  \author{Кройтор, Михаил}%
  \year{2016}%
  \award{1st Place}\tournament{Cupa Bucovinei}%
  \pieces[5+5]{wLg6, sBc5, sBd5, wTh4, sSb3, wBd3, sKe3, sBe2, wKg2, wDe1}%
  \stipulation{\mate2}%
  \solution{%
    1.d4? {\textasciitilde} 2.\queen{}c3\mate; 1... \knight{}c1 2.\queen{}xc1\mate; 1... \knight{}d2 2.\queen{}f2\mate / 2.\queen{}g1\mate, но 1... cxd4 ! \\
     1.\rook{}f4? Цугцванг 1... \knight{}a1 {(S~)} 2.\queen{}c1\mate; 1... \knight{}d2 2.\queen{}f2\mate; 1... \king{}xf4 2.\queen{}g3\mate; 1... d4 2.\rook{}f3\mate, но 1... c4! \\
      1.\queen{}c3! {\textasciitilde} 2.d4\mate; 1... e1=\queen{} {(e1~)} 2.\queen{}{\x}e1\mate; 1... \knight{}c1 {(\knight{}d4)} 2.\queen{}{\x}c1\mate; 1... d4 2.\rook{}e4\mate
  }%
  \comment{%
  В простой форме перемена вступительного и матующего ходов (d4? и \queen{}c3!) украшена интересным ложным следом с жертвой ладьи. Можно также говорить про возврат фигуры в решении, но обычно, в двухходовках, данные элементы носят случайный характер.
  }%
\end{diagram}%
\index{Кройтор Михаил}
\hfill
\begin{diagram}%
  \author{Кройтор, Михаил}%
  \year{2016}%
  \award{6th Place}\tournament{Trofeul "e4 e5" editia a II}%
  \pieces[7+7]{sBb7, wSe7, wKg7, sDa6, sBe6, wBc5, sKe5, sBd4, wTg4, sTe3, wLf3, wBf2, sLb1, wLc1}%
  \stipulation{\mate2}%
  \solution{%
    1.\bishop{}e4! {\textasciitilde} 2.\knight{}g6\mate; 1... \bishop{}xe4 2.f4\mate; 1... \rook{}xe4 2.\rook{}g5\mate; 1... d3 2.\bishop{}b2\mate
  }%
  \comment{%
  Идеальная форма перекрытия Новотного: первым ходом белые перекрывают две дальнобойные фигуры, грозит один мат. Принятие жертвы защищает от мата, но становятся доступным новые маты из-за перекрытия дальнобойных фигур.
  }%
\end{diagram}%
\index{Кройтор Михаил}
\hfill

\pagebreak
\subsection*{Решения}
\markright{}
\addcontentsline{toc}{subsection}{Решения}
\dianamestyle{fullname}
\putsol 
