Данная брошюра представляет моё творчество в шахматной композиции: в ней собраны статьи, когда либо написанные мной, и 120 произведений разных жанров, от задач до этюдов. Статьи были опубликованы в цифровом журнале ``ChessZone'' \cite{ChessZone}, на сайте Crestbook \cite{CrestBook}, в сообществе шахматных композиторов Ru-Chess-Art \cite{RuChessArt}. Если у диаграммы начальной позиции произведения не указан автор, значит позиция была составлена мной.

В возрасте 13 лет познакомился с задачей Дмитрия Кларка, которую в клубе никто не смог решить (\italic{Белые: \king{}е1, \queen{}a3, \knight{}а2; Черные: \king{}а1, \pawn{}\pawn{} b2, b3, мат в два хода}). С этой задачи началось знакомство с шахматной композицией. Однако, не имея знакомых любителей этой области шахмат, развивался я сам. Вкусы сформировались под влиянием книги ``Леонид Куббель'' Владимирова и Фокина \cite{Kubbel}. В начальный период было составлено более 500 задач и этюдов, однако уровень их был низок. И до первого знакомства с шахматным композитором я успел составить около 500 задач и этюдов, которые, в большинстве своём, благополучно после отбросил. Однако, ряд произведений были опубликованы, либо использовались как основа для других задач и этюдов.

Уже будучи студентом, заочно познакомился с Забирохиным Петром, шахматным композитором из Санкт-Петербурга. Он помог опубликовать в 2002 году две задачи в журнале ``Задачи и Этюды''. С тех пор было опубликовано около 200 задач и этюдов. Часть из них была составлена в соавторстве такими композиторами как Туревский Дмитрий, Дидух Сергей, Лебедев Василий и другими; и каждое общение с ними давало мне новый толчок для развития и развивало чувство прекрасного.

Активно публиковаться стал с 2006 года. В 2009 году познакомился с любителями шахматной композиции в Кишинёве, стал принимать участие в конкурсах решений. Став 5 раз подряд чемпионом Молдавии, занялся организацией и судейством соревнований по решению шахматных задач. 

% Предпочитаю популярный стиль: несложное содержание, яркое запоминающееся решение, легкая и естественная позиция. Темы для задач появляются в процессе решения других произведений. Этюды же составляются, чаще всего, по партиям: замечается интересная идея в результате анализа и оборачивается в художественную форму. 

Также отдельная благодарность молдавским шахматным композиторам Гинде Анатолию и уже ушедшему от нас Иванову Альберту.
