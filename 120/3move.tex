\subsection*{Мат в 3 хода}
\markright{}
\addcontentsline{toc}{subsection}{Мат в 3 хода}
\dianamestyle{noname}

\begin{diagram}%
  \author{Кройтор, Михаил}%
  \source{Задачи и этюды}\year{2002}%
  \pieces[5+4]{wLd6, sBd5, wSg5, sKd4, sLg4, sBc3, wLd3, wBc2, wKf2}%
  \stipulation{\mate{}3}%
  \solution{%
    1... \bishop{}{\textasciitilde} 2.\knight{}e6/f3\mate \\
    \bold{1.\bishop{}f1! {\textasciitilde} 2.\bishop{}d3; \\
    \tab 1... \bishop{}d1(f3,h5) 2.\knight{}e6+ \king{}e4 3.\bishop{}d3\mate{}, \\
    \tab 1... \bishop{}h3(f5,e6,d7,c8) 2.\knight{}f3+ \king{}e4 3.\bishop{}d3\mate,\\
    \tab 1... \bishop{}e2! 2.\king{}xf2! \king{}c4 3.\king{}e3\mate}
  }%
  \themes{%
    Anderssen's mate, Switchback, Indian
  }%
  \comment{%
  Первая опубликованая задача (вместе с указанной уже двухходовкой). Центральный вариант с построением белой королевской батареи. Мат такой батареей называется матом Андерссена, в честь знаменитой задачи:
  \chessboard[setfen=8/8/8/8/8/8/8/8 w - - 0 1]
  }%
\end{diagram}%
\index{Кройтор Михаил}
\hfill
\begin{diagram}%
  \author{Кройтор, Михаил}%
  \source{Mat Plus}\year{2007}%
  \pieces[3+3]{wDc6, wTf3, sBe7, sBe6, sKd4, wKg7}%
  \stipulation{\mate{}3}%
  \solution{%
    \bold{1.\rook{}g3!} цугцванг\\ 
    \tab \bold{1... \king{}e5 2.\rook{}g4 \king{}f5 3.\queen{}e4\mate; \\
    \tab 1... e5 2.\rook{}c3 e4 3.\queen{}c5\mate}
  }%
  \comment{%
  }%
\end{diagram}%
\index{Кройтор Михаил}
\hfill
\begin{diagram}%
  \author{Кройтор, Михаил}%
  \source{Mat Plus}\year{2007}%
  \pieces[5+2]{wSf8, wSa7, sSe7, sKb6, wTa3, wKb3, wDf3}%
  \stipulation{\mate{}3}%
  \solution{%
    \bold{1.\queen{}a8! {\textasciitilde} 2.\knight{}e6 {\textasciitilde} 3.\queen{}b8\mate; 2... \knight{}c6 3.\queen{}xc6\mate; \\
          \tab 1... \king{}c7 2.\rook{}a6 {\textasciitilde} 3.\knight{}b5\mate; 2... \knight{}c8 3.\queen{}xc8\mate \\
          \tab 1... \king{}c5 2.\queen{}d8 {\textasciitilde} 3.\knight{}e6\mate / 3.\rook{}a5\mate; \\
          \tab \tab 2... \knight{}c6 3.\knight{}e6\mate; \\
          \tab \tab 2... \knight{}d5 3.\knight{}e6\mate}
  }%
  \comment{%
    Миниатюра с интересным вступительным ходом в угол доски.
  }%
\end{diagram}%
\index{Кройтор Михаил}
\hfill
\begin{diagram}%
  \author{Кройтор, Михаил}%
  \source{Mat Plus}\year{2007}%
  \pieces[6+4]{sLa8, wLe8, sLa7, wKe7, sBc5, wBd5, wBe5, sKc4, wDb2, wSf2}%
  \stipulation{\mate{}3}%
  \solution{%
    \bold{1.\knight{}g4! {\textasciitilde} 2.\bishop{}b5+ \king{}xd5 3.\knight{}f6\mate \\ 
          \tab 1... \king{}xd5 2.\knight{}f6+ \king{}c4 3.\bishop{}b5\mate \\
          \tab 1... \king{}d3 2.\bishop{}g6+ \king{}c4 3.\knight{}e3\mate \\
          \tab 1... \bishop{}c6 2.\bishop{}g6 {\textasciitilde} 3.\knight{}e3\mate}
  }%
  \comment{%
  Варианты заканчиваются правильными матами. Чешская школа.
  }%
\end{diagram}%
\index{Кройтор Михаил}
\hfill
\begin{diagram}%
  \author{Кройтор, Михаил}%
  \source{Mat Plus}\year{2007}%
  \pieces[4+3]{sBd7, wTe7, sBd5, wBe5, sKd4, wDc2, wKf2}%
  \stipulation{\mate{}3}%
  \solution{%
    \bold{1.\rook{}f7!} цугцванг \\ 
    \bold{ \tab 1... d6 2.\rook{}f4+ \king{}xe5 3.\queen{}f5\mate,\\ 
    \tab 1... \king{}xe5 2.\queen{}c5} цугцванг \bold{2... \king{}e6 3.\queen{}e7\mate, \\ 
    \tab \tab 2... \king{}e4 3.\queen{}e3\mate,\\ 
    \tab \tab 2... d6 3.\queen{}e3\mate}
  }%
\end{diagram}%
\index{Кройтор Михаил}
\hfill
\begin{diagram}%
  \author{Кройтор, Михаил}%
  \source{Mat Plus}\year{2007}%
  \pieces[7+9]{wSb8, sLc8, sTh8, sBa7, sBd7, sBh7, sKd6, sBe6, wTh6, wBa5, sBb5, wLg5, wDc3, wKd3, wBe3, sLg3}%
  \stipulation{\mate{}3}%
  \solution{%
    \bold{1. \rook{}h5! {\textasciitilde} 2. \queen{}c5+ \king{}xc5 3. \bishop{}e7\mate; \\ 
    \tab 1... \bishop{}e5 2. \queen{}xe5+ \king{}xe5 3. \bishop{}e7\mate; \\ 
    \tab 1... e5 2. \rook{}h6+ \king{}d5 3. e4\mate}
  }%
  \themes{%
    Active sacrifice
  }%
  \comment{%
    Жертва ферзя в угрозе и вариантах.
  }%
\end{diagram}%
\index{Кройтор Михаил}
\hfill
\begin{diagram}%
  \author{Кройтор, Михаил}%
  \year{2008}%
  \award{3rd Prize}\tournament{ЮК А. Мельничук - 50}%
  \pieces[4+2]{wDf5, wBc4, sKd4, wTc2, sBd2, wKd1}%
  \stipulation{\mate{}3}%
  \solution{%
    1. \rook{}b2? цугцванг 1... \king{}xc4 2. \king{}xd2 \king{}d4 3. \rook{}b4\mate{}, 1... \king{}e3 2. \rook{}b3+ \king{}d4 3. \queen{}d5\mate{}, но 1... \king{}c3! \\
    1. \rook{}a2? цугцванг 1... \king{}xc4 2. \rook{}a4+ \king{}c3/b3 3. \queen{}c2\mate{}, 1... \king{}e3 2. \rook{}a3+ \king{}d4 3. \queen{}d5\mate{}, 1... \king{}c3! \\
    \bold{1. \queen{}g5!} цугцванг \\
    \bold{\tab 1... \king{}e4 2. \rook{}c3 \king{}d4 3. \queen{}e3\mate{}, \\
    \tab 1... \king{}d3 2. \queen{}h4 \king{}e3 3. \rook{}c3\mate{}}
  }%
  \themes{%
    Changed play
  }%
  \comment{%
    Перемена игры в миниатюре при довольно неочевидном решении.
  }%
\end{diagram}%
\index{Кройтор Михаил}
\hfill
\begin{diagram}%
  \author{Кройтор, Михаил}%
  \year{2009}%
  \award{Сomm.}\tournament{Рыбинск 7 дней - 10 лет}%
  \pieces[7+6]{wKc6, sBc5, sKc4, sBd4, wBa3, sBb3, wBc3, sBd3, wBb2, sLd2, wBg2, wDa1, wTf1}%
  \stipulation{\mate{}3}%
  \solution{%
    Иллюзорная игра 1... Bf4 2.\rook{}xf4 \\
    \bold{1.\rook{}f4!} цугцванг \\
    \bold{ \tab 1... \bishop{}xf4 2.\queen{}f1! B{\textasciitilde} 3.\queen{}f7\mate{}, \\
    \tab 1... \bishop{}xc3 (c1) 2.\queen{}(x)c1, 1... \bishop{}e3 2.\queen{}e1}
  }%
  \themes{%
    Active sacrifice, Annihilation
  }%
  \comment{%
    Идея задачи в аннигиляции ``лишней'' белой ладьи.
  }
\end{diagram}%
\index{Кройтор Михаил}
\hfill
\begin{diagram}%
  \author{Кройтор, Михаил}%
  \source{Mat Plus}\year{2010}%
  \pieces[6+3]{sBb7, wDe6, wBd5, wKh5, sBa4, wBc4, sKd4, wTf3, wSe1}%
  \stipulation{\mate{}3}%
  \solution{%
    \bold{1.\queen{}c8! {\textasciitilde} 2.\knight{}c2+ \king{}e5 3.\queen{}e6\mate{}; 2... \king{}e4 3.\queen{}f5\mate{} \\
      1... \king{}e5 2.\rook{}e3+ Kd4 3.\knight{}c2\mate{}; 2... \king{}f4 3.\knight{}g2\mate{}; 2... \king{}f6 3.\queen{}f8\mate{}; 2... \king{}d6 3.\rook{}e6\mate{} \\
      1... \king{}e4 2.\queen{}g4+ \king{}e5 3.\queen{}f4\mate{}}
  }%
  \themes{%
    Meredith
  }%
  \comment{%
    Неожиданный вступительный ход ферзем на край доски.
  }%
\end{diagram}%
\index{Кройтор Михаил}
\hfill
\begin{diagram}%
  \author{Кройтор, Михаил}%
  \year{2010}%
  \award{Special Prize}\tournament{День Шахмат - 2010}%
  \pieces[5+2]{wLg8, wSc7, sBe6, sKd4, wBf3, wKg3, wDc2}%
  \stipulation{\mate{}3}%
  \solution{%
    Иллюзорная игра: 1... e5 2. \knight{}d5 {\textasciitilde} 3. \queen{}c3\mate{}; 1... \king{}e3 2. \queen{}d1 цугцванг e5 3. \knight{}d5\mate{} \\
    \bold{1. \queen{}c1!} цугцванг \\
    \tab \bold{1... e5 2. \knight{}b5+ \king{}d3 3. \bishop{}c4\mate{}; \\
    \tab 1... \king{}d3 2. \knight{}xe6} цугцванг \bold{\king{}e2 3. \knight{}f4\mate{}}
  }%
  \themes{%
    Flight giving and taking key, Changed mates
  }%
\end{diagram}%
\index{Кройтор Михаил}
\hfill
\begin{diagram}%
  \author{Кройтор, Михаил}%
  \source{Mat Plus}\year{2010}%
  \pieces[8+1]{wSd6, wBa3, wBb3, sKd3, wBe3, wBb2, wBd2, wKa1, wDe1}%
  \stipulation{\mate{}3}%
  \solution{%
    \bold{1.\knight{}f7!} цугцванг \\
    \bold{\tab 1... \king{}c2 2.\knight{}e5! \king{}xb3 3.\queen{}d1\mate{}, \\ 
    \tab 1... \king{}e4 2.\queen{}f1! \king{}d5 3.\queen{}c4\mate{}}
  }%
  \themes{%
    Meredith, Model mates
  }%
  \comment{%
    Посвящено Альберту Иванову -- патриарху молдавской шахматной композиции.
  }%
\end{diagram}%
\index{Кройтор Михаил}
\hfill
\begin{diagram}%
  \author{Кройтор, Михаил}%
  \source{Mat Plus}\year{2010}%
  \pieces[3+5]{sBd6, sBe6, sBg5, wKb4, sKd4, sBg4, wLg3, wDe2}%
  \stipulation{\mate{}3}%
  \solution{%
    \bold{1.\queen{}f1!} цугцванг \\
    \bold{\tab 1...d5 2.\queen{}e2 {\textasciitilde} 3.\bishop{}f2(\bishop{}e5)\mate{}; \\
    \tab 1... e5 2.\queen{}e2 (\textasciitilde) d5 3.\bishop{}f2(\bishop{}xe5)\mate{}; \\
    \tab \tab 2... e4 3.\queen{}d2\mate{}; \\
    \tab \tab 2... \king{}d5 3.\queen{}c4\mate{}; \\
    \tab 1... \king{}e4 2. \king{}c3 {\textasciitilde} 3.\queen{}d3\mate{}; \\
    \tab \tab 2... \king{}e3 3.\queen{}d3\mate{}; \\
    \tab \tab 2... \king{}d5 3.\queen{}c4\mate{}; \\
    \tab 1... \king{}e3 2.\king{}c3 \king{}e4 3.\queen{}d3\mate{}}
  }%
  \themes{%
    Meredith
  }%
  \comment{%
    Посвящено Сэму Лойду. Как и в его задачах, впечатляющий первый ход предоставляет черному королю 3 поля.
  }%
\end{diagram}%
\index{Кройтор Михаил}
\hfill
\begin{diagram}%
  \author{Кройтор, Михаил}%
  \source{ChessStar}\year{2010}%
  \award{2nd Сomm.}%
  \pieces[3+3]{wLc6, wKf6, sBc5, sKc4, sBe4, wDe3}%
  \stipulation{\mate{}3}%
  \solution{%
    1.\bishop{}e8? \king{}b4! \\
    \bold{1.\bishop{}a4} цугцванг \\
    \bold{\tab 1... \king{}b4 2.\queen{}b3+ \king{}a5 3.\queen{}b5\mate{}; \\
    \tab 1... \king{}d5 2.\queen{}c3 e3 3.\queen{}d3\mate{}; 2... c4 3.\queen{}e5\mate{}}
  }%
\end{diagram}%
\index{Кройтор Михаил}
\hfill
\begin{diagram}%
  \author{Кройтор, Михаил}%
  \source{ChessStar}\year{2010}%
  \award{3rd Сomm.}%
  \pieces[4+2]{sBd5, wTb4, wBd4, sKe4, wKa3, wDg3}%
  \stipulation{\mate{}3}%
  \solution{%
    \bold{1.\rook{}b7!} цугцванг \\
    \bold{ \tab 1... \king{}f5 2.\rook{}f7+ \king{}e4 3.\rook{}f4\mate{}; \\
    \tab \tab 2... \king{}e6 3.\queen{}g6\mate{}; \\
    \tab 1... \king{}xd4 2.\rook{}b4 + \king{}c5 3.\queen{}c7\mate{}}
  }%
\end{diagram}%
\index{Кройтор Михаил}
\hfill
\begin{diagram}%
  \author{Агапов, Игорь Алексеевич; Капустін, Федір Михайлович; Кройтор, Михаил Васильевич}%
  \year{2011}%
  \award{1st-3rd Prize}\tournament{ЮК М.Матрёнин-64}%
  \pieces[4+3]{wDg7, wKa6, sKe6, wBf4, sBh3, sBh2, wLh1}%
  \stipulation{\mate{}3}%
  \twins{%
    b) Move a6 a5
  }%
  \solution{%
    a) \bold{1. \king{}a5!!} - цугцванг \\
     \bold{\tab 1... \king{}d6 2. \queen{}f7 \king{}c5 3. \queen{}d5\mate{}; \\
     \tab 1... \king{}f5 2. \queen{}h6 \king{}g4 3. \queen{}g5\mate{}} \\
     b) Ka6--a5
     \bold{1. \queen{}h7!} - цугцванг \\
     \bold{\tab 1... \king{}d6 2. \queen{}f7 \king{}c5 3. \queen{}d5\mate{}; \\
     \tab 1... \king{}f6 2. \bishop{}d5 h1=\queen{} 3. \queen{}f7\mate{}}
  }%
  \themes{%
    Changed play
  }%
  \comment{%
    Интересная перемена матов в близнецах при нетривиальных решениях. Задачи были составлены в результате общения в интернет-сообществе ru-chess-art, совместно с Агаповым Игорем и Капустіным Федіром.
  }%
\end{diagram}%
\index{Кройтор Михаил}\index{Капустін Федір}\index{Агапов Игорь}
\hfill
\begin{diagram}%
  \author{Кройтор, Михаил}%
  \source{ChessStar}\year{2011}%
  \award{1st Сomm.}%
  \pieces[4+3]{sBd7, wDf7, sBd6, wBd4, sKe4, wKa2, wTd1}%
  \stipulation{\mate{}3}%
  \solution{%
    \bold{1.\queen{}f1!} цугцванг \\
    \bold{\tab 1... d5 2.\rook{}d3 d6 3.\queen{}f3\mate{}; \\
    \tab 1... \king{}d5 2.\queen{}f3+ \king{}e6 3.\rook{}e1\mate{}}
  }%
\end{diagram}%
\index{Кройтор Михаил}
\hfill
\begin{diagram}%
  \author{Кройтор, Михаил}%
  \source{Кудесник}\year{2011}%
  \award{2nd HM}\tournament{ЮК Алёна Кожакина - 20}%
  \pieces[4+2]{wDh8, sBb5, wTe5, sKd4, wKg4, wLa2}%
  \stipulation{\mate{}3}%
  \solution{%
    \bold{1.\queen{}c8! \\
    \tab 1... \king{}xe5 2.\queen{}c5+ \king{}f6 3.\queen{}g5\mate{}; \\
    \tab \tab 2... \king{}e4 3.\bishop{}b1\mate{} \\
    \tab 1... \king{}d3 2.\bishop{}b1+ \king{}d2 3.\queen{}c2\mate{}; \\
    \tab \tab 2... \king{}d4 3.\queen{}c5\mate{}; \\
    \tab 1... b4 2.\queen{}c4+ \king{}xe5 3.\queen{}f4\mate{}}
  }%
  \themes{%
    Echo mates
  }%
\end{diagram}%
\index{Кройтор Михаил}
\hfill
\begin{diagram}%
  \author{Кройтор, Михаил}%
  \source{Кудесник}\year{2012}%
  \award{5th Prize}\tournament{ЮК Алёна Кожакина - 20}%
  \pieces[5+2]{wSb5, sKd5, wLd4, wTe4, sBh3, wKb1, wDh1}%
  \stipulation{\mate{}3}%
  \solution{%
    \bold{1.\knight{}a7 !} цугцванг \\
    \bold{\tab 1... \king{}d6 2.\rook{}e8 {\textasciitilde} 3.\queen{}c6\mate{}; \\
    \tab 1... h2 2.\bishop{}b6} цугцванг \bold{\king{}d6 3.\queen{}d1\mate{}; \\
    \tab 1... \king{}c4 2.\queen{}xh3} цугцванг \bold{\king{}b4 3.\bishop{}b6\mate{}; \\
    \tab \tab 2... \king{}d5 3.\queen{}e6\mate{}}
  }%
  \comment{%
Два красивых фронтальных мата. Занимательно, что в одном из вариантов матует не батарея \italic{ферзь + ладья}, а батарея \italic{ладья + конь}.
  }%
\end{diagram}%
\index{Кройтор Михаил}
\hfill
\begin{diagram}%
  \author{Кройтор, Михаил}%
  \year{2012}%
  \award{5th Сomm.}\tournament{Гравюра}%
  \pieces[6+3]{wDd8, sBc7, sBc6, wSd5, sKd4, wBe4, wLc2, wBg2, wKc1}%
  \stipulation{\mate{}3}%
  \solution{%
    \bold{1. \queen{}a8! {\textasciitilde} 2. \queen{}xc6 {\textasciitilde} 3. \queen{}c3\mate{}; 2... \king{}e5 3. \queen{}f6\mate{} \\
    \tab 1... cxd5 2. \queen{}xd5+ \king{}e3 3. \queen{}d2\mate{}; \\
    \tab \tab 2... \king{}c3 3. \queen{}c5\mate{}; \\
    \tab 1... c5 2. \queen{}a1+ \king{}c4 3. \queen{}a4\mate{}; \\
    \tab \tab 2... \king{}d4 3. \queen{}c3\mate{}; \\
    \tab 1... \king{}c5 2. \queen{}a3+ \king{}b5 3. \bishop{}d3\mate{}; \\
    \tab \tab 2... \king{}c4 3. \queen{}b4\mate{}; \\
    \tab \tab 2... \king{}d4 3. \queen{}c3\mate{}}
  }%
  \comment{%
    Один из любимых моих приёмов в задачах -- разрушение батареи.
  }%
\end{diagram}%
\index{Кройтор Михаил}
\hfill
\begin{diagram}%
  \author{Кройтор, Михаил}%
  \source{SuperProblem}\year{2012}%
  \award{4th Сomm.}\tournament{TT-60}%
  \pieces[6+2]{sBc3, wBb2, sKc2, wKe2, wTa1, wSb1, wLc1, wTh1}%
  \stipulation{\mate{}3}%
  \solution{%
    \bold{1.\rook{}h4!} цугцванг \\
    \bold{\tab 1... \king{}b3 2.\king{}d3 {\textasciitilde} 3.\rook{}a3\mate{}; \\
    \tab 1... \king{}xc1 2.\rook{}b4 {\textasciitilde} 3.\knight{}a3\mate{}; \\
    \tab \tab 2... cxb2 3.\rook{}c4\mate{}; \\
    \tab 1... cxb2 2.\rook{}c4+ \king{}b3 3.\knight{}d2\mate{}}
  }%
  \themes{%
    Homebase
  }%
  \comment{%
  Задача из тематического турнира с заданием: белые фигуры на начальной позиции.
  }%
\end{diagram}%
\index{Кройтор Михаил}
\hfill
\begin{diagram}%
  \author{Кройтор, Михаил}%
  \source{StrateGems}\year{2013}%
  \award{2nd HM}%
  \pieces[3+2]{wDf7, wLb6, sBe5, sKe4, wKc1}%
  \stipulation{\mate{}3}%
  \twins{%
    b) After key
  }%
  \solution{%
    a) \bold{1.\king{}b2!} цугцванг \\
    \bold{\tab 1... \king{}d3 2. \queen{}f3+ \king{}d2 3. \bishop{}a5\mate{}; 2... \king{}c4 3. \queen{}b3\mate{}}\\
    b) После первого хода белых: \\
    \bold{1.\king{}c1!} цугцванг \\
    \bold{\tab 1... \king{}d3 2. \queen{}d5+ \king{}c3 3. \bishop{}a5\mate{}; 2... \king{}e2 3. \queen{}d1\mate{}}
  }%
  \comment{%
    По задаче F.W.Stork, 1887 -- заменой вступительного хода получен близнец.
  }%
\end{diagram}%
\index{Кройтор Михаил}
\hfill
\begin{diagram}%
  \author{Кройтор, Михаил; Мельничук, Александр Николаевич}%
  \source{StrateGems}\year{2013}%
  \award{5th HM}%
  \pieces[5+4]{wKg6, wDa5, sBd4, wBb3, sLa2, sLb2, wLc2, wLd2, sKa1}%
  \stipulation{\mate{}3}%
  \solution{%
    1.\queen{}a6? цугцванг \\
    \tab 1... \bishop{}c3 2.\queen{}f1+ \king{}b2 3.\queen{}c1\mate{}; \\
    \tab 1... \bishop{}c1 2.\bishop{}xc1 d3 3.\queen{}f6\mate{}; 1... \bishop{}a3 2.\queen{}xa3,\\
    но 1...d3! \\
    1.\queen{}a7? цугцванг \\
    \tab 1... \bishop{}c1 2.\bishop{}xc1 d3 3.\queen{}g7\mate{}; \\
    \tab 1...d3 2.\queen{}g1+ \bishop{}b1 3.\queen{}xb1\mate{}, 2... \bishop{}c1 3.\bishop{}c3\mate{}; \\
    но 1... \bishop{}c3! \\
    \bold{1.\queen{}a8!} цугцванг \\
    \bold{\tab 1... \bishop{}c1 2.\bishop{}xc1 d3 3.\queen{}h8\mate{}; \\
    \tab 1... \bishop{}c3 2.\queen{}h1+ \king{}b2 3.\queen{}c1\mate{}; \\
    \tab 1... \bishop{}a3 2.\queen{}xa3 d3 3.\bishop{}c3\mate{}; \\
    \tab 1... d3 2.\queen{}h1+ \bishop{}c1 / \bishop{}b1 3.\bishop{}c3 / \queen{}xb1\mate{}}
  }%
  \themes{%
    Changed play
  }%
  \comment{%
   Составлено для тематического конкурса - 4 слона в линии.
  }%
\end{diagram}%
\index{Кройтор Михаил}\index{Мельничук Александр}
\hfill
\begin{diagram}%
  \author{Кройтор, Михаил}%
  \source{StrateGems}\year{2013}%
  \award{2nd Сomm.}\tournament{(miniatures)}%
  \pieces[5+2]{wKh7, sBc6, wSc5, wDb4, wBc4, sKd4, wLf1}%
  \stipulation{\mate{}3}%
  \solution{%
    \bold{1.\knight{}d3!} цугцванг \\
    \bold{\tab 1... \king{}e4 2.\queen{}d6 {\textasciitilde} 3.\queen{}g6\mate{}; \\
    \tab \tab 2... \king{}f5 3.\queen{}g6\mate{}; \\
    \tab 1... \king{}e3 2.c5 \king{}f3 3.\queen{}f4\mate{}; \\
    \tab 1... c5 2.\queen{}d2 \king{}e4 3.\queen{}f4\mate{}}
  }%
\end{diagram}%
\index{Кройтор Михаил}
\hfill
\begin{diagram}%
  \author{Кройтор, Михаил}%
  \source{StrateGems}\year{2013}%
  \award{3rd Сomm.}\tournament{(miniatures)}%
  \pieces[4+2]{sKd5, sBe5, wSf5, wLa3, wDc3, wKg1}%
  \stipulation{\mate{}3}%
  \solution{%
    \bold{1.\knight{}h4!} цугцванг\\
    \bold{\tab 1... \king{}e6 2.\queen{}c6+ \king{}f7 3.\queen{}g6\mate{};\\
    \tab 1... e4 2.\queen{}c5+ \king{}e6 3.\queen{}f5\mate{};\\
    \tab 1... \king{}e4 2.\queen{}c4+ \king{}e3 3.\bishop{}c1\mate{}}
  }%
\end{diagram}%
\index{Кройтор Михаил}
\hfill
\begin{diagram}%
  \author{Кройтор, Михаил}%
  \source{StrateGems}\year{2014}%
  \award{Special HM}\tournament{StrateGems}%
  \pieces[5+2]{wLg6, wKb5, sKd4, sBe4, wBd3, wSe3, wDg1}%
  \stipulation{\mate{}3}%
  \solution{%
    \bold{1.\queen{}f2!} цугцванг \\
    \bold{\tab 1... \king{}c3 2.\knight{}c4 \king{}b3 / e3 / exd3 3.\queen{}b2\mate{}; \\ 
    \tab \tab 2... \king{}xd3 3.\queen{}d2\mate{} \\
    \tab 1... \king{}e5 2.\knight{}c4+ \king{}d5 3.\bishop{}f7\mate{} \\
    \tab \tab 2... \king{}e6 3.\queen{}f7\mate{} \\
    \tab 1... \king{}xd3 2.\king{}c5} цугцванг \bold{\king{}c3 3.\queen{}c2\mate{} \\
    \tab 1... exd3 2.\queen{}f4+ \king{}c3 3.\queen{}b4\mate{}}
  }%
\end{diagram}%
\index{Кройтор Михаил}
\hfill
\begin{diagram}%
  \author{Кройтор, Михаил; Мельничук, Александр Николаевич}%
  \source{StrateGems}\year{2014}%
  \pieces[5+5]{wDb6, wTc6, sBg6, sTa5, sKd5, wBf5, sLg5, sBe3, wKh3, wSe2}%
  \stipulation{\mate{}3}%
  \solution{%
    \bold{1.\king{}g2! \textasciitilde 2.\rook{}c5+ \rook{}xc5 3.\queen{}e6\mate{}; 2... \king{}e4 3.\queen{}b1\mate{}; \\
    \tab 1... \king{}e4 2.\queen{}d4+ \king{}xf5 3.\knight{}g3\mate{}; \\
    \tab 1... \rook{}a1 / \rook{}a2 / \rook{}a3 / \rook{}b5 2.\queen{}b5+ \king{}e4 3.\rook{}e6\mate{}; \\
    \tab 1... gxf5 2.\rook{}d6+ \king{}c4 3.\rook{}d4\mate{}; 2... \king{}e5 / \king{}e4 3.\queen{}d4\mate{}}
  }%
\end{diagram}%
\index{Кройтор Михаил}\index{Мельничук Александр}
\hfill
\begin{diagram}%
  \author{Кройтор, Михаил}%
  \year{2017}%
  \award{10th Сomm.}\tournament{МК Руденко}%
  \pieces[6+8]{wKe8, sBc7, sBg7, sBd6, sKe6, wTb5, sBd5, sBh5, wBd4, wBg4, wTf3, sBg3, wLe2, sTe1}%
  \stipulation{\mate{}3}%
  \solution{%
    \bold{1.\rook{}b2! {\textasciitilde} 2.\bishop{}a6 {\textasciitilde} 3.\bishop{}c8\mate{}; \\}
    \bold{\tab 1... \rook{}d1 2.\bishop{}xd1 {\textasciitilde} 3.\rook{}e2\mate{}; \\
    \tab 1... \rook{}f1 2.\bishop{}xf1 {\textasciitilde} 3.\rook{}e2\mate{}; \\
    \tab 1... hxg4 2.\rook{}f4 {\textasciitilde} 3.\bishop{}xg4\mate{}; 2... \rook{}xe2 3.\rook{}xe2\mate{}; \\
    \tab 1... c5 2.\rook{}b7 {\textasciitilde} 3.\rook{}e7\mate{}}
  }%
  \comment{%
    Задача ( в несколько другом виде) пролежала 20 лет. При просмотре своих тетрадей обратил внимание, что задача может быть обогащена вариантами при смене пешечной конфигурации на правой стороне доски.
  }%
\end{diagram}%
\index{Кройтор Михаил}
\hfill
\begin{diagram}%
  \author{Кройтор, Михаил}%
  \source{ChessStar}\year{2019}%
  \pieces[5+2]{wKd7, sBe6, wBd5, sKd4, wBb3, wLg2, wDe1}%
  \stipulation{\mate{}3}%
  \solution{%
    \bold{1.\bishop{}f1! {\textasciitilde} 2.\king{}d6 {\textasciitilde} 3.\queen{}e5\mate{}, \\
    \tab 1... \king{}c5 2.\queen{}c3+ \king{}b6 3.\queen{}c7\mate{}, 2... \king{}xd5 3.\bishop{}g2\mate{}, \\
    \tab 1... \king{}xd5 2.\queen{}e3 {\textasciitilde} 3.\bishop{}g2\mate{} /
3.\bishop{}c4\mate{}, 2... e5 3.\bishop{}c4\mate{}, \\
    \tab 1... exd5 2.\king{}c7 \king{}c5 3.\queen{}c3\mate{}}
  }%
\end{diagram}%
\index{Кройтор Михаил}
\hfill
\begin{diagram}%
  \author{Кройтор, Михаил; Сыгуров, Александр Ильич}%
  \year{2019}%
  \award{2nd HM}\tournament{5th Tourney FRME}%
  \pieces[12+8]{sSb8, sDc8, wSg8, wLh8, wTf7, wBb6, sKd6, sBe6, wTh6, wBa5, sBb5, sBe5, sLa4, wBb4, sBc4, wBe3, wLf3, wSd2, wDe2, wKh2}%
  \stipulation{\mate{}3}%
  \solution{%
    1.\knight{}e7? но 1... \king{}d7! \\
    1.\knight{}e4+? но 1... \king{}c6! \\
    \bold{1.\queen{}g2! {\textasciitilde} 2.\bishop{}xe5+! \king{}xe5 3.\queen{}g3\mate{},\\
    \tab 1... \queen{}c5 2.\rook{}xe6+! \king{}xe6 3.\queen{}g6\mate{},\\
    \tab 1... \queen{}xg8 2.\knight{}xc4+! bxc4 3.\queen{}d2\mate{},\\
    \tab 1... \knight{}c6 2.\knight{}e4+! \king{}d5 3.\knight{}f6\mate{},\\
    \tab 1... \knight{}d7 2.\knight{}e7! e4 3.\knight{}xe4\mate{}, \\
    \tab \tab 2... \knight{}xb6 3.\knight{}f5\mate{}}
  }%
  \themes{%
    Adabashev, Clearance sacrifice, B2
  }%
\end{diagram}%
\index{Кройтор Михаил}\index{Сыгуров Александр}
\hfill
\begin{diagram}%
  \author{Кройтор, Михаил}%
  \year{2019}%
  \award{4th Сomm.}\tournament{Гравюра}%
  \pieces[6+3]{sBc7, sBc6, wTe6, wLg6, wBc5, sKc4, wLa3, wBc2, wKd2}%
  \stipulation{\mate{}3}%
  \solution{%
    \bold{1.\rook{}e8!} цугцванг \\
    \bold{\tab 1... \king{}d5 2.\bishop{}f7+ \king{}d4 3.c3\mate{}, \\
    \tab 1... \king{}b5 2.\bishop{}d3+ \king{}a4 (\king{}a5) 3.\rook{}a8\mate{}, \\
    \tab 1... \king{}d4 2.\rook{}e4+ \king{}d5 3.c4\mate{}}
  }%
\end{diagram}%
\index{Кройтор Михаил}
\hfill

\pagebreak
\subsection*{Решения}
\markright{}
\addcontentsline{toc}{subsection}{Решения}
\dianamestyle{fullname}
\putsol 
