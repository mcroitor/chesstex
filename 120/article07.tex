\subsection*{Субботний вечер}
\markright{}
\addcontentsline{toc}{subsection}{Субботний вечер}% http://crestbook.com/node/825
\dianamestyle{fullname}

\begin{multicols}{2}[]
Вообще-то, блиц -- занятие неблагодарное. Вроде играешь неплохо, голова полна идеями в любой позиции, а тут на тебе -- зеваешь на ровном месте какую-то мелочёвку. Так было и на этот раз. Но расскажу всё по порядку. Это был субботний летний вечер. Жара на улице уже спадала, старички и другие любители шахмат выползали на лавочки ``поразмяться''. Попадались довольно интересные личности. Так, самым пожилым был Милий Алексеевич, 80 лет, но ум его был острее многих. Даром что ли его задачки в районной газете публиковали?! Разминался здесь Модест Петрович: гроза заезжих мастеров. Он был не дурак выпить, однако винные пары на работе мысли не сказывались. Можно и нужно упомянуть других завсегдатаев: Цезарь Антонович, работник банка, любитель азартных игр и ставок; Александр Порфирьевич -- железнодорожник на пенсии, абсолюьно не знающий дебюты, но очень цепкий в защите; часто собирал их всех вместе Владимир Васильевич, стоматолог, с мощными руками и брюшком.

Попадались также люди, которые никогда не играли сами. Эдакие профессиональные болельщики. В их число записался дядя Ваня: пятидесятилетний худющий дядька. Иногда захаживал во двор и я. Вообще-то я так себе шахматист. О таких говорят -- ``много обещающий''. Но я не терял надежды, что в скором будущем будут обо мне говорить эти два слова слитно.

И в этот раз я вышел во двор с твёрдым желанием показать старичкам ``где раки зимуют''. За одним столиком как всегда восседал Модест Петрович. Играли навылет пятиминутки. Проигрывал Модест редко и этот вечер был явно его. 

В четвёртый раз подошла моя очередь -- я сел за доску с желанием подвинуть-таки ``дядю''. Выпало играть чёрными. Модест Петрович разыграл дебют Берда. Он умело играл по всей доске, и после того, как заблокировал ферзевый фланг, начал активно действовать на королевском. В некоторый момент даже белый король отправился атаковать! Однако, не смотря на происки белых, я успешно держал позицию: находил удачные защитительные манёвры, вовремя разменивался. Количество фигур на доске уменьшалось... Партия приобрела следующие очертания:

\begin{center}
\begin{diagram}%
  \author{Петрович --, Модест; Андреевич, Николай}
  \pieces[6+6]{sBa7, sBb6, sBc4, sKe6, sBg5, sLg3, wKd8, wBd2, wBc2, wBa6, wBb5, wSc8}%
  \stipulation{ход чёрных}%
\end{diagram}%
\end{center}\index{Кройтор Михаил}

Позиция для меня казалась уже выигранной: осталось слоном защитить ферзевый фланг, а на королевском есть проходная пешка, которую не остановить. Флажок у белых уже висел, тогда как у меня оставалась еще добрая минута. 

\bold{1... \bishop{}xb8} 

Модест Петрович недолго думая стукнул конём на b6.

\bold{2.\knight{}xb6+!?} 

Эге-ге! -- радостно подумал я, -- Вот и фигура! Конечно, белые получают проходную, но слон её держит. Полезет белый король за офицером -- тут мы и закроем его в уголке! А пешка `жэ' приведёт к победе... 

\bold{2... ab 3.\king{}c8 \bishop{}a7}

\begin{center}
\begin{diagram}%
  \pieces[5+5]{wKc8, wBa6, wBb5, wBd2, wBc2, sKe6, sBg5, sLa7, sBb6, sBc4}%
  \stipulation{После 3... \bishop{}a7}%
\end{diagram}%
\end{center}

 -- Приплыли! -- сказал дядя Ваня. С ним начали соглашаться и остальные наблюдатели. Только Милий Алексеевич резко выбросил: 
 
 -- Всё равно ничья! 

\bold{4. \king{}b7  \king{}d7 5.\king{}xa7 \king{}c7}\\
\begin{center}
\begin{diagram}%
  \pieces[5+4]{wKa7, wBa6, wBb5, wBd2, wBc2, sBb6, sBc4, sKc7, sBg5}%
  \stipulation{После 5... \king{}c7}%
\end{diagram}%
\end{center}

Здесь я уже считал вариант - \italic{6.\king{}a8 g4 7.d4 g3 8.d5 g2 9.d6+ \king{}xd6 10.a7 g1=\queen{}} и белым от мата не уйти. Однако мгновенно последовало 

\bold{6.d3!}

Под ложечкой неприятно засосало. На оставшихся секундах я увидел, что в случае  \italic{6... g4 7. dc g3 8. c5 g2 cb+ 9. \king{}d6 b7 10. g1=\queen{} \king{}a8} есть ферзь против пешек, но непонятно, как выигрывать. А Модесту Петровичу, с его реакцией, не впервой делать 10 ходов за 2 секунды. Однако делать было нечего, -- флажок на часах неприятно приподнялся,  -- я рванул ``гайкой'': 

\bold{6... c3 7. d4 g4 8.d5 g3 9.d6+ \king{}xd6}

\begin{center}
\begin{diagram}%
  \pieces[4+4]{wKa7, wBa6, wBb5, wBc2, sBb6, sKd6, sBg3, sBc3}%
  \stipulation{После 9... \king{}xd6}%
\end{diagram}%
\end{center}

Получалось, что ход белые как-то выиграли. Зрители загудели. Модест мгновенно ответил:

\bold{10.\king{}b7!}

Тут я опять задумался. Потом, бросив взгляд на висящий флажок и пожалев, что пешка на бэ шесть уцелела, я уверенно двинул проходную дальше: 

\bold{10... g2 11.a7 g1=\queen{} 12.a8=\queen{} \queen{}g2+}

\begin{center}
\begin{diagram}%
  \pieces[4+4]{wDa8, wKb7, wBb5, wBc2, sBb6, sBc3, sKd6, sDg2}%
  \stipulation{После 12... \queen{}g2+}%
\end{diagram}%
\end{center}

Тут я приготовился принимать скальп: \italic{13.\king{}a7 \queen{}xa8+ 14.\king{}xa8 \king{}c5} -- а с ним и поздравления. На ход Модеста Петровича я автоматом побил ферзя... 

\bold{13.\king{}xb6! \queen{}xa8}

 - Пат! -- воскликнул дядя Ваня, -- Красота! Неужели чёрные не могли ничего предпринять?! 

\begin{center}
\begin{diagram}%
  \pieces[3+3]{wKb6, wBc2, wBb5, sKd6, sBc3, sDa8}%
  \stipulation{После 13... \queen{}xa8}%
\end{diagram}%
\end{center}

 - Даже если и могли, результат бы не изменился, -- меланхолично заметил Милий Алексеевич, поправляя часы к новой партии. Модест Петрович уступил ему место для дальнейшей партии, сам же направился промочить горло. Хоть я и не вылетел, и остался в игре, результатом партии я доволен не был. Однако в этот вечер Модест Петрович к нам не возвращался, и месть свою мне пришлось отложить на другой раз...
 \end{multicols}