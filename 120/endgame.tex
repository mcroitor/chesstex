\subsection*{Этюды}
\markright{}
\addcontentsline{toc}{subsection}{Этюды}
\dianamestyle{noname}
\begin{diagram}%
  \author{Кройтор, Михаил}%
  \year{2006}%
  \award{HM}\tournament{ЮК Разуменко-70}%
  \pieces[5+5]{sKa8, wSb8, sTd8, sLe8, sTd5, sBg4, wTc2, wTf2, wBg2, wKh1}%
  \stipulation{=}%
  \solution{%
    Ложный след 1. \knight{}c6? \bishop{}xc6 2. \rook{}xc6 g3 3. \rook{}f1 \rook{}h5+ 4. \king{}g1 \rook{}dh8 5. \rook{}f8+! \rook{}xf8 6. \rook{}a6+ \king{}b7 7. \rook{}b6+ \king{}c7 8. \rook{}c6+ \king{}d7 9. \rook{}d6+ \king{}e7 10. \rook{}e6+ \king{}f7 11. \rook{}f6+ \king{}xf6 {0-1}\\
    \bold{1. \knight{}a6! \rook{}h5+ 2. \king{}g1 \rook{}d1+ 3. \rook{}f1 \rook{}xf1+ 4. \king{}xf1 \bishop{}b5+ 5. \king{}f2 \bishop{}xa6 6. \king{}g3} \\ 
    Возможна перестановка ходов 6.\rook{}a2 \king{}b7 7.\king{}g3. \\ 
    \bold{6... \rook{}g5 7. \rook{}a2 \king{}b7 8. \rook{}a4 \bishop{}e2 9. \rook{}e4 \bishop{}d1 10. \rook{}d4 \bishop{}e2 11. \rook{}e4} с преследованием слона ладьёй. Ничья.
  }%
\end{diagram}%
\index{Кройтор Михаил}
\hfill%
\begin{diagram}%
  \author{Кройтор, Михаил}%
  \source{Nona}\year{2006}%
  \award{Сomm.}\tournament{Nona}%
  \pieces[3+5]{wKe8, wLg8, sBa4, sKd4, sBb3, sBh3, sSf2, wTh1}%
  \stipulation{=}%
  \solution{%
    Ложный след 1.\rook{}a1? h2 2.\rook{}xa4+ \king{}c5 3.\rook{}c4+ \king{}b5 4.\rook{}h4 b2 5.\bishop{}h7 h1=\queen{} 6.\rook{}xh1 \knight{}xh1 7.\king{}d7 \king{}c4 8.\king{}d6 \knight{}f2 9.\bishop{}b1 \knight{}d3 10.\king{}c6 \knight{}b4+ 11.\king{}d6 \king{}b3 12.\king{}c5 \knight{}c2 13.\king{}d6 \knight{}a3 14.\bishop{}h7 \knight{}c2 15.\bishop{}g8+ \king{}a3 и чёрные побеждают. \\
    \bold{1.\rook{}h2 \knight{}g4} \\
    или 1... \king{}e3 2.\king{}d7 \king{}f3 3.\king{}c6 \king{}g3 4.\rook{}xh3+ \knight{}xh3 5.\king{}b5 b2 6.\bishop{}h7 \knight{}f2 7.\king{}xa4 \knight{}d1 8.\king{}b3 \king{}f2 9.\king{}c2 с ничьёй.\\
    \bold{2.\rook{}xh3 \knight{}f6+ 3.\king{}d8!}\\
    Проигрывает 3.\king{}f8? \knight{}xg8 4.\king{}xg8 b2 \\
    \bold{\knight{}xg8 4.\rook{}h4+ \king{}c3 5.\rook{}xa4 b2 6.\rook{}a3+ \king{}c4 7.\rook{}a4+ \king{}c5 8.\rook{}a5+ \king{}c6 9.\rook{}a6+ \king{}b7 10.\rook{}a5 b1=\queen{} 11.\rook{}b5+ \queen{}xb5} пат.
  }%
  \comment{%
  Этюд сочетает в себе известные по классическим произведениям манёвры, однако украшен красивой пунтой на третьем ходу.
  }%
\end{diagram}%
\index{Кройтор Михаил}
\hfill%
\begin{diagram}%
  \author{Кройтор, Михаил}%
  \source{Nona}\year{2006}%
  \award{Сomm.}\tournament{Nona}%
  \pieces[6+4]{sTe8, sBa7, wLb7, sBc7, wBa6, wBa5, wBd5, sKc4, wSe4, wKc2}%
  \stipulation{+}%
  \solution{%
    \bold{1.d6! \rook{}xe4! 2.\bishop{}d5+!} \\
    В случае 2.\bishop{}xe4 cxd6 с известной теоретической позицией, где лишний слон не даёт выигрыша. Также 2.dxc7 \rook{}e8 3.c8=\queen{}+ \rook{}xc8 4.\bishop{}xc8 ничья в другой теоретической позиции.\\
    Не помогает 2.d7 \rook{}d4, тоже ничья с лишним белым слоном.\\
    \bold{\king{}xd5 3.d7 \rook{}c4+ 4.\king{}b3 \rook{}b4+! 5.\king{}xb4 \king{}c6 6.d8=\knight{}+!!} и белые побеждают. \\
    Превращение в другие фигуры даёт пат или теоретическую ничью.
  }%
\end{diagram}%
\index{Кройтор Михаил}
\hfill%
\begin{diagram}%
  \author{Кройтор, Михаил}%
  \year{2007}%
  \award{6th-9th HM}\tournament{ЮК "Акобия-70"}%
  \pieces[10+6]{sDc8, wBb7, wDg7, wBd5, wBf5, wBg5, sBh5, sBc4, sTf4, wBh4, wBc3, sBe3, sKg3, wBc2, wBe2, wKh1}%
  \stipulation{+}%
  \solution{%
  Белым грозит мат в один ход, поэтому нет времени забирать чёрного ферзя. Также белый король не может сам себя защитить: 1. \king{}g1? \rook{}f1+! 2. \king{}xf1 \queen{}xf5+ и чёрные ставят мат.\\
    \bold{1. \queen{}e5 \queen{}b8!} \\%
        Красивая контригра чёрных. Быстро они проигрывают после 1... \queen{}xb7 2. \queen{}xe3+ \king{}g4 3. \king{}h2!. Теперь же белый ферзь не может защитить первую горизонталь, поэтому черные надеются на позиционную ничью из шахов, угрозы мата и пата.\\%
    \bold{2. \queen{}xb8 \king{}h3 3. \king{}g1} \\%
        Естественно, нельзя 3. \queen{}xf4 ?! с патом. \\%
    \bold{3... \rook{}g4+ 4. \king{}f1 \rook{}f4+ 5. \king{}e1 \rook{}g4 6. \queen{}f4!} \\%
        Ферзя нужно отдавать на правильном поле. Не помогает белым 6. \queen{}g3+ ?! \rook{}xg3 7. \king{}f1 \king{}h2 8. b8=\queen{} \king{}h1 9. \queen{}xg3 и партия завершается патом. \\%
    \bold{6... \rook{}xf4 7. b8=\queen{}}
    Получается, что белые сбросили пешку b7!
    \bold{7... \rook{}g4 8. \king{}f1 \rook{}f4+ 9. \king{}g1 \rook{}g4+ 10. \king{}h1 \rook{}g2 \rook{}f4 11. \queen{}b1}\\%
    Теперь этот ход, защищающий первую линию, возможен, и белые побеждают.
  }%
\end{diagram}%
\index{Кройтор Михаил}
\hfill%
\begin{diagram}%
  \author{Кройтор, Михаил}%
  \year{2007}%
  \award{3rd-4th HM}\tournament{Neidze 70 JT - MAIN}%
  \pieces[4+5]{sTa8, sKe8, wBa7, sBb7, sLf7, wTb6, sSg6, wKe4, wLc1}%
  \stipulation{=}%
  \solution{%
    \bold{1.\rook{}xb7 \bishop{}d5+! 2.\king{}xd5 \OOO+! 3.\king{}c6 \knight{}e5+ 4.\king{}b6} \\
    Также ничья будет после 4...\rook{}d6+ 5.\king{}c5 \rook{}c6+ 6.\king{}d5 \king{}xb7 7.a8=\queen{}+ \king{}xa8 8.\bishop{}f4 {=}\\
    \bold{4... \knight{}c4+ 5.\king{}c6 \rook{}d6+ 6.\king{}c5 \king{}xb7 7.a8=\queen{}+ \king{}xa8 8.\bishop{}b2! \rook{}d2 9.\bishop{}c3} с ничьей.    
  }%
\end{diagram}%
\index{Кройтор Михаил}
\hfill%
\begin{diagram}%
  \author{Кройтор, Михаил}%
  \year{2007}%
  \award{1st Сomm.}\tournament{JT Dvoretsky-60}%
  \pieces[4+5]{sKf8, sBd6, wKg6, wBa5, wTd5, wSf5, sBg5, sBg3, sBe2}%
  \stipulation{+}%
  \solution{%
    Прямой отыгрыш пешки приводит к ничьей: \italic{1. \rook{}xd6 ? e1=\queen{} 2. \rook{}d8+ \queen{}e8+ 3. \rook{}xe8+ \king{}xe8 4. \knight{}xg3 \king{}d7 5. \knight{}e2 g4 6. \king{}f5 g3 7. \king{}e4 g2 8. \king{}d5 \king{}c7 9. \king{}c5 \king{}b7 10. \king{}b5}. Поэтому пешку надо сохранить. \\
    \bold{1. \rook{}b5! e1=\queen{} 2. \rook{}b8+ \queen{}e8+ 3. \rook{}xe8+ \king{}xe8 4. a6} \\
        Надо торопиться с продвижением пешки, иначе черные успевают её остановить \italic{4. \knight{}xg3 \king{}d7 5. \knight{}e2 g4 6. \king{}f5 g3 7. \king{}e4 g2 8. \king{}d5 \king{}c7} и ничья. \\
    \bold{4... g2 5. a7 g1=\queen{} 6. a8=\queen{}+ Kd7 7. \queen{}b7+ \king{}e6} \\
    Что теперь, делать?\\
    \bold{8. \knight{}d4+! \queen{}xd4 9. \queen{}f7+ \king{}e5 10. \queen{}f5\mate}
    1-0
  }%
\end{diagram}%
\index{Кройтор Михаил}
\hfill%
\begin{diagram}%
  \author{Кройтор, Михаил}%
  \sourcenr{757}\source{Mat Plus}\issue{27}\day{11}\month{7}\year{2007}%
  \award{Сomm.}\tournament{Mat Plus}%
  \pieces[4+6]{wTc8, sKa6, sTe6, wTc5, sTe5, sBg4, sBf3, sBc2, wBg2, wKh1}%
  \stipulation{=}%
  \solution{%
     1.\rook{}xe5 ?! \rook{}xe5 (1...f2 ? 2.\rook{}xe6+ \king{}b7 3.\rook{}xc2 f1=\queen{}+ 4.\king{}h2 g3+ 5.\king{}xg3 \queen{}d3+ 6.\king{}f4 {(\king{}h2, h4)=}) 2.\rook{}xc2 \rook{}e1+ 3.\king{}h2 \rook{}e2 -+) (1.gxf3 ?! \rook{}xc5 -+) \\
    \bold{1.\rook{}xc2 f2} \\
        1...\rook{}e1+ 2.\king{}h2 \rook{}1e2 3.\king{}g3 fxg2 4.\rook{}2c6+ \king{}a5 5.\rook{}c1 = \\
    \bold{2.\rook{}xf2 g3 3.\rook{}f1 Rh5+ 4.\king{}g1 \rook{}eh6 5.\rook{}f6+ ! \rook{}xf6 6.\rook{}a8+ \king{}b7 7.\rook{}a7+ \king{}b8} \\
        7...\king{}b6 8.\rook{}a6+ =\\ 
    \bold{8.\rook{}b7+}
  }%
  \themes{%
    Meredith
  }%
\end{diagram}%
\index{Кройтор Михаил}
\hfill%
\begin{diagram}%
  \author{Дидух, Сергій Ігорович; Кройтор, Михаил Васильевич}%
  \source{Problemaz}\year{2008}%
  \pieces[10+6]{sDd8, wTc6, sBd6, wBd5, wBb4, sKe4, wBg4, sBa3, wLc3, wBg3, sBh3, wSa2, wKd2, wBe2, wBf2, sLh2}%
  \stipulation{+}%
  \solution{%
    \bold{1.\bishop{}f6! \queen{}xf6 2.\knight{}c3+ \king{}e5 3.f4+ \queen{}xf4+! 4.gf+ \bishop{}xf4+ 5.e3! \bishop{}xe3 6.\king{}d3!!} \\
    С двумя вариантами: \\
        a) \bold{6... h2 7.\rook{}c7! h1=\queen{} 8.\rook{}f7} +- \\
        b) \bold{6... a2 7.\rook{}c8! a1=\queen{} 8.\rook{}f8} +-
  }%
  \comment{%
    Первый из этюдов, составленный совместно с Дидух Сергеем.
  }%
\end{diagram}%
\index{Дидух Сергій}\index{Кройтор Михаил}
\hfill%
\begin{diagram}%
  \author{Дидух, Сергій Ігорович; Кройтор, Михаил Васильевич}%
  \year{2009}%
  \award{1st Prize}\tournament{Tel Aviv - 100}%
  \pieces[10+6]{sTa8, wBa7, wBa6, wLb6, wBg6, wBh6, wBg5, wTa4, sBg4, sBd3, sKg3, sBh3, wBd2, wBe2, sTg2, wKe1}%
  \stipulation{+}%
  \solution{%
    1. \bishop{}c7+ \king{}h4 2. g7! 
        (2. exd3? \king{}xg5 3. g7 \king{}xh6 4. \bishop{}e5 h2 5. \bishop{}xh2 g3)
        (2. h7? \rook{}xe2+ 3. \king{}f1 \rook{}ae8 4. \rook{}a1 \rook{}f8+) 
    2... \rook{}e8! 
        (2... \rook{}xe2+ 3. \king{}f1 h2 
            (3... \rook{}ae8 4. \rook{}a1 h2 
                 (4... g3 5. \rook{}a4+) 
            5. \bishop{}xh2 g3 6. a8=\queen{}!) 
        4. \bishop{}xh2 \rook{}c8 5. \rook{}a1 \rook{}xh2 6. h7 \king{}g3 7. h8=\queen{} \rook{}cxh8 8. gxh8=\queen{} \rook{}xh8 9. \king{}g1)
        (2... \rook{}xa7 3. \bishop{}f4! \rook{}xe2+ 4. \king{}d1 \rook{}e8 5. \king{}c1) 
    3. e3 h2 4. \bishop{}xh2 \rook{}c8! 5. \rook{}a1 \rook{}xh2 
        (5... \rook{}e2+!? 6. \king{}f1 \rook{}xh2 7. h7 \king{}g3 8. h8=\rook{} ! \rook{}f8+! 9. gxf8=\bishop{}! 
            (9. gxf8=\knight{}!) 
            (9. \king{}g1? \rook{}g2+ 10. \king{}h1 \rook{}h2+ 11. \rook{}xh2 \rook{}f1+ 12. \rook{}xf1)) 
    6. h7 \king{}g3 7. h8=\rook{}! 
        (7. h8=\queen{} ? \rook{}c1+ 8. \rook{}xc1 \rook{}h1+ 9. \queen{}xh1) 
    7... \rook{}cxh8 8. gxh8=\rook{}! \rook{}xh8 9. a8=\bishop{}! \rook{}xa8 10. a7 \rook{}h8 
        (10... \king{}h2 11. g6 g3 12. g7 g2 13. \king{}f2) 
    11. a8=\bishop{}! 1-0
  }%
  \themes{%
    White underpromotion
  }%
  \comment{%
    Набор слабых превращений. Составлен совместно с Дидух Сергеем.
  }%
\end{diagram}%
\index{Дидух Сергій}\index{Кройтор Михаил}
\hfill%
\begin{diagram}%
  \author{Кройтор, Михаил}%
  \source{Mat Plus}\year{2009}%
  \award{4th Prize}
  \pieces[4+5]{sBe6, sBd5, sBf5, sKa4, wBf4, sTg4, wKe3, wTh2, wSg1}%
  \stipulation{+}%
  \solution{%
    \bold{1.\knight{}f3 d4+! 2.\knight{}xd4 e5 3.fxe5 \rook{}e4+ 4.\king{}d3 \rook{}xe5 5.\king{}c4 \rook{}e3}\\
      (5... \king{}a3 6.\knight{}b5+ {+−}); 
      (5... \king{}a5 6.\knight{}c6+ {+−})\\
    \bold{6.\rook{}h6 \rook{}e5}\\
      (6... \king{}a3 7.\knight{}c2+ {+-});
      (6... \king{}a5 7.\knight{}b3+ {+−})\\
    \bold{7.\rook{}a6+ \rook{}a5 8.\rook{}b6! \rook{}a8}\\
      (8... \king{}a3 9.\rook{}b4 (9.\knight{}b5+) {+−})\\
    \bold{9.\rook{}b4+ \king{}a3}\\
      (9... \king{}a5 10.\knight{}b3 (9.\knight{}c6+) {+-})\\
    \bold{10.\knight{}b5+ \king{}a2 11.\knight{}c3+ \king{}a1 12. \rook{}b1\mate}
  }%
  \themes{%
    Meredith
  }%
  \comment{%
    Первоначальная версия этюда была составлена за полчаса, с целью продемонстрировать, как работает композитор -- под впечатлением книги Умнова.
  }%
\end{diagram}%
\index{Дидух Сергій}\index{Кройтор Михаил}
\hfill%
\begin{diagram}%
  \author{Кройтор, Михаил; Дидух, Сергій Ігорович}%
  \year{2009}%
  \award{2nd HM}\tournament{Olimpiya dunyasi}%
  \pieces[4+5]{sSc8, sKd6, wSe5, sLa4, sBf4, wBh4, wLf3, wKe2, sBg2}%
  \stipulation{=}%
  \solution{%
    \bold{1. \knight{}c4+!} \\
      (1. \knight{}f7+?! \king{}e7 2. \bishop{}xg2 \king{}xf7 3. \bishop{}h3 \knight{}e7 
      4. \king{}f3 \knight{}g6 5. h5 \bishop{}d1+ 6. \king{}e4 \bishop{}xh5) \\
    \bold{1... \king{}c5 2. \bishop{}xg2 \king{}xc4 
    3. \bishop{}h3! \knight{}e7 4. \king{}f3 \knight{}g6 
    5. h5 \bishop{}d1+ 6. \king{}e4 \bishop{}xh5 7. \bishop{}g4!} \\
      (7. \bishop{}e6+?! \king{}c3! 8. \bishop{}f7 \king{}d2! 9. \bishop{}xg6 f3! 
      10. \bishop{}f5 \king{}e2 11. \bishop{}h3 \king{}f2 12. \king{}f4 \king{}g1 13. \king{}g3 f2 {-+}) 
    \bold{7... \bishop{}xg4} {=}
  }%
  \comment{%
    Составлено совместно с Дидух Сергеем.
  }%
\end{diagram}%
\index{Дидух Сергій}\index{Кройтор Михаил}
\hfill%
\begin{diagram}%
  \author{Кройтор, Михаил}%
  \source{ChessStar}\year{2009}%
  \award{6th Сomm.}%
  \pieces[3+3]{wKa8, sBe7, sBf6, wBe4, sKh3, wBf2}%
  \stipulation{=}%
  \solution{%
    \bold{1.e5 fxe5 2.\king{}b7 \king{}g2 3.\king{}c6 e6 4.\king{}d6 e4 5.\king{}e5 \king{}f3 6.\king{}d4 \king{}f4 7.\king{}c5} \\
    Проигрывает 7.\king{}c4? e3 8.fxe3+ \king{}xe3 \\
    \bold{\king{}f3 8.\king{}d4 \king{}f4 9.\king{}c5 e3 10.fxe3+ \king{}xe3 11.\king{}d6} с ничьёй.
  }%
\end{diagram}%
\index{Кройтор Михаил}\index{Кройтор Михаил}
\hfill%
\begin{diagram}%
  \author{Дидух, Сергій Ігорович; Кройтор, Михаил Васильевич}%
  \year{2009}%
  \award{1st HM}\tournament{Kopnin 90 MT}%
  \pieces[6+7]{sKh8, sBg7, wSe6, sBf6, wBg6, wTe5, wKf5, sBc4, wBc3, sTd3, wBf3, sBe2, sBf2}%
  \stipulation{+}%
  \solution{%
    \bold{1. \rook{}a5!} \\
        (1. \rook{}c5? \rook{}xf3+ 2. \knight{}f4 \rook{}xf4+ 3. \king{}e6 \rook{}e4+ 
        4. \king{}f7 \rook{}e7+ 5. \king{}xe7 e1=\queen{}+ 6. \king{}f7 \queen{}e8+ 
        7. \king{}xe8 f1=\queen{} 8. \king{}f7 \queen{}f5 9. \rook{}xf5 {stalemate}) \\
    \bold{1... \rook{}xf3+ 2. \knight{}f4 \rook{}xf4+ 3. \king{}e6 \rook{}e4+ 
    4. \king{}f7 \rook{}e7+ 5. \king{}xe7 e1=\queen{}+ 6. \king{}f7 \queen{}e8+ 
    7. \king{}xe8 f1=\queen{} 8. \king{}f7 \queen{}h1 9. \rook{}b5 \queen{}h2 
    10. \rook{}c5 \queen{}h3 11. \rook{}d5 \queen{}d3 12. \rook{}h5\mate}
  }%
  \comment{%
    Составлено совместно с Дидух Сергеем.
  }%
\end{diagram}%
\index{Дидух Сергій}\index{Кройтор Михаил}
\hfill%
\begin{diagram}%
  \author{Кройтор, Михаил}%
  \year{2012}%
  \award{Сomm.}\tournament{Л.Лошинский-Е.Умнов-100 MT}%
  \pieces[3+7]{sBe7, sKf7, wKh7, wBc6, sBh6, sBh5, sBh4, sBg3, sDa2, wDc1}%
  \stipulation{+}%
  \solution{%
    \bold{1. c7 \queen{}a6} \\
      (1... \queen{}e6 2. \queen{}f1+ \king{}e8 3. \queen{}b5+ \king{}f7 4. \queen{}xh5+ +-) \\
    \bold{2. \queen{}f4+ \king{}e8 3. \queen{}f1! \queen{}c6} \\
      ({main} 3... \queen{}xf1 4. c8=\queen{}+ \king{}f7 5. \queen{}g8+ Kf6 6. \queen{}f8+ +-) \\
      (3... \queen{}c8 4. \queen{}b5+ \king{}f7 5. \queen{}xh5+ \king{}f6 6. \queen{}xh4+ \king{}e6 7. \queen{}g4+) \\
    \bold{4. \queen{}b5 \queen{}xb5 5. c8=\queen{}+ \king{}f7 6. \queen{}g8+ \king{}f6 7. \queen{}g6+ \king{}e5 8. \queen{}xh5+} +-
  }%
  \themes{%
    Active sacrifice
  }%
  \comment{%
    WCCT9 theme
  }%
\end{diagram}%
\index{Кройтор Михаил}
\hfill%
\begin{diagram}%
  \author{Кройтор, Михаил}%
  \year{2012}%
  \award{Сomm.}\tournament{ChessStar}%
  \pieces[3+3]{wTh8, sTb7, sKa6, wKc5, wSe5, sBa3}%
  \stipulation{+}%
  \solution{%
    \bold{1.\rook{}h6+ \king{}a5 2.\knight{}c6+ \king{}a4 3.\rook{}h4+ \king{}b3 4.\knight{}a5+ \king{}c3 5.\rook{}h3+ \king{}b2 6.\rook{}h2+ \king{}b1 7.\knight{}xb7 a2 8.\king{}b4 a1=\queen{} 9.\king{}b3} 1-0
  }%
\end{diagram}%
\index{Кройтор Михаил}
\hfill%
\begin{diagram}%
  \author{Кройтор, Михаил}%
  \year{2012}%
  \award{Сomm.}\tournament{FIDE Olympic Tourney}%
  \pieces[4+3]{sKf8, wKh7, wBd6, sBh4, sDa3, wTe1, wLf1}%
  \stipulation{+}%
  \solution{%
    \bold{1.d7! \queen{}a8} \\
      1... \queen{}e7+ 2.\rook{}xe7 \king{}xe7 3.\bishop{}h3 +- \\
    \bold{2.\bishop{}b5 \king{}f7 3.\rook{}e8 \queen{}d5! 4.\bishop{}c4!} \\
      4.\king{}h6? \queen{}xb5 5.d8=\queen{} \queen{}xe8 =\\
    \bold{4... \queen{}c4 5.\rook{}f8+!} \\
      5.d8=\queen{}? \queen{}c2+ -+ \\
    \bold{5... \king{}xf8 6.d8=\queen{}+ Kf7 7.\queen{}g8+} +-
  }%
\end{diagram}%
\index{Кройтор Михаил}
\hfill%
\begin{diagram}%
  \author{Кройтор, Михаил}%
  \year{2013}%
  \award{50th-62nd Place}\tournament{9 WCCT 2012}%
  \pieces[6+6]{sSb8, sDd8, sBf7, wTa6, sBc6, wBd6, sKe6, wBe5, wLb4, sTh4, wKa2, wDf1}%
  \stipulation{+}%
  \solution{%
    \bold{1. \rook{}a7} \\
      1. \rook{}a5 c5 2. \rook{}xc5 \knight{}d7 3. \rook{}a5 \queen{}c8 = \\
    \bold{1... \knight{}d7} \\
      1... \king{}d5 2. \queen{}xf7+ +-\\
    \bold{2. \rook{}xd7! \king{}xd7} \\
      2... \queen{}xd7 3. \queen{}f6+ \king{}d5 4. \queen{}xh4 +- \\
    \bold{3. \queen{}f5+ \king{}e8 4. e6! fxe6} \\
      4... \queen{}a8+ 5. \bishop{}a3 fxe6 6. \queen{}xe6+ \king{}f8 7. d7+ +- \\
    \bold{5. \queen{}xe6+ \king{}f8 6. \queen{}c8!!} thematic \bold{\queen{}xc8 7. d7+} +-
  }%
  \comment{%
    Темой соревнования была тихая жертва ферзя.
  }%
\end{diagram}%
\index{Кройтор Михаил}
\hfill%
\begin{diagram}%
  \author{Кройтор, Михаил}%
  \year{2013}%
  \award{50th-62nd Place}\tournament{9 WCCT 2012}%
  \pieces[5+4]{wKh8, sDb7, sKf7, wBd6, sBf6, wBd5, sBf5, wDa4, wBe2}%
  \stipulation{+}%
  \solution{%
    \bold{1. \king{}h7! \king{}f8+ 2. d7} \\
      (2. \king{}g6? \queen{}g7+ 3. \king{}xf5 \queen{}g5+ -+) \\
    \bold{2... \queen{}xd5} \\
      (2... \king{}e7 3.\queen{}c6! \queen{}xd7 4. \queen{}xd7+ \king{}xd7 5. \king{}g6 +-) \\
    \bold{3. \queen{}b4+} \\
      (3. \queen{}a3+ \king{}f7 =) \\
    \bold{3... \king{}f7 4. \queen{}b7!! {thematic} \queen{}xb7 5. d8=\knight{}+! \king{}e7 6. \knight{}xb7 \king{}e6 7. \king{}g6 \king{}e5} \\
      (7... f4 8. \knight{}d6 (8.\knight{}a5)) \\
    \bold{8. \knight{}d6! f4 9. \knight{}c4+ \king{}d4 10. \king{}xf6} 1-0
  }%
  \comment{%
    Темой соревнования была тихая жертва ферзя.
  }%
\end{diagram}%
\index{Кройтор Михаил}
\hfill%
\begin{diagram}%
  \author{Кройтор, Михаил}%
  \year{2016}%
  \award{HM}\tournament{JT Valerii Kirillov - 65}%
  \pieces[5+5]{wTh7, sTc6, sKe6, wBg6, wLa5, wKe3, sBc2, wBd2, sLc1, sSh1}%
  \stipulation{=}%
  \solution{%
    Первая тематическая попытка 1.g7? \king{}f7 2.\rook{}h8 \king{}xg7 3.\rook{}xh1 \bishop{}a3 4.\bishop{}c3+ \king{}f7 5.\king{}d3 c1=\queen{} 6.\rook{}xc1 \bishop{}xc1 7.\king{}c2 \bishop{}a3 8.\king{}b3 \bishop{}e7 {-+} \\
    Также нельзя брать коня: 1.\rook{}xh1? \bishop{}b2 {-+} \\
    \bold{1.\rook{}h6! \rook{}c5} \\
     Тематическая попытка 2.\rook{}xh1 \bishop{}b2 3.\bishop{}c3 c1=\queen{} 4.\rook{}xc1 \bishop{}xc1 заканчивается поражением белых. \\
    \bold{2.g7+ \king{}f7 3.\rook{}h8! \king{}xg7 4.\rook{}xh1} \\
     Также к ничьей приводит 4...\bishop{}b2 5.\bishop{}c3+ \bishop{}xc3 6.dxc3 \rook{}xc3+ 7.\king{}d2.
    \bold{4... \bishop{}a3 5.\bishop{}c3+ \king{}f7 6.\king{}d3 c1=\queen{} 7.\rook{}xc1 \bishop{}xc1 8.\king{}c2} \\
     Позиционная ничья. \\
    \bold{\bishop{}a3 9.\king{}b3 \bishop{}c1 10.\king{}c2} \\
     Чёрный слон не может вырваться.
  }%
\end{diagram}%
\index{Кройтор Михаил}
\hfill%
\begin{diagram}%
  \author{Кройтор, Михаил}%
  \year{2016}%
  \award{Сomm.}\tournament{НАУ ЭРА – 10}%
  \pieces[3+2]{sKh8, wLe7, wKf7, wBg2, sSf1}%
  \stipulation{+}%
  \solution{%
    Темой соревнования был ход пешкой с начальной линии на первом ходу. Если начать с шаха: 1.\bishop{}f6+? \king{}h7 2.g4 \knight{}e3 3.g5 \knight{}f5! 4.g6+ \king{}h6 5.\bishop{}e5 \king{}g5 то чёрные добиваются ничьей. \\
    \bold{1.g4 \knight{}e3 2.g5 \knight{}f5 3.\bishop{}b4!} \\
    3.\bishop{}c5? \knight{}h4! 4.\bishop{}f2 \knight{}f3 5.\bishop{}d4+ \knight{}xd4 6.g6 \knight{}e6 = \\
    \bold{3... \knight{}h4! 4.\bishop{}e1! \knight{}f3 5.\bishop{}c3+ \king{}h7 6.g6+} +-
  }%
\end{diagram}%
\index{Кройтор Михаил}
\hfill%
\begin{diagram}%
  \author{Кройтор, Михаил}%
  \source{Unión Argentina de Problemistas de Ajedrez}\year{2019}%
  \award{Сomm.}%
  \pieces[4+3]{wKb7, wBf6, wBb5, sKb4, sSf3, sTd2, wLf2}%
  \stipulation{=}%
  \solution{%
    \bold{1.\bishop{}e3! \rook{}d6 2.f7 \rook{}d7+ 3.\king{}a6 \rook{}xf7 
    4.b6 \knight{}e5 5.b7 \rook{}f8 6.\king{}b6 \knight{}d7+ 
    7.\king{}c7 \knight{}b8 8.\bishop{}c5+! \king{}xc5} {= stalemate} \\
      5...\rook{}f6+ 6.\king{}a7 \knight{}c6+ 7.\king{}b6! \knight{}b8+ 8.\king{}a7 \rook{}f8 9.\bishop{}f4 \knight{}c6+ 10.\king{}b6 {=} \\
      2...\rook{}f6 3.b6 \rook{}xf7+ 4.\king{}a6 \knight{}e5 5.b7 \knight{}d7 6.\king{}a7 \rook{}f8 7.\bishop{}b6 \king{}b5 8.\bishop{}c7 {=} \\
      1...\rook{}d7+ 2.\king{}c8 \rook{}h7 3.b6 \king{}b5 4.b7 \king{}c6 5.b8=\knight{}+ {=} \\
      1...\rook{}d5 2.b6 \king{}b5 3.\king{}c7 {=} \\
    1.f7? \rook{}d7+ 2.\king{}a6 \rook{}xf7 3.b6 \knight{}e5 4.b7 \knight{}d7 5.\bishop{}g1 \rook{}f1! {-+}
  }%
\end{diagram}%
\index{Кройтор Михаил}
\hfill%
\begin{diagram}%
  \author{Кройтор, Михаил}%
  \source{Unión Argentina de Problemistas de Ajedrez}\year{2019}%
  \award{1st HM}%
  \pieces[4+6]{sKa8, sDe8, sBd7, sSc6, wKf6, sLc4, wDf4, wLd2, sSg2, wTb1}%
  \stipulation{=}%
  \solution{%
    \bold{1.\rook{}a1+! \king{}b7 2.\rook{}b1+ \king{}c8 3.\queen{}xc4 \queen{}e5+ 
    4.\king{}g6 \knight{}h4+ 5.\queen{}xh4! \knight{}e7+! 6.\king{}h7 \queen{}f5+ 
    7.\king{}g7 \queen{}g6+ 8.\king{}f8 \queen{}g8+ 9.\king{}xe7 \queen{}d8+ 
    10.\king{}d6 \queen{}xh4 11.\rook{}c1+ \king{}d8 12.\bishop{}g5+! \queen{}xg5 
    13.\rook{}c8+ \king{}xc8} {stalemate} \\
    11...\king{}b7 12.\rook{}b1+ \king{}a6 13.\rook{}a1+ \king{}b5 14.\rook{}b1+ \king{}a4 15.\rook{}b4+ {=} \\
    5.\king{}h7? \queen{}f5+ 6.\king{}g7 \queen{}xb1 7.\queen{}xh4 \queen{}b2+ 8.\queen{}f6 \queen{}xd2 {-+} \\
    3... \queen{}h8+ 4.\king{}f7 \queen{}h7+ 5.\king{}f6 \queen{}xb1 6.\queen{}g8+ \king{}b7 7.\queen{}xg2 {=} \\
    3... \queen{}f8+ 4.\king{}g6 \queen{}d6+ 5.\king{}f7 \queen{}xd2 6.\queen{}a6+ \king{}c7 7.\rook{}b7+ \king{}d6 8.\queen{}a3+ \king{}e5 9.\queen{}c5+ \king{}e4 10.\king{}e8 {=} \\
    1.\queen{}xc4? \queen{}e7+ 2.\king{}f5 Sh4+ 3.\king{}f4 \queen{}d6+ 4.\king{}e4 \queen{}g6+ 5.\king{}d5 \queen{}xb1 {-+}
  }%
\end{diagram}%
\index{Кройтор Михаил}
\hfill%
\begin{diagram}%
  \author{Кройтор, Михаил; Арестов, Павел Михайлович}%
  \source{Шахматная композиция}\year{2019}%
  \award{Сomm.}%
  \pieces[4+3]{sKc8, sBa6, wLd6, wBe6, wBa5, wKc2, sLg1}%
  \stipulation{+}%
  \solution{%
    \bold{1.\bishop{}e7! \bishop{}h2 2.\king{}d3 \bishop{}c7 3.\king{}e4! \bishop{}xa5 
    4.\king{}f5 \bishop{}b6 5.\king{}g6! a5 6.\bishop{}b4! \bishop{}d8 
    7.\bishop{}xa5 \bishop{}xa5 8.e7 \king{}d7 9.\king{}f7} \\
    5.\king{}f6? a5! 6.\king{}f7 a4 {=} \\
    4...\bishop{}e1 5.\bishop{}f6 a5 6.\king{}g6 \bishop{}b4 7.\bishop{}c3! \bishop{}xc3 8.e7 \king{}d7 9.\king{}f7 3.\bishop{}b4? \king{}d8 {=}
  }%
  \comment{%
    Этюд составлен совместно с Арестовым Павлом.
  }%
\end{diagram}%
\index{Кройтор Михаил}\index{Арестов Павел}
\hfill%
\begin{diagram}%
  \author{Кройтор, Михаил}%
  \source{Sinfonie Scacchistiche}\year{2020}%
  \pieces[4+3]{sBb5, sKg4, wBa3, wKc3, sBh3, wBb2, wBh2}%
  \stipulation{+}%
  \solution{%
    \bold{1. b4 \king{}f4} \\
    1... \king{}f3 2. a4 bxa4 3. b5 \king{}g2 4. b6 \king{}xh2 5. b7 \king{}g2 6. b8=\queen{} h2 7. \queen{}b7+ с выигрышем.
    \bold{ 2. \king{}b2!!} \\
    Неожиданный отход короля в угол доски
    \bold{2... \king{}f3} \\ 
    Также белые побеждают и в случае 2... \king{}e4 3. a4 bxa4 4. \king{}a3! \king{}d5 5. \king{}xa4 \king{}c6 6. \king{}a5 и пешка проходит. \\
    \bold{3. a4 bxa4 4. b5 a3+ 5. \king{}a1!! \king{}g2 6. b6 \king{}xh2 7. b7 \king{}g1} \\
    Или 7... \king{}g2, в этом случае белые хорошо известными манёврами отталкивают короля на g1. \\
    \bold{8. b8=\queen{} h2 9. \queen{}g3+ \king{}h1 10. \queen{}f2 a2 11. \queen{}f1\mate{}}
  }%
\end{diagram}%
\index{Кройтор Михаил}
\hfill%

\pagebreak
\subsection*{Решения}
\markright{}
\addcontentsline{toc}{subsection}{Решения}
\dianamestyle{fullname}
\putsol 
