\subsection*{Ладья против Ладьи, Слона и Пешки: на практике}
\markright{}
\addcontentsline{toc}{subsection}{Ладья против Ладьи, Слона и Пешки: на практике}
\dianamestyle{fullname}

Данная статья является естественным продолжением статьи ``Ладья против Ладьи, Слона и Пешки: ничейные мотивы''. Если первая часть была посвящена этюдным способам спасения, то вторая представляет собой ту немногочисленную группу окончаний, в которых люди на практике выкарабкивались ``с дырки на половинку''. Конечно, чаще всего этому позволяли просмотры их соперников.

На практике люди берутся защищать эндшпиль с ладейной пешкой, где слон ``не того цвета''. К сожалению многих, этот эндшпиль сильнейшая сторона без труда выигрывает. Правда, случаются казусы... Хочется дать совет сильнейшей стороне: не надо двигать пешку ``до упора'', так как в этом случае реализация перевеса сильно затрудняется, а зачастую и вовсе становится невозможной.

\begin{multicols}{2}[]
\begin{center}
\begin{diagram}%
  \author{Arnason --, Jon; Mortensen, Erling}
  \year{1989}
  \pieces[3+4]{sKd5, sLe5, wTh4, sTc2, wLf2, wKg2, sBh2}%
  \stipulation{ход белых}%
\end{diagram}%
\end{center}

Материал в позиции на диаграмме не совсем тематичный, так как у белых есть ещё есть слон. Однако, это ненадолго: ``лишний'' слон отдаётся и белые форсируют пат. 

\bold{70. \king{}h1!} 

Чёрные могут попробовать выгнать короля из угла: \italic{70... \rook{}c2+ 71. \king{}g2 \rook{}c2 72. \king{}h1} с повторением позиции.

\bold{\king{}e6 71. \bishop{}g3! \bishop{}xg3} 

Фигуру чёрные вынуждены забрать, иначе белые забирают последнюю пешку с очевидной ничьёй. Теперь идёт ``бешеная'' ладья:

\bold{72. \rook{}e4+ \king{}d5 73. \rook{}d4+ \king{}c5 74. \rook{}d5+ \king{}c6 75. \rook{}d6+ \king{}b7 76. \rook{}b6+ \king{}xb6} 

Пат. 

Вот к чему может привести чрезмерное стремление пешки в ферзи.

\begin{center}
\begin{diagram}%
  \author{Bang --, Andreas; Pedersen, Thomas}
  \year{1996}
  \pieces[2+4]{sTa7, sKf7, sLf6, wKf2, sBh2, wTf1}%
  \stipulation{ход белых}%
\end{diagram}%
\end{center}

Тот же патовый мотив показан на диаграмме 28: 

\bold{45. \king{}g2 \rook{}a2+ 46. \king{}h1 \king{}e6 47. \rook{}xf6+ \king{}xf6}

Если бы черная пешка стояла бы на h3, то белые бы быстро проигрывали.

\begin{center}
\begin{diagram}%
  \author{Basman --, Michael; Hartston, William}
  \year{1968}
  \pieces[4+2]{sTf8, sKh8, wTc7, wKe6, wBh6, wLd3}%
  \stipulation{Ход белых}%
\end{diagram}%
\end{center}

У белых серьёзное материальное и позиционное преимущество, однако они делают единственный ход в этой позиции, который может привести к ничьей!

\bold{80. \king{}e7?? \rook{}f7+} 

После взятия ладьи - пат, на любой другой ход идёт размен ладей с ничьей. 

Победа была близка. Быстрее всего выигрывало \italic{80. \bishop{}g6}. Главное, как говорил мой тренер -- ``держать короля в страхе''. Мат приходит очень скоро. Идея за белых состоит в том, чтобы поставить фигуры на следующие места: король g6, ладья e7, слон e6 и пешка на h7. Следующий вариант демонстрирует, как этого добиться: 

\italic{80... \rook{}a8 81. \king{}f6 \rook{}a6+ 82. \king{}g5 \rook{}a5+ 84. \bishop{}f5 \rook{}a8 85. \rook{}h7+ \king{}g8 86. \rook{}e7 \king{}h8 87. \bishop{}e6 \rook{}f8 88. \king{}h5} или \bishop{}f7 со скорым матом.

\begin{center}
\begin{diagram}%
  \author{Prada Rubin --, Fernando; Rivera Kuzawka, Daniel}
  \year{1995}
  \pieces[4+3]{sTb8, sKh8, sBf7, wKf6, wBh6, wLf5, wTg4}%
  \stipulation{Ход белых}%
\end{diagram}%
\end{center}

Из той же серии следующая позиция, но белые здесь сплоховали.

В данной позиции были сделаны ходы 

\bold{79. \king{}xf7 \rook{}f8+ 80. \king{}e6 \rook{}f6+} 

Белые, увидев, что они теряют пешку, согласились на ничью. А зря! Ведь после \italic{81. \king{}e5 \rook{}xh6 82. \rook{}g1 \rook{}b6 83. \bishop{}e6} черные получают мат максимум в 15 ходов!

\begin{center}
\begin{diagram}%
  \author{Fournier --, Frederic; Desitter, Thierry}
  \year{1991}
  \pieces[4+3]{sKg7, wKg5, wTh5, sTd4, wBh4, wLf3, sBg3}%
  \stipulation{ход белых}%
\end{diagram}%
\end{center}

Как уже было сказано -- чаще всего в практическом эндшпиле встречается у сильнейшей стороны ладейная пешка. Если же у них еще ``неудачный'' слон, то слабейшая сторона вправе предложить размен ладей, как это было в позиции 31. 

\bold{52. \rook{}h6??} 

Выигрывало только \italic{52. \king{}f5}, но ход вполне очевиден. \italic{52... \king{}f8 53. \bishop{}e4 +-} 

\bold{52... g2! 53. \bishop{}xg2} 

Или \italic{53. \rook{}g6+ \king{}h7 54. \king{}f5 \rook{}xh4 =}. 

\bold{53... \rook{}g4+! 54. \king{}xg4 \king{}xh6} 

Вот и ничейная гавань.

\bold{55. \bishop{}e4}

Согласились на ничью.

\begin{center}
\begin{diagram}%
  \author{Herrera --, Martin; Sapienza, Julian}
  \year{2000}
  \pieces[4+3]{wLa7, sKb7, wBb6, wKd6, sTb5, wTe4, sBa3}%
  \stipulation{ход черных}%
\end{diagram}%
\end{center}

В следующей позиции (диаграмма 32) строится интересная позиционная ничья, напоминающая позицию из этюда Nr.26 из предыдущей статьи. 

\bold{60... a2 61. \rook{}a4 \rook{}b2} 

Белая ладья привязана к пешке а2. Потому белые вынуждены подвести короля, чтобы выиграть чёрную пешку. 

\bold{62. \king{}c5 \rook{}h2 63. \rook{}a3 \rook{}g2 64. \king{}b4 \rook{}h2 65. \rook{}a5 \rook{}g2 66. \king{}b3 \rook{}g3+?!}

Проще было курсировать ладьёй по второй горизонтали. 

\bold{67. \king{}xa2 \rook{}g6}

Ничья стала очевидной.

\begin{center}
\begin{diagram}%
  \author{Sax --, Gyula; Kovacevic, Vlatko}
  \year{1982}
  \pieces[4+3]{wTh8, sKf7, sBg7, wLh7, wBh5, wKf4, sTd1}%
  \stipulation{ход чёрных}%
\end{diagram}%
\end{center}

Из той же оперы эндшпиль на диаграмме 33. Он показывает, как же все-таки можно сконструировать ``домик'' королю слабейшей стороны.

\bold{56... g5+!}

Медлить нельзя! Иначе просто пешку белые заберут в более выгодной для себя редакции! 

\bold{57. hxg6+} 

Или \italic{57. \king{}xg5 \king{}g7} с ничьёй.

\bold{57... \king{}g7}

Вот и ``домик''. Позиция ничейна. Партия еще продолжалась \italic{58. \rook{}g8+ \king{}h6 59. \rook{}c8 \king{}g7 60. \bishop{}g8 \rook{}f1+ 61. \king{}e5 \rook{}e1+ 62. \king{}d6 \king{}xg6 63. \bishop{}d5 \king{}g5 64. \rook{}f8 \king{}g4 65. \bishop{}f3+ \king{}g5 66. \king{}d5 \rook{}a1 67. \king{}e5 \rook{}a5+ 68. \bishop{}d5 \king{}g4 69. \rook{}g8+ \king{}h5 70. \king{}e6 \king{}h4} и соперники согласились с мирным исходом. 

Следующие 3 примера демонстрируют интересные защитительные ресурсы, которые могут встретится в партии.

\begin{center}
\begin{diagram}%
  \author{Teufel --, Juergen; Kestler, Hans Guenther}
  \year{1968}
  \pieces[3+4]{sBf7, wBg5, sKa3, sLb3, wTf3, wKf2, sTa1}%
  \stipulation{ход белых}%
\end{diagram}%
\end{center}

Слон чёрных связан, ладья перекрыта. Пользуясь тем, что чёрный король мешает своим фигурам, белые разменивают пешки.

\bold{48. g6!} 

Замечательный удар! Черная пешка переводится на линию, на которой её не защитить! 

\bold{48... \rook{}a2+ 49. \king{}g1 fxg6 50. \rook{}g3 \king{}b4 51. \rook{}xg6}

Пешка разменена, ничья очевидна.

В интересный пат удалось залезть гроссмейстеру Гагунашвили в одной из своих партий (диаграмма 35).

\begin{center}
\begin{diagram}%
  \author{Vedder --, Richard; Gagunashvili, Merab}
  \year{2004}
  \pieces[4+2]{wLe5, wKf5, wBg4, sKh4, wTg3, sTa2}%
  \stipulation{ход черных}%
\end{diagram}%
\end{center}

Фигуры белых расположены кучно, и всё бы ничего, если бы не был запатован чёрный король. Поэтому слабейшая сторона пытается сбросить ладью или, при случае, попробовать забрать пешку.

В партии последовало: 

\bold{70... \rook{}f2+ 71. \king{}e6} 

Понятно, белые не согласны на \italic{71... \bishop{}f4 72. \rook{}xf4+! \king{}xf4} пат.

\bold{\king{}g5 72. \king{}d5 \rook{}h2 73. \bishop{}d6?} 

Ошибка. Выигрывало \italic{73. \bishop{}d4 \king{}f4 74. \rook{}g1} и белые наладили координацию фигур. 

\bold{73... \rook{}h4 74. \bishop{}e7+ \king{}f4 75. \bishop{}xh4}

Красивый пат! Он мне настолько понравился, что я захотел составить этюд с таким окончанием. Но проверка базы этюдов показала существование такового... Моё разочарование было неимоверным!

\begin{center}
\begin{diagram}%
  \author{Schulz --, Klaus Juergen; Nikolac, Juraj}
  \year{1986}
  \pieces[3+5]{wKe3, sKf7, wTc6, sLe5, sTb4, sBg4, sBc3, wLe1}%
  \stipulation{ход белых}%
\end{diagram}%
\end{center}

А в следующем окончании (диаграмма 36) белые не смогли воспользоваться предоставившейся возможностью...

Плохо без двух пешек. Потому белые решают отдать слона: 

\bold{87. \bishop{}xc3 \rook{}b3 88. \king{}e4?} 

Неудачная реакция. Отход королем на вторую горизонталь не позволял дать чёрным выигрывающий шах -- \italic{88. \king{}f2 \bishop{}xc3 89. \king{}g3} -- и последняя пешка легко бы забиралась. Теперь белые пожимают плоды своей невнимательности: 

\bold{88... \bishop{}xc3 89. \king{}f4 \rook{}b4+} 

Вот про это я и говорил!

\bold{90. \king{}g3 \bishop{}e5+} 

и белые прекратили своё сопротивление.

Наконец, партия двух гроссмейстеров. Она приводится как демонстрация того, что иногда бывает сложно найти выигрышный план за сильнейшую сторону.

\begin{center}
\begin{diagram}%
  \author{Eljanov --, Pavel; Malakhov, Vladimir}
  \year{2017}
  \pieces[4+2]{sKh8, wKg5, wTd4, wLd3, sTb8, wBh6}%
  \stipulation{ход белых}%
\end{diagram}%
\end{center}

\bold{61. \rook{}d7 \rook{}g8+ 62. \king{}h5 \rook{}b8} 

На чём основывается защита чёрных и как теперь выигрывать белым? Чёрные удерживают позицию шахами по 6-ой линии и с поля g8. Поэтому, для победы белым необходимо было перевести слона на e6, что достигалось следующим образом: \italic{63. \rook{}h7+ \king{}g8 64. \rook{}e7! \king{}h8 65. \bishop{}c4} после чего белый король без проблем попадает на g6, не боясь пата.

\bold{63. \bishop{}c4 \rook{}e8 64. \bishop{}d3 \rook{}b8 65. \bishop{}f5 \rook{}a8 66. \rook{}e7 \rook{}b8 67. \king{}g6 \rook{}g8+ 68. \rook{}g7 \rook{}e8 69. \rook{}h7+ \king{}g8} 

Счастье было близко в случае \italic{70. \rook{}f7 \king{}h8 71. \bishop{}d7} со скорым матом.

\bold{70. \rook{}d7 \king{}h8 71. \bishop{}d3 \rook{}e6+} 

Здесь белые не выдержали напряжения

\bold{72. \king{}f7? \rook{}d6}

и теперь ничья очевидна.
\end{multicols}

Какой можно подвести итог? В принципе, сдаваться никогда не поздно: необходимо научиться и в таком эндшпиле доставить максимум трудностей своему сопернику, выискивать малейшие шансы. Успехов вам в эндшпиле!