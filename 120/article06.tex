\subsection*{О паровозах}
\markright{}
\addcontentsline{toc}{subsection}{О паровозах}
\dianamestyle{fullname}

Очень часто, особенно у шахматистов-практиков, возникает сомнение в естественности начальной позиции этюда. Поэтому является, как мне думается, хорошим тоном, чтобы ``неестественность'' возникала в процемме решения. Попробуем рассмотреть тему правдоподобности на базе такого шахматного феномена как ``паровоз''. Паровозом я называю строеные пешки, что, на самом деле, в шахматной партии не редкость.

Данная статья преследует сразу две цели: призыв к поиску предшественников и определение, когда этюд, при наличии предшественника, имеет право на существование.

\begin{multicols}{2}[]
Подобный вопрос появился после изучения ряда пешечных этюдов, в числе которых оказался следующий:

\begin{center}
\begin{diagram}%
  \author{Кок, Теодорус}
  \year{2004}
  \pieces[2+4]{sBg5, wBd3, sBg3, sBg2, sKd1, wKg1}%
  \stipulation{ход черных, белые выигрывают}%
\end{diagram}%
\end{center}\index{Кок Теодорус}

Несложное решение, с акцентом на слабое превращение пешки:

\bold{1... \king{}e2 2.d4 \king{}f3 3.d5 \king{}g4 4.d6 \king{}h3 5.d7 g4 6.d8=\rook{}!}

Превращение в ферзя приводит к пату и ничьей.

Мне вспомнилось давнее предложение моего друга Василия Лебедева составить этюд с таким же финалом. Правда, тогда мы с ним сначала решили поискать предшественников... Что мог бы сделать и Кок. Более того, это могли бы сделать и Шлёссер, Сельман, Ульрихсен и многие другие! Ведь целесообразность поиска предшественников заключается не только в том, чтобы не оказаться неоригинальным, но также и в том, чтобы ознакомиться с путями развития идеи. Добавление некоторой тонкости этюду может сделать его вполне самостоятельным произведением.

Итак, парад предшественников!\\

Первым, и самым главным зачинателем следует назвать Гербстмана. Еще в 1939 году он опубликовал следующий этюд:

\begin{center}
\begin{diagram}%
  \author{Гербстман, Александр}
  \year{1939}
  \pieces[4+5]{sKa6, wBa5, sBb5, wBa4, sBb4, sBb3, sBa2, wBe2, wKa1}%
  \stipulation{выигрыш}%
\end{diagram}%
\end{center}\index{Гербстман Александр}

Решение его следующее:

\bold{1.e4 \king{}xa5}

Или \italic{1... bxa4 2.e5}

\bold{2.e5 \king{}xa4}

\italic{2... \king{}b6 3.e6 \king{}c7 4.axb5}

\bold{3.\king{}b2}

Естественно, не \italic{3.e6?? \king{}a3} и белые умудряются получить мат. Теперь чёрные пытаются залезть в пат:

\bold{3... a1=\queen{}+ 4.\king{}xa1 \king{}a3 5.e6 b2+ 6.\king{}b1 b3 7.e7 b4 8.e8=\rook{}!} с победой.

Уже тогда этот финал был представлен более развёрнуто и богаче. 

Следующим был Бо Линдгрен.

\begin{center}
\begin{diagram}%
  \author{Линдгрен, Бо}
  \year{1945}
  \pieces[2+5]{sBb5, sKa4, wBe4, sBb3, sBb2, sLa1, wKb1}%
  \stipulation{выигрыш}%
\end{diagram}%
\end{center}\index{Линдгрен Бо}

Здесь этюдист отличился тем, что добавил чёрным слона. Целесообразность добавления фигуры состоит в унификации решения: после слабого превращения выигрыш достигается строго единственным образом.

Начальная позиция вызывает серьезные опасения в умении чёрных играть в шахматы. 

Решение:

\bold{1.e5 \king{}b4! 2.e6 \king{}a3! 3.e7 b4 4.e8=\rook{}! \king{}a4 5.\rook{}e5 \king{}a3 6. \rook{}a5\mate{}}

Смотрел этот пат и Хильдебранд, но введенная им вступительная игра не отвечает на вопрос, откуда же у черных ``паровоз''? Развитие идеи могло бы идти по пути страивания черных пешек. Это попытался сделать Яноши. Наиболее правдоподобная версия его этюда оказалась с нерешаемостью.

\begin{center}
\begin{diagram}%
  \author{Яноши, Эрвин}
  \year{1962}
  \pieces[5+6]{sBc7, wBa6, sBb5, wKd4, sBa3, sBb3, wBc3, wBe3, wSb2, sKg2, sLa1}%
  \stipulation{выигрыш}%
\end{diagram}%
\end{center}\index{Яноши Эрвин}

Сомнительна позиция белого коня, -– с какой это стати он закупорил чёрного слона? Грудью на амбразуру?!

Белые легко получают ферзя, однако взамен получают сильную проходную, с которой приходится бороться.

\bold{1.a7 c5+}

Пешка стремится занять более удобную позицию. Взятие слоном не помогало чёрным - \italic{1... \bishop{}xb2 2.a8=\queen{}+ \king{}f2 3.\queen{}h1 c5+ 4.\king{}d3 c4+ 5.\king{}d4 b4 6.\king{}xc4}.

\bold{2.\king{}d3!}

В случае \italic{2.\king{}xc5} ничья должна была достигаться следующим образом: \italic{2... \bishop{}xb2 3.a8=\queen{}+ \king{}f2 4.\king{}b4 a2 5.\king{}xb3 a1=\queen{} 6.\queen{}xa1 \bishop{}xa1 7.e4 b4! 8.\king{}xb4 \king{}e3 9.e5 \king{}d3 10.e6 \bishop{}xc3+}. Однако, продолжая \italic{4.\queen{}d5! a2 5.\queen{}d2+} белые выигрывали.

\bold{2... axb2 3.a8\queen{}+ \king{}g1}

Чтобы не пустить ферзя на h1.

\bold{4.\queen{}g8+ \king{}f2}

Необходимо дать шах, чтобы отбросить чёрного короля от пешки e3 –- и только потом забрать можно на b3. Кстати, обратите внимание на роль пешки с5: чёрные её бросили вперёд,имея в виду вилку на короля и ферзя.

\bold{5.\queen{}f7+ \king{}e1 6.\queen{}xb3 c4+ 7.\king{}c2 cxb3+ 8.\king{}b1}

Знакомая позиция!

\bold{8... \king{}d2 9.e4 \king{}xc3 10.e5 \king{}b4 11.e6 \king{}a3 12.e7 b4 13.e8=\rook{}! \king{}a4 14.\rook{}e5 \king{}a3 15.\rook{}a5\mate}

Впоследствии Яноши исправил этюд –- просто убрал вступительную игру. И не смотря на то, что первоначальная версия этюда оказалась неправильной, заслуга его в другом –- он показал, как ``на фабриках делают паровозы''!

Удачное развитие ``паровозной'' темы сделал Николай Кралин. Если во всех предыдущих этюдах игра заканчивалась слабым превращением, то в его позиции на этом дело не остановилось. Маэстро добавил превращение в другую слабую фигуру –- коня! Но посмотрим, что же он такое сделал.

\begin{center}
\begin{diagram}%
  \author{Кралин, Николай}
  \year{1999}
  \pieces[6+8]{sBa7, wBa6, wBd6, sBe5, sBf5, sBg5, sBh5, wBe4, wBh4, wKf3, wLg3, sKh3, sBg2, sLh1}%
  \stipulation{выигрыш}%
\end{diagram}%
\end{center}\index{Кралин Николай}

\bold{1.\king{}f2 f4! 2.d7}

Плохо брать \italic{2.hxg5? h4!} и здесь уже победа будет за чёрными.

\bold{2... fxg3+ 3.\king{}g1 g4 4.d8=\rook{}!}

Знакомо? Да-да, в точности игра из предыдущих этюдов... Почему плохо превращать в ферзя –- понятно. А вот превращение в коня опровергается так: \italic{4.d8=\knight{}? \king{}xh4 5.\knight{}f7 \king{}h3 6.\knight{}xe5 h4}.

\bold{4... \king{}xh4 5.\rook{}g8!}

Нельзя выпустить чёрного короля. Например, \italic{5.\rook{}d7? \king{}g5! 6.\rook{}xa7 h4 7.\rook{}g7+ \king{}f4! 8.a7 h3 9.\rook{}h7 h2+ 10.\rook{}xh2 gxh2+ 11.\king{}xh2 g1=\queen{}+ 12.\king{}xg1 \bishop{}xe4} с ничьей.
Теперь идёт форсированная, но не лишенная прелести игра:

\bold{5... \king{}h3 6.\rook{}g6! \king{}h4 7.\rook{}g7 \king{}h3 8.\rook{}b7! h4 9.\rook{}b6 axb6 10.a7 b5 11.a8=\knight{}!}

Еще одно слабое превращение!

\bold{11... b4 12.\knight{}c7 b3 13.\knight{}e6 b2 14.\knight{}g5\mate{}}

Замечательная игра!\\

Завершить обзор мне хотелось бы своей попыткой построения ``паровоза''. Она коротка, но, надеюсь, содержательна.

\begin{center}
\begin{diagram}%
  \author{Кройтор, Михаил; Лебедев, Василий}
  \year{2008}
  \pieces[3+5]{wBd5, sBg5, sBf4, sBg4, sKh4, wSe2, sBf2, wKh2}%
  \stipulation{выигрыш}%
\end{diagram}%
\end{center}\index{Кройтор Михаил}\index{Лебедев Василий}

Плохо сразу подводить короля: \italic{1.\king{}g2? f3+ 2.\king{}xf2 fxe2 3.d6 g3+ 4.\king{}xe2 \king{}h3 5.d7 g2} с ничьей. Надо строить ``паровоз''!

\bold{1.\knight{}g3! fxg3+ 2.\king{}g2 f1=\queen{}+ 3.\king{}xf1 \king{}h3 4.\king{}g1 g2 5.d6 g3 6.d7 g4 7.d8=\rook{}!}

Конечно, и последний этюд тоже внушает опасения -– как у черных появилась пара сдвоенных пешек? Однако на этот вопрос ответить уже легче -– так, пешка на f2 пришла с e3, взяв какую-то фигуру... В общем, можно попытаться развить игру в позиции, чтобы она заиграла новыми красками.
\end{multicols}