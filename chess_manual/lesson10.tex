\chapter{Как правильно начинать партию. Типичные ошибки в дебюте.}

\section{Методические указания}

Объясняя детям правила развития в дебюте, сразу показывай те им пример. Попросите записать основные правила. Акцентируйте внимание детей на том, что не все соперники будут ходить согласно этим правилам. Это значит, что они должны найти, как ``использовать'' промашки соперника.

\section{Содержание занятия}

Шахматная партия делится на 3 части: 

\begin{enumerate}
\item \textbf{Начало игры}, которое называется \textbf{Дебют} (от французского début -- начало, возникновение);
\item \textbf{Середина игры} -- \textbf{Миттельшпиль} (от немецкого mitte -- середина и spiel -- игра);
\item \textbf{Конец игры} -- \textbf{Эндшпиль} (от немецкого ende -- конец и spiel -- игра).
\end{enumerate}

Дебютом называется начальная стадии игры и длится она примерно 10-15 ходов. Основная ``цель'' дебюта -- быстро и правильно развить фигуры, а также помешать развитию соперника. До изучения теории шахматных дебютов важно знать основные правила развития в дебюте:

\begin{enumerate}
\item Лучше начинать партию с хода центральной пешки (королевской или ферзевой) е2-е4 либо d2–d4. Это помогает захватить центр и развить свои фигуры.
\item Первыми надо развивать лёгкие фигуры: коней и слонов.
\item Фигуры лучше расположены в центре -- имеют больше возможностей.
\item Важно позаботиться о безопасности короля -- сделайте рокировку.
\item Попытайтесь контролировать центр.
\item Если есть возможность развить фигуру и напасть одновременно: ходите так.
\end{enumerate}

Пример возможного развития в дебюте:

\newchessgame[id=DebutSample]
\mainline{1. e4 e5} 
Правило 1 -- начинаем с центральных пешек. 
\mainline{2. Nf3} 
Правила 2,3 и 6 -- выводим лёгкую фигуру ближе к центру и одновременно нападаем на пешку е5.
\mainline{2... Nc6} 
Выводим лёгкую фигуру и защищаем пешку.
\mainline{3. Bc4 Bc5} 
Обе стороны развивают ближе к центру слонов и освобождают пространство для рокировки.
\mainline{4. d3}
Открываем дорогу второму слону. 
\mainline{4... Nf6 5. O-O O-O} 
И белые, и чёрные прячут короля.
\mainline{6. Bg5 d6 7. Nbd2 Bd7}
все лёгкие фигуры выведены, рокировка сделана, центральные пешки выдвинуты – далее следует развить тяжёлые фигуры (ферзя и ладей) и переходить к следующей стадии, середине игры. 

А вот что НЕ НУЖНО делать в дебюте:
\begin{enumerate}
\item выводить рано ферзя.
\item ходить 2 раза одной и той же фигурой без крайней необходимости.
\item двигать больше, чем 2-3 пешки в дебюте.
\item раскрывать короля.
\item отдавать просто так фигуры и пешки.
\end{enumerate}

Ошибки в дебюте могут привести к быстрому проигрышу. Вот несколько примеров типичных дебютных ошибок:

\begin{itemize}
\item Копирование ходов соперника: около 100 лет назад известный шахматист Сэм Лойд поспорил с одним любителем, который утверждал, что копируя ходы соперника может играть на равных с чемпионом. Вот, что получилось: 

\newchessgame
\mainline{1. d4 d5 2. Qd3 Qd6 3. Qh3?? Qh6?? 4. Qxc8#}

Чёрные опоздали с матом на 1 ход (диаграмма 40).
 
\begin{center}
\begin{tabular}{ c }
\chessboard \\
Д. 40 мат в зеркальной партии \\
\end{tabular}
\end{center}

\item Раннее развитие ферзя: Проблема раннего развития ферзя в том, что на самую сильную фигуру будут всё время нападать.

\newchessgame
\mainline{1. e4 e5 2. Nf3 Qf6?!} 

Первая ошибка чёрных -- рано ввели в игру ферзя.

\mainline{3. Bc4 Qg6?!}

Вторая ошибка -- повторение ходов одной и той же фигурой.

\mainline{4. O-O Qxe4 5. Bxf7+!}

Диаграмма 41: попросите детей найти наилучшее продолжение после 5... \king{}xf7, объясните им, что рано выведенный ферзь очень часто попадает под ``вилку'' и теряется.
 
\begin{center}
\begin{tabular}{ c c }
\chessboard & \chessboard[setfen=rnbk1bnr/pppp1Bpp/5q2/8/8/8/PPPP1PPP/RNBQR1K1 w] \\
Д. 41 после 5-го хода белых & Д. 42 мат в 1 ход \\
\end{tabular}
\end{center}

В партии последовало

\mainline{5... Kd8 6. Nxe5 Qxe5 7. Re1 Qf6}

Найдите мат в 1 ход в получившейся позиции (диаграмма 42).
\end{itemize}

\section{Домашнее задание}

Разберите следующую партию 
\newchessgame
\mainline{1. e4 e5 2. Nf3 f6 3. Nxe5 fxe5 4. Qh5+ Ke7 5. Qxe5+ Kf7 6. Bc4+ Kg6 7. Qf5+ Kh6 8. d4+ g5 9.h4 d6 10. hxg5+ Kg7 11. Qf7#}
и ответьте на вопросы:
\begin{enumerate}
\item Почему чёрные не успели развить фигуры?
\item Какие ошибки они совершили?
\end{enumerate}
Как следует играть после: 3... \queen{}e7 4. \knight{}f3 \queen{}xe4+? 5. \bishop{}e2
