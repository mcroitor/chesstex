\chapter{Рокировка}
\section{Методические указания}

Следует уделить особое внимание рокировке, так как дети не сразу усваивают, в каких случаях рокировка невозможна. Кроме того, они часто неправильно делают рокировку: сначала передвигают ладью, а потом короля. В этом случае считается, что надо делать ход ладьёй.

\section{Содержание занятия}

Рокировка –- перемещение короля и одной из ладей того же цвета, находящихся в начальной позиции, по крайней горизонтали. Рокировка выполняется следующим образом: король перемещается с начального поля на два поля по направлению к ладье, стоящей на начальном поле, затем ладья переставляется через короля на следующее за ним соседнее поле. На диаграмме 19 обе стороны могут делать рокировки как на королевском фланге, так и на ферзевом.

\begin{center} 
\begin{tabular}{ c c }
\chessboard[
setfen=r3k2r/pppbqpp1/1bnp1n1p/4p3/4P3/PBNPBN2/1PPQ1PPP/R3K2R,
showmover=false] 
&
\chessboard[
setfen=r3k1nr/pp2qpp1/1b1pp2p/1b6/4P3/PBQ2N2/1PP2PPP/R1B1K2R,
showmover=false] \\
\textbf{Д. 19 рокировки} & \textbf{Д. 20 возможна ли рокировка?}
\end{tabular}
\end{center}

Рокировка невозможна: 
\begin{enumerate}
\item Если король ранее, до рокировки, делал ход; 
\item С той ладьей, которая ранее, до рокировки делала ход.
Рокировка временно невозможна в следующих случаях (Д. 20): 
\item Если атаковано одной из фигур соперника поле, на котором или стоит король, или которое он должен пересечь, или которое он должен занять; 
\item Если между королем и ладьей, с которой может быть произведена рокировка, находится фигура.
\end{enumerate}

\section{Домашнее задание}

На следующих диаграммах проверьте, если возможна рокировка. Если возможна, то в какую сторону? Если рокировка невозможна, укажите причины.

\begin{center} 
\begin{tabular}{ c c }
\chessboard[setfen=8/8/8/8/8/8/8/8 w] 
&
\chessboard[setfen=8/8/8/8/8/8/8/8 w] \\
\textbf{Д. 4.1 Ход белых} & \textbf{Д. 4.2 Ход белых} \\
\chessboard[setfen=8/8/8/8/8/8/8/8 w] 
&
\chessboard[setfen=8/8/8/8/8/8/8/8 w] \\
\textbf{Д. 4.3 Ход белых} & \textbf{Д. 4.4 Ход белых}
\end{tabular}
\end{center}