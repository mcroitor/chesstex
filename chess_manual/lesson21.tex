\chapter{Атака на нерокировавшего короля}

\section{Методические указания}

Напомните детям что такое атака и какова её цель (мат либо получение большого материального перевеса). Целью этого занятия является показать детям, что нерокировавший король часто находится в опасности. При анализе партий, предлагайте детям самим найти лучший ход.

\section{Содержание занятия}

Атака на короля – один из самых быстрых планов для выигрыша, а король в центре может стать очень соблазнительной мишенью. Однако надо учитывать несколько моментов, таких как расположение своих фигур и фигур противника, владение центром, безопасность собственного короля. Оцените позицию и вперёд!

Давайте разберём несколько партий знаменитых шахматистов и посмотрим, как они проводили атаку на короля.

Андерсен – Кизерицкий, Лондон 1851

1.е4 е5 2.f4 ef 3.Сс4 Фh4+ 4.Крf1 b5 5.C:b5 Кf6 6.Кf3 (выводим коня с темпом – нападаем на ферзя, рано введённого в игру) 6… Фh6 7.d3 Кh5 8.Кh4 Фg5 (лучше 8… Сb7) 9.Кf5 c6 10.g4 Кf6 11.Лg1! cb 12.h4 Фg6 13.h5 Фg5 14.Фf3 (угрожая поймать ферзя 15.С:f4, обратите внимание детей, что чёрный ферзь, который рано бросился в атаку, теперь находится в серьёзной опасности) 14…Кg8 (конь вынужден вернуться на исходную позицию) 15.C:f4 Фf6 16.Кс3 Сс5 17.Кd5 (очень сильно 17.d4) 17…Ф:b2 18.Cd6 (в настоящее время доказано, что лучшим продолжением атаки было 18.Се3) 18…С:g1? (шансы на защиту давало 18…Ф:а1+ 19.Кре2 Фb2!) 19.e5! Ф:а1+ 20.Кре2 Ка6 (упорнее 20…Са6) 21.К:g7+ Крd8 22.Фf6+ К:f6 23.Се7\# Партия показывает, как опасно играть одной – двумя фигурами, пренебрегая развитием.

Рассмотрим ещё одну партию:

Морфи – любитель, Нью Орлеан 1858
1.e4 e5 2.Кf3 Кc6 3.Cc4 Кf6 4.d4 ed 5.Кg5 d5 6.ed К:d5? (правильно 6…Фе7+ 7.Фе2 Кb4, нападая на пешки c2 и d5 или 7.Крf1 Ke5 8.Ф:d4 K:c4 9.Ф:с4 h6 10.Кf3 Фc5 и чёрные разманивают атакующие фигуры, успевая вовремя «спрятать» короля) 7.0-0! (белые «прячут» своего короля и развивают ладью) 7… Ce7 (если 7…Се6 8.Ле1 Фd7, то 9.К:f7 Кр:f7 10.Фf3+ Крg8 11.Л:е6) 8.К:f7! (первая жертва, призванная завлечь короля в центр на открытое пространство) 8… Кр:f7 9.Фf3+ Кре6 (лучше было отдать фигуру путём 9…Сf6 10.C:d5+ Ce6 11.C:e6+ Кр:е6, но после 12.Сf4 у белых атака) 10.Кс3! (в этой партии активно используется такой тактический удар, как связка) 10… dc 11.Лe1+ Кe5 (теперь оба коня связаны) 12.Cf4 Cf6 13.C:e5 C:e5 14.Л:е5+! Кр:е5 15.Ле1+ (все белые фигуры участвуют в атаке, чёрные же либо находятся на исходных позициях, либо под нападением) 15… Крd4 16.C:d5 Ле8 (и другие ходы не спасают, например 16…cb 17.Ле4+) 17.Фd3+ Крс5 18.b4+ Кр:b4 19.Фd4+ с форсированным матом. В этой партии Морфи продемонстрировал основной метод атаки на короля – максимальное вскрытие.

\section{Домашнее задание}

Запишите финальную позицию партии Морфи – любитель. Найдите форсированный мат на все возможные продолжения чёрных. Решения запишите.