\chapter{Эндшпиль: проходная пешка}

\section{Методические указания}


\section{Содержание занятия}

Часто пешечный эндшпиль сводится к решению задачи – догонит или не догонит король пешку противника, рвущуюся в ферзи. На этот вопрос помогает ответить понятие квадрата пешки и связанное с ним правило:

Квадратом пешки называется воображаемый квадрат в сторону короля противника, образованный отрезками, равными по длине расстоянию пешки до поля превращения. 

Если король при своём ходе попадает в квадрат пешки, то он пешку задерживает.

Пример квадрата пешки изображен на диаграмме 71 (отмечен крестиками).
 
 
Д. 71 квадрат пешки
 
Д. 72 ход черных
 
При своём ходе чёрные входят в квадрат пешки, поэтому они догоняют пешку:

1… Kpf5 2.c6 Kpe6 3.c7 Kpd7

При своём ходе белые выигрывают, потому что черный король не попадает в квадрат пешки:

1.c6 Kpf6 2.c7 Kpe7 3.c8Ф с победой.

Однако, с пешкой в начальной позиции правило квадрата пешки не действует, потому что свой первый ход пешка может сделать через клетку.

Поэтому, не смотря на очередь хода, черные, в позиции, изображенной на диаграмме 107, отыграть пешку не могут: 1… Kpc2 2.b4 и т.д.
 
Д. 73 Ничья независимо от очереди хода.

Правило квадрата пешки имеет большое значение в оценке пешечных окончаний. Например, позицию, изображенную на диаграмму 108, можно сразу определить как ничейную, независимо от очереди хода. Это можно утверждать с уверенностью, потому что у каждой из сторон есть защищённые проходные пешки. Короли же не могут выходить из квадратов пешек, иначе соперник проведёт пешку в ферзи. Ни одна из сторон не может усилить свою позицию.
В заключение хочется привести известный этюд Рети, в котором, на первый взгляд, опровергается правило квадрата пешки:

Рихард Рети, 1921
 
Д. 74 ничья

Визуально видно, что черный король находится в квадрате пешки с6, белому же королю не хватает 3-х ходов для достижения черной пешки. Возникает вопрос, как же можно добиться ничьей? Здесь надо сказать об одном важном принципе шахмат, позволяющем добиваться результата, а именно – создавать множественные угрозы. Ничья в этюде Рети достижима в двух случаях: если король догонит пешку или если белые смогут провести свою пешку в ферзи. Поэтому, в решении, белые преследуют обе цели сразу:

1.Kpg7! h4 2.Kpf6! h3 3.Kpe6! h2 4.c7 Kpb7 5.Kpd7 h1Ф 6.c8Ф+ с ничьей. А что бы произошло, если бы черные попробовали сначала забрать белую пешку? Тогда бы белые попадали в квадрат пешки, и партия закончилась бы ничьей:
2… Kpb6 3.Kpe5! (белым нужен еще ход, чтобы догнать пешку – 3… Kp:c6 4.Kpf4 и ничья) 3… h3 4.Kpd6 h2 5.c7 Kpb7 6.Kpd7 h1Ф 7.c8Ф+ опять с ничьей.

\section{Домашнее задание}

Оцените позиции. Укажите примерный план игры.
 
 
18.1 ход черных
 
18.2 ход белых
 
 
18.3 ход белых
 
18.4 ход белых
