\chapter{История появления и развития шахмат}
\section{Методические указания}

Расскажите детям легенду возникновения шахмат и немного из истории развития. В качестве литературного источника можно использовать книгу Михаила Садовяну.

\section{Содержание занятия}
Шахматы являются одной из самых древних игр, их история насчитывает не менее 1500 лет. Никто не знает кто и когда на самом деле придумал шахматы, но, согласно самой распространённой легенде, шахматы возникли в Индии: однажды индийскому правителю –- Радже –- стало скучно и он обещал наградить того, кто его развлечёт. Один брамин пришёл к Радже и показал ему интересную игру –- 2 войска, белое и чёрное, сражаются на специальном поле. Игра очень понравилась Радже и он спросил, чего хочет брамин за своё изобретение. Хитрый брамин попросил столько пшеничных зёрен, сколько окажется на шахматной доске, если на первую клетку положить одно зерно, на вторую -- два зерна, на третью - четыре зерна и т. д. Попросите детей посчитать количество зерён. Оказалось, что такого количества зерна нет на всей планете (оно равно 264 - 1~1,845x1019 зёрен, чего достаточно, чтобы заполнить хранилище объёмом 180 км).
 
В V – VI веке нашей эры в Индии возникла первая известная нам игра, похожая на шахматы –- чатуранга. В этой игре было всего 4 фигуры: король, колесница (прародитель ладьи), слоны, конница и пехота. Играли в чатурангу вчетвером и чтобы выиграть необходимо было уничтожить всё войско соперников (Д. 1). Из Индии шахматы попали на Арабский Восток, где и получили современное название: Шах Мат, что в переводе означает "Властитель Повержен".

\begin{center}
\begin{tabular}{ c }
\chessboard[setfen=BP2krnb/NP2pppp/RP6/KP6/6PK/6PR/pppp2PN/bnrk2PB, showmover=false] \\
\textbf{Д. 1 начальная позиция чатуранги} \\
\end{tabular}
\end{center}

В VIII –- IX веке шахматы попали в Европу и с XV века правила игры практически не менялись.

Несколько знаменательных дат из истории шахмат:
\begin{itemize}
\item 1119 год -- Состоялась первая партия по переписке между королём Англии Генрихом I и королём Франции Людовиком VI. 
\item 1490 год -- Впервые использовано правило "взятия на проходе". 
\item 1550 год -- В Италии создан первый шахматный клуб. 
\item 1575 год -- Первый шахматный турнир, который прошел при королевском дворе в Мадриде. 
\item 1744 год -- Первый сеанс игры вслепую против двух игроков (Филидор в Париже).
\item 1763 год -- Появление музы шахмат Каиссы в поэме Уильяма Джонса.
\end{itemize}
 
\section{Домашнее задание}

Попросите детей написать свою историю появления шахмат, на основании рассказанной легенды.
