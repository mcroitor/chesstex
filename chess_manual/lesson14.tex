\chapter{Основы миттельшпиля}

\section{Методические указания}

Следует обратить внимание на то, что позиционная игра, оценка позиции являются достаточно сложными для понимания начинающими шахматистами. Намного интересней им тактика (также неотъемлемая часть игры в миттельшпиле), однако следует объяснить детям, что основы позиционной игры важны не меньше, чем тактика и, умея правильно оценивать позицию, создавать план и противодействовать плану соперника – намного легче выиграть. Запаситесь терпением и уделите этим темам отдельное внимание. Перед тем, как показать детям решение примеров, спросите, есть ли у них идеи (учите детей самостоятельно искать решения, анализировать их и принимать правильные решения) и проанализируйте их предложения.

\section{Содержание занятия}

Миттельшпиль – это вторая, после дебюта, стадия шахматной партии. После того, как вы вывели все фигуры и сделали рокировку – начинаются активные боевые действия в миттельшпиле. Что же нужно сделать, чтобы выиграть партию? Давайте рассмотрим наиболее важные элементы игры в середине игры. 

\subsection{Центр}

Давайте вспомним, что такое малый центр (поля e4, d4, e5, d5) и большой центр (добавляются поля c3, c4, c5, c6, d6, e6, f6, f5, f4, f3, e3 и d3). Вспомните основные правила в начале партии: лучше начинать партию с ходя центральной пешки (королевской или ферзевой) е2-е4 либо d2 – d4; фигуры лучше расположены и имеют больше пространства в центре; попытайтесь контролировать центр. Зачем мы следовали этим правилам в дебюте? Для того, чтобы в середине игры наши фигуры имели больше пространства, могли перемещаться по всей доске, а фигуры противника оказались закрыты и малоподвижны. В такой ситуации намного легче провести атаку на короля и завершить партию ещё в миттельшпиле. 

\begin{center}
\begin{tabular}{ c }
Ортега – Корчной, Гавана 1963 \\
\chessboard[setfen=8/8/8/8/8/8/8/8 b] \\
Д. 57 1... ? План за чёрных \\
\end{tabular}
\end{center}

Давайте рассмотрим позицию на диаграмме:  у чёрных явное преимущество во владении центром, пешки d4 и e4 опасно продвинулись вперёд и контролируют поля в лагере белых, чёрный ферзь намного активнее белого. В партии последовало: 1...d3! 2. Cd e3!  (открываем диагональ ферзю и слону и создаём угрозу мата на g2) 3.Cf3 ef+ 4.Кр:f2 Кg4+! 5.Крg1 (5.C:g4 Фg2+и у чёрных решающее преимущество) 5… Фd4+ 6.Ce3 К:е3 7.Ф:е3 Ф:е3+ 8.Л:e3 Cc5  и у чёрных лишнее качество в эндшпиле. Таким образом, контроль центра позволил чёрным провести форсированную атаку и выиграть материал, достаточный для выигрыша партии в эндшпиле.

Один из сильнейших теоретиков шахмат Арон Нимцович говорил: ``Нужно с максимальным внимание относится к стратегии центра; надо избегать всякого преждевременного ``поворота'' в сторону флангов (из опасения неприятельского нападения на центр) и стараться оперировать под знаком централизации!''.

\subsection{Расположение фигур}

Одним из основных правил игры в миттельшпиле является улучшение положения своих фигур и ухудшение положения фигур соперника. Рассмотрим детально сильные и слабые  стороны расположения некоторых фигур:

Слон -- проявляет силу на открытых диагоналях, на которых можно использовать его ``дальнобойность''.  Для ограничения слона противника часто полезно поставить против него пешечный клин (``рогатку''). Чтобы улучшить такого ограниченного слона, можно или подорвать этот клин пешкой, или перевести слона на другую диагональ. 

Примеры:

\begin{center}
\begin{tabular}{ c }
Алёхин – Васич, 1931 \\
\chessboard[setfen=8/8/8/8/8/8/8/8 w] \\
Д. 58 Ход белых \\
\end{tabular}
\end{center}

В этой позиции белые слоны явно расположены лучше чёрного «закрытого» слона. Давайте попробуем воспользоваться хорошим расположением слонов: 1.Ф:е6+ (жертва ферзя, белый слон на а3 отрезает возможность короля уйти на f8 либо закрыться от шаха путём Фе7) 1…fe 2. Cg6\#

\begin{center}
\begin{tabular}{ c }
\chessboard[setfen=8/8/8/8/8/8/8/8 w] \\
Д. 59 Пример пешечного клина \\
\end{tabular}
\end{center}
 
Оцените силу белых и чёрных слонов. 

«Конь на середине доски, защищённый пешкой и гарантированный от нападения неприятельской пешки, сильнее слона и по силе почти равен ладье», - Зигберт Тарраш.

\begin{center}
\begin{tabular}{ c }
Фишер – Болбочан, Стокгольм 1962 \\
\chessboard[setfen=8/8/8/8/8/8/8/8 w] \\
Д. 60 ход белых \\
\end{tabular}
\end{center}

Ход белых. Поле d5 – идеально подходит для белого коня, однако его защищает чёрный конь. В партии последовало: 1.С:b6 Ф:b6 2.Кd5! – обратите внимание, в этой позиции белый конь значительно лучше чёрного слона. Белый конь расположен в самом центре, контролирует поля в лагере противника, защищён своей пешкой и ни одна чёрная пешка не может на него напасть.

Ладья – ``Когда дело доходит до открытого боя, здесь ладья очень опасна. Выключить ладью соперника из игры хотя бы на время – большая удача'', - Анатолий Мацукевич.

Ладья, будучи фигурой дальнобойной, лучше расположена на открытых и полуоткрытых линиях. Открытая линия – вертикаль, на которой нет пешек, полуоткрытая – вертикаль, на которой есть пешки только одного цвета.


\begin{center}
\begin{tabular}{ c }
Менчик – Томас, Лондон 1932 \\
\chessboard[setfen=8/8/8/8/8/8/8/8 w] \\
Д. 61 ход белых \\
\end{tabular}
\end{center}

В партии последовало: 1.Ф:h7+ Кр:h7 2.Лh1\# - белая ладья заняла открытую линию и поставила мат. 

Очень важно следить за действиями соперника, ведь он тоже хочет выиграть! Одним из важных элементов шахматной партии является профилактика – поиск и анализ сильных ходов соперника и принятие мер для пресечения этих возможностей.

Ещё одним важным элементом при поиске хода - это создание угроз! Нападайте, ищите удары и комбинации, играйте активно! Чемпион Мира Х. Р. Капабланка говорил: ``захватывайте инициативу при всяком удобном случае. Владеть инициативой – это уже преимущество'', а знаменитый шахматный тренер Марк Дворецкий, считает, что ``развивать инициативу – означает находить объекты для нападения, тем самым принуждая соперника к обороне''.


\section{Домашнее задание}

Оцените позицию (расположение фигур, открытые линии) и предложите план игры.
 
\begin{center}
\begin{tabular}{ c c }
\chessboard[setfen=8/8/8/8/8/8/8/8 w] & \chessboard[setfen=8/8/8/8/8/8/8/8 w] \\
14.1 ход белых & 14.2 ход белых \\
\end{tabular}
\end{center}
