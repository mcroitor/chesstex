\chapter{Классификация дебютов. Итальянская партия.}

\section{Методические указания}

Расскажите о классификации дебютов. Покажите детям 1 вариант из итальянской партии. Попросите записать его.

После того, как покажете детям вариант итальянской партии, попросите их разыграть его между собой, как за белых, так и за чёрных.

Расскажите детям, что теория дебютов очень обширна и, каждый может выбрать вариант, который ему нравится и подходит, но пока им стоит научиться правильно начинать партию и играть даже в незнакомых позициях.


\section{Содержание занятия}

Существует множество различных дебютов и шахматисту необязательно знать их все, поэтому мы расскажем вам только об общей классификации:

\begin{itemize}
\item Открытые -- характерные первые ходы \emph{1.е4 е5}, примеры таких дебютов: итальянская партия, дебют четырёх коней и т.д.
\item Закрытые -- первый ход белых не 1. е4, например ферзевый гамбит и др. -- \emph{1. d4}, английское начало \emph{1. с4} и др.
\item Полуоткрытые -- на ход белых \emph{1.е4}, чёрные отвечают как угодно, кроме 1... е5, примеры: сицилианская защита \emph{1... с5}, французская защита \emph{1... е6} и т.д.
\end{itemize}

Итальянская партия -- один из древнейших дебютов, относится к открытым началом. На итальянском звучит, как Джокко Пьяно -- тихое начало.

\newchessgame
\mainline{1. e4 e5 2. Nf3 Nc6 3. Bc4 Bc5}

\begin{center}
\begin{tabular}{ c }
\chessboard \\
\textbf{Д. Итальянская партия} \\
\end{tabular}
\end{center}

Именно эти 3 хода являются характерными для итальянской партии. Как белые, так и чёрные, придерживаются идеи захвата центра и быстрого развития фигур.

\mainline{4. d3}

Спокойный ход, защищает пешку е4 и открывает дорогу чернопольному слону. 

\mainline{4... Nf6}

Развивает коня ближе к центру и готовит рокировку.

\mainline{5. Nc3 d6}

С той же идеей, что ход белых 4. d3

\mainline{6. Be3}

Белые развивают слона с ``темпом'': на ответ чёрных \emph{6... \bishop{}g4}, белые заберут слона с поля с5 и сдвоят чёрные пешки линии ``с''. Ход \emph{6... \bishop{}:е3} является сомнительным, так как сдвоенные по линии ``е'' белые пешки контролируют практически все центральные поля и, после рокировок, ладья на f1 получает полуоткрытую линию и возможность атаки на короля.

\mainline{6... Bb6} 

Сдвоение пешек по линии ``b'' не так страшно для чёрных: пешка приближается к центру и открывается линия для ладьи. 

\mainline{7. h3} 

Так называемая ``форточка'' для короля после рокировки, в которую всегда можно убежать в случае шаха, а также защита от хода \emph{\bishop{}g4} со связыванием коня. 

\mainline{7... Be6 8. Bb3 h6 9. O-O}

\begin{center}
\begin{tabular}{ c }
\chessboard \\
\textbf{Д. Ход чёрных} \\
\end{tabular}
\end{center}

Развитие окончено и партия переходит в следующую стадию.

\section{Домашнее задание}

Выучить показанный вариант партии. Разыграть примерную партию.
