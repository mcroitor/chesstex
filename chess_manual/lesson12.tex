\chapter{Тактические удары}

\section{Методические указания}

Перед началом занятия напомните детям, что такое нападение, шах, атака. Данная тема привлекательна для детей, потому что благодаря тактическим ударам они могут добиться преимущества: выиграть фигуру или поставить мат.

\section{Содержание занятия}

В шахматах существуют приёмы, называемые тактическими ударами, которые позволяют добиться преимущества:
•	Двойной удар
•	Двойной шах
•	Открытое нападение
•	Открытый шах
•	Связка
Эти приёмы характеризуются тем, что они создают сразу несколько угроз, от которых противнику сложнее защититься. Рассмотрим их по порядку.
Двойной удар – некоторая фигура нападает на 2 фигуры сразу. Частным случаем двойного удара является вилка – нападение коня или пешки на 2 фигуры или более. Пример вилки показан на диаграмме 48. Белые, ходом 1.Kf5+, нападают конём на короля и слона одновременно. Черные не могут защитить одновременно обе фигуры, поэтому слона они теряют.
 
 
Д. 48 ход белых
 
Д. 49 ход белых
 
Другой пример двойного удара показан на диаграмме 49: ферзь, ходом 1.Фс1+ атакует короля и ладью. Черные должны увести короля из под шаха, поэтому следующим ходом белые забирают ладью.
Открытое нападение – один из тактических приёмов, характеризующийся наличием специального механизма – батареи. Батареей называется взаимное расположение дальнобойной фигуры (ферзя, ладьи или слона) и разноходящей с ней фигуры того же цвета. Некоторая фигура (перекрывающая) перекрывает дальнобойную фигуру (перекрываемую). В результате ухода перекрывающей фигуры или пешки открывается линия действия перекрываемой фигуры и эта открывшаяся фигура создает какую-то угрозу.
 
 
Д. 50 Ход белых
 
Д. 51 ход белых
 
В позиции, изображенной на диаграмме 50, можно заметить, что любой отскок белого коня грозит нападением белого слона на черного коня. Это и есть действие батареи – открытое нападение. Чтобы добиться выигрыша, белые производят одновременное нападение конём на слона – 1.Kd6 Черные не могут защитить одновременно две фигуры, поэтому одну из них они теряют.
Частным случаем открытого нападения является открытый шах (диаграмма 51).
Двойной шах – есть нападение на короля двумя фигурами сразу. Это возможно при помощи специальной конструкции, называемой батареей. Является частным случаем открытого шаха (Д. 52).
 
 
Д. 52 двойной шах
 
Д. 53 ход белых
 
В этой позиции на черного короля направлен белый слон, но между ними стоит белая ладья. Если ладья уходит на любое поле, то объявляется открытый шах. В случае отхода ладьи на поля с6 или g3 открытый шах сопровождается последующим выигрышем ферзя. Однако в этой позиции есть более сильный ход: 1.Лс8++ х с двойным шахом (от ладьи и слона) и матом.
Связка – тактический приём, при котором дальнобойная фигура (называемая связывающей) нападает на фигуру или пешку соперника (связываемая фигура), за которой, на линии нападения расположена другая неприятельская фигура (равнозначная или более ценная) либо важный пункт. Пример связки показан на диаграмме 53. 
В этой позиции связывающей фигурой является ферзь, а связанной – черный слон. На диаграмме показывается пример абсолютной связки, потому что черный слон не может двигаться – так как открывается шах черному королю. В этой позиции белые могут сделать другую связку, которая приводит к выигрышу фигуры. Попросите детей найти её. К выигрышу приводит ход 1.Фа2 со связкой ладьи.


\section{Домашнее задание}

Определить и нанести тактический удар.
 
 
12.1 ход белых
 
12.2 ход белых
 
 
12.3 ход белых
 
12.4 ход белых
 
 
12.5 ход белых
 
12.6 ход белых
 
 
12.7 ход белых
 
12.8 ход белых
