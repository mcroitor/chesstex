\chapter{Реализация материального преимущества}

\section{Методические указания}

Одной из частых проблем начинающих (и не только) шахматистов – это неумение реализовать полученное преимущество. Мы не станем останавливаться на реализации позиционного преимущества, а рассмотрим только сложности в реализации материального. В начале занятия спросите у детей:

\begin{itemize}
\item Какова цель партии?
\item Что такое материальное преимущество?
\item Достаточно ли для выигрыша партии забрать все фигуры противника?
\item Каков результат партии, если на доске мат? А если пат?
\end{itemize}

\section{Содержание занятия}

Часто случается так, что одна из сторон получила материальный перевес в виде фигуры. В данном случае самым простым способом реализации материального преимущества будет размен всех оставшихся равноценных фигур, чтобы потом заматовать одинокого короля. При этом следует быть внимательным и не поставить пат. Другим способом является прямая атака на короля. Давайте рассмотрим несколько примеров.

Шлехтер – Маршалл, Вена 1908
 
Д. 90 ход белых

В партии последовало: \bold{27.Ф:d5} (предлагаем размен) \bold{Фе7 28.Фd6 Ф:d6} (или 28… Фg5 29.Фb8+ Крh7 30.Ce4+ попросите детей найти все возможные ответы чёрных и последующий выигрыш за белых (на 30…g6 31.Лf7X; на 30…Фg6 просто забираем ферзя; на 30…f5 31.ef+ g6 и либо 32.Фc7+ либо 32.f7)) \bold{29.ed Лd2 30.C:b7 Л:d6 31.Крg2 g6 32.Лf2 Крg7 33.Лс2} и чёрные сдались. Попросите детей «доиграть» эту позицию. Если белые не смогут выиграть: найдите ошибку и проанализируйте причину. 
Вполне возможен был и другой путь реализации преимущества: \bold{27.С:d5} угрожая пешке на f7. Не проходит 27…Фh3 ввиду 28.С:f7+ Крh8 29.Фd8+ и 30.Фg8X
Выигрывает и указанное Таррашем красивое продолжение \bold{27.Л:f7}, однако не стоит показывать его детям. Лучше реализовать просто и надёжно.

\section{Домашнее задание}

