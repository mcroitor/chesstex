\chapter{Матование одинокого короля: линейный мат; мат ферзём и королём.}

\section{Методические указания}

Повторите с детьми геометрию шахматной доски -- вертикали и горизонтали. Объясните что такое ``крайняя линия''.

Матование одинокого короля лучше показывать на демонстрационной доске, подробно объясняя принцип построения мата.

После объяснения темы, посадите детей парами и дайте им попрактиковаться в построении мата, меняясь местами: один ставит мат, другой играет одиноким королём. При этом, попросите детей считать количество ходов до мата.

\section{Содержание занятия}

Одинокому королю возможно поставить мат следующими фигурами / минимальными сочетаниями фигур: двумя ладьями, ферзём, одной ладьёй, двумя слонами, слоном и конём. Маты одним слоном или одним конём невозможны теоретически, мат двумя конями невозможен практически. Мат слоном и конём отличается сложной геометрией и входит в учебный план для шахматистов II – I разрядов.

\subsection*{Линейный мат.}

Начнём мы с самого простого: линейный мат. Линейный ставится тяжёлыми фигурами: чаще двумя ладьями, реже ладьёй и ферзём либо двумя ферзями. В первую очередь необходимо определить крайнюю линию, на которую мы будем оттеснять короля: если это линии ``a'' либо ``h'', то мат мы ставим по вертикалям, если это ``1'' либо ``8'' линии, то по горизонталям. Обычно выбирается ближайшая к королю слабейшей стороны линия. Принцип построения мата достаточно прост. Допустим, мы выбираем линейный мат по горизонталям. Одна ладья ``отрезает короля'' от противоположной части доски, а вторая ставит шах по горизонтали. Король вынужден отступить. Теперь вторая ладья отрезает, а первой мы ставим шах (диаграмма 31). 
 
\begin{center}
\begin{tabular}{ c c }
\chessboard[setfen=8/8/8/1k4R1/7R/8/8/7K b,
color=red,
pgfstyle=straightmove,
markmoves={b5-c6, h4-h6}] & 
\chessboard[setfen=4k2R/6R1/8/8/8/8/8/K7 b] \\
\textbf{Д. 22 И так до последней вертикали} & \textbf{Д. 23 линейный мат}
\end{tabular}
\end{center} 
 
И так до последней горизонтали (диаграмма 32), в результате чего королю соперника некуда отступать. Получается линейный мат.
Важно, чтобы дети, увлёкшись, не ``зевнули'' одну из фигур. При нападении короля на ладью (см. диаграмму 18) необходимо отвести ладью по той же линии, на которой она находится.
 
\begin{center} 
\begin{tabular}{ c }
\chessboard[
setfen=8/8/8/6k1/7R/R7/8/7K,
color=red,
pgfstyle=straightmove,
markmoves={h4-b4}] \\
\textbf{Д. 24 построение линейного мата}
\end{tabular}
\end{center} 

\subsection*{Мат ферзём и королём.}

Существует несколько способов матования одинокого короля ферзём и королём. В первую очередь необходимо оттеснить короля на крайнюю линию. Покажите детям примеры позиций, к которым они должны стремиться (диаграмма 34).
 
\begin{center} 
\begin{tabular}{ c }
\chessboard[setfen=k1K4k/6Q1/Q7/8/8/2K3K1/1Q5Q/1k5k,
markregions={a1-d4, a5-d8, e1-h4, e5-h8},showmover=false] \\
\textbf{Д. 25 различные маты ферзем}
\end{tabular}
\end{center}

Принцип построения мата: ферзь оттесняет короля слабейшей стороны на выбранную крайнюю линию ``ходом коня''.  Рассмотрим позицию на диаграмме 35. 
 
\begin{center} 
\begin{tabular}{ c }
\chessboard[setfen=K7/8/8/8/5Q2/3k4/8/8 b] \\
\textbf{Д. 26 ход черных}
\end{tabular}
\end{center}

Оттеснение будет выглядеть следующим образом: 1. ... \king{}с3 2. \queen{}е4 \king{}b3 3. \queen{}d4 \king{}с2 4. \queen{}е3 \king{}b2 5. \queen{}d3 \king{}с1 6. \queen{}е2
 
 
\begin{center} 
\begin{tabular}{ c c }
\chessboard[setfen=K7/8/8/8/8/8/4Q3/2k5 b] 
&
\chessboard[setfen=K7/8/8/8/8/8/2Q5/k7 b] \\
\textbf{Д. 27 после 6-го хода белых} & \textbf{Д. 28 пат}
\end{tabular}
\end{center} 
 
Далее подходим королём на третью горизонталь и ставим один из матов показанных выше. Распространённой ошибкой детей является следующая позиция, изображенная на диаграмме 37.

Напомните детям, что такое пат и уточните, что королю слабейшей стороны всегда необходимо оставлять ходы.

\section{Домашнее задание}

Ход белых. Мат в 2 хода.
 
\begin{center} 
\begin{tabular}{ c c }
\chessboard[setfen=2K5/7R/8/8/8/5R2/1k6/8 w] 
&
\chessboard[setfen=2R5/k7/5K2/8/8/8/8/5R2 w] \\
\textbf{7.1 мат в 2 хода} & \textbf{7.2 мат в 2 хода} \\
\chessboard[setfen=1k6/R4K2/6R1/8/8/8/8/8 w] 
&
\chessboard[setfen=8/8/1Q6/8/8/k2K4/8/8 w] \\
\textbf{7.3 мат в 2 хода} & \textbf{7.4 мат в 2 хода} \\
\chessboard[setfen=8/8/8/8/8/k1K3Q1/8/8 w] 
&
\chessboard[setfen=7k/8/5K2/8/8/8/2Q5/8 w] \\
\textbf{7.5 мат в 2 хода} & \textbf{7.6 мат в 2 хода} \\
\end{tabular}
\end{center} 

