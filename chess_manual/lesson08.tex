\chapter{Матование одинокого короля: мат ладьёй и королём.}

\section{Методические указания}

Смотрите методические указания к теме ``Матование одинокого короля: линейный мат; мат ферзём и королём''.

\section{Содержание занятия}

Алгоритм матования одинокого короля ладьёй и королём сложнее, чем мат ферзём. Для оттеснения короля соперника к краю доски одной ладьи недостаточно; необходима помощь короля. Сначала покажите детям позицию, к которой они должны стремиться (диаграмма 29).
 
\begin{center} 
\begin{tabular}{ c }
\chessboard[setfen=R7/8/8/k1K5/8/6k1/8/4r2K,pgfstyle=border,markregions={a1-d8,e1-h8},showmover=false] \\
\textbf{Д. 29 примеры матов ладьёй} \\
\end{tabular}
\end{center} 

Далее, на примере, объясните принцип матования. На диаграмме 30 король слабейшей стороны находится в центре. 
 
\begin{center} 
\begin{tabular}{ c }
\chessboard[setfen=8/8/8/4k3/8/7K/R7/8 w] \\
\textbf{Д. 30 ход белых} \\
\end{tabular}
\end{center} 

Процесс матования ладьей и королём можно разбить на несколько шагов:

\begin{enumerate}
\item 	Выбираем крайнюю линию и отрезаем короля. В данном случае, ``отрезаем'' короля по 4 горизонтали: 

\newchessgame[setfen=8/8/8/4k3/8/7K/R7/8, moveid=1w]
\mainline{1. Ra4}

и будем оттеснять его на 8 горизонталь. Идентичный алгоритм используется и для матования по вертикалям. Король слабейшей стороны чаще всего стремится напасть на ладью

\mainline{1... Kd5}

\item Очень важно взаимодействие короля и ладьи. Подходим нашим королём по 3 горизонтали. Короли должны встать один напротив другого (оппозиция), причём это необходимо сделать, оставляя  за собой право хода. 

\mainline{2. Kg3 Kc5 3. Kf3 Kb5}

\item Частой ошибкой начинающих является ``увлечение погоней'' и дети забывают о том, что ладью можно и потерять. Отходить ладьёй необходимо по той же горизонтали (вертикали).

\mainline{4. Rh4}

\item Погоня за королём:

\mainline{4... Ka5 5. Ke3 Kb5 6. Kd3 Kc5}

 – получаем позицию, в которой возможно занять оппозицию (стать напротив), но теряя при этом ход. В этой ситуации стоит сделать ``выжидающий ход'' ладьёй
  
\mainline{7. Rg4}

\item Как только короли заняли оппозицию:

\mainline{7... Kd5}

следует шах по горизонтали (вертикали).

\mainline{8. Rg5+}

Таким образом, белый король ``не пускает чёрного короля'' на пятую горизонталь, а ладья оттесняет с четвертой. 
 
\begin{center} 
\begin{tabular}{ c }
\chessboard \\
\textbf{Д. 31 после 8-го хода белых} \\
\end{tabular}
\end{center} 

\item Следует 

\mainline{8... Ke6}

Подходим белым королём уже по 4-ой горизонтали ``на ход коня'' – 

\mainline{9. Kd4 Kf6 10. Ra5} 

и повторяем вышеописанное до мата на последней горизонтали. 

\mainline{10... Kg6 11. Kf4 Kh6 12. Kg4 Kg6 13. Ra6+ Kf7 14. Kg5 Ke7 15. Kf5 Ke8 16. Ra7 Kf8 17. Ke6 Kg8 18. Kf6 Kh8 19. Kg6 Kg8 20. Ra8#}
\end{enumerate}

\section{Домашнее задание}

\begin{center} 
\begin{tabular}{ c c }
\chessboard[setfen=8/8/8/3k2R1/8/3K4/8/8 w] &
\chessboard[setfen=4K2k/4R3/8/8/8/8/8/8 w] \\
\textbf{8.1 Мат в 2 хода} & \textbf{8.2 Мат в 3 хода} \\
\chessboard[setfen=8/8/8/5k2/8/3K4/8/1R6 w] &
\chessboard[setfen=8/5R2/8/2K5/8/8/6k1/8 w] \\
\textbf{8.3 Поставьте мат} & \textbf{8.4 Поставьте мат} \\
\end{tabular}
\end{center} 
