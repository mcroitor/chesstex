\chapter{Ход пешки}
\section{Методические указания}

Повторите ход пешкой и взятие.

\section{Содержание занятия}

Пешка, атакующая поле, которое пересекла пешка соперника, продвинутая с начальной позиции сразу на два поля, может взять эту продвинутую пешку, как если бы последний ее ход был только на одно поле. Это действие может быть совершено только очередным ходом и называется взятием ``на проходе''. На диаграмме 15 показано взятие на проходе.

\begin{center}
\begin{tabular}{ c c }
\chessboard[setfen=4k3/8/8/1pP5/8/8/8/4K3,
pgfstyle=straightmove,
color=gray,markmoves={b7-b5},
color=red,markmoves={c5-b6},
showmover=false] 
&
\chessboard[setfen=4k3/P7/4K3/8/8/8/8/8,showmover=false] \\
\textbf{Д. 15 взятие на проходе} & \textbf{Д. 16 превращение пешки в ферзя} \\
\end{tabular}
\end{center}
 
Пешка, которая достигает самой последней горизонтали от своей начальной позиции, должна быть заменена на том же поле на ферзя, ладью, слона или коня того же цвета, что является частью одного хода. Эта замена пешки называется превращением, действие новой фигуры начинается сразу с момента ее появления на шахматной доске. Выбор игроком фигуры для превращения не ограничивается фигурами, которые были взяты ранее до превращения. На диаграммах показаны различные ситуации, в которых пешка может превращаться в разные фигуры. Например, на  диаграмме 16, пешка, походив с поля a7 на поле a8, превращается в ферзя и ставит мат черному королю.

На диаграмме 17 показан пример, когда превращение в сильнейшую фигуру не выигрывает партию, а приводит лишь к ничьей: после превращения пешки c7 в ферзя у черного короля нет ходов. На доске пат, то есть ничья. Победы можно достигнуть, превратив пешку в ладью: \textbf{1. c8=\rook{} \king{}a6 2. \rook{}a8\mate{}}
Диаграмма 18 демонстрирует случай, когда полезно превратить пешку в коня. В этом случае белые следующим ходом забирают ферзя черных и выигрывают: \textbf{1. d8=\knight{}+} и 2. \textbf{\knight{}xf7}

\begin{center}
\begin{tabular}{ c c }
\chessboard[setfen=8/k1P5/2K5/8/8/8/8/8,showmover=false]
&
\chessboard[setfen=8/3P1q2/2k5/8/8/4B3/4K3/8,showmover=false] \\
\textbf{Д. 17 превращение пешки} & \textbf{Д. 18 превращение пешки} \\
\end{tabular}
\end{center}
 
\section{Домашнее задание}

\begin{enumerate}
\item Укажите возможные ходы пешками после указанного под диаграммой хода.


\begin{center}
 \begin{tabular}{ c c }
\chessboard[
setfen=4k3/2p5/1P6/3PK3/8/8/8/8,
showmover=false]
&
\chessboard[
setfen=4k3/8/1P6/2pPK3/8/8/8/8,
showmover=false] \\
\textbf{3.1} 0... \king{}f8-e8 & \textbf{3.2} 0... c7-c5 \\
\chessboard[
setfen=2r1k3/1P6/8/2pPK3/8/8/8/8,
showmover=false] 
& 
\chessboard[
setfen=1r2kb2/1P6/8/2pP4/8/K7/8/8,
showmover=false] \\
\textbf{3.3} 0... \rook{}c7-c8 & \textbf{3.4} 0... \king{}f7-e8 \\
\end{tabular}
\end{center}

\end{enumerate}
