В современном мире интеллект является одним из самых значимых социальных качеств личности, обилие информации предполагает необходимость индивидуума быстро воспринимать огромный поток информации, анализировать его и выделять значимые для себя элементы. Формирование таких качеств, как концентрация, умение предвидеть последствия действий и событий, логическое мышление, креативность, умение быстро сориентироваться в ситуации и принять решение, воля, стремление к самосовершенствованию, память, коммуникативные навыки являются необходимостью в современном мире. Шахматы – один из древнейших и действенных способов развития целого спектра качеств и навыков, необходимых в жизни.
Интеллектуальное развитие детей происходит главным образом в дошкольных и школьных образовательных учреждениях. Не случайно в большинстве культур систематическое обучение детей начинается в возрасте 5 -7 лет. Этот возраст, согласно Ж.Пиаже, знаменует собой переход от дооперационального мышления к мышлению на уровне конкретных операций. Дети приобретают эту более сложную и тонкую форму мышления в процессе активного исследования физической среды, задавая себе вопросы и, в основном, самостоятельно находя на них ответы.

Задаваясь фундаментальным вопросом, нужны ли нам шахматы, необходимо окунуться во множество психопедагогических и социальных исследований, изучить мнение общества и даже культуру разных народов. Мы же остановимся только на общих моментах: шахматы являются одной из самых древних игр, которая из "игры для знати" стала доступна всем желающим. Во все времена шахматы высоко ценились знаменитыми личностями разных профессий: Маркс и Энгельс, Калинин, Ворошилов, Фрунзе, Робеспьер, Суворов, Пушкин, Толстой, Сервантес, Байрон и Вольтер, Бетховен, Менделеев, Йорга и многие другие любили и практиковали эту игру.

В последние десятилетия было проведено множество исследований, доказывающих положительную воспитательно-образовательную роль шахмат, без каких-либо негативных последствий.

Почему шахматы полезны в начальной школе?
\begin{enumerate}
\item в шахматы играет более 500 млн. человек в 167 странах – это второе место по распространенности после футбола.
\item это универсальная игра – научиться играть можно независимо от пола, возраста, расовой или социальной принадлежности.
\item это интеллектуальная игра, посредством которой развиваются множество интеллектуальных и социальных качеств.
\item эта игра не требует больших затрат.
\end{enumerate}

Исследования в области влияния обучения шахматам на развитие детей дошкольного и младшего школьного возраста установили:
\begin{enumerate}
\item В 1974/76 годах профессор психологии Dr.Christiaen проводил исследования в школах Бельгии. Оказалось, что дети, занимающиеся шахматами, добились значительно лучших показателей на государственных экзаменах по всем предметам.
\item В 1977/79 годах профессор Фанг проводил подобный эксперимент в школах Гонконга. Результат показал, что по таким предметам, как математика, физика и химия, шахматисты в среднем получили оценки на 15% выше, чем остальные ученики.
\item В 1979/83 годах министерство образования Венесуэлы ввело обучение шахматам для 4500 второклассников. Исследователи, изучив результаты эксперимента, сделали вывод, что методическое обучение игре в шахматы ускоряет интеллектуальное развитие детей. Правительство Венесуэлы постановило внедрить уроки шахмат во всех школах.
\item В 1990/91 годах профессор Фридман изучал влияние шахмат на интеллект ребёнка. Исследования проводились в Бельгии. Они показали, что занятия шахматами способствовали увеличению интеллекта на 21\% по шкале Wechsler.
\item В 1979/83 годах профессор Фергюсон изучал влияние шахмат на умственные способности детей. В исследованиях участвовали одарённые ученики школ Пенсильвании. Было установлено, что те школьники, которые начали изучать шахматы, увеличили свои умственные способности за год на 17.3\%, в то время как остальные на 4.6\% по шкале Watson-Glaser.
\item В 1990/92 и в 1995/96 годах известный профессор психологии д-р Маргулис проводил исследования в школах Нью-Йорка и Лос-Анджелеса. Результаты исследований показали, что школьники, изучающие шахматы, на государственных экзаменах по чтению получили значительно более высокие оценки, чем другие школьники в городе и в среднем по стране.
\item В 1994/97 годах, Джеймс Липтрап проводил исследования в школах Техаса. Ученики, занимающиеся шахматами, вдвое улучшили свои показатели по чтению и математике по сравнению с остальными школьниками.
\item В 1997 году профессора Маргулис и Спит изучали влияние шахмат на интеллектуальные способности школьников Нью-Йорка. Школьники, изучающие шахматы, получили 91.4\%, остальные школьники 64.4\% в соответствии с Intelligence scale.
\end{enumerate}

В настоящее время более 30 стран (такие, как Армения, Испания, Италия, Турция, Азербайджан и др.) включили шахматы в школьные программы. С каждым годом список этих стран увеличивается.
