\chapter{Матование одинокого короля: мат двумя слонами.}

\section{Методические указания}

Напомните детям что такое диагональ и виды диагоналей.

Смотрите методические указания к теме ``Матование одинокого короля: линейный мат; мат ферзём и королём''.

\section{Содержание занятия}

Для того, чтобы поставить мат двумя слонами необходимо оттеснить короля в угол доски. Покажите детям пример мата двумя слонами (Д. 32)
 
\begin{center}
\begin{tabular}{ c }
\chessboard[setfen=k7/2K5/8/8/3BB3/8/8/8 b] \\
Д. 32 мат двумя слонами \\
\end{tabular}
\end{center}

Решите позицию на диаграмме 33. 
 
\begin{center}
\begin{tabular}{ c c }
\chessboard[setfen=8/7B/8/8/8/BK6/8/k7 w] & \chessboard[setfen=8/8/8/7B/3k4/8/5K1B/8 w] \\
Д. 33 Белые начинают и ставят мат в 1 ход &
Д. 34 Матование 2 слонами \\
\end{tabular}
\end{center} 
 
Для матования двумя слонами необходимо взаимодействие всех трёх фигур: слонов и короля. Два слона хорошо дополняют друг друга, так как один играет по белым полям, а другой – по черным. Принцип построения мата в данном случае достаточно прост (см. Д. 34):

\begin{enumerate}

\item Два слона располагаются на соседних диагоналях, отрезая короля соперника от части доски. 

\newchessgame[setfen=8/8/8/7B/3k4/8/5K1B/8, moveid=1w]
\mainline{1. Bf3 Kd3 2. Bd5 Kc4}

Черному королю невыгодно отходить на вторую линию, так как в этом случае он быстрее оттесняется к краю доски (2... \king{}d2 3. \bishop{}e4; 2... \king{}c2 3.\bishop{}d5)

\item Королём оттесняется король соперника в угол:

\mainline{3. Ke3 Kc5 4. Kd3 Kb4}

\item Как только черный король отходит от линий воздействия слонов (Д. 35) – отсекаем новые линии слонами.
\end{enumerate}

Главное, не увлечься и не поставить пат, например, как на диаграмме 36. Шаги 1-3 повторяются до тех пор, пока король соперника не будет оттеснён к краю доски.

\mainline{5. Bd4 Kb5 6. Kc3 Ka5 7. Be2 Ka4} (см. Д. 37)
  
\begin{center}
\begin{tabular}{ c c }
\chessboard[setfen=8/8/8/4B3/1k6/3K1B/8/8 w] & \chessboard[setfen=8/8/8/8/k7/2B5/1K2B3/8 b] \\
Д. 35 отсекание черного короля слонами &
Д. 36 пат двумя слонами \\
\end{tabular}
\end{center} 
 
Теперь необходимо последовательно отогнать черного короля в угол, отрезая по клеточке слонами. Только не надо забывать, что король соперника может обойти вашего короля.
 
\begin{center}
\begin{tabular}{ c }
\chessboard[setfen=8/8/8/8/k2B4/2K5/4B3/8 w] \\
Д. 37 загоняем короля в угол \\
\end{tabular}
\end{center} 

\mainline{8. Bb6 Ka3 9. Bb5 Ka2 10. Kc2 Ka3 11. Bc5+ Ka2 12. Bc4+ Ka1 13. Bd4#}

\section{Домашнее задание}

\begin{center}
\begin{tabular}{ c c }
\chessboard[setfen=k7/8/BK6/2B5/8/8/8/8 w] & \chessboard[setfen=8/5K1k/8/8/5B2/8/4B3/8 w] \\
Д. 38 Мат в 2 хода & Д. 39 Мат в 2 хода \\
\end{tabular}
\end{center}
