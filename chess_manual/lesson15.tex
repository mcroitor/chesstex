\chapter{Оценка позиции. План игры.}

\section{Методические указания}

Смотрите методические указания к теме «Основы миттельшпиля». Попросите детей записать элементы оценки позиции. Напомните детям шкалу ценностей шахматных фигур. При оценивании ответов на домашнее задание обратите внимание, что в позиции могут быть определены несколько планов. Оценивайте логичность планов, а не конкретные варианты.

Начиная с этого занятия желательно давать ученикам на дом проанализировать сыгранные ими партии – это стимулирует рост юного шахматиста.

\section{Содержание занятия}

Для того, чтобы выработать стратегию игры (придумать план игры), необходимо правильно оценивать позицию. Оценка позиции состоит из нескольких этапов:

1.	Непосредственные угрозы. Если на доске стоит мат в несколько ходов, то все остальные оценки не имеют значения.

2.	Материальная оценка. Включает в себя прямой подсчет сил белых и черных фигур по шкале ценности фигур. Различные учебники приводят различные шкалы, однако в общности они совпадают и соответствуют следующим утверждениям: 

a.	пешка < конь < слон < ладья < ферзь
b.	конь чуть слабее слона
c.	две легкие фигуры примерно равны по силе ладье с пешкой
d.	две ладьи чуть сильнее ферзя
e.	ладья, легкая фигура и пешка примерно равны ферзю

Примерные шкалы приведены в таблице ниже.

Фигура	Шкала 1	Шкала 2
Конь 	3	2.5
Слон	3	3
Ладья	5	5
Ферзь	9	10

3.	Развитие фигур. Сторона, у которой фигуры развиты и расположены активней – имеет преимущество. Под активностью фигур понимают следующие факторы:
a.	Фигуры, воздействующие на центр доски сильнее
b.	Ладьи на открытых линиях сильнее, так как у них больше пространства для манёвров.
c.	Связанные ладьи сильнее, так как они поддерживают друг друга
d.	Слоны также лучше играют по открытым линиям
e.	Конь на краю доски очень плохо стоит
f.	Фигуры, направленные на вражеского короля стоят лучше, так как позволяют атаковать короля.

4.	Пешечная структура. Пешки играют важную роль в защите и атаке. Поэтому их взаимодействие очень важно. Так, сдвоенные пешки – это недостаток. Две связанные пешки сильнее раздвоенных. Вообще, чем меньше пешечных островков, тем лучше. Изолированные пешки часто становятся объектом атаки в силу их малой подвижности и отсутствия надежной защиты.

Таковы простейшие принципы оценки позиции. На основе оценки позиции можно строить план дальнейшей игры. Обычно, выявив преимущества и недостатки расположения своих фигур и фигур противника, в результате дальнейшей игры шахматист пытается избавиться от своих слабостей, создать слабости в лагере соперника или уже использовать существующие слабости позиции. При составлении плана игры шахматист рассматривает ход или варианты, которые позволяют достичь намеченной цели. При расчете вариантов желательно также рассматривать оценку возникающих позиций.

Рассмотрим применение описанных правил на примерах.

Боголюбов - Тарраш
Karlsbad, 1923
 
Д. 62 Ход белых

Непосредственных угроз нет как со стороны черных, так и со стороны белых. Наблюдается также материальное равенство. Однако, расположение фигур у обеих сторон существенно разнится: в то время, как у белых все фигуры развиты и центр в их распоряжении, черные ютятся на краю доски. Незавидны расположения черного ферзя и коня на g8. Следует отметить, что оба белых коня уже готовы атаковать бастионы черного короля, равно как белые ферзь и слон. Поэтому появляется идея вскрывать королевский фланг. Самым логичным кажется вскрытие пешкой по линии h – это позволяет подключить к атаке и ладью. Появляется план – продвижение пешки h3-h4-h5 c последующим разменом. 
Что и произошло в партии:

16.h4 Rad8 17.h5 dxe5 18.dxe5 b5 19.cxb5 Qxb5 20.hxg6 fxg6 21.Nh4 и в конце концов белые выиграли.

Теперь попробуйте самостоятельно оценить следующую позицию и составить план игры.

Ананд – Карпов, 
Лозанна 1998
 
Д. 63 Ход белых

Попробуйте ответить на следующие вопросы:

1.	Есть ли у белых прямые угрозы? А у черных?
2.	Подсчитайте баланс фигур. На чьей стороне преимущество?
3.	Определите безопасность королей
4.	Оцените развитие и положение фигур. Какие преимущества и недостатки есть у белых? А у черных?
5.	Чьи пешки расположены лучше?
6.	Создайте план игры.

Ответы на поставленные вопросы будут следующими: 

1.	Прямых угроз нет ни у белых, ни у черных. 
2.	На доске стоит материальное равенство.
3.	Оба короля рокировались и укрыты пешками.
4.	Фигуры в большинстве своём развиты. Белый слон на с1 активней черного слона на с8, так как может быть легко введен в игру, тогда как черный слон стеснён своими же пешками. Черный конь на d5 занимает очень активную позицию и не может быть согнан с неё.  Белая ладья е1 заняла активную позицию. На доске ведется активная борьба за центр.
5.	У белых есть слабость в виде пешки изолированной пешки d4, так как защищать её белые должны фигурами. Черные пешки расположены лучше.
6.	Из указанных преимуществ и недостатков и складывается дальнейшая игра. Первым делом надо рассмотреть вариант с разменом активного коня черных. В результате данного размена (1.K:d5 K:d5 2.C:d5 Ф:d5) у черных получается больше преимуществ и возможностей: преимущество двух слонов и возможность атаки изолированной пешки d4. Поэтому, перед разменом можно усилить атаку на коня ходом 1.Фb3 c угрозой 2. K:d5 K:d5 3.C:d5 Ф:d5 3.Ф:d5 ed, что приводит к позиции, более благоприятной для белых. Можно попробовать усилить позицию коня ходом 1.Ке5. Этот ход препятствует также развитию слона черных.

Теперь, некоторые рекомендации по составлению плана игры. Как уже было отмечено, необходимо пользоваться преимуществами своей позиции, улучшать свою позицию, пользоваться недостатками позиции фигур соперника либо ухудшать их позицию. Например, если у вас материальное преимущество, то в такой ситуации выгодно разменивать равноценные фигуры и переходить в эндшпиль с лишним материалом. Если есть угроза, то ставьте мат или выигрывайте материал. Если угрожают вам: защищайтесь, не дайте сопернику так легко выиграть. Если король соперника в центре или открыт, то ищите возможность атаковать его и поставить мат. Переведите свои фигуры поближе к нему. Если только ваши фигуры в центре: постарайтесь оттеснить фигуры противника ещё дальше. Обязательно ищите хорошие поля для своих фигур. Играйте активно, но не забывайте, что ваш соперник тоже будет создавать угрозы!

\section{Домашнее задание}

Оцените позицию. Составьте план игры. 
 
15.1 ход черных
 
15.2 ход белых
 
 
15.3 ход белых
 
15.4 ход черных
 
