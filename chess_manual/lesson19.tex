\chapter{Тактические приёмы: отвлечение и завлечение}

\section{Методические указания}

Перед началом занятия рекомендуется повторить тактический приём «уничтожение защиты», а также различные тактические удары и понятия форсированного варианта, жертвы и комбинации.

\section{Содержание занятия}

Одним из часто встречающихся на практике тактических приемов является отвлечение (фигуры). Оно применяется тогда, когда требуется отвлечь какую-либо фигуру или пешку противника от защиты важного поля, линии или другой фигуры. В результате отвлечения фигуры (пешки) мы можем воспользоваться получившимся ослаблением. Данный приём похож на приём «уничтожение защиты» по идее, однако при его осуществлении защищающая фигура соперника не уничтожается.
 
Д. 75 ход белых. отвлечение

В качестве примера рассмотрим позицию на диаграмме 75. При внимательном изучении становится заметно, что если бы черного ферзя не было на последней горизонтали, то белые ставили бы мат 1.Лd8x? Однако забрать ферзя у нас нет возможности. Черный ферзь выполняет ещё одну функцию – защищает ладью. Это видно по варианту 1.Лd8+ Ф:d8 2.Ф:a7. Это наводит на мысль, что ферзь не может справляться с двумя обязанностями сразу – он перегружен. Поэтому первым ходом можно пожертвовать ферзя за ладью:

1.Ф:a7!

Теперь, если черные откажутся брать ферзя, то останутся без ладьи. А если примут жертву, то у последней горизонтали не останется защитников (отвлечение!):

1… Ф:a7 Лd8x

В следующем примере (диаграмма 76) от мата на g7 черных опять защищает ферзь. 
 
 
Д. 76 ход белых
 
Д. 77 ход белых
 
Отвлечь его можно прямым нападением, с угрозой забрать:

1.Ле8! Ф:е8 2.Ф:g7x

Попытка игнорировать угрозу завершается потерей ферзя – 1… Лg6 2.Л:f8+ Kp:f8 Ф:c7 с преимуществом.

Комбинации на отвлечение можно проходить не только ради мата. Также это действенное оружие при выигрыше материала или при проведении пешки (диаграмма 77).

При равенстве материалов у белых определенное преимущество. Оно заключается в том, что черная пешка остановлена белым королём, а белую пешку можно тормозить только черным слоном. Следует заметить, что после размен пешки на слона лишает все надежды на победу (одним слоном мат не поставить). После естественного вступления:

1.g6 Cf8

становится видно, что для победы белым необходимо отвлечь черного слона от защиты поля g7. Это достигается двойным ударом:

2.Сf7+! C:f7 3.g7 ~ g8Ф

И белые побеждают.

Если в тактическом приёме на отвлечение атакующая сторона заставляла защищающуюся сторону убирать защиту с некоторого поля, то c помощью тактического приема завлечение можно заставить фигуру противника занять невыгодную для нее позицию. На этой позиции завлеченная фигура подвергается атаке и гибнет, либо существенно мешает координации собственных сил, из-за чего гибнут другие фигуры. Особенно эффективен данный прием, когда объектом его применения становится король. Обычно завлечение используется для создания двойного удара или для выигрыша времени.
 
 
Д. 78 ход белых. Завлечение
 
Д. 79 ход белых
 
В позиции на диаграмме 78 черные грозят матом. Победу белым могут принести только активные действия.

1.Лh8+!

Жертва ферзя ради выигрыша времени. Черный король завлекается под шах, в результате чего он получает мат:

1… Кр:h8 2.Фh1+ Kpg8 3.Фh7x

В следующем примере (Д. 79) черный ферзь защищает от линейного мата. Но защита эта приводит к тому, что после завлечения ферзь выигрывается двойным ударом:

1.Лh8+! Ф:h8 2.Лh1+

Как правило, завлечение проявляется в сочетании с другими тактическими средствами, а также может применяться несколько раз. Как, например, в позиции на диаграмме 80.
 
Д. 80 ход белых.

В этой позиции комбинация завлечения проводится 3 раза.

1.Ла5!

Первая жертва с целью завлечения черного ферзя под вилку в варианте 1… Ф:а5 2.Кс6+. Черные отказываются принять жертву, на что следует второе завлечение:

1… с5 2.Л:с5! Ф:с5

От второй жертвы уже нельзя отказаться. Теперь видно, что король и ферзь завлечены под двойной удар пешкой. Это будет третья жертва на завлечение, в результате которой белые ставят вилку на короля и ферзя и выигрывают (из-за оставшейся проходной пешки):

3.d4+! Kp:d4 4.Ke6+ или 3… Ф:d4 4.Kc6+ c победой.

\section{Домашнее задание}

Проведите комбинацию на отвлечение
 
 
19.1 ход белых
 
19.2 ход белых
 
Проведите комбинацию на завлечение
 
 
19.3 ход белых
 
19.4 ход белых
 
