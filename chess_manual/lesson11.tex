\chapter{Тактические приёмы. Понятие комбинации.}

\section{Методические указания}

Для развития комбинационного зрения учащихся следует научить их рассчитывать форсированные варианты без передвижения фигур.

\section{Содержание занятия}

Шахматная комбинация является тем самым элементом, из-за которого шахматную игру называют искусством. Необычность решений в шахматной партии, красота жертв ради достижения результата – это именно то, за что все любят шахматы. Именно комбинация делает шахматные партии красивыми и эффектными.

В рамках данного занятия ученик знакомится с понятием комбинации. Рассматриваются составляющие комбинации, такие как форсированный вариант, жертва, тактический приём.

\begin{itemize}
\item Размен –- снятие с доски равноценных фигур. Термин ``размен'' имеет и более глубокий смысл: операция для достижения определенных целей. Размен предпринимается с целью: выиграть темп; не терять времени на отступление; уничтожить силы противника, защищающие важные пункты; отдать свою ``плохую'' фигуру за ``хорошую'' противника. С помощью размена можно ослабить пешечное расположение противника, создать в его лагере слабости. Размен предпринимается и с целью захватить контроль над открытой линией или перевести игру в выгодный эндшпиль. Последнее часто является способом реализации материального преимущества. Размен иногда является и одним из средств защиты —- обороняющаяся сторона путем разменов стремится ослабить атаку противника.
 
\begin{center} 
\begin{tabular}{ c }
\chessboard[setfen=r2q1rk1/ppp2ppp/1bnpbn2/3Np1B1/2B1P3/3P3P/PPP2PP1/RN1Q1RK1 w] \\
\textbf{Д. 43 Выгодный размен, ход белых}
\end{tabular}
\end{center}

На диаграмме 43 могут совершить выгодный размен: \textbf{1.Nxb6 ab 2.Bxe6 fe} , в результате которого у черных портится пешечная структура. Однако еще лучше будет обменять коней – \textbf{1.Nxf6+ gf 2.Bh6}. В результате этого размена у черных ослабилась защита короля, что позволяет белым его атаковать. Например, следующим образом: \textbf{2... Re8 3.Nh4 Bxc4 ? 4.Qg4+ Kh8 5.Qg7\#}

\item Форсированный вариант -– вынужденная последовательность ходов. Вынужденность хода определяется одноходовыми угрозами, такими как угроза мата, шах, потеря материала. Представляет собой последовательность разменов, нападений, шахов и защит от них.
\item Жертва -– отдача материала с целью получения какой-то выгоды, будь то мат, отыгрыш (выигрыш) материала, получение позиционного преимущества.
\item Комбинация -– форсированный вариант, сопровождающийся жертвой и заканчивающийся к выгоде активной (атакующей или контратакующей) стороны.
\end{itemize}
 
\begin{center} 
\begin{tabular}{ c }
\chessboard[setfen=r4r1k/6p1/4p1P1/8/6p1/qPp5/2P5/1K1QR1R1 w] \\
\textbf{Д. 44 мат в 6 ходов}
\end{tabular}
\end{center}

В позиции на диаграмме 47 белые форсировано (последовательно объявляя шахи) ставят мат в 6 ходов:

\textbf{1.Rh1+ Kg8 2.Rh8+! Kxh8 3.Rh1+ Kg8 4.Rh8+! Kxh8}

После того как белые расчистили форсировано первую горизонталь, в игру вступает ферзь:

\textbf{5.Qh1+ Kg8 6.Qh7\#}

Как видно, на втором и четвертом ходах белые отдали ладью с целью выигрыша времени. Это и есть жертва. Сам вариант представляет собой комбинацию: в нем белые пожертвовали материал, черные были вынуждены делать свои ходы и в результате белые поставили мат.

Конечно, форсированный вариант бывает без жертвы, как, например, в позиции на диаграмме 48.
 
\begin{center} 
\begin{tabular}{ c }
\chessboard[setfen=r4r1k/2q2p2/ppn3p1/2p5/8/5N1P/PPRQ1PP1/3R2K1 w] \\
\textbf{Д. 45 Ход белых. Белые выигрывают}
\end{tabular}
\end{center}

Белые в несколько ходов выигрывают качество и пешку, вынуждая черных делать единственные ходы:

\textbf{1.Qh6+ Kg8 2.Ng5}

Грозит мат черным \emph{3.Qh7\#} . У черных единственная защита

\textbf{2... f6 3.Qxg6+ Qg7 4.Qxg7+ Kxg7 5.Ne6+}

И у белых выигранная позиция. Черные могли попробовать отказаться от варианта на третьем ходу, однако качество они всё равно теряют –- \textbf{3... Kh8 4.Qh6+ Kg8 5.Ne6}.
 
\begin{center} 
\begin{tabular}{ c }
\chessboard[setfen=4k3/5pp1/p3qb2/1p6/8/2N2QP1/PP3P2/6K1 b] \\
\textbf{Д. 46 ход черных. промежуточный ход}
\end{tabular}
\end{center}

Следует отметить, что вынуждение ответа соперника –- понятие относительное. Только непосредственная угроза королю (то есть шах) вынуждает обязательную его защиту. Все остальные угрозы могут быть временно проигнорированы созданием более серьёзной угрозы. Например, после хода 1... С:c3 (диаграмма 49) белые могут не сразу забирать слона, а сначала объявить шах –- 2.Фа8+ и забрать слона на следующем ходу. Такой ход, создающий в ответ на угрозу более серьёзную угрозу, называется промежуточным ходом.

Бывают жертвы без форсированных вариантов. В этом случае говорят о ловушках. Под ловушкой понимают провоцирование соперника на продолжение, кажущееся выгодным, которое в действительности является ошибочным. Пример ловушки виден на диаграмме 47.
 
\begin{center} 
\begin{tabular}{ c }
\chessboard[setfen=rnbqkb1r/pp2pppp/3p1n2/2p5/4P3/5N1P/PPPP1PP1/RNBQKB1R w] \\
\textbf{Д. 47 ход белых}
\end{tabular}
\end{center}

В этой позиции белые играют \textbf{1.с3}, жертвуя пешку е4. В случае, если черные примут жертву, то они потеряют фигуру: \textbf{1... Nxе4 2.Qа4+} и \textbf{3.Qxе4}. Однако черные не обязаны брать пешку, они могут спокойно завершать развитие своих фигур.

\section{Домашнее задание}

Рассчитать и записать форсированный вариант.

\begin{center} 
\begin{tabular}{ c c }
\chessboard[setfen=rnbqk2r/pp2bppp/4pn2/2pp4/2PP4/P1N1P3/1P3PPP/R1BQKBNR w] 
&
\chessboard[setfen=r2r2k1/ppp1qp2/2n1p1p1/8/3P4/P4N1P/1P3PP1/R2QK2R b] \\
\textbf{11.1 ход белых} & \textbf{11.2 ход черных} \\
\chessboard[setfen=r2r3k/ppp1qp2/2n1p1p1/8/3P4/P4N1P/1P1Q1PP1/R3K2R w] 
&
\chessboard[setfen=4r3/3b1p2/5k2/8/4B3/8/6KP/4R3 b] \\
\textbf{11.3 ход белых} & \textbf{11.4 ход черных} \\
\end{tabular}
\end{center} 
