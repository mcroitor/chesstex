\chapter{Тактические приёмы. Понятие комбинации.}

\section{Методические указания}

Для развития комбинационного зрения учащихся следует научить их рассчитывать форсированные варианты без передвижения фигур.

\section{Содержание занятия}

Шахматная комбинация является тем самым элементом, из-за которого шахматную игру называют искусством. Необычность решений в шахматной партии, красота жертв ради достижения результата – это именно то, за что все любят шахматы. Именно комбинация делает шахматные партии красивыми и эффектными.

В рамках данного занятия ученик знакомится с понятием комбинации. Рассматриваются составляющие комбинации, такие как форсированный вариант, жертва, тактический приём.

\begin{itemize}
\item Размен –- снятие с доски равноценных фигур. Термин ``размен'' имеет и более глубокий смысл: операция для достижения определенных целей. Размен предпринимается с целью: выиграть темп; не терять времени на отступление; уничтожить силы противника, защищающие важные пункты; отдать свою ``плохую'' фигуру за ``хорошую'' противника. С помощью размена можно ослабить пешечное расположение противника, создать в его лагере слабости. Размен предпринимается и с целью захватить контроль над открытой линией или перевести игру в выгодный эндшпиль. Последнее часто является способом реализации материального преимущества. Размен иногда является и одним из средств защиты —- обороняющаяся сторона путем разменов стремится ослабить атаку противника.
 
\begin{center} 
\begin{tabular}{ c }
\chessboard[setfen=8/8/8/8/8/8/8/8 w] \\
\textbf{Д. 43 Выгодный размен, ход белых}
\end{tabular}
\end{center}

На диаграмме 43 могут совершить выгодный размен: 1.K:b6 ab 2.C:e6 fe , в результате которого у черных портиться пешечная структура. Однако еще лучше будет обменять коней – 1.Kf6+ gf 2.Ch6. В результате этого размена у черных ослабилась защита короля, что позволяет белым его атаковать. Например, следующим образом: 2… Лe8 3.Kh4 C:c4? 4.Фg4+ Kph8 5.Фg7x

\item Форсированный вариант -– вынужденная последовательность ходов. Вынужденность хода определяется одноходовыми угрозами, такими как угроза мата, шах, потеря материала. Представляет собой последовательность разменов, нападений, шахов и защит от них.
\item Жертва -– отдача материала с целью получения какой-то выгоды, будь то мат, отыгрыш (выигрыш) материала, получение позиционного преимущества.
\item Комбинация -– форсированный вариант, сопровождающийся жертвой и заканчивающийся к выгоде активной (атакующей или контратакующей) стороны.
\end{itemize}
 
\begin{center} 
\begin{tabular}{ c }
\chessboard[setfen=8/8/8/8/8/8/8/8 w] \\
\textbf{Д. 44 мат в 6 ходов}
\end{tabular}
\end{center}

В позиции на диаграмме 47 белые форсировано (последовательно объявляя шахи) ставят мат в 6 ходов:

1.Лh1+ Kpg8 2.Лh8+! Kp:h8 3.Лh1+ Kpg8 4.Лh8+! Kp:h8

После того как белые расчистили форсировано первую горизонталь, в игру вступает ферзь:

5.Фh1+ Kpg8 6.Фh7x

Как видно, на втором и четвертом ходах белые отдали ладью с целью выигрыша времени. Это и есть жертва. Сам вариант представляет собой комбинацию: в нем белые пожертвовали материал, черные были вынуждены делать свои ходы и в результате белые поставили мат.

Конечно, форсированный вариант бывает без жертвы, как, например, в позиции на диаграмме 48.
 
\begin{center} 
\begin{tabular}{ c }
\chessboard[setfen=8/8/8/8/8/8/8/8 w] \\
\textbf{Д. 45 Ход белых. Белые выигрывают}
\end{tabular}
\end{center}

Белые в несколько ходов выигрывают качество и пешку, вынуждая черных делать единственные ходы:

1.Фh6+ Kpg8 2.Kg5

Грозит мат черным 3.Фh7x . У черных единственная защита

2… f6 3.Ф:g6+ Фg7 4.Ф:g7+ Kp:g7 5.Ke6+

И у белых выигранная позиция. Черные могли попробовать отказаться от варианта на третьем ходу, однако качество они всё равно теряют – 3… Kph8 4.Фh6+ Kpg8 5.Ke6.
 
\begin{center} 
\begin{tabular}{ c }
\chessboard[setfen=8/8/8/8/8/8/8/8 b] \\
\textbf{Д. 46 ход черных. промежуточный ход}
\end{tabular}
\end{center}

Следует отметить, что вынуждение ответа соперника –- понятие относительное. Только непосредственная угроза королю (то есть шах) вынуждает обязательную его защиту. Все остальные угрозы могут быть временно проигнорированы созданием более серьёзной угрозы. Например, после хода 1... С:c3 (диаграмма 49) белые могут не сразу забирать слона, а сначала объявить шах –- 2.Фа8+ и забрать слона на следующем ходу. Такой ход, создающий в ответ на угрозу более серьёзную угрозу, называется промежуточным ходом.

Бывают жертвы без форсированных вариантов. В этом случае говорят о ловушках. Под ловушкой понимают провоцирование соперника на продолжение, кажущееся выгодным, которое в действительности является ошибочным. Пример ловушки виден на диаграмме 47.
 
\begin{center} 
\begin{tabular}{ c }
\chessboard[setfen=8/8/8/8/8/8/8/8 w] \\
\textbf{Д. 47 ход белых}
\end{tabular}
\end{center}

В этой позиции белые играют 1.с3 , жертвуя пешку е4. В случае, если черные примут жертву, то они потеряют фигуру: 1… К:е4 2.Фа4+ и 3.Ф:е4. Однако черные не обязаны брать пешку, они могут спокойно завершать развитие своих фигур.

\section{Домашнее задание}

Рассчитать и записать форсированный вариант.

\begin{center} 
\begin{tabular}{ c c }
\chessboard[setfen=8/8/8/8/8/8/8/8 w] 
&
\chessboard[setfen=8/8/8/8/8/8/8/8 b] \\
\textbf{11.1 ход белых} & \textbf{11.2 ход черных} \\
\chessboard[setfen=8/8/8/8/8/8/8/8 w] 
&
\chessboard[setfen=8/8/8/8/8/8/8/8 b] \\
\textbf{11.3 ход белых} & \textbf{11.4 ход черных} \\
\end{tabular}
\end{center} 

