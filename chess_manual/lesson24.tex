\chapter{Относительная ценность фигур}

\section{Методические указания}

Напомните детям о ценности фигур, измеряемой в пешках или пунктах. Однако необходимо обратить внимание детей на то, что иногда это общее правило ``не работает''.  Так пешка на предпоследней горизонтали, в случае, когда её нельзя остановить, может резко поменять соотношение сил на доске. Более того, иногда выгодней превратить пешку в коня или слона, а не ферзя. Также важно, чтобы дети не переоценивали силу коня, который им нравится обычно больше слона. Напомните им, как легко можно поставить мат двумя слонами, а вот двумя конями его поставить нельзя.

\section{Содержание занятия}

Начнём мы с понятия ``нетождественный размен''. Это размен неравноценных фигур. Спросите детей, если мы поменяем коня на слона, то это будет равный размен или белые/чёрные будут иметь преимущество? А если ладью на слона? А ферзя на две ладьи? Или коня и пешку на ладью? Попросите детей посчитать пункты и аргументировать свой ответ.

Расставьте несколько позиций с различными сочетаниями фигур и попросите детей оценить, у кого есть материальное преимущество.

Например, позиции на диаграммах 86, 87.
 
\begin{center}
\begin{tabular}{c c}
\chessboard[setfen=2kr4/ppp4p/3r2n1/8/2P2n2/1Q5P/PP3PP1/5RK1 w] &
\chessboard[setfen=r5k1/pbp2pbp/1p3np1/3p4/3P4/1PP2N2/P1B2PPP/R3K2R w] \\
\textbf{Д. 86 ход белых} & \textbf{Д. 87 ход черных} \\
\end{tabular}
\end{center}
 
Расставьте следующую позицию:
 
\begin{center}
\begin{tabular}{c}
\chessboard[setfen=8/8/1PP3k1/8/8/4R3/7K/8] \\
\textbf{Д. 88 ход белых} \\
\end{tabular}
\end{center}

\begin{itemize}
\item Попросите оценить позицию. У чёрных явное материальное преимущество.
\item Спросите детей, могут ли чёрные остановить пешки от превращения? Нет.
\item Как изменится соотношение сил после превращения 1ой из пешек в ферзи?
\end{itemize}

Рассадите детей парами, путём жеребьёвки определите цвет и попросите их разыграть следующую позицию:
 
\begin{center}
\begin{tabular}{ c }
\chessboard[setfen=6KQ/r7/6k1/8/4P3/8/8/8] \\
\textbf{Д. 89 ход белых} \\
\end{tabular}
\end{center}

После разыгрывания, оцените результативность и покажите правильное продолжение.

Чёрные угрожают поставить мат, естественная защита \emph{1.Kf8}, на что логичным является продолжение \emph{1... Rа8+ 2.Kе7 R:h8 3.e5} белые пытаются использовать свой единственный шанс –- провести пешку \emph{3... Kf5}. Нападаем на пешку \emph{4.е6 Rh6} и пешку не спасти. Чёрные выигрывают.

\clearpage

\section{Домашнее задание}

\begin{center}
\begin{tabular}{ c c}

Оцените позицию: & 
Оцените и разыграйте позицию: \\
\chessboard[setfen=8/4k3/3b3Q/8/8/6r1/5K2/8] & 
\chessboard[setfen=8/1n1k4/8/P7/8/8/8/7K] \\
24.1 ход белых &
24.2 ход белых \\
\end{tabular}
\end{center}
