\chapter*{План занятий}

\begin{description}
\item [Урок 1.] История появления и развития шахмат
\item [Урок 2.] Правила шахмат
\item [Урок 3.] Ход пешки
\item [Урок 4.] Рокировка
\item [Урок 5.] Шахматная этика
\item [Урок 6.] Шахматная нотация
\item [Урок 7.] Матование одинокого короля: линейный мат; мат ферзём и королём.
\item [Урок 8.] Матование одинокого короля: мат ладьёй и королём.
\item [Урок 9.] Матование одинокого короля: мат двумя слонами.
Конкурс решения позиций
\item [Урок 10.] Как правильно начинать партию. Типичные ошибки в дебюте.
\item [Урок 11.] Тактические приёмы. Понятие комбинации
\item [Урок 12.] Тактические удары
\item [Урок 13.] Тактические приёмы: уничтожение защиты
Конкурс решения позиций
\item [Урок 14.] Основы миттельшпиля
\item [Урок 15.] Оценка позиции. План игры.
\item [Урок 16.] Шахматная этика.
Тестирование
Турнир
\item [Урок 17.] Эндшпиль: оппозиция
\item [Урок 18.] Эндшпиль: проходная пешка
\item [Урок 19.] Тактические приёмы: отвлечение и завлечение
\item [Урок 20.] Тактические приёмы: перекрытие и освобождение линии
Конкурс решения позиций
\item [Урок 21.] Атака на нерокировавшего короля
\item [Урок 22.] Атака на короля при разносторонних рокировках
\item [Урок 23.] Классификация дебютов. Итальянская партия.
\item [Урок 24.] Относительная ценность фигур
\item [Урок 25.] Реализация материального преимущества
\item [Урок 26.] Ферзь против пешки
\item [Урок 27.] Тактические приёмы: блокирование и освобождение поля
\item [Урок 28.] Контригра
Конкурс решения позиций
\item [Урок 29.] Шахматная композиция
\item [Урок 30.] Чемпионы мира
\item [Урок 31.] Типичные комбинации
\item [Урок 32.] Комбинации с сочетанием идей
Тестирование
Квалификационный турнир
\end{description}