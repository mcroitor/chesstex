\chapter{Шахматная этика}

Теодор Рузвельт сказал, что "воспитать человека интеллектуально, не воспитав нравственно, -- значит вырастить угрозу для общества".

\section{Методические указания}

Уделите отдельное внимание правилам поведения, как во время шахматной партии, так до и после неё. Недостаточно просто рассказать детям об этих правилах, важно на протяжении всего обучения напоминать о них и следить за их выполнением. Шахматная Этика или Кодекс воспитывает в детях взаимоуважение, учит общаться, контролировать свои эмоции.

Необходимо объяснить детям, зачем нужны эти правила и почему следует их соблюдать. После каждого правила попросите детей продемонстрировать это: например, приветствие соперника. 

\section{Содержание занятия}

Мы приведём несколько основных правил поведения во время шахматной партии. Важно применять эти правила даже во время дружеских и тренировочных партий:

\begin{enumerate}
\item \emph{Перед началом партии поприветствуйте соперника рукопожатием.}

Обязательное рукопожатие до партии говорит о взаимном уважении и устанавливает рамки соперничества только за шахматной доской. Девизом ФИДЕ (Всемирной Шахматной Организации) является Gens Una Sumus –- Мы Одна Семья!

\item \emph{Тронул – ходи!}

Правила ФИДЕ гласят: если игрок дотронулся до своей фигуры, то он обязан сделать ей любой возможный ход; если игрок дотронулся до фигуры соперника –- он обязан её побить; если невозможно сделать ход фигурой, до которой игрок дотронулся либо взять фигуру соперника –- игрок делает любой другой ход. Данное правило очень важно для начинающих, они должны понимать, что у каждого действия есть последствие и, если он дотронулся до ферзя и единственный ход –- это потерять его, то обратного пути нет. Применяя его в тренировочных партиях, дети учатся сначала думать, а не "хвататься за фигуры".
 
\item \emph{Скажи "Поправляю!"}

Бывают ситуации, когда фигура не очень чётко стоит на поле. Тогда игрок может её поправить, но и тут есть 2 важных правила:

\begin{itemize}
\item поправлять фигуры можно только во время своего хода: таким образом мы показываем своё уважение к сопернику и не отвлекаем его от обдумывания
\item перед тем, как поправить фигуру, необходимо сказать "Поправляю!", чтобы вас не заставили сделать ход этой фигурой.
\end{itemize}

\item \emph{Отпустил фигуру – ход сделан!}

Если игрок сделал ход на доске и отпустил фигуру –- ход считается сделанным. До этого момента можно сделать и другой ход, но только этой же фигурой. Данное правило также воспитывает в детях умение предвидеть и принимать последствия своих решений.

\item \emph{Проиграл – поздравь соперника!}

Слово "Поздравляю" из уст проигравшего и протянутая для рукопожатия рука –- это признак рыцарского, взрослого и адекватного восприятия. Проигрыш –- это опыт из которого необходимо сделать выводы, исправить ошибки и, в следующий раз, вы обязательно одержите победу. Не менее важно контролировать свои эмоции и выиграв партию. Следует с уважением отнестись к поражению соперника и пожелать ему удачи в дальнейшем. Излишняя показная гордость может быть наказана уже в следующей партии.

\item \emph{Соблюдай тишину!}

Шахматная партия –- это соревнование умов и лишний шум отвлекает от обдумывания, уменьшает концентрацию. Разговоры во время партии должны сводится к предложению ничьи, поправлению фигур, и, если есть необходимость, можно обратится к судье (в нашем случае, к преподавателю).

\end{enumerate}

И последнее, но очень важное правило: \emph{Будьте взаимно вежливы!}

\section{Домашнее задание}

прописать