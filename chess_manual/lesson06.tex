\chapter{Шахматная нотация}

\section{Методические указания}

В рамках занятия требуется объяснить ученикам, почему необходимо хорошо знать шахматную доску. Необходимо обратить внимание учеников на правильное расположение доски: у игроков в левом углу доски всегда черная клетка. Провести несколько упражнений по определению цвета поля по его координатам.

Рассказать о способах записи позиций. Записать несколько позиций. 

Рассказать о различных формах записи шахматной партии. Продемонстрировать полную нотацию, краткую нотацию, рассказать о записи партий на различных языках. Партию полной нотацией лучше записывать в столбик, а короткой нотацией лучше записывать в строчку. Сначала рекомендуется ученикам записывать партию полной нотацией. Показать специальные символы, для обозначения оценок ходов / позиции.

\section{Содержание занятия}

Под шахматной нотацией понимается форма записи шахматной партии. Она применяется для обозначения полей доски, для записи позиций и шахматных партий. Вертикали нумеруются латинскими буквами от a до h, а горизонтали – цифрами от 1 до 8. Название каждого поля определяется обозначениями пересекающихся в этом поле горизонтали и вертикали. Таким образом, левый угол доски со стороны белых будет записываться как а1.

В настоящее время, в шахматной литературе, для обозначения шахматных фигур используют международную алгебраическую нотацию, в которой используются специальные обозначения шахматных фигур (см. таблицу 1). 

Для обозначения шахматных фигур используют большую первую букву названия фигуры: конь – К, король – Кр (добавляют букву р чтобы отличать от коня), слон – С и т.д. Сокращения, принятые в различных странах, указаны в таблице 1.

\begin{center}
\emph{Таблица 1} Обозначения шахматных фигур в различных языках
\begin{tabular}{ | l | c | c | c | c | } 
\hline
Название фигуры & Фигура & Русский & Румынский & Английский \\ \hline
король & \king & Кр & R & K \\ \hline
ферзь & \queen & Ф & D & Q \\ \hline
Ладья & \rook & Л & T & R \\ \hline
Слон & \bishop & С & N & B \\ \hline
Конь & \knight & К & C & N, S \\ \hline
пешка & \pawn & П & P & P \\ \hline
\end{tabular}
\end{center}

Для указания фигуры на поле записывают сокращение этой фигуры, а потом само поле. Например, Ке2 означает, что на поле е2 стоит конь, Крg7 – король на g7.

При записи позиции нотацией сначала перечисляют белые фигуры, потом черные. Если есть несколько одинаковых фигур, то при их перечислении достаточно указать только имя первой. Обычно фигуры перечисляют по их положению на горизонталях, слева направо.

\emph{Пример1:} запись начальной позиции:

{\indent \textbf{Белые:} Кре1, Фd1, Ла1, h1, Cc1, f1, Kb1, g1, пп. a2, b2, c2, d2, e2, f2, g2, h2

\textbf{Черные:} Кре8, Фd8, Ла8, h8, Cc8, f8, Kb8, g8, пп. a7, b7, c7, d7, e7, f7, g7, h7}

\emph{Пример 2:} позиция с диаграммы 20 записывается следующим образом:

{\indent \textbf{Белые:} Крg1, Ла1, f1, Ce3, пп. b4, c2, d3, e4, f2, g2, h3 (11)

\textbf{Черные:} Крc8, Лd8, h8, Kf6, пп. a6, b7, c7, e5, f7, g7, h7 (11)

Ход белых}

Иногда, после записи положений фигур указывается их количество (как в примере 2), для проверки.
 
\begin{center}
\begin{tabular}{ c }
\chessboard[setfen=2kr3r/1pp2ppp/p4n2/4p3/1P2P3/3PB2P/2P2PP1/R4RK1 w] \\
\textbf{Д. 21} \\
\end{tabular}
\end{center}

Существует две формы записи партии: длинная нотация, которую иногда называют полной нотацией, и короткая нотация (краткая нотация).

При записи хода длинной нотацией сначала указывается фигура, которая делает ход, потом указывается, с какого поля она ходит, и через черточку указывается поле, на которое она ходит. В случае взятия, между полями, вместо черточки пишется крестик.

\emph{Пример 3:} если ферзь ходит с поля d1 на поле g4, то ход записывается следующим образом. Фd1-g4. Если конь с поля f3 берет пешку на е5, то ход записывается как Kf3xe5.

Замечание: В русской записи взятие обозначается двоеточием (:), а мат – буквой х.

При записи хода пешкой указываются только поля, на которых пешка бывает. Например, если ходит пешка с поля с6 на поле с7, то ход записывается следующим образом: с7-с6.

Рокировка имеет свою собственную форму записи. Длинная рокировка обозначается как О-О-О, Короткая рокировка обозначается как О-О. 

\begin{center}
\emph{Таблица 2} специальные шахматные символы
\begin{tabular}{ | c | l | c | l | }  
\hline
Обозначение & Пояснение & Обозначение & Пояснение \\ \hline
- & простой ход & ! & Сильный ход \\ \hline
x & Взятие & ? & Ошибка \\ \hline
+ & Шах & !! & Очень сильный ход \\ \hline
\# & Мат & ?? & Грубая ошибка \\ \hline
+- & У белых выиграно & += & У белых преимущество \\ \hline
-+ & У черных выиграно & =+ &У черных преимущество \\ \hline
?! & Сомнительный ход & !? & Ход с идеей \\ \hline
= & Ничья & ~ & неясная позиция \\ \hline
++ & двойной шах & e.p. & Взятие на проходе \\ \hline
\end{tabular}
\end{center}

\emph{Пример 4:} партия, записанная длинной нотацией:
\begin{center} 
\textbf{Osswald,Cornelia - Beau,Annette, Baden, 1985}
\begin{tabular}{ l l }
1.e2-e4 c7-c6 & 7.h2-h3 Сg4xf3 \\
2.d2-d4 d7-d5 & 8.Фd1xf3 f6xe5 \\
3.e4-e5 Сc8-f5 & 9.Фf3-h5+ g7-g6 \\
4.Кg1-f3 Кb8-d7 & 10.Сd3xg6+! h7xg6 \\
5.Сc1-f4 f7-f6 & 11.Фh5xg6\# \\
6.Сf1-d3 Сf5-g4 & \\
\end{tabular}
\end{center}
 
Для оценки ходов или позиции используются специальные шахматные символы, которые пишутся после хода (таблица 2). Например, в примере 4 10 ход белых отмечен как сильный ход.

В записи хода короткой нотацией указывается фигура, которая ходит, и поле, на которое ходит эта фигура. В случае, если на поле может пойти две одинаковые фигуры, то указывается название фигуры, потом вертикаль, с которой уходит фигура, потом поле, на которое фигура ходит. Если же вертикали у фигур совпадают, указывается горизонталь.

\emph{Пример 5:} Ход коня с поля c6 на поле e5 будет записываться как Кce5 (Д. 22).
 
\begin{center}
\begin{tabular}{ c }
\chessboard[setfen=4k3/8/2N5/8/8/5N2/8/4K3] \\
\textbf{Д. 22 Ход Kc6-e5} \\
\end{tabular}
\end{center}

Превращение пешки записывается как ход пешки, только в конце добавляется обозначение фигуры, в которую она превращается. Иногда ход от обозначения превращенной фигуры отделяется знаком равенства. Например, если пешка ходила с поля с7 на поле с8 и превратилась в ферзя, то ход запишется в длинной нотации как с7-с8=Ф; в короткой нотации с8Ф.

\emph{Пример 6:} партия, записанная короткой нотацией:

\textbf{Napoleon I - Madame de Remusat, Malmaison Castle Malmaison Castle, 1804}

{\indent 1.Kc3 e5 2.Kf3 d6 3.e4 f5 4.h3 fxe4 5.Kxe4 Kc6 6.Kfg5 d5 7.Фh5+ g6 8.Фf3 Кh6 9.Кf6+ Kрe7 10.Кxd5+ Kрd6 11.Кe4+ Kрxd5 12.Сc4+ Kрxc4 13.Фb3+ Kрd4 14.Фd3\# }

\section{Домашнее задание}

\begin{enumerate}
\item Записать позицию нотацией:
 
\begin{center}
\begin{tabular}{ c }
\chessboard[setfen=r1bbk2r/1pp2ppp/p4n2/4p3/1n2P3/1BN1BN2/PPP2PPP/R3K2R b KQkq - 0 10] \\
\textbf{Д. 6.1 Ход чёрных} \\
\end{tabular}
\end{center}

\item Расставить позицию:

{\indent \textbf{Белые:} Крg1, Фd1, Ла1, f1, Cc1, d3, Kb1, f3, пп. a2, b2, c3, e5, f2, g2, h2 (15)

\textbf{Черные:} Крg8, Фd8, Ла8, f8, Cc8, c5, Kc6, e7, пп. a7, b7, d5, e6, f7, g7, h7 (15)

ход белых}

\item Сыграть и записать партию дома.
\end{enumerate}