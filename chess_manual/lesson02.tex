\chapter{Правила шахмат}
\section{Методические указания}

Данная тема является объёмной, поэтому рекомендуем разделить её на 2 урока и совместить, например, с темой ``История появления и развития шахмат''. В первой части расскажите о геометрии шахматной доски и ходах фигур, а вторую часть посвятите таким понятиям, как шах, мат и пат.

Следует обратить внимание детей на правильные названия шахматных фигур: ферзь, но не королева или дама; ладья, но не тура; слон, но не офицер. Также необходимо правильно располагать доску –- угловое поле доски слева от игрока должно быть черным.

\section{Содержание занятия}

В шахматы играют на доске, размером 8х8 клеток (всего 64 клетки, см. Д. 2). Шахматная доска имеет свою геометрию, а именно 3 вида линий: вертикали, горизонтали и диагонали (см. Д. 3). Попросите детей посчитать количество вертикалей и горизонталей, найти самую большую и маленькую белую и чёрную диагональ. 

\begin{center}
\begin{tabular}{ c c }
\chessboard[setfen=8/8/8/8/8/8/8/8,showmover=false]
&
\chessboard[setfen=8/8/8/8/8/8/8/8,
color=red,
pgfstyle=text,
text={\large $\leftarrow$ горизонталь $\rightarrow$},
markregion=a1-h1,
pgfstyle={[rotate=90]text},
text={\large $\leftarrow$ вертикаль $\rightarrow$},
markregion=a1-a8,
pgfstyle={[rotate=45]text},
text={\large $\leftarrow$ д и а г о н а л ь $\rightarrow$},
markregion=a1-h8,
showmover=false] \\
\textbf{Д. 2 шахматная доска} & \textbf{Д. 3 геометрия шахматной доски}\\
\end{tabular}
\end{center}
 
У каждой вертикали и горизонтали есть имя – это латинские буквы и цифры соответственно. Каждая клетка, или шахматное поле, имеет своё ``имя'', которое состоит из буквы и цифры, на пересечении которых оно находится (см. Д. 4). Попросите детей назвать ``имена'' нескольких полей.

Левая часть доски со стороны белых называется ферзевым флангом, правая часть -- королевским флангом. Малым центром доски называются поля d4, d5, e4, e5. Большим центром называется квадрат, ограниченный полями с3, с6, f3 и f6.

На шахматной доске располагаются 16 белых фигур: пешки, ладьи, кони, слоны, ферзь и король; и столько же черных фигур (см. Д. 5). Играющий белыми фигурами начинает партию, ходы делаются по очереди. При расстановке фигур, обратите внимание детей на то, как правильно расположить короля и ферзя. Расскажите им о правиле ``ферзь любит свой цвет'', то есть белый ферзь на белом поле, а чёрный -- на чёрном и, соответственно, короли рядом.

\begin{center}
\begin{tabular}{ c c }
\chessboard[setfen=8/8/8/8/8/8/8/8,showmover=false]
&
\chessboard[setfen=rnbqkbnr/pppppppp/8/8/8/8/PPPPPPPP/RNBQKBNR w] \\
\textbf{Д. 4 шахматные поля} & \textbf{Д. 5 Начальная позиция} \\
\end{tabular}
\end{center}
 
Ладья может стать на любое поле вертикали или горизонтали, на которых она стоит. На Д. 6 ладья может пойти на любое из полей a5, b5, d5, e5, f5, c4, c3, c6, c7, c8; а также взять черную ладью на поле g5.

\begin{center}
\begin{tabular}{ c c }
\chessboard[
setfen=8/8/8/2R3r1/8/8/2P5/8,
pgfstyle=straightmove,
color=red,
markmoves={c5-a5, c5-c8, c5-g5, c5-c3},
showmover=false]
&
\chessboard[
setfen=8/8/8/4B3/8/6p1/1P6/8,
pgfstyle=straightmove,
color=red,
markmoves={e5-b8, e5-c3, e5-h8, e5-g3},
showmover=false] \\
\textbf{Д. 6 ходы ладьёй} & \textbf{Д. 7 ходы слона} \\
\end{tabular}
\end{center}
 
Слон может ходить на любое поле по диагоналям, на которых он стоит. На диаграмме 7 видно, что слон может пойти на поля b8, c7, d6, c3, d4, f6, g7, h8, f4, а также взять пешку g3.

Ферзь может ходить на любое поле по вертикали, горизонтали или диагонали, на которых он стоит. Его ходы совпадают с ходами слона и ладьи одновременно (Д. 8).

При выполнении ходов слон, ладья или ферзь не могут передвигаться через какие-либо стоящие на их пути фигуры.

\begin{center}
\begin{tabular}{ c c }
\chessboard[
setfen=8/8/8/5n2/8/3Q2R1/8/8,
pgfstyle=straightmove,
color=red,
markmoves={d3-a3, d3-a6, d3-d8, d3-f5, d3-f3, d3-f1, d3-d1, d3-b1},
showmover=false]
&
\chessboard[
setfen=8/8/8/8/3N4/8/8/8,
pgfstyle=knightmove,
color=red,
markmoves={d4-b5, d4-c6, d4-e6, d4-f5, d4-f3, d4-e2, d4-c2, d4-b3},
showmover=false] \\
\textbf{Д. 8 ходы ферзя} & \textbf{Д. 9 ходы коня} \\
\end{tabular}
\end{center}
 
Конь может пойти на одно из полей, ближайших к тому, на котором он стоит, но не на той же самой горизонтали, вертикали или диагонали. Ход коня похож на букву "Г" (Д. 9). 

\begin{center}
\begin{tabular}{ c }
\chessboard[
setfen=8/8/8/2p5/3P4/8/4P3/8,
pgfstyle=straightmove,
color=red,
markmoves={d4-c5, d4-d5, e2-e3, e2-e4},
showmover=false] \\
\textbf{Д. 10 ходы пешкой} \\
\end{tabular}
\end{center}

Пешка может ходить вперед на свободное поле, расположенное непосредственно перед ней на той же самой вертикали. С начальной позиции она может ходить как на одно поле, так на два поля по той же самой вертикали, если оба эти поля не заняты. Взятие пешкой совершается следующим образом: на поле, занимаемое фигурой соперника, расположенное перед ней по диагонали на соседней вертикали ставится пешка, одновременно снимая фигуру соперника с шахматной доски. На Д. 10 видно, что белая пешка с поля d4 может ходить на поле d5, а также забрать пешку с5. Пешка с поля е2 может пойти на поле е3 или на поле е4.

\begin{center}
\begin{tabular}{ c }
\chessboard[
setfen=8/8/8/8/4K3/8/8/8,
pgfstyle=straightmove,
color=red,
markmoves={e4-e5, e4-f5, e4-f4, e4-f3, e4-e3, e4-d3, e4-d4, e4-d5},
showmover=false] \\
\textbf{Д. 11 ходы короля} \\
\end{tabular}
\end{center}

Король – самая важная фигура. Король ходить может на любое поле вокруг себя, если оно не занято своей фигурой или не атаковано фигурой соперника (Д. 11). Ещё одно важное правило: короли не встречаются, то есть король не может пойти на клетку рядом с королём соперника.

Цель любого шахматиста в партии –- выигрыш короля соперника, или, если быть более точным – поставить мат (напасть таким образом на короля соперника, чтобы соперник не смог его защитить). Нападение на короля какой-либо фигурой называется шахом. От шаха можно защититься тремя способами (Д. 12):

\begin{itemize}
\item Увести короля из под шаха;
\item Забрать напавшую на короля фигуру;
\item Перекрыться (поставить фигуру между королём и фигурой соперника).
\end{itemize}

Если защититься от шаха невозможно, то король получил мат, игра заканчивается.

\begin{center}
\begin{tabular}{ c c }
\chessboard[
setfen=Q3k3/3p4/2b2K2/8/r3b2N/2k5/5k1P/K6K,
pgfstyle=border,
markregions={a1-d4, a5-h8, e1-h4},
showmover=false]
&
\chessboard[
setfen=k3K2k/R7/2NK2Q1/7R/4B3/1K2B2K/1Q6/1k5k,
pgfstyle=border,
markregions={a1-d4, a5-d8, e1-h4, e5-h8},
showmover=false] \\
\textbf{Д. 12 Защита от шаха} & \textbf{Д. 13 мат} \\
\end{tabular}
\end{center}
 
Другими словами, мат –- это шах от которого нет защиты. На диаграмме 13 приведены примеры мата.

Для достижения победы часто необходимо получить материальное преимущество, то есть забрать как можно больше фигур соперника. Для того, чтобы примерно оценить у кого преимущество в партии используют простейший подсчет материала. Считается, что пешка –- самая слабая фигура, конь и слон стоят три пешки, ладья –- пять пешек и ферзь –- девять пешек. В связи с тем, что с потерей короля игра заканчивается, стоимость короля не учитывается.

Не разрешается ходить фигурой на поле, занятое фигурой того же цвета. Если фигура ходит на поле, занятое фигурой партнера, последняя берется и снимается с шахматной доски, что является частью того же самого хода. Считается, что фигура нападает на фигуру партнера, если фигура может совершить взятие на этом поле. Расставьте детям произвольную позицию и попросите их найти возможные взятия.

Результат игры
Игра считается завершившейся победой одного из игроков в следующих случаях:

\begin{itemize}
\item После объявления  мата королю соперника; 
\item После заявления соперника о своем поражении в партии.
\end{itemize}

Партия считается закончившейся вничью:

\begin{itemize}
\item Если игрок, который должен ходить, не имеет никакого, разрешенного правилами игры, хода, а его король не находится под шахом. Такую ситуацию в игре называют "патом";
\item Если возникла позиция, когда ни один из игроков не может заматовать короля соперника любой серией возможных ходов (например, если на доске не осталось фигур, способных поставить мат); 
\item По соглашению между игроками во время партии. 
\end{itemize}

Партия может закончиться вничью:
\begin{itemize}
\item Если в ходе игры на шахматной доске возникла или может возникнуть, по крайней мере, в третий раз идентичная (одинаковая) позиция; 
\item Если последние 50 последовательных ходов были сделаны игроками без ходов какой-либо пешки, и без взятия какой-либо фигуры. 
\end{itemize}

\begin{center}
\begin{tabular}{ c }
\chessboard[
setfen=k3Q1nk/1R6/2K3K1/8/1R3Q2/3K3k/p7/k4K2,
pgfstyle=border,
markregions={a1-d4, a5-d8, e1-h4, e5-h8},
showmover=false] \\
\textbf{Д. 14 пример пата} \\
\end{tabular}
\end{center}

\section{Домашнее задание}

\begin{enumerate}
\item Нарисуйте шахматную доску.
\item На диаграммах ниже (подготовьте заранее домашнее задание для каждого ребёнка на отдельном листе) стрелочками укажите, какие чёрные фигуры может взять белая фигура.

\begin{center}
\begin{tabular}{ c c }
\chessboard[
setfen=8/1b6/8/3R3q/8/1n6/8/3r4,
showmover=false]
&
\chessboard[
setfen=8/7n/r5n1/8/8/2pB3q/8/8,
showmover=false] \\
\textbf{2.1} & \textbf{2.2} \\
\chessboard[
setfen=8/8/1q6/1r3p2/3N4/8/2b4n/8,
showmover=false]
&
\chessboard[
setfen=6n1/6n1/8/3Q3r/1r6/8/8/2bb4,
showmover=false] \\
\textbf{2.3} & \textbf{2.4} \\
\end{tabular}
\end{center}

\begin{center}
\begin{tabular}{ c c }
\chessboard[
setfen=8/8/4p3/2n1Kb2/5r2/3q4/8/8,
showmover=false] 
&
\chessboard[
setfen=8/8/8/8/2rp4/3P4/8/8,
showmover=false] \\
\textbf{2.5} & \textbf{2.6} \\
\end{tabular}
\end{center}
 
\item На диаграммах ниже определите, где шах, где пат, а где мат. В случае шаха укажите возможные способы защиты.

\begin{center}
 \begin{tabular}{ c c }
\chessboard[
setfen=6k1/5pp1/4p3/8/8/2N3P1/4PP2/1r3K2,
showmover=false]
&
\chessboard[
setfen=8/8/7b/8/8/k1n5/2PR4/2K5,
showmover=false] \\
\textbf{2.7} & \textbf{2.8} \\
\chessboard[
setfen=R3k3/1R6/8/8/8/8/8/4K3,
showmover=false] 
& 
\chessboard[
setfen=R2bk1K1/1R6/8/8/8/8/8/8,
showmover=false] \\
\textbf{2.9} & \textbf{2.10}\\
\end{tabular}
\end{center}

\end{enumerate}