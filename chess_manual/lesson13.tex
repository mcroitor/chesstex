\chapter{Тактические приёмы: уничтожение защиты}

\section{Методические указания}

В начале занятия необходимо повторить с учениками понятия форсированного варианта, комбинации, а также виды тактических ударов. 

В конце урока можно провести конкурс решения позиций, так как ученики уже знают простейшие тактические удары. Пример позиций для конкурса приведен в конце занятия.

\section{Содержание занятия}

Часто случается, что неприятельская фигура защищает какой-то важный пункт, захват которого решает партию. Для того, чтобы добраться до этого пункта, защищающую фигуру разменивают. Данный приём называется уничтожением защиты. Он демонстрируется на диаграмме 54:
 
 
Д. 54 ход белых
 
Д. 55 ход белых
 
Следует обратить внимание на расположение фигур: ладья, конь и ферзь готовы атаковать вражеского короля, тогда как черные фигуры ютятся на последней горизонтали. Черного короля защищает только пешка g7 – без неё был бы мат в два хода: 1.Фf7+ Kph8 2.Kg6x Поэтому белые сначала уничтожают защитника:

1.Л:g7+! Kp:g7 2.Фf7+ Kph8 3.Kg6x

Похожая ситуация и на диаграмме 55. Не смотря на то, что белому королю грозит мат в один ход и их ферзь под боем, они побеждают. Следует заметить, что черного короля от мата 1.Лh1x защищает опять только пешка. Поэтому белые уничтожают эту пешку, отдав за неё самую сильную фигуру – ферзя:

1.Ф:h7+!! Kp:h7 2.Лh1x

Конечно, можно уничтожать не только защитника короля, но и какой-либо фигуры, как это можно заметить на диаграмме 56.
 
Д. 56 ход белых

Черный конь занимает хорошую позицию, но его защищает только пешка. Это позволяет провести следующую маленькую комбинацию:

1. K:d5!

Уничтожая защитника коня. Попытка отыграть пешку ни к чему не приводит – 1… K:f2 2.Kp:f2 и коня нельзя забирать из-за мата. Приходится довольствоваться черным потерей пешки.

1… Л:d5 2.Л:e4

После чего у белых решающее преимущество.

\section{Домашнее задание}

Выиграйте, уничтожив защитника. 
 
13.1 ход белых
 
13.2 ход белых
  
13.3 ход белых
