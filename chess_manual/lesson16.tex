\chapter{Шахматная этика}

\section{Методические указания}

Смотрите методические указания к теме ``Шахматная этика'', 1ое занятие.

В конце занятия проведите тест на 10 минут, для проверки и закрепления знаний учеников. Примеры тестов можно найти в приложении.

В рамках занятия желательно провести первый квалификационный турнир, в результате которого ученики могут получить свою первую категорию – 4-й разряд. Об организации шахматных турниров смотрите приложение.

\section{Содержание занятия}

На прошлом занятии по шахматной этике мы говорили о правилах поведения во время партии. Все эти правила очень важны и применяются, как во время дружеских и тренировочных встреч, так и обязательны во время турниров. Но существует и несколько специальных турнирных правил, которые дети должны знать.

Начнём мы с вежливости, если игрок не хочет продолжать партию и просто встаёт и уходит, не сдав партию, то судья может наказать его за неспортивное поведение: поставить проигрыш в данной партии и даже дисквалифицировать из турнира.

Другим важным правилом является соблюдение тишины: все жалобы и претензии надо говорить судье, а не своему сопернику.

Большинство турниров проводятся с использованием шахматных часов. (покажите детям шахматные часы и объясните принцип их работы). После того, как соперники поприветствовали друг друга, шахматист, играющий чёрными фигурами, запускает часы. Часы надо переключать той же рукой, которой сделан ход, даже если это неудобно. Нажимать на кнопку надо аккуратно, ни в коем случае нельзя бить по часам: время от этого быстрее не пойдёт, а судья может вас наказать.

Начинающие шахматисты иногда, по невнимательности, делают невозможные ходы. В этом случае, игрок должен позвать судью и указать на такой ход. Игрок, сделавший запрещённый ход, получает замечание и обязан сделать регламентированный ход. При 3ем невозможном ходе, судья засчитывает поражение. Объясните детям, что они должны быть очень внимательны, иначе могут проиграть партию даже с большим материальным перевесом.

Во время партии можно предложить ничью, но делать это нужно сразу после сделанного хода и перед нажатием на часы. Однажды сделанное предложение нельзя взять обратно. Соперник может принять или отклонить предложение ничьи.

Очень важно, что во время партии нельзя ни с кем разговаривать, советоваться, смотреть в книги или конспекты, разговаривать по телефону –- за всё это вы сразу получаете поражение в партии. Также нельзя покидать игровой зал. Если вам очень нужно выйти – спросите разрешение у судьи.

И, ни в коем случае, нельзя отвлекать соперника и мешать ему обдумывать ход!

Игрок, который закончил партию, становится зрителем. Ни в коем случае нельзя комментировать партии других игроков, подсказывать, разговаривать в зале или сообщать о падении флажка. Даже закончив партию, мы должны уважать друг друга, и вести себя с другими так, как нам бы хотелось, чтобы они относились к нам.

\section{Домашнее задание}

Проанализировать одну из сыгранных в турнире партий.