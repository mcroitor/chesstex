\chapter*{Учебный план}

Группы начальной подготовки: рекомендуемый возраст 7 – 9 лет (начальные классы)

\section*{СПЕЦИФИЧЕСКИЕ КОМПЕТЕНЦИИ:}

\begin{enumerate}
\item Развитие когнитивных процессов через формирование качеств, необходимых для игры в шахматы;
\item Развитие личности, способной интегрироваться в социум, конструктивно взаимодействуя с окружающими;
\item Знание терминологии, правил шахмат.
\item Развитие психомоторики и умение применять методы восстановления посредством физических упражнений, медитации 
\end{enumerate}

\begin{center}
\begin{longtable}{ p{0.25\textwidth} | p{0.1\textwidth} | p{0.5\textwidth} }
Название темы & Кол-во уроков & Ожидаемый результат \\ \hline
История шахмат & 2 & Должны знать:

\begin{itemize}
\item историю возникновения шахмат
\item историю развития шахмат
\end{itemize} \\ \hline
Шахматные правила & 3 & Должны знать:

\begin{itemize}
\item геометрию шахматной доски
\item названия фигур и их ходы 
\item относительную стоимость фигур
\item определения шахматных терминов (шах, мат, пат, рокировка, взятие на проходе, превращение пешки)
\item результат игры
\item случаи, когда рокировка запрещена
\end{itemize}

Должны уметь:

\begin{itemize}
\item Правильно расположить доску
\item Ходить фигурами
\item определять шах, мат, пат
\item защищаться от шаха
\item делать рокировку
\item делать взятие на проходе
\end{itemize}

Обобщение

\begin{itemize}
\item Должны уметь играть партии по шахматным правилам от начала и до конца
\end{itemize} \\ \hline
Шахматная нотация & 1 & Должны знать
\begin{itemize}
\item Обозначения полей
\item Обозначения шахматных фигур
\item Специальные символы (взятие, шах, мат, хороший ход, плохой ход и т.д.)
\item Виды нотаций
\end{itemize}
Должны уметь
\begin{itemize}
\item Записать позицию
\item Прочитать запись позиции
\item Записать партию полной и краткой нотацией 
\item Прочитать партию в полной и краткой нотации
\end{itemize} \\ \hline
Матование одинокого короля & 3 & Должны знать
\begin{itemize}
\item Принципы построения мата
\end{itemize}
Должны уметь
\begin{itemize}
\item Ставить линейный мат
\item Ставить мат ферзём и королём
\item Ставить мат ладьёй и королём
\item Ставить мат двумя слонами
\end{itemize} \\ \hline
Тактические приёмы. Комбинация & 8 & Должны знать
\begin{itemize}
\item Понятия тактического приёма, форсированного варианта, жертвы, комбинации
\item Базовые тактические удары (связка, двойной удар, двойной шах, открытое нападение)
\item Базовые тактические приёмы (уничтожение защиты, отвлечение, завлечение, перекрытие линии, освобождение линии, блокирование, освобождение поля)
\end{itemize}
Должны уметь
\begin{itemize}
\item Определять вид тактического удара
\item Определять используемый тактический приём
\item Решать простые комбинации с использованием базовых тактических ударов и приёмов
\end{itemize}
Обобщение
\begin{itemize}
\item Ученики должны применять полученные знания и умения в партии
\item Должны уметь рассчитывать варианты
\end{itemize} \\ \hline
Дебют & 2 & Должны знать
\begin{itemize}
\item Классификацию дебютов
\item Как правильно начинать партию
\item Типичные дебютные ошибки
\item Итальянскую партию
\end{itemize}
Должны уметь
\begin{itemize}
\item Правильно начинать партию
\item Разыгрывать итальянскую партию
\end{itemize} \\ \hline
Миттельшпиль & 7 & Должны знать
\begin{itemize}
\item Понятие большого и малого центра
\item Понятие пешечной структуры
\item Мобильность фигур
\item Понятие контригры
\end{itemize}
Должны уметь
\begin{itemize}
\item Проводить атаку на короля
\item Составлять план игры
\item Создавать контригру
\item Реализовывать материальное преимущество
\end{itemize} \\ \hline
Эндшпиль & 3 & Должны знать
\begin{itemize}
\item Понятие оппозиции
\item Правило квадрата
\item Пешечный прорыв
\end{itemize}
Должны уметь
\begin{itemize}
\item Занимать оппозицию
\item Рассчитывать квадрат пешки
\end{itemize}
Обобщение
\begin{itemize}
\item Ученики должны уметь разыгрывать элементарные пешечные окончания
\end{itemize} \\ \hline
Шахматная этика & 2 & Должны знать
\begin{itemize}
\item Элементарные правила поведения за доской
\item Правила поведения на соревнованиях
\end{itemize}
Должны уметь
\begin{itemize}
\item Воспринимать адекватно результат партии
\item Уважать соперника
\end{itemize} \\ \hline
Шахматная композиция & 1 & Должны знать:
\begin{itemize}
\item Что такое шахматная композиция
\item Разницу между задачей и этюдом
\item Простейшие шахматные термины
\end{itemize}
Должны уметь:
\begin{itemize}
\item Решать двухходовые задачи
\item Решать одновариантные многоходовые задачи
\item Решать простые этюды
\end{itemize} \\ \hline
Практика & 36 & \begin{itemize}
\item Промежуточные турниры
\item Тестирование 
\item Конкурсы решения позиций
\item Общая физическая подготовка
\end{itemize} \\ \hline
ИТОГО: & 68 & \\
\end{longtable}
\end{center}
