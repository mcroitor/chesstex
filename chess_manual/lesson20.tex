\chapter{Тактические приёмы: перекрытие и освобождение линии}

\section{Методические указания}

В начале занятия повторите с учениками тактические удары, а также тактические приёмы, которые они изучили. В конце занятия рекомендуется провести конкурс решения позиций – это позволит оценить текущие знания учеников а также будет способствовать закреплению материала.

\section{Содержание занятия}

Бывает, что некоторая дальнобойная фигура препятствует нанесению решающего удара, но ни забрать, ни отвлечь её мы не можем. Иногда в этом случае может помочь перекрытие. Под перекрытием понимают отрезание активной фигуры от театра действий. Простейший пример перекрытия представлен на диаграмме 81.
 
 
Д. 81 ход белых. Перекрытие
 
Д. 82 ход белых
 
После хода 1.d4 черная ладья отрезается от защиты пешки, белые забирают на h5 и получают материальное преимущество.

Перекрытие может проходить и с жертвой. Например, на диаграмме Д. 82 белым мешает поставить мат на g7 черная ладья. Поэтому её необходимо отрезать от поля g7:

1.Лb7! C:b7 2.Фg7x
 
 
Д. 83 ход белых
 
Д. 84 ход белых. Освобождение
 
Часто используется перекрытие для проведения пешки. Если в позиции на диаграмме 83 белые сыграют 1.d7?, то после 1… Лd1 пешка задерживается и партия заканчивается вничью. Это случается из-за того, что черная ладья успевает стать позади пешки и остановить её. Поэтому необходимо заранее перекрыть эту линию:

1.Лd5!! cd 2.d7

Теперь белую пешку не остановить. Она превратится в ферзя и принесёт белым победу в партии.

Если смысл комбинации на перекрытие в выключении некоторой фигуры противника для того чтобы она нам не мешала, то в комбинациях на освобождение идея в том, что мешает победе уже некоторая наша фигура. И для того, чтобы добиться решающего преимущества ею жертвуют. Например, на диаграмме 84 белый конь мешает нанести решающий удар 1.Фh7 с матом. Поэтому его следует убрать с темпом:

1.Ke7+! C:e7 2.Фh7+ Kpf8 3.Фh8x

Как видно, комбинация на открытие линии похожа по принципу на открытый удар.
 
Д. 85 ход белых

В других ситуациях своя фигура может занимать какое-то важное поле. И в этом случае можно ей пожертвовать или отойти с темпом. Обратите внимание на диаграмму 132. Можно заметить, что белый ферзь занимает выгодное поле g6 для коня, где он делает вилку на короля и ферзя. Это наводит на мысль, что надо убрать ферзя с темпом:

1.Ф:е8+ Л:е8 2.Kg6+ ~ K:h4

И у белых преимущество в фигуру и пешку.

\section{Домашнее задание}

Проведите комбинацию на перекрытие
 
 
20.1 ход белых
 
20.2 ход белых
 
Проведите комбинацию на открытие линии (поля)
 
 
20.3 ход белых
 
20.4 ход белых
 