\chapter{Атака на короля при разносторонних рокировках}

\section{Методические указания}

Напомните детям виды рокировок. Попросите детей представить или расставить на доске 2 позиции: в одном случае рокировки односторонние, а в другом разносторонние. Задайте детям вопрос: Как вы думаете, в каком случае легче начать атаку на короля? Почему? Правильно, при разносторонних рокировках, к атаке могут подключиться и пешки, не раскрывая при этом собственного короля.

\section{Содержание занятия}

Когда противники сделали рокировки на разных флангах, возникает возможность атаки на короля. Необходимо рассмотреть несколько элементов: защищённость собственного короля, расположение фигур, своих и соперника, владение центром, возможность пешечного штурма. Оцените возможности, наметьте план и вперёд!
Давайте рассмотрим пример атаки из партии одного из чемпионов мира Хосе Рауля Капабланки с знаменитым мастером Яновским. Партия была сыграна в Петербурге в 1914 году.
1.е4 е5 2.Кf3 Кc6 3.Cb5 a6 (испанская партия) 4.C:c6 dc 5.Кс3 Сс5 (лучше 5…f6) 6.d3 Cg4 7.Ce3 C:e3 (неудачный размен) 8.fe Фе7 9. 0-0 0-0-0 10.Фе1 Кh6 11.Лb1 (намечая план пешечной атаки) 11… f6 12.b4 Кf7 13.a4 C:f3 14.Л:f3 b6 15.b5 cb 16.ab a5 (вскрытие линий предупреждено, но ценой ослабления важного пункта d5) 17.Kd5 Фс5 18.с4 (теперь задача белых подготовить d3-d4 и с4-с5) 18… Кg5 19.Лf2 Ke6 20.Фс3 Лd7 21.Лd1 Крb7 22.d4 Фd6 23.Лс2 ed 24.ed Kf4? (ошибка в трудном положении) 25.с5 К:d5 26.ed Ф:d5 27.c6+ и вскоре чёрные сдались. Успеху пешечного штурма – наиболее распространённого плана атаки при разносторонних рокировках – способствовало ослабленное пешечное прикрытие чёрного короля.

\section{Домашнее задание}

Предложите возможное продолжение атаки за белых.
Алёхин – Минделло,
Голландия 1933 год
 
22.1 ход белых
