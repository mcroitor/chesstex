\section{Решения задач Кройтор М.В.}

\begin{enumerate}
\item \textbf{``Задачи и этюды'', 2002	, Специальный Почетный отзыв}. Первая опубликованная и отмеченная задача. Первым ходом представляется два поля черному королю. \textbf{1.\knight{}g5!} с угрозой \textbf{2.\knight{}xf3\mate{}} И 4 варианта: \textbf{1. ... \king{}e3 2.\knight{}e6\mate{}; 1. ... \king{}e5 2.\bishop{}g7\mate{}; 1. ... \rook{}d3 2.\queen{}c5\mate{}; 1. ... \bishop{}c3 2.\queen{}e4\mate{}}

\item \textbf{ ``Рыбинск 7 дней'', 2006, 2ой Похвальный Отзыв } Популярная задача, самой интересной попыткой решения является 1.\king{}xg6 ?? с патом! В решении белый король уводится от зоны боевых действий:
\textbf{1.\king{}g7!} цугцванг! \textbf{1...\king{}f5 2.\queen{}h3\mate{}; 1...g5 2.\knight{}h6\mate{}}
   
\item \textbf{ ``Mat Plus'', 2007 } Задача интересна обилием попыток, затрудняющих решение: 

1.\bishop{}b5?? цугцванг 1...e4 2.\queen{}c4\mate{}, опровержение 1...c4!

1.\queen{}f3+?? 1...\king{}c4 2.\queen{}b3\mate{}, опровержение 1...e4!

1.\queen{}e2?? цугцванг 1...c4 2.\queen{}xe5\mate{};  1...d3 2.\queen{}xd3\mate{}, опровержение 1...e4!

1.\queen{}f5?? цугцванг  1...c4 2.\queen{}xe5\mate{};  1...d3 2.\queen{}xd3\mate{}, опровержение 1...\king{}c4!

1.\queen{}b5?? цугцванг 1...d3 2.\queen{}xd3\mate{};  1...e4 2.\bishop{}b3\mate{}, опровержение 1...\queen{}e4!

1.\queen{}a6?? (2.\bishop{}c6\mate{}); 1...d3 2.\queen{}xd3\mate{};  1...c4 2.\queen{}c6\mate{}, опровержение  1...\king{}e4!

В действительной игре белые предоставляют черному королю два поля. \textbf{1.\queen{}g3!} цугцванг. С пятью вариантами: \textbf{1...\king{}e4 2.\bishop{}c6\mate{}; 1...\king{}c4 2.\queen{}b3\mate{}; 1...c4 2.\queen{}xe5\mate{}; 1...d3 2.\queen{}xd3\mate{};  1...e4 2.\queen{}g8\mate{}}
   
\item \textbf{ ``Mat Plus'', 2009 } Эта задача была опубликована в редакции Андрея Журавлева. Интересна роль черного слона, который добавляет тактики в ложных следах (1.\rook{}xe4+?, опровержение  1...\king{}xe4+!)

Решение: \textbf{1.\rook{}e3!} с угрозой \textbf{2.\rook{}xe4\mate{}} Жертвуются две фигуры. \textbf{1...\rook{}xd4 2.\queen{}g5\mate{}; 1...\king{}xf4 2.\queen{}g3\mate{}; 1...\king{}xd4 2.\queen{}a1\mate{}; 1...\knight{}xf4 2.\queen{}g7\mate{}}
      
\item \textbf{ ``ChessStar'', 2012, 1ый-2ой Приз }  Перемена матов в форме близнецов в миниатюре. a) \textbf{1.\rook{}g8!} цугцванг. \textbf{1...\king{}xd6 2.\rook{}g6\mate{}; 1...cd6 2.\rook{}c8\mate{}}
          
b) \rook{}g6 --> \rook{}f6  \textbf{1.\rook{}f5! }цугцванг. \textbf{1...\king{}xd6 2.\queen{}f6\mate{}; 1...cd6 2.\queen{}c1\mate{}}
      
\item \textbf{ ``ChessStar'', 2014, 3ий Похвальный Отзыв, миниатюры } \textbf{1.\queen{}g7!} цугцванг. \textbf{1...\king{}d3 2.\queen{}c3\mate{}; 1...\king{}f4 2.\queen{}e5\mate{}; 1...\king{}f2 2.\queen{}g1\mate{}; 1...\king{}d2 2.\queen{}c3\mate{}; 1...c4 2.\queen{}d4\mate{}}

\item \textbf{ ``Cupa Bucovinei 2015, E4 E5'', 2016, 3 Место } \textbf{1.\rook{}e4!} с угрозой \textbf{2.\rook{}e5\mate{}} Жертва ладьи, черным развязывается ферзь, даётся поле королю. \textbf{1...\bishop{}xe4 2.\queen{}a2\mate{}; 1...\queen{}xe4 2.\queen{}g5\mate{}; 1...\king{}xe4 2.d5\mate{}}
      
\item \textbf{ ``Cupa Bucovinei 2015, E4 E5'', 2016, 4 Место } \textbf{1.\knight{}e4!} с угрозой \textbf{2.\bishop{}d6\mate{}.  1...\bishop{}xe4 2.\rook{}xe4\mate{}; 1...\bishop{}xc5 2.d4\mate{}; 1...\king{}f4 2.\bishop{}xe3\mate{}}
      
\item \textbf{ ``Гравюра-2016'', 2016, 1ый Почетный Отзыв } Не смотря на легкость позиции, у задачи богатая игра.  В попытках черный сло проявляет большую активность: 1.\rook{}e3? \bishop{}c4!;  1.\knight{}e5? \bishop{}c4!; 1.\knight{}h4? \bishop{}g4!; 1.\knight{}f2-d1? \bishop{}xd1!

В действительной игре жертвуется ферзь: \textbf{1.\knight{}e4!} с угрозой \textbf{2.\queen{}c3\mate{}; 1...\bishop{}d3 2.\queen{}c5\mate{}; 1...de4 2.\rook{}d8\mate{}}

\item \textbf{ ``Московский конкурс - 2016'', 2016, 4ый Похвальный Отзыв}  В задаче проходит выбор развязывания черной фигуры. В ложном следе развязывается ладья. 1.\knight{}b7? с угрозой 2.\knight{}c5\mate{}; 1...\rook{}d8+ 2.\knight{}xd8\mate{}, опровержение 1...\rook{}b5! со связыванием развязывающей фигуры. В решении развязывается черный слон: \textbf{1.\knight{}c4!} с угрозой \textbf{2.\knight{}d2\mate{}; 1...\bishop{}d6+ 2.\knight{}xd6\mate{}}

\item \textbf{ ``Задачи и этюды'', 2002 } Одна из первых опубликованных задач. Демонстрирует известную (со времен Андерсена) идею построения королевской батареи. Иллюзорная игра: 1... \bishop{}~ 2.\knight{}e6/f3\mate{}. Действительная игра:  \textbf{1.\bishop{}f1!}, с угрозой \textbf{2.\bishop{}d3)}, которая запускает иллюзорную игру.  \textbf{1... \bishop{}d1(f3,h5) 2.\knight{}e6+ \king{}e4 3.\bishop{}d3\mate{}; 1... \bishop{}h3(f5,e6,d7,c8) 2.\knight{}f3+ \king{}e4 3.\bishop{}d3\mate{}}. Центральный вариант: \textbf{1... \bishop{}e2! 2.\king{}xf2! \king{}c4 3.\king{}e3\mate{}}
   
\item \textbf{ ``Рыбинск 7 дней'', 2006, 1ый Почетный Отзыв } 1.\bishop{}f4! цугцванг.
      1...\king{}e4 2.\queen{}c4+ \king{}f3 3.Se5 \mate{} 
      1...\king{}xf4 2.\king{}f2 цугцванг. \king{}e4 3.\queen{}c4 \mate{} 
      1...\king{}e2 2.\queen{}d5 цугцванг. \king{}e1 3.\queen{}d2 \mate{}
                  
\item \textbf{ ``Mat Plus'', 2007 } 1. \rook{}h5! ~2. \queen{}c5+ \king{}xc5 3. \bishop{}e7\mate{}
  1. ... \bishop{}e5 2. \queen{}xe5+ \king{}xe5 3. \bishop{}e7\mate{}
  1. ... e5 2. \rook{}h6+ \king{}d5 3. e4\mate{}
  
\item \textbf{ ``ЮК А. Мельничук - 50'', 2008, 3ий Приз } 1. \rook{}b2? waiting
1... \king{}:c4 2. \king{}:d2 waiting \king{}d4 3. \rook{}b4\mate{}
1... \king{}e3 2. \rook{}b3+ \king{}d4 3. \queen{}d5\mate{}
1... \king{}c3! 

1. \rook{}a2? waiting
1... \king{}:c4 2. \rook{}a4+ \king{}c3/b3 3. \queen{}c2\mate{}
1... \king{}e3 2. \rook{}a3+ \king{}d4 3. \queen{}d5\mate{}
1... \king{}c3! 

1. \queen{}g5! waiting
1... \king{}e4 2. \rook{}c3 waiting \king{}d4 3. \queen{}e3\mate{}
1... \king{}d3 2. \queen{}h4 waiting \king{}e3 3. \rook{}c3\mate{}

\item \textbf{ ``Рыбинск 7 дней'', 2009, Похвальный Отзыв на равных. } Задача на тему аннигиляции. Иллюзорная игра проходит, если черные отдают слона на поле f4 -- 1... \bishop{}f4 2.\rook{}xf4. В действительной игре белым приходится жертвовать ладью на этом поле: \textbf{ 1.\rook{}f4!} цугцванг. \textbf{ 1... \bishop{}xf4 2.\queen{}f1! \bishop{}~ 3.\queen{}f7\mate{}; 1... \bishop{}xc3 (c1) 2.\queen{}(x)c1; 1... \bishop{}e3 2.\queen{}e1}.
  
\item \textbf{ ``День Шахмат - 2010'', 2010, Специальный приз } Перемена игры в трехходовке, в форме миниатюры. Иллюзорная игра -- 1. ... e5 2. \knight{}d5 с угрозой 3. \queen{}c3\mate{}; 1. ... \king{}e3 2. \queen{}d1 цугцванг e5 3. \knight{}d5\mate{}. Решение: \textbf{ 1. \queen{}c1!} Цугцванг. \textbf{ 1. ... e5 2. \knight{}b5+ \king{}d3 3. \bishop{}c4\mate{};  1. ... \king{}d3 2. \knight{}xe6} цугцванг \textbf{\king{}e2 3. \knight{}f4\mate{}}

\item \textbf{ ``Mat Plus'', 2010} Задача посвящена Иванову Альберту Федотовичу. 1.\knight{}f7! 
  1...\king{}c2 2.Se5! \king{}xb3 3.\queen{}d1\mate{}
  1...\king{}e4 2.\queen{}f1! \king{}d5 3.\queen{}c4\mate{}

\item \textbf{ ``Mat Plus'', 2010} Задача посвящена Сэму Лойду 1.\queen{}f1!
  1...d5 2.\queen{}e2 ~ 3.\bishop{}f2(\bishop{}e5)\mate{}
  1...e5 2.\queen{}e2 (~) d5 3.\bishop{}f2(\bishop{}xe5)\mate{}
      2...e4 3.\queen{}d2\mate{}
      2...\king{}d5 3.\queen{}c4\mate{}
  1...\king{}e4 2.\king{}c3 ~ 3.\queen{}d3\mate{}
      2...\king{}e3 3.\queen{}d3\mate{}
      2...\king{}d5 3.\queen{}c4\mate{}
  1...\king{}e3 2.\king{}c3 \king{}e4 3.\queen{}d3\mate{}

\item \textbf{ ``http://chessstar.com'', 2010, 2ой Похвальный Отзыв } Try: 1.\bishop{}e8? \king{}b4!
1.\bishop{}a4 ZZ 
  1...\king{}b4 2.\queen{}b3+ \king{}a5 3.\queen{}b5\mate{}
  1...\king{}d5 2.\queen{}c3 
    2... e3 3.\queen{}d3\mate{} 
    2... c4 3.\queen{}e5\mate{}

\item \textbf{ ``http://chessstar.com'', 2010, 3ий Похвальный Отзыв } 1.\rook{}b7 ZZ 
  1...\king{}xd4 2.\rook{}b4+ \king{}c5 3.\queen{}c7\mate{}
  1...\king{}f5 2.\rook{}f7+ 
    2... \king{}e6 3.\queen{}g6\mate{} 
    2... \king{}e4 3.\rook{}f4\mate{}

\item \textbf{ ``http://chessstar.com'', 2011, 1ый Похвальный Отзыв} 1.\queen{}f1! – zz;
  1... d5 2.\rook{}d3 d6 3.\queen{}f3\mate{},
  1... \king{}d5 2.\queen{}f3+ \king{}e6 3.\rook{}e1\mate{},

\item \textbf{ ``ЮК А. Кожакиной - 20, Кудесник'', 2011, 2ой Почетный Отзыв } 1. \queen{}c8! waiting
1... b4 2. \queen{}c4+ \king{}:e5 3. \queen{}f4\mate{}
1... \king{}d3 2. \bishop{}b1+
  2... \king{}d4 3. \queen{}c5\mate{}
  2... \king{}d2 3. \queen{}c2\mate{}
1... \king{}:e5 2. \queen{}c5+
  2... \king{}e4 3. \bishop{}b1\mate{}
  2... \king{}f6 3. \queen{}g5\mate{}

\item \textbf{ ``Гравюра'', 2012, 5ый Похвальный Отзыв } 1. \queen{}a8! [2. \queen{}xc6 [3. \queen{}c3\mate{}]
  2. ... \king{}e5 3. \queen{}f6\mate{}]
1. ... cxd5 2. \queen{}xd5+
  2. ... \king{}e3 3. \queen{}d2\mate{}
  2. ... \king{}c3 3. \queen{}c5\mate{}
1. ... c5 2. \queen{}a1+
  2. ... \king{}c4 3. \queen{}a4\mate{}
   2. ... \king{}d4 3. \queen{}c3\mate{}
1. ... \king{}c5 2. \queen{}a3+
  2. ... \king{}b5 3. \bishop{}d3\mate{}
  2. ... \king{}c4 3. \queen{}b4\mate{}
  2. ... \king{}d4 3. \queen{}c3\mate{}

\item \textbf{ ``ЮК Алена Кожакина - 20'', 2012, 5ый Приз } 1.\knight{}a7 ! цугцванг.
    1...\king{}d6 2.\rook{}e8 ~ 3.\queen{}c6 \mate{} 
    1...h2 2.\bishop{}b6 ZZ \king{}d6 3.\queen{}d1 \mate{} 
    1...\king{}c4 2.\queen{}xh3 ZZ
            2...\king{}b4 3.\bishop{}b6 \mate{} 
            2...\king{}d5 3.\queen{}e6 \mate{}

\item \textbf{ ``TT-60, SuperProblem'', 2012, 4ый Похвальный Отзыв} 1.\rook{}h4! 
1...\king{}xc1 2.\rook{}b4 threat: 3.\knight{}a3\mate{} 
  2...cxb2 3.\rook{}c4\mate{} 
1...cxb2 2.\rook{}c4+ \king{}b3 3.\knight{}d2\mate{} 
1...\king{}b3 2.\king{}d3 threat: 3.\rook{}a3\mate{}

\item \textbf{ ``StrateGems'', 2013 2ой Почетный Отзыв } Близнецы-маятник. a) 1.\king{}b2! ZZ
  1. ... \king{}d3 2. \queen{}f3+
   2. ... \king{}d2 3. \bishop{}a5\mate{}
   2. ... \king{}c4 3. \queen{}b3\mate{}

b) После первого хода белых: 1. \king{}c1! ZZ
 1. ... \king{}d3 2. \queen{}d5+
  2. ... \king{}c3 3. \bishop{}a5\mate{}
  2. ... \king{}e2 3. \queen{}d1\mate{}

\item \textbf{ ``StrateGems'', 2013 5ый Почетный Отзыв } Задача составлена совместно с Мельничуком Александром.
1.\queen{}a6? (zz)
1...\bishop{}c3 2.\queen{}f1+ \king{}b2 3.\queen{}c1\mate{}
1...\bishop{}c1 2.\bishop{}xc1 d3 3.\queen{}f6\mate{}
(1...\bishop{}a3 2.\queen{}xa3)
1...d3!

1.\queen{}a7? (zz)
1...\bishop{}c1 2.Bxc1 d3 3.\queen{}g7\mate{}
1...d3 2.\queen{}g1+ \bishop{}b1 3.\queen{}xb1\mate{}, 2...\bishop{}c1 3.\bishop{}c3\mate{}
1...\bishop{}c3!

1.\queen{}a8! (zz), 
1...\bishop{}c1  2.\bishop{}xc1  d3  3.\queen{}h8\mate{}
1...\bishop{}c3  2.\queen{}h1+  \king{}b2  3.\queen{}c1\mate{}
1...\bishop{}a3  2.\queen{}xa3  d3  3.Bc3\mate{}
1...d3 2.\queen{}h1+ \bishop{}c1/\bishop{}b1 3.\bishop{}c3/\queen{}xb1\mate{}

\item \textbf{ ``StrateGems (миниатюры)'', 2013 2ой Похвальный Отзыв } 1.\knight{}d3! (zz)
  1...\king{}e4 2.\queen{}d6 ~ 3.\queen{}g6\mate{}
      2... \king{}f5 3.\queen{}g6\mate{}
  1...\king{}e3 2.c5 \king{}f3 3.\queen{}f4\mate{}
  1...c5 2.\queen{}d2 \king{}e4 3.\queen{}f4\mate{}

\item \textbf{ ``StrateGems (миниатюры)'', 2013 3ий Похвальный Отзыв } 1.\knight{}h4! - zz
  1...\king{}e6 2.\queen{}c6+ \king{}f7 3.\queen{}g6\mate{}  (model)
  1...e4 2.\queen{}c5+ \king{}e6 3.\queen{}f5\mate{}  (model)
  1...\king{}e4 2.\queen{}c4+ \king{}e3 3.\bishop{}c1\mate{}

\item \textbf{ ``StrateGems'', 2014, Специальный Почетный Отзыв } 1.\queen{}f2! ZZ
  1...\king{}xd3 2.\king{}c5 
  1...\king{}e5 2.\knight{}c4+
  1...\king{}c3 2.\knight{}c4
  1...exd3 2.\queen{}f4+

\item \textbf{ ``МК Руденко'', 2017, 10ый Похвальный Отзыв } 1.\rook{}b2! threat: 2.\bishop{}a6 threat: 3.\bishop{}c8 \mate{}
    1...\rook{}d1 2.\bishop{}xd1 threat: 3.\rook{}e2 \mate{}
    1...\rook{}f1 2.\bishop{}xf1 threat: 3.\rook{}e2 \mate{}
    1...hg4 2.\rook{}f4 threat: 3.\bishop{}xg4 \mate{}
            2...\rook{}xe2 3.\rook{}xe2 \mate{}
    1...c5 2.\rook{}b7 threat: 3.\rook{}e7 \mate{}

\item \textbf{ ``Задачи и этюды'', 2010, Приз } 1. \queen{}g8! 
1... c2 2. \queen{}f7+ (~)
  2... \king{}g5 3. \bishop{}f4+ \king{}h4 4. \queen{}h7\mate{}
  2... \king{}e4 3. \queen{}f3+ \king{}d4 4. \queen{}d3\mate{}
1... \king{}f6 2. e4 waiting c2 3. e5+ \king{}f5 4. \queen{}g4\mate{}
1... \king{}e4 2. \queen{}g5 [3. \queen{}e5\mate{}]
  2... \king{}d4 3. \queen{}c5+ \king{}e4 4. \queen{}e5\mate{}

\item \textbf{ ``JT Mohamed Moubarak Ryan 55 (миниатюры)'', 2011, Почетный Отзыв } 1.\queen{}f3! {- цугцванг}
   1...\king{}e6 2.\knight{}b6 \king{}e5 3.\knight{}c4+ \king{}e6 4.\knight{}c5 \mate{}
   1... e6 2.\queen{}f2 \king{}e4 3.\knight{}d6 +
                   3...\king{}d3 4.\queen{}d2 \mate{}
                   3...\king{}d5 4.\queen{}c5 \mate{}
                   3...\king{}e5 4.\queen{}d4 \mate{}
           2...\king{}d5 3.\queen{}d4 + \king{}c6 4.\knight{}a7 \mate{}

\item \textbf{ ``ЮК Алена Кожакина - 20'', 2012, Похвальный Отзыв } 1.\queen{}e7 ! threat: 2.\bishop{}e4 + \king{}b5 3.\king{}b2 ZZ
                    3...\king{}a4 4.\bishop{}c6 \mate{} 
                    3...\king{}b6 4.\queen{}b7 \mate{} 
                    3...\king{}a6 4.\queen{}b7 \mate{} 
                    3...\king{}c4 4.\queen{}c5 \mate{} 
            2...\king{}b6 3.\queen{}b7 \mate{} 
    1...\king{}d5 2.\bishop{}b5 ZZ \king{}d4 3.\queen{}e6 ZZ \king{}c3 4.\queen{}c4 \mate{}

\item \textbf{ ``ЮК А.Кожакина-20'', 2012, 5ый Почетный Отзыв } 1.\queen{}e8! 
1...\king{}d6 2.Sa5 \king{}c7 3.\queen{}c6+ 
    3...\king{}d8 4.\bishop{}f6\mate{} 
    3...\king{}b8 4.\queen{}b7\mate{} 
  2...\king{}c5 3.\queen{}b8 \king{}d5 4.\queen{}e5\mate{} 
1...\king{}c4 2.\queen{}e5 \king{}b4 3.\king{}c2 
    3...\king{}a4/\king{}a3 4.\queen{}a5\mate{} 
    3...\king{}c4 4.\queen{}c5\mate{} 
  2...\king{}xb3 3.\queen{}b5+ \king{}a3 4.\bishop{}b2\mate{} 
  2...\king{}d3 3.Sd4 threat: 4.\queen{}e2\mate{} 
    3...\king{}c4 4.\queen{}b5\mate{}

\item \textbf{ ``TT-35, SuperProblem'', 2012, 1ый Похвальный Отзыв } 1.\king{}c3! (2.\queen{}f7 d5 3.\queen{}f2 d6 4.\king{}b3 \king{}d3 5.\queen{}e3\mate{})
  1... \king{}d5 2.\queen{}f3+ \king{}e6 3.\bishop{}g7 \king{}e7 4.\queen{}f6+ \king{}e8 5.\queen{}f8\mate{}

\item \textbf{ ``ТТ Ясиноватая 21'', 2012, 2ый Почетный Отзыв } Задача составлена совместно с Туревским Дмитрием.
1.\knight{}e6-d8 e7*d8=\bishop{}   2.Bc7-e5 \bishop{}d8-b6 \mate{}
1.\bishop{}c7-d8 e7*d8=\knight{}   2.\knight{}e6-c5 \knight{}d8-c6 \mate{}

\item \textbf{ ``Гравюра'', 2016, 5ый Похвальный Отзыв } 1.Sf4-h5 \rook{}f1-f4 +   2.\king{}h4-g5 \rook{}f8*f5 \mate{} {display-departure-rank}
1.Sf5-h6 \rook{}f8-f5   2.\king{}h4-g4 \rook{}f1*f4 \mate{} {display-departure-rank}

\item \textbf{ ``ТТ Ясиноватая-21'', 2012, Специальный Почетный Отзыв } 1.\bishop{}c5-e7 f6*e7   2.\rook{}d7-d3 e7-e8=\knight{}  3.\rook{}d8-d4 \knight{}e8-f6 \mate{} 
1.\rook{}d7-e7 f6*e7   2.\rook{}d8-d3 e7-e8=\bishop{}   3.\bishop{}c5-d4 \bishop{}e8-c6 \mate{}

\item \textbf{ ``http://popovgl.narod.ru'', 2009, 5ый Приз } 1.\king{}b3-c4 \king{}h8-g7 2.\queen{}a3-d3 \queen{}a1-a5 3.\king{}c4-d4 \king{}g7-f6 4.\queen{}c1-c4 \queen{}a5-e5 \mate{}

\item \textbf{ ``Mat Plus'', 2010 } I) 1.\knight{}b4 \bishop{}d2+ 2.\king{}b2 \rook{}xf1 3.\knight{}a2 \bishop{}e1 4.\king{}a1 \bishop{}c3\mate{}
II) 1.\knight{}xa3 \bishop{}e3 2.Sb1 \rook{}a8 3.\king{}b2 \bishop{}a7 4.\king{}a1 \bishop{}d4\mate{}

\item \textbf{ ``JT Юрий Белоконь-60'', 2011, Похвальный Отзыв }  Задача составлена совместно с Жилко Дмитрием. 1.\knight{}d7 c4 2.\bishop{}c7 c5 3.\rook{}b3+ \king{}c4 4.\rook{}b7 d5\mate{} 
1.\king{}d5 \king{}b3 2.\knight{}c6 \king{}c2 3.\rook{}e6 \king{}d3 4.\bishop{}d6 c4\mate{}

\item \textbf{ ``8° ECSC, \king{}iev'', 2012, 2ой Почетный Отзыв } 1... \rook{}b1\mate{}
1)  1.\king{}d1 \king{}b1 2.\king{}e2 \king{}:c2 3.\king{}e1 \king{}d3 4.\king{}d1 \rook{}b1\mate{}
2)  1.\rook{}c3 \rook{}b1+ 2.\king{}c2 \rook{}e1 3.\rook{}d3 \rook{}e4 4.\king{}c3 \rook{}c4\mate{}

\item \textbf{ ``TT-32, SuperProblem'', 2012, 7 Место} 1.\knight{}b7 O-O-O 2.O-O-O \rook{}f1 3.\king{}b8 \rook{}f6 4.\king{}a8 \rook{}xb6 5.\rook{}b8 \rook{}a6\mate{}

\item \textbf{ ``Проблеміст України'', 2011 } 1.b7-b6 b3-b4   2.\king{}d3-c4 b4-b5 3.\king{}c4-b4 \king{}a1-a2   4.\king{}b4-a5 \king{}a2-b3 5.d2-d1=\bishop{}+ \king{}b3-c4   6.\bishop{}d1-a4 b2-b4\mate{}

\item \textbf{ ``Mat Plus'', 2010 } 1.\king{}e4-f3 \king{}d1-c1   2.\king{}f3-e2 \king{}c1-b1   3.\king{}e2*d2 \king{}b1-a1   4.\king{}d2-c2 \king{}a1-a2   5.d3-d2 \king{}a2-a1   6.d2-d1=\queen{} + \king{}a1-a2   7.\queen{}d1-d8 \king{}a2-a1   8.\king{}c2*b3 \king{}a1-b1   9.\king{}b3-a4 \king{}b1-a2  10.\queen{}d8-a5 b2-b3 \mate{}

\item \textbf{ ``TT-59, SuperProblem'', 2012, 2ой Похвальный Отзыв } 1.\rook{}h2! \king{}xg3 2.\king{}g1 \king{}f3 3.\rook{}g2 g3 4.\king{}f1 g4 5.\rook{}f2+ gxf2 6.\queen{}e5 g3 7.\queen{}d4 g2\mate{}

\item \textbf{ ``ЮК Акобия-70'', 2007, 6ой-9ый Почетный Отзыв } 1. \queen{}e5 
(1. \king{}g1 ? \rook{}f1+ 2. \king{}xf1 \queen{}xf5+ !9) 
1... \queen{}b8 !
(1... \queen{}xb7 2. \queen{}xe3+ \king{}g4 3. \king{}h2 ! !8) 
2. \queen{}xb8 \king{}h3 3.\king{}g1 
(3. \queen{}xf4 !? {stalemate}) 
3... \rook{}g4+ 4. \king{}f1 \rook{}f4+ 5.\king{}e1 \rook{}g4 6. \queen{}f4 ! 
(6. \queen{}g3+ !? \rook{}xg3 7. \king{}f1 \king{}h2 8. b8=\queen{} \king{}h1 9. \queen{}xg3 { stalemate}) 
6... \rook{}xf4 7. b8=\queen{} \rook{}g4 8. \king{}f1 \rook{}f4+ 9. \king{}g1 \rook{}g4+ 10. \king{}h1 \rook{}g2 
({Main} 10... \rook{}f4 11. \queen{}b1 !8 {possible because wP is not on b7}) 
11. f6 \rook{}xe2 12. \queen{}c8+ \king{}g3 13. \queen{}c7+ \king{}h3 14. \queen{}d7+ \king{}g3 15. \queen{}d6+ \king{}h3 6. \queen{}e6+ \king{}g3 17. \queen{}e5+ \king{}h3 18. \queen{}f5+ \king{}g3 19. \king{}g1 {(\queen{}f1)+- The good play with logical elements, but there are many technical pawns... In spite of this, the young author made some success.} 1-0

\item \textbf{ ``JT Dvoretsky-60'', 2007, 1ый Похвальный Отзыв } 1. \rook{}b5 ! (1. \rook{}xd6 ? e1=\queen{} 2. \rook{}d8+ \queen{}e8+ 3. \rook{}xe8+ \king{}xe8 4. \knight{}xg3 \king{}d7 5. \knight{}e2 g4 6. \king{}f5 g3 7. \king{}e4 g2 8. \king{}d5 \king{}c7 9. \king{}c5 \king{}b7 10. \king{}b5 !1) 1... e1=\queen{} 2. \rook{}b8+ \queen{}e8+ 3. \rook{}xe8+ \king{}xe8 4. a6 (4. \knight{}xg3 \king{}d7 5. \knight{}e2 g4 6. \king{}f5 g3 7. \king{}e4 g2 8. \king{}d5 \king{}c7 !1) 4... g2 5. a7 g1=\queen{} 6. a8=\queen{}+ \king{}d7 7. \queen{}b7+ \king{}e6 8. \knight{}d4+ \queen{}xd4 9. \queen{}f7+ \king{}e5 10. \queen{}f5\mate{} 1-0

\item \textbf{ ``Mat Plus'', 2009, 4ый Приз } 1.\knight{}f3 d4+! 2.\knight{}xd4 e5 3.fxe5 \rook{}e4+ 4.\king{}d3 \rook{}xe5 5.\king{}c4 \rook{}e3
  5...\king{}a3 6.\knight{}b5+ +−
  5...\king{}a5 6.\knight{}c6+ +−
6.\rook{}h6 \rook{}e5
  6...\king{}a3 7.\knight{}c2+ +-
  6...\king{}a5 7.\knight{}b3+ +−
7.\rook{}a6+ \rook{}a5 8.\rook{}b6! \rook{}a8 
  8...\king{}a3 9.\rook{}b4/\knight{}b5+ +−
9.\rook{}b4+ \king{}a3
  9...\king{}a5 10.\knight{}b3/\knight{}c6+ +−
10.\knight{}b5+ \king{}a2 11.\knight{}c3+ \king{}a1 12. \rook{}b1\mate{}

\item \textbf{ ``Tel Aviv - 100'', 2009, 1ый Приз } Этюд, составленный совместно с Сергеем Дидухом.
1. \bishop{}c7+ \king{}h4 2. g7! 
    (2. exd3? \king{}xg5 3. g7 \king{}xh6 4. \bishop{}e5 h2 5. \bishop{}xh2 g3)
    (2. h7? \rook{}xe2+ 3. \king{}f1 \rook{}ae8 4. \rook{}a1 \rook{}f8+) 
2... \rook{}e8! 
    (2... \rook{}xe2+ 3. \king{}f1 h2 
        (3... \rook{}ae8 4. \rook{}a1 h2 
             (4... g3 5. \rook{}a4+) 
        5. \bishop{}xh2 g3 6. a8=\queen{}!) 
    4. \bishop{}xh2 \rook{}c8 5. \rook{}a1 \rook{}xh2 6. h7 \king{}g3 7. h8=\queen{} \rook{}cxh8 8. gxh8=\queen{} \rook{}xh8 9. \king{}g1)
    (2... \rook{}xa7 3. \bishop{}f4! \rook{}xe2+ 4. \king{}d1 \rook{}e8 5. \king{}c1) 
3. e3 h2 4. \bishop{}xh2 \rook{}c8! 5. \rook{}a1 \rook{}xh2 
    (5... \rook{}e2+!? 6. \king{}f1 \rook{}xh2 7. h7 \king{}g3 8. h8=\rook{} ! \rook{}f8+! 9. gxf8=\bishop{}! 
        (9. gxf8=\knight{}!) 
        (9. \king{}g1? \rook{}g2+ 10. \king{}h1 \rook{}h2+ 11. \rook{}xh2 \rook{}f1+ 12. \rook{}xf1)) 
6. h7 \king{}g3 7. h8=\rook{}! 
    (7. h8=\queen{} ? \rook{}c1+ 8. \rook{}xc1 \rook{}h1+ 9. \queen{}xh1) 
7... \rook{}cxh8 8. gxh8=\rook{}! \rook{}xh8 9. a8=\bishop{}! \rook{}xa8 10. a7 \rook{}h8 
    (10... \king{}h2 11. g6 g3 12. g7 g2 13. \king{}f2) 
11. a8=\bishop{}! 1-0

\item \textbf{ ``Kopnin 90 MT'', 2009, 1ый Почетный Отзыв } Этюд, составленный совместно с Сергеем Дидухом.
 1. \rook{}a5! 
    (1. \rook{}c5? \rook{}xf3+ 2. \knight{}f4 \rook{}xf4+ 3. \king{}e6 \rook{}e4+ 4. \king{}f7 \rook{}e7+ 5. \king{}xe7 e1=\queen{}+ 6. \king{}f7 \queen{}e8+ 7. \king{}xe8 f1=\queen{} 8. \king{}f7 \queen{}f5 9. \rook{}xf5 {stalemate})
1... \rook{}xf3+ 2. \knight{}f4 \rook{}xf4+ 3. \king{}e6 \rook{}e4+ 4. \king{}f7 \rook{}e7+ 5. \king{}xe7 e1=\queen{}+ 6. \king{}f7 \queen{}e8+ 7. \king{}xe8 f1=\queen{} 8. \king{}f7 \queen{}h1 9. \rook{}b5 \queen{}h2 10. \rook{}c5 \queen{}h3 11. \rook{}d5 \queen{}d3 12. \rook{}h5\mate{}

\item \textbf{ ``FIDE Olympic Tourney'', 2012, Похвальный Отзыв } 1.d7! \queen{}a8 
  (1... \queen{}e7+ 2.\rook{}xe7 \king{}xe7 3.Bh3 +-) 
2.\bishop{}b5 \king{}f7 3.\rook{}e8 \queen{}d5! 4.\bishop{}c4! 
  (4.\king{}h6? \queen{}xb5 5.d8\queen{} \queen{}xe8 =)
4... \queen{}c4 5.\rook{}f8+! 
  (5.d8\queen{}? \queen{}c2+ -+) 
5... \king{}xf8 6.d8\queen{}+ \king{}f7 7.\queen{}g8+ +-

\item \textbf{ ``Л.Лошинский-Е.Умнов-100 MT'', 2012, Похвальный Отзыв } 1. c7 \queen{}a6 
  (1... \queen{}e6 2. \queen{}f1+ \king{}e8 3. \queen{}b5+ \king{}f7 4. \queen{}xh5+ +-) 
2. \queen{}f4+ \king{}e8 3. \queen{}f1! \queen{}c6 
  ({main} 3... \queen{}xf1 4. c8=\queen{}+ \king{}f7 5. \queen{}g8+ \king{}f6 6. \queen{}f8+ +-) 
  (3... \queen{}c8 4. \queen{}b5+ \king{}f7 5. \queen{}xh5+ \king{}f6 6. \queen{}xh4+ \king{}e6 7. \queen{}g4+) 
4. \queen{}b5 \queen{}xb5 5. c8=\queen{}+ \king{}f7 6. \queen{}g8+ \king{}f6 7. \queen{}g6+ \king{}e5 8. \queen{}xh5+ +-

\item \textbf{ ``9° WCCT'', 2012 - 2013, 50-62 Место } 1. \rook{}a7 
  (1. \rook{}a5 c5 2. \rook{}xc5 \knight{}d7 3. \rook{}a5 \queen{}c8 =) 
1... \knight{}d7 
  (1... \king{}d5 2. \queen{}xf7+ +-)
2. \rook{}xd7! \king{}xd7 
  (2... \queen{}xd7 3. \queen{}f6+ \king{}d5 4. \queen{}xh4 +-) 
3. \queen{}f5+ \king{}e8 4. e6! fxe6 
  (4... \queen{}a8+ 5. \bishop{}a3 fxe6 6. \queen{}xe6+ \king{}f8 7. d7+ +-) 
5. \queen{}xe6+ \king{}f8 6. \queen{}c8!! {thematic} \queen{}xc8 7. d7+ +- 1-0

\item \textbf{ ``9° WCCT'', 2012 - 2013, 50-62 Место } 1. \king{}h7! \king{}f8+ 2. d7 
  (2. \king{}g6? \queen{}g7+ 3. \king{}xf5 \queen{}g5+ -+) 
2... \queen{}xd5 
  (2... \king{}e7 3.\queen{}c6! \queen{}xd7 4. \queen{}xd7+ \king{}xd7 5. \king{}g6 +-) 
3. \queen{}b4+ 
  (3. \queen{}a3+ \king{}f7 =) 
3... \king{}f7 4. \queen{}b7!! {thematic} \queen{}xb7 5. d8=\knight{}+! \king{}e7 6. \knight{}xb7 \king{}e6 7. \king{}g6 \king{}e5 
  (7... f4 8. \knight{}d6 (8.\knight{}a5)) 
8. \knight{}d6! f4 9. \knight{}c4+ \king{}d4 10. \king{}xf6 1-0

\item \textbf{ ``ЮК Разуменко-70'', 2006, Почетный Отзыв } 1. \knight{}a6! 
  (1. \knight{}c6? \bishop{}xc6 2. \rook{}xc6 g3 3. \rook{}f1 \rook{}h5+ 4. \king{}g1 \rook{}dh8 5. \rook{}f8+! \rook{}xf8 6.
\rook{}a6+ \king{}b7 7. \rook{}b6+ \king{}c7 (7... \king{}xb6?) 8. \rook{}c6+ \king{}d7 9. \rook{}d6+ \king{}e7 10. \rook{}e6+ \king{}f7 11.\rook{}f6+ \king{}xf6) 
1... \rook{}h5+ 2. \king{}g1 \rook{}d1+ 3. \rook{}f1 \rook{}xf1+ 4. \king{}xf1 \bishop{}b5+ 5. \king{}f2 \bishop{}xa6 6. \king{}g3 
 (6.\rook{}a2 \king{}b7 7.\king{}g3 minor dual)
\rook{}g5 7. \rook{}a2 \king{}b7 8. \rook{}a4 \bishop{}e2 9. \rook{}e4 \bishop{}d1 10. \rook{}d4 \bishop{}e2 11. \rook{}e4 1/2-1/2

\item \textbf{ ``Mat Plus'', 2007, Похвальный Отзыв } 1.\rook{}xc2 (1.\rook{}xe5 !? \rook{}xe5 (1...f2 ? 2.\rook{}xe6+ \king{}b7 3.\rook{}xc2 f1=\queen{}+ 4.\king{}h2 g3+ 5.\king{}xg3 Qd3+ 6.\king{}f4 {(\king{}h2, h4)=}) 2.\rook{}xc2 \rook{}e1+ 3.\king{}h2 \rook{}e2 -+) (1.gxf3 !? \rook{}xc5 -+) 1...f2 (1...\rook{}e1+ 2.\king{}h2 \rook{}1e2 3.\king{}g3 fxg2 4.\rook{}2c6+ \king{}a5 5.\rook{}c1 =) 2.\rook{}xf2 g3 3.\rook{}f1 \rook{}h5+ 4.\king{}g1 \rook{}eh6 5.\rook{}f6+! \rook{}xf6 6.\rook{}a8+ \king{}b7 7.\rook{}a7+ \king{}b8 (7...\king{}b6 8.\rook{}a6+ =) 8.\rook{}b7+ 1/2-1/2

\item \textbf{ ``Olimpiya dunyasi'', 2009, 2ой Почетный Отзыв } Этюд, составленный совместно с Сергеем Дидухом.
1. \knight{}c4+! 
  (1. \knight{}f7+?! \king{}e7 2. \bishop{}xg2 \king{}xf7 3. \bishop{}h3 \knight{}e7 4. \king{}f3 \knight{}g6 5. h5 Bd1+ 6. \king{}e4 \bishop{}xh5) 
1... \king{}c5 2. \bishop{}xg2 \king{}xc4 3. \bishop{}h3! \knight{}e7 4. \king{}f3 \knight{}g6 5. h5 \bishop{}d1+ 6. \king{}e4 \bishop{}xh5 7. \bishop{}g4! 
  (7. \bishop{}e6+?! \king{}c3! 8. \bishop{}f7 \king{}d2! 9. \bishop{}xg6 f3! 10. \bishop{}f5 \king{}e2 11. \bishop{}h3 \king{}f2 12. \king{}f4 \king{}g1 13. \king{}g3 f2 -+) 
7... \bishop{}xg4 =

\item \textbf{ ``JT Valerii Kirillov - 65'', 2016, Почетный Отзыв } Этюд был опубликован в редакции Сергея Дидуха. Логический этюд, украшенный попыткой 1.g7? \king{}f7 2.\rook{}h8 \king{}xg7 3.\rook{}xh1 \bishop{}a3 4.\bishop{}c3+ \king{}f7 5.\king{}d3 c1=\queen{} 6.\rook{}xc1 \bishop{}xc1 7.\king{}c2 \bishop{}a3 8.\king{}b3 \bishop{}e7 и черные выигрывают. Совсем слабо сразу забрать черного коня: 1.\rook{}xh1? \bishop{}b2 -+. Белые должны вынудить черную ладью занять неудобную позицию. 

\textbf{1.\rook{}h6! \rook{}c5} Иначе белые с выгодой продвигают свою пешку. Тематическая попытка 2.\rook{}xh1 \bishop{}b2 3.\bishop{}c3 c1=\queen 4.\rook{}xc1 \bishop{}xc1 -+. \textbf{ 2.g7+ \king{}f7 3.\rook{}h8! \king{}xg7 4.\rook{}xh1 \bishop{}a3  5.\bishop{}c3+ \king{}f7 6.\king{}d3 c1=\queen{} 7.\rook{}xc1 \bishop{}xc1 8.\king{}c2} с интересной позицией вечного преследования \textbf{ \bishop{}a3 9.\king{}b3 \bishop{}c1 10.\king{}c2 } - ничья. Дополнительный вариант:  4... \bishop{}b2 5.\bishop{}c3+ \bishop{}xc3 6.dc3 \rook{}xc3+ 7.\king{}d2 =

\end{enumerate}