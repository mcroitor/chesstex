\chapter{ОППОЗИЦИЯ}
\section{Теория}

Умение пешки превращаться в любую фигуру своего цвета (за исключением короля) является серьёзным фактором, зачастую переломляющим ход борьбы. Например, на следующей диаграмме, при всём материальном преимуществе черных, выиграть они не могут, так как ни слон, ни два коня мат поставить не могут.

\begin{center}
\chessboard[\diagramsize,setfen=8/8/8/8/2n5/2k1n3/8/2K5,showmover=false]

\textbf{Д. 1.1. ничья}
\end{center}

Однако преимущество в пешку может оказаться решающим. Это связано с замечательным свойством пешки, которая может превратиться в любую фигуру того же цвета (за исключением короля) по достижении последней (первой, в случае черных фигур) горизонтали и принести этим победу.

\begin{center}
\chessboard[\diagramsize,setfen=4k3/8/3KP3/8/8/8/8/8,showmover=false]

\textbf{Д. 1.2. ход белых}
\end{center}

На \textbf{диаграмме 1.2} белые легко побеждают, проведя пешку в ферзи: \textbf{1.e6-e7 \king{}e8-f7 2.\king{}d6-d7} и \textbf{3.e7-e8=\queen{}} со скорым матом. Как видно, иногда лишняя пешка означает больше, чем два коня. Однако не  всегда лишняя пешка помогает одержать победу. Так, при ходе черных на \textbf{диаграмме 1.2} они добиваются ничьей: \textbf{1... \king{}e8-d8 2.e6-e7+ \king{}d8-e8 3.\king{}d6-e6} пат. Даже при других ходах белые не смогли бы усилить свою позицию. Возникает вопрос - в каких случаях сильнейшая сторона выигрывает и когда закономерен ничейный результат? На это помогает ответить понятие \emph{оппозиции}.

\emph{\textbf{Оппозицией} называется противостояние королей на одной линии или диагонали (диагональная оппозиция), при расстоянии королей в одно поле (ближняя оппозиция), три, пять или семь полей (дальняя оппозиция).}

\begin{center}
\chessboard[
	\diagramsize,
	pgfstyle=cross,
	color=blue,
	markfields={c6,d6,e6},
	setfen=8/3k4/8/3K4/3P4/8/8/8,
	showmover=false]

\textbf{Д. 1.3. оппозиция}
\end{center}

Сторона, занимающая оппозицию, обычно получает преимущество (см. \textbf{диаграмму 1.3}), так как оппозиция, во-первых, препятствует обходу короля противника, а во-вторых, позволяет прорваться собственному королю. Так и в случае на \textbf{диаграмме 1.2}. Однако следующий пример показывает, что одного владения оппозицией для победы недостаточно (\textbf{диаграмма 1.4}).

\begin{center}
\chessboard[\diagramsize,setfen=8/8/2k5/8/4P3/3K4/8/8,showmover=false]

\textbf{Д. 1.4. ход белых}
\end{center}

При своём ходе белые занимают оппозицию: \textbf{1.\king{}d3-с4} , однако, после ходов \textbf{1 \king{}c6-d6 2.\king{}c4-d4 \king{}d6-e6} белому королю, для удержания оппозиции, мешает собственная пешка. После \textbf{3.e4-e5 \king{}e6-e7 4.\king{}d4-d5 \king{}e7-d7} оппозицией владеют уже черные. 

\begin{center}
\chessboard[\diagramsize,setfen=8/8/2k5/8/8/3K4/4P3/8,showmover=false]

\textbf{Д. 1.5. Чем закончиться партия?}
\end{center}

При своём ходе белые занимают оппозицию и побеждают, так как им не мешает собственная пешка. Партия может продолжаться следующим образом: 
\textbf{1.\king{}d3-e4}

диагональная оппозиция! Возможно также и \emph{1.\king{}d3-c4}, и даже \emph{1.\king{}d3-d4} -- почему?

\textbf{1 \king{}c6-d6 2.\king{}e4-f5 \king{}d6-e7 3.\king{}f5-e5!}
Опять занимая оппозицию, белые оттесняют короля соперника.
\textbf{3 \king{}e7-f7 4.\king{}e5-d6 \king{}f7-e8}
А вот черные завладеть оппозицией не могут, так как у белых есть пешка, которая всегда может передать ход: \emph{4... \king{}f7-f8 5.d2-d4}

\begin{center}
\chessboard[\diagramsize,setfen=4k3/8/3K4/8/8/8/4P3/8,showmover=false]

\textbf{Д. 1.6. Ход белых}
\end{center}

В получившейся позиции у белых есть столько запасных ходов, что они побеждают независимо от очереди хода.

\textbf{4.e2-e3!?}
Этот хитрый ход сделан с таким расчетом, чтобы пешка оказалась на поле е6 при черном короле на d8. В случае \emph{4.e2-e4 \king{}e8-d8 5.e4-e5 \king{}d8-e8 6.e5-e6??} получалась бы знакомая нам по \textbf{Диаграмме 1.2} позиция, в которой, при своём ходе, чёрные занимают оппозицию и добиваются ничьей.
\textbf{4 \king{}e8-d8 5.e3-e4 \king{}d8-e8 6.e4-e5 \king{}e8-d8 7.e5-e6 \king{}d8-e8 8.e6-e7}
с победой.

Можно сформулировать следующее правило:

\emph{В окончании король с пешкой против короля сильнейшая сторона побеждает, если король находится впереди пешки и владеет оппозицией (или есть запасные ходы пешкой для занятия оппозиции).}

Однако и из этого правила есть исключения, например, при ладейной пешке (см. \textbf{диаграмму 7}).

\begin{center}
\chessboard[\diagramsize,setfen=1k6/8/1K6/8/8/8/P7/8,showmover=false]

\textbf{Д. 1.7. Ход белых}
\end{center}

В данной ситуации белые не могут одержать победу, потому что черный король будет всё время передвигаться по полям а8 и b8, а белые не могут его оттуда оттеснить: \textbf{1.a2-a4 \king{}b8-a8 2.a4-a5 \king{}a8-b8 3.a5-a6 \king{}b8-a8 4.a6-a7} - и чёрный король не имеет ходов, пат.

\vfill
\pagebreak

\section{Позиции для анализа}

\begin{tabular}{ c c }
\chessboard[
\diagramsize,
setfen=8/2k5/4P3/3K4/8/8/8/8,
showmover=false]
 &
\chessboard[
\diagramsize,
setfen=8/5k2/5P2/5K2/8/8/8/8,
showmover=false]
 \\
\textbf{Задание 1.1: ход черных} & \textbf{Задание 1.2: ход белых}  \\
\chessboard[
\diagramsize,
setfen=k7/2K5/8/1P6/8/8/8/8,
showmover=false]
 &
\chessboard[
\diagramsize,
setfen=8/8/3k4/8/3K4/8/4P3/8,
showmover=false]
 \\
\textbf{Задание 1.3: ход белых} & \textbf{Задание 1.4: ход белых}  \\
\end{tabular}
