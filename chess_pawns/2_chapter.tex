\chapter{КВАДРАТ ПЕШКИ}
\section{Теория}

Часто пешечный эндшпиль сводится к решению задачи -- догонит или не догонит король пешку противника, рвущуюся в ферзи. На этот вопрос помогает ответить понятие \textbf{квадрата пешки} и связанное с ним правило:

\emph{\textbf{Квадратом пешки} называется воображаемый квадрат в сторону короля противника, образованный отрезками, равными по длине растоянию пешки до поля превращения.}

\emph{Если король при своём ходе попадает в квадрат пешки, то он пешку задерживает.}

Пример квадрата пешки изображен на диаграмме 1 (отмечен крестиками).

\begin{center}
\chessboard[
	pgfstyle=cross,
	color=blue,
	markfields={c6,c7,c8,d5,e5,f5,f6,f7,f8},
	setfen=8/8/8/2P3k1/8/8/8/6K1]

\textbf{Д. 2.1. Квадрат пешки}
\end{center}

При своём ходе чёрные входят в квадрат пешки, поэтому они догоняют пешку:

\textbf{1.\king{}g5-f5 2.c5-c6 \king{}f5-e6 3.c6-c7 \king{}e6-d7}

При своём ходе белые выигрывают, потому что черный король не попадает в квадрат пешки:

\textbf{1.c5-c6 \king{}g5-f6 2.c6-c7 \king{}f6-e7 3.c7-c8=\queen{}} с победой.

Однако, с пешкой в начальной позиции правило квадрата пешки не действует, потому что свой первый ход пешка может сделать через клетку.

\begin{center}
\chessboard[setfen=8/8/8/8/8/7K/1P6/3k4]

\textbf{Д. 2.2. ход черных}
\end{center}

Поэтому, не смотря на очередь хода, черные, в позиции, изображенной на диаграмме 2, отыграть пешку не могут: \textbf{1 \king{}d1-c2 2.b2-b4} и т.д.

\begin{center}
\chessboard[setfen=8/8/8/p3kPp1/Pp4P1/1K6/8/8]

\textbf{Д. 2.3. ничья независимо от очереди хода}
\end{center}

Правило квадрата пешки имеет большое значение в оценке пешечных окончаний. Например, позицию, изображенную на диаграмму 3, можно сразу определить как ничейную, независимо от очереди хода. Это можно утверждать с уверенностью, потому что у каждой из сторон есть защищённые проходные пешки. Короли же не могут выходить из квадратов пешек, иначе соперник проведёт пешку в ферзи. Ни одна из сторон не может усилить свою позицию.

В заключение хочется привести известный этюд Рети, в котором, на первый взгляд, опровергается правило квадрата пешки:

\begin{center}
\chessboard[setfen=7K/8/k1P5/7p/8/8/8/8]

\textbf{Д. 2.4. Рихард Рети, 1921, ничья}
\end{center}

Визуально видно, что черный король находится в квадрате пешки с6, белому же королю не хватает 3-х ходов для достижения черной пешки. Возникает вопрос, как же можно добиться ничьей? Здесь надо сказать об одном важном принципе шахмат, позволяющем добиваться результата, а именно - создавать множественные угрозы. Ничья в этюде Рети достижима в двух случаях: если король догонит пешку или если белые смогут провести свою пешку в ферзи. Поэтому, в решении, белые преследуют обе цели сразу:

\textbf{1.\king{}h8-g7! h5-h4 2.\king{}g7-f6! h4-h3 3.\king{}f6-e6! h3-h2 4.c6-c7 \king{}a6-b7 5.\king{}e6-d7 h2-h1=\queen{} 6.c7-c8=\queen{}+} с ничьей. А что бы произошло, если бы черные попробовали сначала забрать белую пешку? Тогда бы белые попадали в квадрат пешки, и партия закончилась бы ничьей:

\textbf{2. \king{}a6-b6 3.\king{}f6-e5!} (белым нужен еще ход, чтобы догнать пешку -- \emph{3 \king{}b6:c6 4.\king{}e5-f4} и ничья)

\textbf{3 h4-h3 4.\king{}e5-d6 h3-h2 5.c6-c7 \king{}b6-b7 6.\king{}d6-d7 h2-h1=\queen{} 7.c7-c8=\queen{}+} опять с ничьей.

\vfill
\pagebreak

\section{Позиции для анализа}
\begin{tabular}{ c c }
\chessboard[setfen=8/8/6k1/8/1P4K1/8/8/8]
 &
\chessboard[setfen=8/8/8/p7/3kPK2/8/8/8]
 \\
\textbf{Задание 2.1: ход черных} & \textbf{Задание 2.2: ход белых}  \\
\chessboard[setfen=8/8/8/8/8/P4p2/k6K/8]
 &
\chessboard[setfen=8/6p1/k1P2p1p/7K/8/8/8/8]
 \\
\textbf{Задание 2.3: ход белых} & \textbf{Задание 2.4: ход белых}  \\
\end{tabular}
