\chapter{ПЕШЕЧНЫЙ ПРОРЫВ}
\section{Теория}

Один из важнейших принципов реализации преимущества в пешечном эндшпиле -- создание проходной пешки.  Для начала несколько определений: 

Пешка называется \emph{проходной}, если она не может быть остановлена пешками противника.

Пешка называется \emph{кандидатом в проходные} если на её пути нет пешки противоположного цвета, но такие пешки есть на соседних вертикалях.

Пешка называется \emph{блокированной} (заблокированной) если у неё нет ходов (то есть перед ней стоит фигура).

На диаграмме 3.1 белая пешка е4 является проходной, черная пешка b5 является кандидатом в проходные.

\begin{center}
\chessboard[
	\diagramsize,
	pgfstyle=straightmove,
	color=blue,
	markmoves={e4-e8},
	setfen=8/8/8/ppk5/4P3/P2K4/8/8,
	showmover=false]

\textbf{Д. 3.1. проходная пешка}
\end{center}

Появление проходной пешки даёт следующие преимущества:
\begin{itemize}
\item проходная пешка может превратиться в фигуру, что позволяет получить материальное преимущество или поставить мат;
\item король противника вынужден отвлечься на проходную пешку, что позволяет прорваться к пешкам противника и уничтожить их.
\end{itemize}

Поэтому необходимо уметь создавать проходные пешки. Обычно проходная пешка может появиться при реализации пешечного большинства. О пешечном большинстве (или меньшинстве) говорят, когда у одной из сторон расположено больше (меньше) пешек напротив пешек противника. Например, в позиции на диаграмме 3.1 у черных на ферзевом фланге пешечное большинство -- пешки а5, b5 против пешки а3.

\begin{center}
\chessboard[
	\diagramsize,
	setfen=8/6pp/k7/8/K7/8/5PPP/8,
	showmover=false]

\textbf{Д. 3.2. ход белых}
\end{center}

\begin{center}
\chessboard[
	\diagramsize,
	setfen=8/6pp/8/4kPPP/8/8/6K1/8,
	showmover=false]

\textbf{Д. 3.3. ход белых}
\end{center}

Не смотря на то, что проходная пешка является серьёзной силой, следует помнить об уже описанном \emph{правиле квадрата}. На диаграмме 3.4 черный король находится в квадрате пешек. Однако, после последовательной жертвы белых пешек оставшиеся черные пешки мешают своему королю. \textbf{1. f5-f6! g7:f6 2.g5-g6! h7:g6 3.h5-h6! +-} и нет хода \emph{3... \king{}e5-f6}.

\begin{center}
\chessboard[
	\diagramsize,
	setfen=8/ppp5/8/PPP5/8/5k2/8/5K2,
	showmover=false]

\textbf{Д. 3.4. ход белых}
\end{center}

Классическая позиция на тему пешечного прорыва показана на диаграмме 3.5. 
\textbf{ 1.b5-b6! a7:b6 2.c5-c6! b7:c6 3.a5-a6 +-} или \textbf{ 1... c7:b6 2.a5-a6! b7:a6 3.c5-c6 +-}

\begin{center}
\chessboard[
	\diagramsize,
	setfen=8/4pp2/8/3k1PP1/1p4P1/pK6/8/8,
	showmover=false]

\textbf{Д. 3.5. ход белых}
\end{center}

\textbf{ 1.g5-g6! f7:g6 2.f5:g6! \king{}d5-e6 3.g4-g5! +-} и не смотря на то, что черный король в квадрате пешки, он не может к ней подойти. Нет выжидательного хода (цугцванг!), поэтому черные должны выйти королём из квадрата пешки и они проигрывают.

\vfill
\pagebreak
\section{Позиции для анализа}
