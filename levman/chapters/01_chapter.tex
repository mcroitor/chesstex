\chapter{Первый ход белых}

Двухходовая шахматная задача, как явствует из самого задания, состоит из двух ходов белых: это значит, что в начальном положении белые делают какой-то ход (вступительный), и как бы черные не защищались, белые своим вторым ходом дают мат черному королю. При таком ограниченном количестве ходов само собой понятно, что первый ход белых -- ключ к задаче -- играет огромную роль. В этой главе мы и займемся подробным рассмотрением первого хода белых.

Каждая задача предназначена для того, чтобы ее решали, -- поэтому автор задачи стремится выбрать такой первый ход, который не бросался бы в глаза решающему. Автор старается сделать первый ход как можно более трудным и скрытым, -- пусть решающие хорошенько поломают голову, прежде чем им удастся напасть на правильный путь.
Многие начинающие любители, пытаясь составить задачу, заставляют белых на 1-ом же ходу дать шах черному королю. Такой первый ход никуда не годится: во-первых, его легко найти, так как каждый решающий первым делом ищет, нельзя ли как-нибудь объявить черным шах, а, во-вторых, на шах у черных не может быть много защит, так что никакой более или менее интересной игры в задаче не получится. Поэтому обязательным условием каждой двухходовой (а также и многоходовой) задачи является тихий первый ход, то есть такой первый ход, которым черному королю не объявляется шаха.

Но этого мало: белые на первом ходу не только не должны шаховать, но не должны и бить какую-нибудь черную фигуру. Ведь если сразу же побить черную фигуру, то этим самым сильно ослабляется сопротивляемость черных. Исключения допускаются только в отношении черных пешек: ново-американская школа допускает в порядке исключения взятие черной пешки на первом ходу, -- но только в том случае, если иного хорошего хода нет и если тема задачи искупает такую вольность.

Первый ход должен быть \so{единственным}, то есть в задаче не может быть еще какого-нибудь другого первого хода, который также дает возможность заматовать черных на втором ходу. Иными словами, задача должна иметь только одно решение. Если к цели ведут два или больше первых хода, то говорят, что задача допускает побочное решение, и такая задача вообще отпадает.

Таковы три основных нормальных требования, которые предъявляются в настоящее время к любой шахматной задаче, а в частности и к двухходовке. Но есть и четвертое -- не столь важное, как первые три, но все же очень важное. Белые не должны на первом ходу превращать свою пешку в фигуру. Избегать этого нужно по той же причине, но какой нельзя первым ходом бить черную фигуру: не подобает белым \so{грубыми} средствами увеличивать свое преимущество перед противником. Правда, и в отношении этого требования (как и в отношении второго) делают иногда исключения, -- но исключения только подтверждают общее правило.

Кроме этих чисто внешних, формальных требований, к первому ходу предъявляются еще и идейные требования, то есть первый ход должен заключать в себе какую-нибудь идею, какой-то замысел, какую-то тонкость.

Нужно сказать, что в разное время на этот счет существовали различные представления. Композиторы чешской школы понимали идейность и красоту первого хода совсем не так, как, например, представители немецкой школе или современные проблемисты. Так, художественная школа считала, что наилучшим и наикрасивейшим является такой первый ход, которым жертвуется сильная белая фигура или отдается свободное поле черному королю, или же черным дается возможность объявить шах белому королю. Немецкая школа обращала преимущественно внимание на трудность и скрытость первого хода и на скрытые в нем стратегические моменты: скажем, если первым ходом освобождается или пересекается какая-нибудь важная линия или же если у белой фигуры, делающей первый ход, есть много полей отступления, но почему-то она должка встать именно на одно, точно установленное поле.

Ново-американская школа не придает первому ходу такого исключительного значения как художественная или немецкая, хотя и она предъявляет к нему большие требования. Она очень ценит и пожертвование фигуры на первом ходу, и предоставление черному королю свободных полей, и стратегические моменты немецкой школы. Она со своей стороны стремится сделать первый ход \so{тематичным}, то есть непосредственно связанным с самой темой задачи (например, если тема задачи -- развязывание белого коня, то очень хорошо, если в начальном положении этот конь еще свободен и лишь после первого хода белых оказывается связанным). Вообще же ново-американская школа считает, что в задаче первый и второй ходы совершенно равноправны и нельзя жертвовать темой или полнотой идеи во имя «красивого» первого хода.

Рассмотрим же по порядку, какие требования в смысле идейности и красоты предъявляем мы к первому ходу белых.

\textbf{Пожертвование фигуры.} У композиторов художественной школы тема эта в свое время была самой любимой. Действительно, пожертвование сильной фигуры на 1-ом ходу выглядит очень эффектно, и начинающие составители всегда первым делом берутся за задачи с таким вступлением. Рассмотрим же несколько примеров пожертвований и сравним, как они осуществляются в задачах разного направления.

Возьмем, для начала, пожертвование сильнейшей белой фигуры -- ферзя. В задаче № 1 это пожертвование проведено в типично чешском стиле. Задача построена на Zugzwang, -- это значит, что белые собственно еще ничем не грозят, но на любой ход черных возникает возможность мата. Таким образом, жертва ферзя не содержит какой-нибудь грубой угрозы, это тихая жертва ферзя, найти которую далеко не так просто. Само собой разумеется, что пожертвование ферзя приводит к правильному мату (1. ... \rook{}:b4 2.\knight{}gf3\mate). Хотя в задаче имеются еще два правильных мата (1. ... \rook{}:d2 2.f3\mate и 1 ... \rook{}е:f2 2.d4\mate), при связанной черной ладье, но без пожертвования ферзя на первом ходу эта изящная задача потеряла бы свое значение.

\begin{center}
 \begin{tabular}{ c c }
\textbf{№ 1. K. A. Л. Куббель} & \textbf{№ 2. А. Эллерман} \\
III пр. конк. <<Die Schachwelt>>, 1911 & I пр. <<G.С.>>, январь 1916 \\
\chessboard[
\diagramsize,
setfen=8/8/3Q4/5p2/5r2/1B4B1/3PrPpN/1K2k1N1,
label=false,
showmover=false]
& 
\chessboard[
\diagramsize,
setfen=B1N5/4p3/2n1r3/R1Bk4/4p2R/8/8/4K2Q,
label=false,
showmover=false] \\
\textbf{Mar в 2 хода (1.\queen{}b4).} & \textbf{Мат в 2 хода (1.\queen{}f3).}
\end{tabular}
\end{center}

Совсем по-иному проведена жертва ферзя в задаче № 2, совсем другое содержание вложено в эту задачу, характерную для новой школы. Она тоже построена на Zugzwang, но цель ее -- не правильные маты, а встречная стратегическая игра черных и белых фигур. У черного короля два свободных поля -- с4 и с5. Ha 1. ... \king{}с4 последует 2.\knight{}b6\mate, на 1. ... \king{}е5 2.\rook{}h5\mate. Центральные же варианты: 1. ... ef+ 2.\bishop{}е3!\mate (ладья h4 держит поле с4, а ладья а5--поле е5) и 1. ... \rook{}e5 2. \queen{}b3\mate (ладья сама загородила выход своему королю). В этой задаче жертва ферзя не самоцель, а лишь красивое вступление к дальнейшей игре, которая имеет ценность сама по себе.

Приведем два примера с пожертвованием белой ладьи.

Первый ход в задаче № 3 действительно чрезвычайно эффектен -- сплошной фейерверк! Посудите сами: одна ладья уже находится под боем, и тем не менее белые первым ходом жертвуют и вторую ладью и ферзя. Эта типичная чешская задача содержит, конечно, несколько правильных матов ( 1. ... \king{}:d6 2.\bishop{}f4\mate; 1. ... \king{}:f6 2.\queen{}:a1\mate; 1. ... \bishop{}:d6 2.\bishop{}с3\mate ), но все ее идейное содержание именно в пожертвовании фигур.

\begin{center}
 \begin{tabular}{ c c }
\textbf{№ 3. Л. Б. 3алкинд.} & \textbf{№ 4. К. Мэнсфилд} \\
II пр. конк. <<Die Schachwelt>>, 1912 & II приз, <<G.C.>>, декабрь 1916 \\
\chessboard[
\diagramsize,
setfen=1b2K3/8/3Rp1pp/2p1k3/6P1/6p1/3B4/r4R1Q,
label=false,
showmover=false]
& 
\chessboard[
\diagramsize,
setfen=4b3/K2p2nq/4PRrp/1NN1k3/6PB/1Q6/6pR/b6B,
label=false,
showmover=false] \\
\textbf{Maт в 2 хода (1.\rook{}f6).} & \textbf{Мат в 2 хода (1.\rook{}f1).}
\end{tabular}
\end{center}

Первый ход в задаче № 4 внешне не столь эффектен, -- жертвуется только одна ладья, -- но какое в ней богатство мысли, какая насыщенность содержанием! Ладья f6 уходит вниз по линии f, чтобы сделать возможным мат слоном на g3. Но почему ей надлежит пойти на f1, а не задержаться на f3 или f2? Оказывается, что на 1.\rook{}f3 черные ответят 1. ... g1\queen{}! а на 1.\rook{}f2 1. ... gh\knight{}!, и белые не могут дать мата. При ходе же 1.\rook{}f1 на 1. ... g1\queen{} или 1. ... gf\queen{} (\knight{}) последует 2.\queen{}d5\mate, а на 1. ... gh\knight{} 2.\rook{}е2\mate. В этой задаче пожертвование ладьи не является самоцелью, а лишь мотивировкой дальнейшей игры.

На этих примерах мы можем видеть, какая разница существует в подходе к пожертвованию со стороны композиторов различных направлений. Мы, сторонники новой школы, полагаем, что пожертвование фигуры на первом ходу, конечно, украшает задачу, но не может служить стержнем, темой, существом задачной комбинации. К задаче мы предъявляем гораздо больше требования; она должна быть содержательной не только в отношении первого хода белых, но также и в отношении ответа черных и заключительного хода белых.

\textbf{Свободные поля.} \so{Свободным} мы называем такое поле, на которое может вступить черный король, то есть, иными словами, такое поле около черного короля, которое не занято черной фигурой и на которое не действует никакая белая фигура. Чем больше таких свободных полей у короля, тем труднее дать ему мат на втором ходу и тем содержательнее задача. Поэтому хорошим первым ходом считается такой, который даёт королю одно или несколько свободных полей для отступления. Рассмотрим два примера: обе задачи (№№ 5 и 6) построены на Zugzvang, то есть на любой ход чёрных у белых есть уже готовый ответ. В задаче № 5 белые делают просто выжидательный ход, давая при этом королю свободное поле. Но комбинация эта слишком элементарна, чтобы она могла нас удовлетворить. Гораздо интереснее задача № 6, где белые не имеют простого выжидательного хода. После 1.\queen{}а2 они не только отдают королю поле b5, но и отказываются от мата 2.\queen{}:b7\mate в ответ на 1. ... b6-b5. Вместо этого они на 1. ... b5 сыграют 2.\knight{}с4\mate!, пользуясь тем, что чёрная пешка сама заняла свободное поле. Кроме того, на ход 1. ... \king{}а7 белые ответят уже не 2.\queen{}:b7\mate, а 2.\knight{}с6\mate. 

\begin{center}
 \begin{tabular}{ c c }
\textbf{№ 5. Г. Xескот.} & \textbf{№ 6. Д. O'Kиф.} \\
Конк. Герм. Шахм. Союза, 1910 & 2 поч. отз. <<G.С.>>, апрель 1918 \\
\chessboard[
\diagramsize,
setfen=8/3B1N2/3P2K1/3k4/1Rb5/3rp3/2QnN3/8,
label=false,
showmover=false]
& 
\chessboard[
\diagramsize,
setfen=8/1p6/kp6/N2Q4/8/3p4/3Bp3/4K3,
label=false,
showmover=false] \\
\textbf{Мат в 2 хода (1.\bishop{}е8).} & \textbf{Мат в 2 хода (1.\queen{}а2).}
\end{tabular}
\end{center}

Эти две небольшие задачи, в которых так различно представлен замысел (предоставление свободного поля королю при Zugzwang’e), дают нам возможность высказать наше отношение и к \so{свободным} полям, предоставляемым на первом ходу. Это предоставление не может быть самоцелью (за редкими исключениями), а должно углубить и украшать стратегическую тему задачи. Примером такого предоставления свободных полей могут служить задачи № 58 и № 94.

\textbf{Стратегические темы.} Посмотрим теперь, какие стратегические идеи может таить в себе первый ход белых. Остановимся прежде всего на \so{открытии линий}.

Возьмем простейший пример: белая фигура занимает линию, по которой дальнобойная черная фигура могла бы дать шах белому королю. И вот на первом ходу белые уводят эту фигуру, очищая линию для шаха черных.

\begin{center}
 \begin{tabular}{ c c }
\textbf{№ 7. А. Уайт.} & \textbf{№ 8. С. С. Левман.} \\
<<G.С.>>, март 1920. & <<Magyar Sakkvilag>>, 1927. \\
\chessboard[
\diagramsize,
setfen=3R3K/8/5Q2/8/5N2/1B6/2PB4/1qbk4,
label=false,
showmover=false]
& 
\chessboard[
\diagramsize,
setfen=5R1K/5N1b/rp4kN/8/7p/5B1p/1B4q1/5RQ1,
label=false,
showmover=false] \\
\textbf{Мат в 2 хода (1.\queen{}а1).} & \textbf{Мат в 2 хода (1.\rook{}а1).}
\end{tabular}
\end{center}

В задаче № 7 белый ферзь уводится с линии а1--h8, открывая возможность черному ферзю и слону попасть на эту диагональ и объявить шах белому королю. Но белый ферзь не покидает линии, а движется по ней до конечного пункта. На 1. ... \queen{}:а1+ последует 2.сЗ\mate. На 1. ... \queen{}b2 + 2.\bishop{}сЗ\mate и на 1. ... \bishop{}b2+ 2.сЗ\mate.

В задаче № 8 белые своим первым ходом уводят свою фигуру с той линии, по которой в будущем будет двигаться другая белая фигура, но не просто уводят, а заставляют ее проделать весь путь по этой линии -- до конечного пункта. Значение этого движения ладьи становится вполне ясным лишь после ответа черных 1. ... \queen{}gЗ!, на что последует 2.\queen{}b1\mate. Белые отводят ладью на край доски (нельзя 1.\rook{}f2 из-за 1. ... \queen{}:g1), а следом за ней движется белый ферзь, чтобы, встав вплотную около ладьи, объявить мат.

В этой задаче ладья играет, однако, известную роль сама по себе: она матует в двух вариантах (при 1. ... b5 2.\rook{}:а6\mate и 1. ... \queen{}:g1 2.\rook{}:g1\mate). Возможны, однако, такие положения, когда белая фигура, отводимая на край доски с целью открыть линию для другой, движущейся вслед белой фигуре, не играет сама никакой роли, даже не участвует ни в одном мате. Такое движение двух белых получило название \so{бристольской} темы. В двухходовках она встречается очень редко. В качество примера интересного осуществления этой идеи в двухходовке приведём задачу № 9.

\begin{center}
\begin{tabular} {c}
\textbf{№ 9. Ф. Симхович} \\
III пр. <<Leipziger Tageblatt>>, 1925. \\
\chessboard[
\diagramsize,
setfen=3Q3r/5rb1/3n1qB1/4NPN1/p4k2/4R1Rp/K3PnPB/8,
label=false,
showmover=false] \\
\textbf{Мат a 2 хода (1.\rook{}аЗ).}
\end{tabular}
\end{center}

В этой задаче ладья е3 на первом ходу уходит на самый угол доски, -- на первый взгляд это движение непонятно. Почему не встать ладьей на d3 или с3, -- ведь и в том и в другом случае создается угроза 2. \rook{}g4\mate -- с матом. Тонкий смысл первого хода становится ясен лишь при ответе 1. ... \queen{}с6+. В начальном положении задачи, когда ладья стоит на еЗ, белые на этот шах могли бы ответить просто 2.\knight{}:е6\mate, теперь же это невозможно, но зато белые матуют ходом 2.\rook{}g3-b3\mate!!, пользуясь тем, что конь g5 защищен ферзем d8. Таким образом, в этой задаче проведена <<бристольская тема>> в ее чистом виде, так как ладья на аЗ самостоятельной роли не играет. Остальные варианты этой прекрасной задачи таковы: 1. ... \king{}:е5 2.\rook{}gd3\mate; 1. ... \queen{}:g6 2.\rook{}gf3\mate; 1. ... \queen{}:е5 2.\rook{}af3\mate; 1. ... \knight{}d3 2.\knight{}:d3\mate.

\textbf{Связывание и развязывание.} Белые могут своим первым ходом связать свою или чужую фигуру, а также и развязать любую фигуру. Но развязать первым ходом свою фигуру, вообще говоря, нехорошо -- это слишком грубый способ для достижения мата. Поэтому развязывание белой фигуры на первом ходу встречается сравнительно редко. Так же редко встречается и связывание на первом ходу черной фигуры: такой ход сразу обезоруживает черных, лишает их возможности защищаться. Поэтому нам предстоит рассмотреть только два случая: 1) связывание белой фигуры и 2) развязывание черной.

\begin{center}
 \begin{tabular}{ c c }
\textbf{№ 10. А. Эллерман.} & \textbf{№ 11. Э. Э. Вестбери.} \\
<<G.C.>>, март 1919 & II пр. <<G.С.>>, март 1917. \\
\chessboard[
\diagramsize,
setfen=5N2/1p2p2n/1Rb1pkBP/1Q4r1/1K5B/P3P3/8/8,
label=false,
showmover=false]
& 
\chessboard[
\diagramsize,
setfen=b3B3/p7/Rb1k4/1nRp1P2/5P2/4Qn2/5K2/8,
label=false,
showmover=false] \\
\textbf{Мат в 2 хода (1.\king{}а5).} & \textbf{Мат в 2 хода (1.\rook{}с8).}
\end{tabular}
\end{center}

Что значит связанная фигура? Это значит, что какая-нибудь фигура так стоит в отношении своего короля, что либо совсем не может двигаться, либо может ходить только по линии связывания. Связать свою фигуру на первом ходу, значит поставить её на линию между своим королем и дальнобойной фигурой противника.

Пример связывания белой фигуры мы видим в задаче № 10. В начальном положении белый ферзь свободен и может двигаться в любом направлении. После первого же хода он связан и может двигаться только по линии связки, то есть на поля с5, d5, е5, f5 и g5. Таким образом, первый ход белых в этой задаче содержит момент связывания белой фигуры. На 1. ... с5 белые ответят 2. \queen{}f1. Этот ход становится возможным потому, что на линию связки встала черная фигура, и теперь белый ферзь развязан. Другое развязывание белого ферзя получится при ходе 1. ... \bishop{}d5, на что последует 2.\queen{}b2.

\begin{center}
 \begin{tabular}{ c c }
\textbf{№ 12. П. Неунывайко.} & \textbf{№ 13. Р. Виндль.} \\
Журн. <<64>>, 1927 & II пр. <<G.С.>>, апрель 1919. \\
\chessboard[
\diagramsize,
setfen=b2bNN1K/r4pRB/r3p3/2Rnk3/1p3p1B/4p3/1q6/3Q4,
label=false,
showmover=false]
& 
\chessboard[
\diagramsize,
setfen=8/4p3/Kp6/p2N2Q1/Rq2k3/2P5/3pB3/8,
label=false,
showmover=false] \\
\textbf{Мат в 2 хода (1.\rook{}с4).} & \textbf{Мат в 2 хода (1.с4).}
\end{tabular}
\end{center}

Второй пример связывания белой фигуры мы находим в задаче № 11. В начальном положении белый ферзь свободен, но после первого хода он уже связан. Развязывается он лишь при ответах черных 1. ... \knight{}d4 или 1. ... d4.

В задаче № 12 мы находим развязывание черной фигуры на первом ходу. Белая ладья при своем отходе освобождает черного коня, который получает возможность играть во все стороны. Здесь мы имеем прямое развязывание черного коня.

В задаче № 13 мы видим уже не прямое, а косвенное развязывание черной фигуры: состоит оно в том, что на линию связки становится белая фигура, и благодаря этому черная фигура получает свободу.

\begin{center}
 \begin{tabular}{ c c }
\textbf{№ 14. Г. Гвиделли.} & \textbf{№ 15. К. Грабовский.} \\
<<G. С.>>, 1917. & II пр. <<G. С.>>, февраль 1918. \\
\chessboard[
\diagramsize,
setfen=8/5BRp/5B2/2N1qPn1/1P1k4/2p1Np1p/2p2Q2/r2bK1nR,
label=false,
showmover=false]
& 
\chessboard[
\diagramsize,
setfen=2N1brnB/5k2/4p1Rp/8/4B3/8/8/7K,
label=false,
showmover=false] \\
\textbf{Мат в 2 хода (1.\bishop{}а2).} & \textbf{Мат в 2 хода (1.\rook{}g2).}
\end{tabular}
\end{center}

\textbf{Тема заграждения.} Тема заграждения состоит в том, что белая фигура делает такой первый ход, который преграждает путь определенной черной фигуре и тем не даст ей возможность сделать защитительный ход.

При первом взгляде на позицию (№ 14) может показаться, что задача решается при любом отходе \bishop{}f7 по линии а2--g8. На самом же деле решает только ход 1.\bishop{}а2, так как в противном случае черные будут иметь возможность защититься от угрозы 2.\rook{}е7\mate{} ходом 1. ... \rook{}а6 или 1 ... \rook{}а7.

Еще тоньше проведена эта тема в задаче № 15. Почему нужно отступить ладьей g6 именно на g2, а не на g3, g4 или g1? А дело в том, что на угрозу белых 2. \bishop{}g6\mate черные могут ответить 1. ... \bishop{}е6! и если белая ладья на первом же ходу но станет на g2, то белый слон окажется связанным и не сможет дать мата.

\begin{center}
 \begin{tabular}{ c }
\textbf{№ 16. Г. Гвиделли.} \\
<<G.С.>>, окт. 1914. \\
\chessboard[
\diagramsize,
setfen=7B/r2ppN2/P5Q1/5P2/R6n/1B1kp1b1/K2ppr2/N3n3,
label=false,
showmover=false] \\
\textbf{Мат в 2 хода (1. \bishop{}b2).}
 \end{tabular}
\end{center}

\textbf{Антикритический ход.} В заключение остановимся на так называемом \so{антикритическом} ходе. Для этого нужно выяснить сначала, что такое \so{критическое} поле. В задаче № 16 таким является поле d4 -- потому что мат ладьей на d4 не проходит из-за того, что при этом для короля освобождается поле сЗ. После того как слон h8 встал на b2, мат ладьей на d4 уже возможен. При своем движении с h8 на b2 слон как бы перешагнул через критическое поле, -- поэтому-то такой ход называется \so{антикритическим}. Но оказывается, что в задаче имеются еще два критических поля. Действительно, при ходе 1. ... \rook{}f4 белые не могут дать мат ходом 2.\knight{}е5\mate, так как при этом белый слон отрезается от поля сЗ. Но после того как слон перешагнул через критическое поле (е5), этот мат становится возможным. Но этого мало. При ответе 1. ... \knight{}f3 белые, до своего первого хода, не могут матовать ходом 2.f6\mate, так как поле сЗ опять-таки становится свободным. После же первого хода белых, после того как слон перешагнул через критическое поле f6, этот мат становится возможным. Таким образом, ход 1.\bishop{}b2 является антикритическим не только в отношении поля d4, но также и в отношении полей с5 и f6.
