\chapter{Полусвязывание}

Тема полусвязывания появилась и стала разрабатываться сравнительно недавно, немногим больше 10 лет тому назад. Собственно говоря, комбинация, лежащая в основе этой темы, была и ранее известна, но на нее не обращали никакого внимания, -- до той поры, пока ново-американская школа не показала, как много нового и интересного таит в себе эта тема.

Состоит эта тема в следующем. На одной линии с черным королем стоит дальнобойная белая фигура (ферзь, ладья, слон), а между ними стоят какие-нибудь две черные фигуры. Пока обе фигуры стоят на этой линии, каждая из них может свободно двигаться и уйти с этой линии. Но вот одна из этих фигур, в целях защиты или просто в силу необходимости сделать ход, сошла с этой линии. Что сталось со второй черной фигурой? Она оказалась связанной. То же случилось бы с первой черной фигурой, если бы отошла вторая. Таким образом, каждая из них как бы полусвязана. Комбинация же состоит в том, что при отходе одной из черных фигур белые имеют возможность матовать только потому, что оставшаяся на линии черная фигура связана. Поясним примером:

В задаче № 105 линией полусвязки является диагональ а1~h8. Между белым слоном и черным королем стоят две черные фигуры, ладья и конь: Если двинется ладья, то конь окажется связанным; если же отойдет конь, то связанной окажется ладья. Таким образом, мы имеем в этой
 
\begin{center} 
 \begin{tabular}{ c c }
\textbf{№ . .} & \textbf{№ . .} \\
. & . \\
\chessboard[
\diagramsize,
setfen=2K5/4B3/8/1R6/kpQb4/3R4/n1r1r3/3B4,
label=false,
showmover=false]
& 
\chessboard[
\diagramsize,
setfen=2NN2nn/2K2p2/8/Q1B1k3/1pr1P3/2B4b/4RR2/8,
label=false,
showmover=false] \\
\textbf{Мат в 2 хода (1. \rook{}d5).} & \textbf{Мат в 2 хода (1. \king{}b6).}
 \end{tabular}
\end{center}

№ 105. К. Г р а б о в с к и й, 
II пр. „G. С.“, марі- 1916.
Мат в 2 хода (1. Кс7).	

задаче такую позицию, которая может привести к полусвязыванию ладьи или коня. Но пройдет ла эта комбинация, мы еще не знаем. Это мы увидим лишь в процессе решения.

Итак, после первого хода белые грозят 2. Kg6. Как защищаться черным? Двигать коня д7 нехорошо, так как на любой его отход последует 2. <J>g8. Значит, нужно защищаться ладьей. На 1. ... Ag3 последует 2, Фа8і Почему стал возможен этот мат? Во-первых, потому, что при отходе ладьи конь связан (это и есть полусвязка!), но кроме того еще и потому, что ладья перекрыла слона Ь2. Если же ладья защититься от мата ходом Лсб, то белые сыграют 2. Лсі81 -- снова полусвязка и перекрытие другого черного слона (при 1. ... ЛсЗ 2. Лсі8 не проходит из-за ответа 2. ... Сс4). Отсюда мы видим, что комбинация полусвязывания проходит в этой задаче дважды.

Но комбинация эта проходит лишь при движении одной из черных фигур, стоящих в полусвязке, ладьи, -- при движении ;ко коня связка ладьи не имеет значения, т. е. комбинация полусвязывания не проходит. Такое полусвязывание, которое проходит лишь в результате движения одной из полусвязанных фигур, называется н е п о л н ы м.

В задаче № 106 мы видим ужо п о л н о е полусвязывание. Задача эта, построенная на цугцванге, принадлежит родоначальнику темы полусвязывания, английскому композитору К. Мэнсфильду. Мэнсфильд дал ряд великолепных образцов обработки этой темы, некоторые из которых мы приведем в дальнейшем. В указанной выше задаче в полусвязке стоят два черных коня. При любом движении коня (4 (кроме 1. ... К : g6) белые матуют ходом 2. Og4 (при связанном коне f2), а на 1. ... К : g6 последует 2. g4 (опять полусвязка). При любом отходе коня f2 (кроме 1. Кс4) белые матуют ходом 2. ФсіЗ (ири связанном коне f4), а на 1. . .. Ке4 последуют 2. Феб (четвертая полусвязка). В этой задаче комбинация получается в результате движения обеих полусвязанных фигур, т. е. является ПОЛНОЙ.

Уже из этих первых задач на тему полусвязывание мы можем сделать один существенный вывод: комбинация полусвязывания редко проходит в своем элементарном виде, т. е. редко бывает, чтобы одна из полусвязанных черных фигур, покидая линию полусвязки, т о л ь к о связывала вторую черную фигуру. Обычно, она при этом делает еще кое - что: перекрывает какую-нибудь другую фигуру, блокирует поле, дает мат и пр. Таким образом, тема полусвязывания сочетается с другими темами ново-американской школы. В дальнейшем, рассматривая различные комбинации с полусвязыванием, мы будем указывать, с какими идеями переплетается тома полусвязывания.

Какие черные фигуры могут создать комбинацию полусвязки? Все, начиная от ферзя и кончая пешкой. При этом этом могут быть и одинаковые фигуры (два коня, две ладьи) и различные (конь и ладья, слон и пешка и пр.), а полусвязывание может происходить и по диагонали и по линии (горизонтальной или вертикальной).

Рассмотрим несколько случаев полусвязывания одинаковых черных фигур.

\begin{center} 
 \begin{tabular}{ c c }
\textbf{№ . .} & \textbf{№ . .} \\
. & . \\
\chessboard[
\diagramsize,
setfen=2K5/4B3/8/1R6/kpQb4/3R4/n1r1r3/3B4,
label=false,
showmover=false]
& 
\chessboard[
\diagramsize,
setfen=2NN2nn/2K2p2/8/Q1B1k3/1pr1P3/2B4b/4RR2/8,
label=false,
showmover=false] \\
\textbf{Мат в 2 хода (1. \rook{}d5).} & \textbf{Мат в 2 хода (1. \king{}b6).}
 \end{tabular}
\end{center}
№ 108. A. Map н
III поч. отз. „G. С.“, аирель 1919.
Мат в 2 хода (1. Кра2).	    № 107. С. М л о т к о в с к и й. и А. Эллерман.
    II вр. копк. „Brisbane Courier", 1926.
Мат в 2 хода (1. Фіі2).

В первой из этих задач (№ 107) мы имеем диагональное полусвязывание двух ладей в самом элементарном виде. В самом деле, защищаясь от мата ферзем на ЬЗ, черные могут играть либо 1. ... Лс2 на что последует 2. Ф: е2, либо 1. ... Ad3 2. Фсб, либо 1, ... Лс5 2. Фс4. Конечно, при таком ограниченном количестве белых и черных фигур провести эту тему полисс весьма трудно. Но nor п задаче № 108 мы находим прекрасную обработку этой темы, соединенной с другими темами новой школы.

Белые грозят 2. Ф '• d6. Черные могут защищаться обеими ладьями. Но на 1. ... Ло7 -- d7 (или еб) последует 2. Ксб. Интереснее игра второй ладьи: 1. ... Ad6: d2 (или d3 и пр.) 2. Фс7, 1. ... Ad6—d7 2. Ксб. 1. ... Лсб! 2. КЬ7. В последнем варианте мы имеем сочетание полусвязывания с блокированием. Эта игра двух черных ладей, стоящих в полусвязке, соединена в задаче с игрой черного ферзя: 1. ... Фс5 2. Ке4! и 1. ... ©d5 2. Kd3! (развязывание белого коня плюс перекрытия черных ладей). Обе темы так тесно переплетены в задаче, что трудно сказать, какая из них является основной -- полусвязывание или развязывание.

Тему горизонтального полусвязывания двух черных ладей мы находим в прекрасной задаче Эллермана (№ 109). Белые грозят 2. Ф!4. Идейные защиты черных таковы: 1. ... Лс4 2. Kd3 (полусвязывание плюс перекрытие слона аб), 1. ... ЛсЗ 2. d4 (полусвязывание плюс перекрытие слона аі) и 1. ... Ad4 2. Ксб (полусвязывание). Задача имеет еще 'несколько вариантов, но уже без полусвязывания.

Прелестная комбинация с полусвязыванием двух слонов (конечно, в отношении слонов возможно лишь горизонтальное или вертикальное полусвязывание) проведена в задаче № 110. Правда, комбинация полусвязки проходит только в двух вариантах, но варианты сами по себе великолепны. Грозит 2. Лс7. Ha 1. ... Cd5! последует 2. ФЬ7! (полусвязка плюс перекрытие ладьи) и на 1. ... Cd6 2. Ф^4 (полусвязка плюс перекрытие той же ладьи). Только найдя оба эти варианта, можно понять смысл и тонкость первого хода. Слон должен отступить именно на Ь2. так как на с5 или f4 он будет мешать двигаться белому ферзю в тематических вариантах.

\begin{center} 
 \begin{tabular}{ c c }
\textbf{№ . .} & \textbf{№ . .} \\
. & . \\
\chessboard[
\diagramsize,
setfen=2K5/4B3/8/1R6/kpQb4/3R4/n1r1r3/3B4,
label=false,
showmover=false]
& 
\chessboard[
\diagramsize,
setfen=2NN2nn/2K2p2/8/Q1B1k3/1pr1P3/2B4b/4RR2/8,
label=false,
showmover=false] \\
\textbf{Мат в 2 хода (1. \rook{}d5).} & \textbf{Мат в 2 хода (1. \king{}b6).}
 \end{tabular}
\end{center}
№ 109. А. Эллерман.
II пр. „С. С.“, ноябрь 1917.
Мат в 2 хода (1. Kpg3).
	№ 110. Г. Гвиделли.
I пр. „G. С.“, янпарь 1917.
Мат в 2 хода (1. Ch2),

В задачс Г. Гвидслли кы имоем лишь два варианта с полусвязыва- нкек. В Лгг 111, принадлежащем знамонитоку итальянскому композитору Алъборто Мари, зта тоудная тема проведена в четырех прскрасных вариантах.

\begin{center} 
 \begin{tabular}{ c c }
\textbf{№ . .} & \textbf{№ . .} \\
. & . \\
\chessboard[
\diagramsize,
setfen=2K5/4B3/8/1R6/kpQb4/3R4/n1r1r3/3B4,
label=false,
showmover=false]
& 
\chessboard[
\diagramsize,
setfen=2NN2nn/2K2p2/8/Q1B1k3/1pr1P3/2B4b/4RR2/8,
label=false,
showmover=false] \\
\textbf{Мат в 2 хода (1. \rook{}d5).} & \textbf{Мат в 2 хода (1. \king{}b6).}
 \end{tabular}
\end{center}
№ 111. A. Мари.
I np. KOIIK. „Brisbnnc Courier", 1922.
Mar в 2 хода (1. Cf6).
	Мат в 2 хода (1. Ag3).


     Гроаит 2. Л(14. Чорнме могуг яащиідаться обоими слоиами. Ня 1. . . . Сс4 послодуот 2. Kf4, так как слон <12 сияяпн, а поло с4 яаблокировано, что Ідаот воэможность выключить ладыо Ь4. Ha 1. ... СсЗ последует 2. С : с4—опягь полусвязывание, на втот с перекрыгием ладьи с2. На 1. ... Cf4 бельте отвстяг 2. ®g2!, пользуясь полусвязкой слона, а также перекрытием ладьи f5, а на 1. ... СеЗ 2. Ф: f5. Тонкость первого хода становигся ясна из варианта 1. ... Ксб 2. Ф^8,—необходимо первым же ходом перекрыть линию f, чтобы отрезать черную ладью от поля f7.

    Примером вертикального полусвязывания двух коней может служить приводенная выше аадача № 106. Диагональную полуснязку их мы находим в задаче № 112. Угроза вдесь 2. ФЫ. Главную игру дает конь d5. Ha 1. ... Кс7~Ь последует 2. С : с7, на 1. . . . К : f6 4- 2. С : S6, ка
1. ... К: f4 2. С: f4, а на 1. ... КеЗ 2. Cd6. Ha 1. ... Kd4 также по- следует 2. С : <14, а на 1. ... К: сЗ 2. С: сЗ.

    Очснь трудна тема полусняяынапия двух псшск. Всс же ссть немало задач, где тема эта хорошо проведена. В качссгве примера приведсн задачу известного датского композитора К. A. К. Ларсена (№ 113): в ней полусвязывание черных пешек переплетается с моменгами их превра- іцекия в различные фигуры. Защищаясь от угрозы 2. КеЗ, черные могуг сыграть 1. ... flKl, на что последует 2. Лсі. При защите 1. . . . dlK белые играіот 2. Kel, а при ходе 1. следуег 2. Фс7.

\begin{center} 
 \begin{tabular}{ c c }
\textbf{№ . .} & \textbf{№ . .} \\
. & . \\
\chessboard[
\diagramsize,
setfen=2K5/4B3/8/1R6/kpQb4/3R4/n1r1r3/3B4,
label=false,
showmover=false]
& 
\chessboard[
\diagramsize,
setfen=2NN2nn/2K2p2/8/Q1B1k3/1pr1P3/2B4b/4RR2/8,
label=false,
showmover=false] \\
\textbf{Мат в 2 хода (1. \rook{}d5).} & \textbf{Мат в 2 хода (1. \king{}b6).}
 \end{tabular}
\end{center}
№ 113. К. A. К. Ларсои.
IV пр. „G. С.“, ноябрь 1920.
Мат в 2 хода (1. Kg4).	№ 114. К. В э т н е й.
II up. ,,G. С.“, май 1920.
Mar в 2 хода (1. Ag8).

    Познакомипшись с полусвязыванием одинаковых черных фигур, мы псреходим к тсм комбинациям, которые возникают в результате полу- связывания различных фигур. Исчерпать все эти сочетания мьг, конечно,
№ 118. K. A. K. Л a p c e II. II up. ,,G. C.“, фсврпль 1920.
Мат в 2 хода (1. Kf3).
	Мат в 2 хода (1. Kpc5)


    B задачс № 118 мы видим сочотание у;ке трех тсм: нолусвязки, раз- вязывания и сще шахов на вскрышку. Такое сочетание чрезвычайно трудно осуществимо, но в этой задаче оно очень хорошо и искусно представлено. Грозит 2. ФГ4, но первым своим ходом белые связали коня с15 и сдслали нозможнмми открытые шахи коием со сгороны чорпых. Но на безразличный шах конем (напр. 1. ... Ка5 + ) иоследует просго 2. С: с2ч (полусвязка). Поэгому черныс должны так огойги консм, чтобы одновременно защититься or этого мата. Для эгого им нужно попасгь конем на е5. Но на 1. ... Кс5 последует 2. КсЗ! (развязыванио с полу- связкой). Ha 1. . . . d3-f-6eAbie отвечают 2. Л: еЗ (полусвязка), Ha 1. . .. е5 послодуот 2. Kf6. Прскрасноя и трудная зпдача!
    В вадаче № 119 к эгим трсм томам добавлеи сщо один момснг: раэвязываіше черной фигуры. В самом деле, посло 1-го хода бслых гро- зит 2. f4. Как защиідаться чсрным от этоіі угрозы? Прсждс псего, путем шаха. Но на 1. .. . с6 + или 1. ... cb-f- послодует просто 2. Л: е7 (гю-
№ 120. Л. А. И о а о іі. Конкурс жури. „64", 1925. (Исправлснпая).
лусвязка). Поэтому черныс играют 1. . .. с5+, развязывая чсрного коня d5 и гірепягсгвуя указанному выше мату, Но своим ходом черкые развязали также белого коня Ьб, который и дает мат" 2. Kd7! Итак, в ва- рианте 1. ... с5~Ь 2. Kd7 мы имеем четыре момснта: шах на вскрышку,
азвязьівание белой фигуры, развязыванне чорной фигуры и полусвязку. Ізумительное достижениеі Правда, задача получилась восьма громоздкой, но ведь и замыссл исключительной трудности. Есть в задачо и ещс один
крлснпміі HnpniuiT I, ... Ko5 (рааняаывая черного коня c!5) 2. Kc4 (вто-
рое оазБязывание белого коня).
     Всликолсгшую обряботку сходиой, но ощс болоо орипшалыюн томм мы находим іі задачс JN1’ 120. Болыо гроаяг матом Kd6. Защититься от этой угрозы бслыо могут разными способами,—Б первую очередь, раз- вязывая своого ферзя. Но на 1. ... Cd5 послсдуег 2. Фс!3!, так как белый фсрзь развязан, а конь Ь4 стоиг в полуевязке. Таким образом, мы имеем здссъ сочетание трох тем развязывания чсрной фигуры, раз- вязывания бслой фигуры и комбинации полусвязки. Но ато жо сочетание проводено сще в одном вариантс: 1. ... Kd5 2. ФЫ! Прочие варианты:
1. ... Kpf5 2. © ; с5.—1	А.: сб 2. K:g5.

\begin{center} 
 \begin{tabular}{ c c }
\textbf{№ . .} & \textbf{№ . .} \\
. & . \\
\chessboard[
\diagramsize,
setfen=2K5/4B3/8/1R6/kpQb4/3R4/n1r1r3/3B4,
label=false,
showmover=false]
& 
\chessboard[
\diagramsize,
setfen=2NN2nn/2K2p2/8/Q1B1k3/1pr1P3/2B4b/4RR2/8,
label=false,
showmover=false] \\
\textbf{Мат в 2 хода (1. \rook{}d5).} & \textbf{Мат в 2 хода (1. \king{}b6).}
 \end{tabular}
\end{center}
№ 121. К. Ш с п п а р д. 
III пр. „G. С.“, май 1919.
Мат в 2 хода (1. ФЬ2).
	№ 122. А. П. Г у л я е в.
„Шахматы", 1927.
Мат в 2 хода (1. Kpd7).


     До сих пор мы зкакомили наших читателсй с такими задачами, где проводктся лиаіь одна система полусвязывания (системок полусвязы- вания мы называем позицию, где между дальнобойной белой фигурой и чсркым королем стоят двс черныс фигуры, которыс своей игрой соз- дают комбинацшо полусвязки). Ко таких систем в одной задаче может быгь и больше. Для примера берем задачу № 121. Здесь две таких систсмы: первая по линии g, а вторая no третьсй линии. Задача по- строена на Zugzwang. Тематическис ваоианты таковы: 1. ... Ag4 СЛ 2. К : !і5—1. ... Лс5 2. © : с5.—1. ... Ла5 (Ь5, с5 и т. д.) 2. Ке4,—1. ... f2-i- 2. © : f2.—1 ... с2 2. К : е2. Итого—пять полусвязок. Ha 1. .. . Кр.Ч послсдуст 2. ФЬ8.
     По лусвязка можст быть нс только открыгон, но и скрытой, замаски- рсванной. Пример такой замаскированной системы полусвязки мы нахо- дим в задачс Лга 122. Бслыо грозят 2. Кс7. Чсрные в цслях заіциты долгкнь! играть конем Ь5, связывая коня сб. Но на 1. ... Kd4 послсдует
2. © : (7 (развязыванио с полусвязкой), а на 1. ... Kdo 2. ФГЗ (то же самоо). Полусвязка обнаружг.иастся здссь лншь в самыіі последниіі мо- мснт,—послс псрвого хода болых она находится в замаскированком видс.
     Заканчивая главу о волусвязываикк, мы считаем кеобходимым ука- зать нг то, что мы и в малой степени но исчерпали всех комбинаций, скрьі?ых в этой иктереснейцісй тскс. Хотя на поприщс полусвязывашія сдглано и досткгкуто ужо очснь много, но мы огнюдь не счигаом ату область коиой иіхолы ксчорпанной. Здссъ возможна ещо большаи. и про- дуктивная работа.
     3 заключение помещаем две задачи К. Мэксфильда, где приведено максимальное холичсстао полусаязывавий при ішгерссной и острой игре.

\begin{center} 
 \begin{tabular}{ c c }
\textbf{№ . .} & \textbf{№ . .} \\
. & . \\
\chessboard[
\diagramsize,
setfen=2K5/4B3/8/1R6/kpQb4/3R4/n1r1r3/3B4,
label=false,
showmover=false]
& 
\chessboard[
\diagramsize,
setfen=2NN2nn/2K2p2/8/Q1B1k3/1pr1P3/2B4b/4RR2/8,
label=false,
showmover=false] \\
\textbf{Мат в 2 хода (1. \rook{}d5).} & \textbf{Мат в 2 хода (1. \king{}b6).}
 \end{tabular}
\end{center}
№ 123. К. Мэнсфильд.	т д-
r	1 пр. конк. „Ы Ajedrez Ar#entino“, 1926.
	I up* коик. „Hampsbiro Post*1, 1915.	

