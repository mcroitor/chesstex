\chapter{Шахи белому королю.}

Конечно, и в задачах старых школ мы часто встречаем такую тему: защищаясь от грозящего мата, черные дают в свою очередь шax белому королю, но, несмотря на это, белые все же матуют своего противника на следующем ходу. Но только проблемисты новой школы всесторонне разработали эту тему, углубили ее и обогатили таким содержанием,. которое и не снилось старым мастерам.

Для того, чтобы полнее познакомить наших читателей с темой шаха, нам придется разбить ее на три группы: к первой мы отнесем те случаи, когда черная фигура непосродственно объявляет шах белому королю, ко второй -- те задачи, где шах получается на вскрышку, т. е. черная фигура при своем движении открывает стоящую за ней дальнобойную черную фигуру (ферзя, ладью, слона), которая дает шах белому королю, и, наконец, к третьей -- те позиции, где черные объявляют белым двойной шах, т. е. одновременно шах двумя фигурами.

Рассмотрим задачу № 46.

\begin{center} 
 \begin{tabular}{ c c }
\textbf{№ 46. С. С. Левман.} & \textbf{№ 47. Л. Ротштейн.} \\
III пр. конк. <<Известия>>, 1923--25. & I пр. <<G. С.>>, февраль. 1919. \\
\chessboard[
\diagramsize,
setfen=1bB2n2/6pq/2pp3p/2Bk1P1R/b7/1P1NKp2/4P3/3Q4,
label=false,
showmover=false]
& 
\chessboard[
\diagramsize,
setfen=n7/1p1P4/4q3/RB1kP1Nn/1P2R1b1/3p4/3K2Q1/8,
label=false,
showmover=false] \\
\textbf{Мат в 2 хода (1. \king{}f4)} & \textbf{Мат в 2 хода (1. \king{}сЗ)}
 \end{tabular}
\end{center}

В начальном положении белому королю не грозит ни один шах со стороны черных. Своим первым ходом, создающим угрозу 2. е4\mate{}, белый король становится на такое поле, с которого ему грозит сразу 5 шахов. Но на каждый из этих шахов у белых есть в запасе достаточный ответ. И действительно, на 1. ... \queen{}:f5+ последует 2. \rook{}:f5\mate{}, на 1. ... \knight{}eб+ 2. \bishop{}:еб\mate{}, на 1. ... \knight{}g6+ или 1. ... g5+ 2. fg\mate{}. Все четыре шаха со стороны черных -- простые шахи. Ho есть еще пятый шах: 1. ... dc + . Это уже \so{шах на вскрышку}, так как шахует не пешка, а слон, стоящий позади пешки, но теперь распространивший свое влияние и на поле f4. В ответ на этот шах белые играют 2. \knight{}е5\mate{} (мат на вскрышку), пользуясь тем, что черные заблокировали поле с5 \footnote{О блокировании см. в VI главе.}.

В задаче № 47 мы также имеем два простых (или \so{прямых}) шаха: 1. ... \queen{}:е5+ и 1. ... \queen{}сб+. В обоих случаях черный ферзь блокирует одно из полей около черного короля, и это дает возможность белым дать мат: в первом случае при помощи 2. \rook{}d4\mate{} и во втором -- 2. \bishop{}с4\mate{}. Эта задача показывает, в чем состоит комбинация, лежащая в основе темы шаха: белые, правда, позволяют черным дать шах своему (белому) королю, но этим самым черные как-то ухудшают свое положение -- либо блокируют поле, либо снимают защиту с какого-нибудь поля, либо перекрывают какую-нибудь свою фигуру и так далее.

Мы не считаем необходимым останавливаться подробнее на задачах с прямыми шахами и переходим ко второй группе, где черные дают шax на вскрышку. В этой области есть гораздо больше интересных и заслуживающих внимание комбинаций.

Шах на вскрышку может произойти от движения черной ладьи, слона, коня, пешки и, наконец, самого черного короля. Рассмотрим все эти случаи.

В задаче № 48 мы имеем очень простую комбинацию: ладья b7 может отойти с шахом, открывая слона а8, но если она двинется на линии b, то будет матовать \knight{}е7 (1. ... \rook{}b6 2. \knight{}сб\mate{}.-- 1. ... \rook{}b5 2. \knight{}d5\mate{}), а если она пойдет по 7-ой линии, то будет матовать \knight{}b4 (1. ... \rook{}с7. 2. \knight{}сб\mate{}. -- 1. ... \rook{}d7 2. \knight{}d5\mate{}). При всей простоте эта комбинация, однако, очень занятна.

\begin{center} 
 \begin{tabular}{ c c }
\textbf{№ 48. Г. Гвиделли.} & \textbf{№ 49. Андерсон.} \\
Поч. отз. ,<<G. С.>>, 1915. & I пр. <<G. C.>>, окт. 1917.\\
\chessboard[
\diagramsize,
setfen=bn1BB1R1/pr2N3/8/8/RN5k/1p6/8/5K2,
label=false,
showmover=false]
& 
\chessboard[
\diagramsize,
setfen=2n4B/3QPp2/bbp1N1p1/2RP1kPr/3r1P2/3N3P/3P4/1B1n1RK,
label=false,
showmover=false] \\
\textbf{Maт в 2 хода (1. \king{}g2).} & \textbf{Maт в 2 хода (1. \rook{}b5).}
 \end{tabular}
\end{center}

Из этой простой комбинации видно, однако, в чем соль, в чем смысл этой темы: белые дают возможность \so{отходящей} черной фигуре (в данном примере -- ладье) сделать несколько ходов, но на \so{различные} движения этой фигуры следуют и различные маты. Чем больше таких движений отходящей фигуры (мы предлагаем назвать её вскрывающей фигурой) и чем больше количество разнообразных матов, следующих в ответ на шахи черных, тем богаче представлена тема и тем ярче и интереснее задача.

В задаче № 49 \so{вскрывающей} фигурой также служит ладья, но игра её и ответные маты гораздо интереснее, чем в предыдущей задаче. На какое-нибудь безразличное отступление ладьи (напр., 1. ... \rook{}а4+ ) белые ответят 2. \knight{}е6-с5\mate{}. Но ведь ладья может сыграть так, чтобы помешать этому мату, скажем, 1. ... \rook{}:d5+. Теперь 2. \knight{}с6-с5\mate{} не проходит из-за ответа 2. \rook{}:d7. Ho пользуясь тем, что ладья на d5 связана, белые играют 2. \knight{}d3-f2\mate{}! (но не 2. \knight{}d3-с5, так как на это последует 2. ... \rook{}аЗ). Если же черные сыграют 1. ... \rook{}:f4+ , то белые сыграют 2. \knight{}d3-с5 (нельзя ни 2. \knight{}d3-f2 из-за 2. \rook{}с4, ни 2. \knight{}е6-c5 из-за того, что белая пешка g5 требует защиты). У черных есть, однако, еще один маневр 1. ... \rook{}е4! Все предыдущие маты теперь уже нс проходят, но зато белые могут сыграть 2. \knight{}d4\mate{}, пользуясь тем, что черная ладья заблокировала поле d5. Трудная и содержательная задача, показывающая, какие богатейшие комбинации таит в себе тема шаха на вскрышку.

В задачах № 50 и № 51 \so{вскрывающей} фигурой является слон. Обе задачи принадлежат знаменитому и безвременно скончавшемуся итальянскому композитору Г. Гвиделли.

В № 50 черный слон имеет много полей для отхода с шахом. Но если слон побьёт на b8 или пойдёт на с7,  то он этим самым снимет удар с ладьи b4 и белые дадут мат ходом 2. \knight{}c6\mate{}. Если же слон пойдёт по линии a3--f8, то он перестаёт влиять на поле f4 и кроме того отнимает у черного короля поле e5 (белый слон b8!), и белые матуют ходом \knight{}f4-e6. Поэтому чёрный слон играет 1. ... \bishop{}e5!, держа под прицелом обе батареи. Но этим ходом он блокирует поле e5, и белые дают мат при помощи шаха навскрышку 2. d6\mate{}!

\begin{center}
 \begin{tabular}{ c c }
\textbf{№ 50. Г. Гвиделли.} & \textbf{№ 51. Г. Гвиделли.} \\
Поч. отз. <<Hampshire Post>>, 1915 & I пр. <<G. C.>>, 1916. \\
\chessboard[
\diagramsize,
setfen=BB6/3p2q1/K1Rb1rp1/1p1P1P2/1R1NkNQ1/8/5P2/7b,
label=false,
showmover=false]
& 
\chessboard[
\diagramsize,
setfen=4K3/3Q2p1/6p1/3pBR1p/2pRb1kr/2r3p1/2p1q1Nn/3B2N1,
label=false,
showmover=false] \\
\textbf{Maт в 2 хода (1. \rook{}c3).} & \textbf{Maт в 2 хода (1. \bishop{}f4).}
 \end{tabular}
\end{center}

Вторая задача открывается блестящим вступительным ходом белых: этот ход не только вскрывает линию f для будущих шахов, но и развязывает слона e4, то есть \so{вскрывающую} фигуру. Грозит 2.\rook{}g5\mate{}. На 1. ... \bishop{}:d5+ последует 2.  \bishop{}e3\mate{}!!, так как черный слон связан и не может взять белую ладью. На 1. ... \bishop{}d3 белые играют 2. \knight{}e3\mate{}, пользуясь тем, что перекрыта черная ладья, а на 1. ... \bishop{}f3 или \bishop{}:g2 2. \rook{}e5\mate{}. Изящная и великолепно задуманная задача!

Мы переходим теперь к задачам, где вскрывающей фигурой является черный конь. Нужно сказать, что эта тема допускает еще больше интересных комбинаций, чем шахи навскрышку ладьёй или слоном. Кроме того, конь может давать открытый шах не только по горизонтали или вертикали, как слон, но также и по диагонали, как ладья.

Приведем две небольшие задачи, характеризующие такие комбинации с конём.

\begin{center}
 \begin{tabular}{ c c }
\textbf{№ 52. Ф. А. Л. Кускоп.} & \textbf{№ 53. С. С. Левман.} \\
 I пр. << G. C.>>, 1925 & Поч. отз. <<Известия>>, 1925\\
\chessboard[
\diagramsize,
setfen=4R3/1B6/8/8/1KN1n2r/5P2/1R1Pk2P/1Q6,
label=false,
showmover=false]
& 
\chessboard[
\diagramsize,
setfen=1R2n1bb/6nq/p7/1N5p/kB6/P1R5/2P5/K2B4,
label=false,
showmover=false] \\
\textbf{Maт в 2 хода (1. \knight{}e3).} & \textbf{Maт в 2 хода (1. \rook{}c4).}
 \end{tabular}
\end{center}

В № 52 грозит 2. \queen{}f1. Черный конь может отойти с шахом на вскрышку, но на любое его отступление, кроме \knight{}:d2, последует 2. d4\mate{}, так как пешка f3 уже защищена слоном. На 1. ... \knight{}:d2 последует 2.\knight{}g4\mate{}! -- белые пользуются тем, что чёрный конь связан.

В № 53 конь даёт открытый шах уже не по горизонтали, а по диагонали. Грозит 2. \knight{}c3\mate{}. На 1. ... \knight{}f5 белые отвечают 2. c3\mate{}, пользуясь тем, что конь перекрыл ферзя, а на 1. ... \knight{}e6 2. \bishop{}c3\mate{}, так как конь перекрыл слона.

От этих простейших комбинаций перейдём к более сложным.

В прекрасной задаче Эллермана (№ 54) конь d4 имеет возможность отступить с открытым шахом на любое из 8 полей. Нетрудно, однако, увидеть, что при отступлениях его на b5, b3, c2, e2 и f3 белые дают мат слоном на вскрышку (2. \bishop{}c4\mate), так как коньне может закрыть своего короля от ладьи  e8. Но если конь отступит на c6, то перекроет ладью c7, и белые сыграют 2. \knight{}c4, а на 1. ... \knight{}f5 последует 2. \queen{}e4, так как конь перекрыл слона h7. И, наконец, при 1. ... \knight{}:e6 белые пользуются тем, что конь связан, и сыграют 2. d4\mate{}. Как тонко и в то же время просто скомбинированы эти варианты.

\begin{center} 
 \begin{tabular}{ c c }
\textbf{№ 54. А. Эллерманн} & \textbf{№ 55. К. Мэнсфильд.} \\
I пр. <<G. С.>>, окт. 1916 & I пр. <<G. С.>>, март 1917. \\
\chessboard[
\diagramsize,
setfen=4R3/2rp3b/3pB2r/N3k1P1/K2nNqP1/2PP3p/7B/7Q,
label=false,
showmover=false]
& 
\chessboard[
\diagramsize,
setfen=1B4n1/7N/K1pN4/5P2/R1n2k2/2P1R2P/4b1B1/2Q5,
label=false,
showmover=false] \\
\textbf{Мат в 2 хода (1. Сс4).} & \textbf{Мат в 2 хода (1. Сс4).}
 \end{tabular}
\end{center}

Еще красивее игра в задаче № 55. Прекрасный вступительный ход развязывает черного коня, который может отступить куда угодно, давая открытый шах по диагонали. Ha 1. ... \knight{}:еЗ белые ответят 2. \knight{}b5\mate{}!, на 1. ... \knight{}:d6 последует великолепный мат слоном на вскрышку 2. \bishop{}dЗ\mate{}!, на 1. ... \knight{}с5 2. \rook{}d3\mate, а на 1. ... \knight{}d2 2. \knight{}с4\mate. Замечательно, что в ответ на открытые шахи черных белые объявляют мат также при помощи открытых шахов -- ладьей, слоном и конем.

Вскрывающей фигурой можот быть и черная пешка. В качестве примера приведем две задачи: в первой пешка дает открытый шах по горизонтали, во второй -- по диагонали.

В № 56 грозит 2. \knight{}f6\mate{}. Пешка d7 может сделать три различных хода, давая шах на вскрышку. Ha 1. ... dе последует 2. \knight{}е7\mate{}, так как теперь черная пешка ушла с линии d и не можег перекрыть белого слона, встав на d5. Если черные сыграют 1. ... dc, то белые ответят 2. \bishop{}с7\mate{}!, -- так как пешка не может уже бить ферзя (но не 2. \bishop{}g7 из-за 2. ... \bishop{}е5). А вот если черные пойдут 1. ... d6, то последует 2. \bishop{}g7\mate{}, так как пешка перекрыла своего же слона.

В задаче № 57 пешка d7 также имеет 3 хода. Ha 1. ... d6 белые ответят 2. \knight{}сб\mate{}. Ha 1. ... d5 этот ответ уже не годится, так как теперь слон b5 развязан, но зато черные заблокировали поле d5 и белые дают мат 2. \bishop{}d7\mate, Ha 1. ... de белые отвечают 2. \queen{}:b5\mate{}.
 
\begin{center} 
 \begin{tabular}{ c c }
\textbf{№ 56. К. Мэнсфильд.} &  \textbf{№ 57. Л. Б. Залкинд.}\\
 \textbf{и Д. В. Чандлер.} & \\
<<G. С.>>, 1914. & II пр. конк. <<64>>, 1925. \\
\chessboard[
\diagramsize,
setfen=1b6/r2p1Q1K/2B3p1/1p1NB2b/4kP2/2P4R/4r3/8,
label=false,
showmover=false]
& 
\chessboard[
\diagramsize,
setfen=1b1NK3/p2pR3/2R1B2n/Qb2k1P1/8/Bp3P2/4N3/1rr5,
label=false,
showmover=false] \\
\textbf{Мат в 2 хода (1. Фсб).} &  \textbf{Мат в 2 хода (1. Лс3).}
 \end{tabular}
\end{center}

Мы переходим теперь к последней черной фигуре, могущей давать шахи на вскрышку, -- к черному королю. Тема эта представляет большие трудности, и хороших задач, где она была бы полно выражена, немного. В приведенной ниже задаче № 58 черный король имеет возможность отступить на 3 поля, давая открытый шax белому королю. Ha 1. ... \king{}сЗ последует 2. \knight{}f6-с4\mate{}, на 1. ... \king{}с5 2. \knight{}f2-е4\mate{}, a на 1. ... \king{}е5 2. \knight{}fб-е4\mate{}. В этой задаче нужно отметить великолепный первый ход белых, отдающий черному королю все три тематические поля, то есть именно те поля, вступив на которые черный король дает шах на вскрышку белому королю.

До сих пор мы рассматривали только такие задачи, где играет лишь одна вскрывающая фигура. А между тем композиторам ново-американской школы удалось создать ряд прекрасных производений и на более сложную тему: в задаче участвуют две вскрывающие фигуры одновременно. Приведем в качестве примера 3 таких аадачи.
 
\begin{center} 
 \begin{tabular}{ c c }
\textbf{№ 58. А. Г. Стэббс.} & \textbf{№ 59. А. М. Спарке.} \\
I пр. <<G. С.>>, февраль 1918. & III пр. <<G. С.>>, дек. 1916. \\
\chessboard[
\diagramsize,
setfen=8/2Q5/4bNP1/1P1p1p1p/K2k3r/3P1R2/2p2N2/b3r1B1,
label=false,
showmover=false]
& 
\chessboard[
\diagramsize,
setfen=2nr1q2/2p1Pp2/2N1P3/Qb1k4/1BR1R3/n1p5/2p5/4bK1B,
label=false,
showmover=false] \\
\textbf{Мат в 2 хода (1. Сс5).} & \textbf{Мат в 2 хода (1. Ф7).}
 \end{tabular}
\end{center}

В задаче № 59 мы имеем один шах на вскрышку (1. ... fe) и один прямой шах (1. ... \bishop{}:с4). Первый ход, развязывающий слона b5 и жертвующий коня с6, очень хорош.

Несравненно интереснее и ярче задача № 60, где проведено 4 шаха на вскрышку черными конями. Белые грозят 2. \knight{}b5\mate{} (двойной шах и мат). Единственная защита черных -- встречные шахи. С шахами могут отойти оба коня. При ходе 1. ... knight{}е5:d7 (f7, f3) белые играют 2. \bishop{}сб\mate{}. При шахе 1. ... \knight{}:с4 2. \knight{}с4\mate{} (пользуясь тем, что черный конь связан). На 1. ... \knight{}g7-е8 белые матуют ходом 2. \bishop{}f7\mate{} (конь e5 связан), а на 1. ... \knight{}сб последует 2. \knight{}f7\mate{} (конь e5 опять-таки связан).

\begin{center} 
 \begin{tabular}{ c c }
\textbf{№ 60. Г. Гвиделли.} & \textbf{№ 61. Гартонг.} \\
\textbf{и Э. Вестбери} & \\
I пр. <<G. С.>>, май 1916. & I пр. <<G. С.>>, 1926. \\
\chessboard[
\diagramsize,
setfen=7B/b2RK1nq/2pN2p1/Q1P1np1p/2Bk2r1/3p1P2/2b1R3/8,
label=false,
showmover=false]
& 
\chessboard[
\diagramsize,
setfen=8/BK1p4/p4p1q/2R1NB2/1p1k2p1/2N5/n3nPp1/rrb1Rb2,
label=false,
showmover=false] \\
\textbf{Мат в 2 хода (1. \queen{}b4).} & \textbf{Мат в 2 хода (1. \king{}:a6).}
 \end{tabular}
\end{center}

Прекрасно обработана эта тема и в задаче № 61. Здесь шахи на вскрышку проходят по трем линиям (1. ... \knight{}а:c3 2. \rook{}а5\mate{}. -- 1. ... fe 2. \rook{}с6\mate{}. -- 1. ... \knight{}есЗ 2. \rook{}b5. -- 1. ... Кe~ 2. \rook{}с4\mate{}), причем трижды используется момент блокирования полей сЗ и е5. Преимущество задачи Гартонга заключается и в том, что в начальном положении на одинственный возможный шах (1. ... bc) есть уже готовый мат 2. \rook{}b5, между тем как в задаче № 60 в начальном положении на грозящие со стороны черных шахи у болых нет ответа.

В заключение остановимся на тех задачах, где черные, защищаясь, объявляют белому королю двойной шax. Таких задач, вообщо говоря немного, так как тема эта чрезвычайно трудна. Мы ограничимся двумя наиболее яркими примерами.

В старой задаче Хескота (№ 62) тема эта проведена прекрасно. На 1. ... fg++ (двойной шах) белые отвечают 2. \king{}e5\mate{}! На шах же 1. .... hg белые выполняют угрозу 2. \king{}f5\mate{}.

В задаче № 63 белые своим первым ходом дают возможность коню с4 отойти куда угодно, давая шах на вскрышку. Однако, при каком-кибудь безразличном отходе коня (напр. 1. ... \king{}:g5) белые играют 2. \rook{}d3\mate{}! и дают мат. Поэтому черный конь бьет ладью на d6, давая том самым двойной шах (ферзем и конем), на что последует 2. \king{}f6\mate{} -- и мат.
 
\begin{center} 
 \begin{tabular}{ c c }
\textbf{№ 62. Г. Xескот.} & \textbf{№ 63. A. Кремер.} \\
I пр. <<Sydney Herald>>, 1907 & <<Deutsche Tageszeitung>>, 1920. \\
\chessboard[
\diagramsize,
setfen=6B1/1pr2qr1/5pNp/2p2R1P/Ppk2KR1/2N4Q/P4B2/b7,
label=false,
showmover=false]
& 
\chessboard[
\diagramsize,
setfen=N4BB1/2N4p/3p4/1Pk2KQ1/1r2n3/3RP1p1/1rq5/2R5,
label=false,
showmover=false] \\
\textbf{Maт в 2 хода (1. \rook{}f5-g5).} & \textbf{Maт в 2 хода (1. \rook{}:d6).	 }
 \end{tabular}
\end{center}

Подводя итоги главе, посвященной теме шаха (белому королю), мы считаем необходимым обратить внимание наших читателей на одно важное обстоятельство. В лучших задачах на эту тему белому королю в начальном положении не грозят никакие шахи. Возможность таких шахов появляется лишь в результате вступительного хода белых. Если же эти шахи возможны уже в начальном положении, то на любой из них у белых имеется в задаче готовый мат. Это условие должно обязательно соблюдаться во всех задачах, разрабатывающих тему шаха как прямого, так и открытого.

Мы полагаем, что в области открытых шахов многое еще не достигнуто, многое требует еще изучения и разработки, хотя среди произведений на эту тему имеется уже сейчас немало великолепных достижений. Особенно много в этой области поработал Г. Гвиделли, давший ряд первоклассных задач на эту тему.
