\chapter{Блокирование.}

Тема блокирования известна уже давно, -- много раз она встречалась в произведениях представителей старых школ. Ново-американская школа углубила самое понятие блокирования, усложнила его и с его помощью чрезвычайно обогатила комплекс задачных идей.

Блокированием называется такой ход черных, когда какая-нибудь черная фигура становится вплотную около черного короля и тем препятствует ему вступить на занятое ею поле. В практической партии можно блокировать пешку, то есть занять поле впереди нее и тем мешать ее продвижению. В задачах термин блокирования относится только к королю. Черная фигура, ограничивающая поле передвижения черного короля, называется \so{блокирующей}, а поле, занятое этой фигурой, \so{блокированным}.

Каков же ближайший результат блокирования? До 6локирующего хода черный король мог отступить на блокируемое поле (и тем спастись от мата), -- теперь же после блокирующего хода, он этой возможности уже не имеет. И благодаря этому белые дают мат на втором ходу.

Рассмотрим несколько случаев блокирования.
 
 \begin{center} 
 \begin{tabular}{ c c }
\textbf{№ 64. А. В. Галицкий.} & \textbf{№ 65. Э. Э. Вестбери.} \\
<<Московские ведомости>>, 1900. & Похв. отз. конк. <<Тidskrift>>, 1909. \\
\chessboard[
\diagramsize,
setfen=4N3/2ppp3/N6Q/3k2p1/B5K1/3P4/4P3/8,
label=false,
showmover=false] & 
\chessboard[
\diagramsize,
setfen=2nK4/p1P5/B1k5/2pN1N2/2n1P3/p7/1p5Q/1r6,
label=false,
showmover=false] \\
\textbf{Мат в 2 хода (1. \queen{}g7).} & \textbf{Мат в 2 хода (1. \queen{}c2).}
 \end{tabular}
\end{center}

В № 64, после первого хода белых, черные находятся в Zugzwang-e, то есть вынуждены сделать какой-то ход, ухудшающий их положение. В самом деле, белые пока не грозят никаким матом на следующем ходу. Но черные вынуждены двигаться.

Скажем, двинулась пешка с на одно поле : с7--с6. Что же получилось? Поле с6 \so{заблокировано}, и белый слон может дать мат 2. \bishop{}bЗ\mate{}. До движения пешки на с6 слон должен был охранять это поле, но теперь, после блокирования, в этом нет нужды. Ту же картину мы увидим и при иных движениях черных пешек: на 1. ... с5 последует 2. \knight{}а6-с7\mate{} (поле с5 заблокировано), на 1. ... d6 2. \knight{}с8:с7, на 1. ... е5. 2. \queen{}f7\mate{}, а на 1. ... е6 2. е4\mate{}. В первых четырех случаях черные, блокируя поле, вместе с тем освобождают какую-нибудь белую фигуру oт необходимости охранять это поле. В последнем же варианте (1. ... е6) они просто блокируют свободное поле е6. Эта задача прекрасно иллюстрирует сущность простого блокирования полей вокруг черного короля черными пешками.

Ho блокировать могут не только пешки, но и все другие черные фигуры. В № 65 эту работу берут на себя черные кони. После 1-го хода белых черные должны защищаться от угрозы 2. \queen{}а4\mate{}. Защищаться они могут конями, но нетрудно заметить, что, защицаясь, кони становятся на клетки b6 и d6, то есть блокируют своего короля. Становясь на b6, они освобождают коня d5, а занимая поле d6, освобождают коня f5. И действительно, на 1. ... \knight{}с4-b6 последует 2. \knight{}b4\mate{}, на 1. ... \knight{}с4-d6 2. \knight{}d4\mate{}, на 1. ... \knight{}с8-b6 2. \knight{}d5-с7\mate{} и на 1. ... \knight{}c8-d6 2. \knight{}f5-c7\mate{}.

\begin{center} 
 \begin{tabular}{ c c } 
\textbf{№ 66. Ф. А. Л. Кускоп.} & \textbf{№ 67. Г. В. Беттман.} \\
II пр. конк. <<G. С.>>, январь 1916. &  <<G. С.>>, май 1920. \\
\chessboard[
\diagramsize,
setfen=8/2p5/2R5/Pn1k4/1P2R3/2b5/5PB1/6K1,
label=false,
showmover=false] & 
\chessboard[
\diagramsize,
setfen=8/8/4K3/8/2N5/B3R3/ppkp2Q1/r1b2N2,
label=false,
showmover=false] \\
\textbf{Maт в 2 хода (1. f4).} & \textbf{Maт в 2 хода (1. \queen{}c6).}
 \end{tabular}
\end{center}

Блокировать могут также слоны, ладьи, ферзь. Мы не будем приводить специальных примеров блокирования этими фигурами, так как читатель встретится с ними в дальнейшем при рассмотрении данной темы.

До сих пор мы видели, что блокирующая фигура либо просто препятствует королю уйти на бывшее свободным поле, либо освобождает еще белую фигуру от необходимости охранять блокируемое поле. В № 66 мы видим кое-что новое. Защищаясь от угрозы 2. \rook{}e4-с4\mate{} черные играют 1. ... \bishop{}d4+. При этом они не только блокируют поле d4, но и объявляют шaх белому королю. Ответ белых -- 2. \rook{}e3\mate{}! -- становится возможным потому, что ладья освободилась oт необходимости защищать поле d4. При защите 1. ... \knight{}d4 белые играют 2. \rook{}е5\mate{}, а при 1. ... \knight{}d6 2. \rook{}с5\mate{}.

Новый момент в блокировании мы находим и в задаче № 67. Задача построена на Zugzwang. У черного короля -- два свободных поля: b1 и d1. При движении пешек черные не только блокируют одно из этих полей, но и превращают свою пешку в какую-нибудь фигуру. Этот момент превращения черных пешек имеет здесь большое значение. Так, на 1. ... b1=\knight{} последует 2. \queen{}а4\mate{}, но этот мат станет невозможен, если черная пешка b превраnится не в коня, а в ферзя (или ладью). Ha 1. ... b1=\queen{} последует уже новый мат 2. \knight{}b2\mate{}, который не был возможен при превращении пешки в коня. To же самое происходит и на пункте d1. На 1. ... d1=\queen{} последует 2. \knight{}d2\mate{}, а на 1. ... d1=\knight{} 2. \queen{}c4\mate{}.

Рассмотрим теперь внимательно задачу К. A. К. Ларсона (№ 68). На первый взгляд мы находим в ней обычное блокирование трех полей, но на самом деле это нечто совсем новое.

Белые грозят 2. \queen{}e6\mate{}. На 1. ... \knight{}с5 последует простой мат 2. \knight{}сЗ\mate{}. Конь освободился от необходимости держать поле с5 и матует на сЗ. Но вот черные играют 1. ... \knight{}d4. Они блокируют поле d4, но ладья b4, которая освободилась благодаря этому oт необходимости защищать поле d4, не может матовать. Мат дает слон -- 2. \bishop{}с4\mate{}. Что же случилось? Слон матует, перекрыв ладью, так как поле d4 заблокировано и не нуждаетсяв защите ладьи. Точно также на 1. ... \bishop{}d6 последует 2. \bishop{}c6\mate{}, причем белый слон перекрывает ладью b6, освобожденную от необходимости держать поле d6. Здесь мы имеем \so{блокирование}, осложненное перекрытием белой фигуры: такое блокирование мы предлагаем назвать \so{сложным блокированием}, в отличие от простого блокирования, с которым мы имели дело в предыдущих примерах.

При сложном блокировании перекрываемая белая фигура (в задаче № 68 это обе белые ладьи) нужна: она как-бы поддерживает матующую фигуру. Но возможны и такие случаи, когда перекрываемая белая фигура становится совсем ненужной. Примеры этого мы находим в задаче №69.

\begin{center} 
 \begin{tabular}{ c c } 
\textbf{№ 68. К. А. К. Ларсен.} & \textbf{№ 69. Д. К. Д. Вайнрайт.} \\
II пр. конк. <<G. С.>>, январь 1916. &  <<G. С.>>, май 1920. \\
\chessboard[
\diagramsize,
setfen=Kb4n1/8/1R6/1B1k4/NR6/1nQp4/8/7b,
label=false,
showmover=false] & 
\chessboard[
\diagramsize,
setfen=4Q3/p7/Nb1pR1NK/1r1k4/2n4R/4p3/8/2r4b,
label=false,
showmover=false] \\
\textbf{Maт в 2 хода (1. \queen{}c1).} & \textbf{Maт в 2 хода (1. \rook{}f6).}
 \end{tabular}
\end{center}

В самом деле, при защите черных 1. ... \knight{}e5 белые отвечают 2.\knight{}e7\mate{}! Посмотрим на матовую картину и отдадим себе отчет в том, какую роль здесь сыграл ферзь. В самом мате он не участвует, он отрезан от черного короля, он уже не нужен, он как бы выключен из игры. Такое сложное блокирование, при котором перекрываемая фигура становится совершенно ненужной, мы называем \so{выключением}. В этой задаче мы имеем и второе выключение после 1. ... \bishop{}d4 белые играют 2.\knight{}f4\mate{} , выключая ладью h4.

\begin{center} 
 \begin{tabular}{ c c }
\textbf{№ 70. А. И. Финк и Ва-Танэ.} & \textbf{№ 71. С. С. Левман.} \\
I пр. <<G.C.>>, 1920. &  Поч. отз. <<Mid-week Sports Reference>>>, 1926. \\
\chessboard[
\diagramsize,
setfen=K7/2Rpp3/N1P2p2/3kPQ2/Rnp5/Bn1rpP2/2B1N3/8,
label=false,
showmover=false] &
\chessboard[
\diagramsize,
setfen=3N2B1/r2p4/5RKB/1n2k3/3p2R1/2n5/8/8,
label=false,
showmover=false] \\
\textbf{Мат в 2 хода (1. \rook{}c8).} & \textbf{Мат в 2 хода (1. \rook{}b6).}
 \end{tabular}
\end{center}

Мы видим из этих примеров, как много интересных моментов содержит тема блокирования,- эта старая тема, которую разрабатывали многие композиторы старых школ, но которая полностью развернулось и выяаила все свои возможности лишь в последнее время, когда за нее взялись модернисты. Мы приведем еще несколько примеров на эту тему, которые должны показать, какие блестящие успехи достигнуты проблемистами ново-американской школы в области блокирования.

В задаче № 70 проведено максимальное количество блокирований: целых 8, из которых 7 простых и 1 выключение. Задача построена очень искуссно, на полном Zugzwang-е, причем первым ходом меняется 1 мат в ответ на 1. ... \knight{}:с6).

В задаче № 71 мы видим четыре сложных блокирования: из них два вьключения. Эти четыре блокирования достигнуты с минимальной затратой сил. В задаче всего 12 фигур, и все 4 идейных варианта сделаны очень четко.
 
\begin{center} 
 \begin{tabular}{ c c }
\textbf{№ 72. А. Боттаччи.} & \textbf{№ 73. К. A. К. Ларсен.} \\
I пр. <<Alfiere di Re>>, 1921. & I пр. конк. <<Brisbane Courier>>, 1925. \\
\chessboard[
\diagramsize,
setfen=bb6/2ppR3/7K/1n5R/3k1N2/n1r3NQ/1P6/5B2,
label=false,
showmover=false] &
\chessboard[
\diagramsize,
setfen=2R5/Kn5b/8/1p6/2NkPPp1/R5p1/p1pNp1B1/r2Q2q1,
label=false,
showmover=false] \\
\textbf{Мат в 2 хода (1. \queen{}h4).} & \textbf{Мат в 2 хода (1. \knight{}e5).}
 \end{tabular}
\end{center}
 
Одной из лучших задач на тему блокирования нужно признать задачу №72. В отличие от предыдущей, мы имеем в ней 5 сложных блокирований (максимум для данного времени!), из них 4 выключения. Белые грозят 2. \queen{}f6\mate{}. Ha 1. ... \rook{}e3 2. \knight{}e6\mate{}!, на 1. ... \knight{}с4 2. \knight{}g3-e2\mate{}, а на 1. ... c5 2.\knight{}f5\mate{}! На 1. ... \rook{}c5 белые играют \knight{}d5\mate{}. В задаче имеется еще ряд интересных вариантов: 1. ... c6 2. \rook{}e4\mate{}; 1. ... d5 2. \knight{}d3\mate{}!
	
Из прекрасной задачи А. Боттаччи мы можем извлечь еще некоторые интересные моменты. Прежде всего мы видим, как тонко в некоторых вариантах играют черные, защищаясь от грозящего им мата. Ходы 1. ... \rook{}d3! или 1. ... d5 не так-то легко найти, так как они представляют собой скрытую защиту (перекрытие линии действия белых фигур). Это черта, характерная для всей новейшей школы; богатство и тонкость защиты, а не только нападения.

Второе -- соединение темы блокирования с другими (например, с темой перекрытия в вариантах 1. ... сб и 1. .. . d5). В задачной литературе мы имеем ряд произведений, где тема блокирования является уже не центральной, не главной, а как бы побочной. В центре задачи стоит другая тема, а комбинации с блокированием как бы украшают и обогащают задачу. В качестве примера приведем задачу К. A. К. Ларсена, знаменитого датского композитора, много поработавшего в области блокирования. В центре задачи -- движения пешек с2 н е2, которые превращаются в коней и перекрывают ладью и ферзя. Но на ряду с этим мы находим в задаче 3 выключения (1. ... \queen{}е3 2. \knight{}b3\mate{}! -- 1. ... \bishop{}:е4 2. \knight{}df3\mate{}! -- 1. ... \knight{}с5 2. \knight{}с6\mate{}).

Изучив все эти примеры и познакомившись с достижениями в области обработки темы блокирования (простого и сложного), читатель может задать себе вопрос: a не исчерпаны ли уже возможности, заложенные в этой теме? Стоит ли дальше работать в этом направлении?

Mы полагаем, что темы блокирования (особенно сложного) еще далеко не исчерпаны и заслуживают самого пристального изучения и внимания со стороны проблемистов. С одной стороны, может продолжатъся работа по достижению максимумов (то есть максимального количества блокирований в одной задаче). С другой стороны, еще более широкие перспективы открываются перед тем, кто попытается сочетать комбинации сложного блокирования с другими темами и комбинациями ново-американской школы.
