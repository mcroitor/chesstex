В основу данного сборника легли результаты чемпионатов республики Молдова по составлению шахматных задач. Материалы были предоставлены старейшиной композиторского цеха Ивановым Альбертом Федотовичем. Как будет видно, в чемпионатах принимали участие композиторы с задачами из различных жанров. Это связано с тем, что в Молдове малое количество людей интересуется таким видом искусства как шахматная композиция и, соответственно, еще меньше людей составляет задачи и этюды.

Позиции, при подготовке сборника, прошли компьютерную проверку. Часть задач (и призовые тоже) оказалась с дефектами. Составитель сборника приложил все возможные усилия по сохранению задач, однако часть из них так и не смог исправить. Однако эти позиции всё равно входят в сборник, сразу по нескольким мотивам: во-первых, в целях сохранения истории; во-вторых, с целью демонстрации представляемых ими идей; в-третьих, в надежде, что читатели сборника предложат свои исправления.

Надеюсь, что данная брошюра окажет своё влияние на развитие шахматной композиции в Молдове. Автор пытался писать комментарии к решениям задач на доступном для любителей шахматной композиции уровне. Варианты записаны в английской нотации, в которой K – король, Q – ферзь, R – ладья, B – слон, S – конь.
