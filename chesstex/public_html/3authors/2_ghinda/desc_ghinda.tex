\section{Гиндэ Анатолий Алексеевич}

GHIND\u{A}  Anatolie Alexei, n\u{a}scut la 02 august 1951 \^{i}ntr-o familie de pedagogi din satul Ruse\c{s}tii Noi, raionul Ialoveni (anterior raionul Kotovsk, recent H\^{a}nce\c{s}ti). \^{I}n familia de 5 copii – (doi b\u{a}ie\c{t}i \c{s}i trei fete) Anatolie a fost al treilea copil.

\^{I}n 1966 a absolvit 8 clase, continu\^{a}nd \^{i}nv\u{a}\c{t}\u{a}tura la Colegiul Pedagogic din C\u{a}l\u{a}ra\c{s}i, pe care l-a absolvit \^{i}n 1970.  A \^{i}nceput activitatea pedagogic\u{a} \^{i}n localitatea de ba\c{s}tin\u{a}, Ruse\c{s}tii Noi ca \^{i}nv\u{a}\c{t}\u{a}tor la clasele primare, ca dup\u{a} 2 luni s\u{a} fiu \^{i}nrolat la serviciul militar. Re\^{i}ntors din armat\u{a}, \^{i}n 1972 am intrat la facultatea de matematic\u{a} a Institutului Pedagogic de Stat ``Ion Creang\u{a}'' din Chi\c{s}in\u{a}u. Dup\u{a} absolvire, \^{i}n anul 1978 am revenit la \c{s}coala medie din localitate activ\^{a}nd ca instructor superior de pionieri, citind \c{s}i c\^{a}teva ore de matematic\u{a}. Dup\u{a} c\^{a}teva cursuri de reprofilare am citit \c{s}i ore de muzic\u{a} \c{s}i art\u{a} plastic\u{a}. Aproape 10 ani am activat prin cumul la gradini\c{t}a de copii ca conduc\u{a}tor muzical. 
	
\^{I}n 1980 sunt c\u{a}s\u{a}torit. Am dou\u{a} fiice, doi nepoti. Din 1985, c\^{a}nd a fost introdus informatica ca obiect de studii, din nou dup\u{a} cursuri de reprofilare activez \c{s}i p\^{a}n\u{a} \^{i}n prezent ca profesor de informatic\u{a} \c{s}i educa\c{t}ie plastic\u{a} la Liceul Teoretic din Ruse\c{s}tii Noi.
	
Antologia de pasiuni nu este limitat\u{a}, deaceea nu pot pune accent pe una sau dou\u{a} ocupa\c{t}ii preferate. Prefer s\u{a} nu fiu indiferent de tot ce m\u{a} \^{i}nconjoar\u{a}: muzica, desenul, \c{s}ahul, fotografia, s\u{a} me\c{s}teresc, s\u{a} citesc, s\u{a} creez ceva nou, s\u{a} rezolv din matematic\u{a}, programare, \c{s}ah, studii psihologice, s\u{a} cultiv plante \^{i}n gr\u{a}din\u{a}, s\u{a} admir natura, s\u{a} apreciez \c{s}i s\u{a} pre\c{t}uiesc frumosul, s\u{a} promovez valoarea.
	
Ca s\u{a} nu fiu rezervat, \c{t}in s\u{a} constat c\u{a} \c{s}ahul a fost o pasiune mai veche. Am \^{i}nv\u{a}\c{t}at s\u{a} joc \^{i}n clasaa a 7-a. Deoarece nu prea erau pasiona\c{t}i de \c{s}ah, nu aveam cu cine juca, am preferat s\u{a} rezolv probleme de \c{s}ah din ziare \c{s}i reviste. Astfel am participat la majoritatea campionatelor republicane de dezleg\u{a}ri a compozi\c{t}iilor de \c{s}ah, care mai apoi mi-a inspirat dorin\c{t}a de a compune probleme de \c{s}ah. La \^{i}nceput pentru mine, apoi am \^{i}ncercat \c{s}i pentru concursuri interna\c{t}ionale ca Rom\^{a}nia, Polonia, Croa\c{t}ia, Rusia. P\^{a}n\u{a} \^{i}n prezent sunt create peste 200 de probleme de \c{s}ah (mat direct \^{i}n 2,3, invers, ajutor \^{i}n 2-8 mut\u{a}ri) dar \c{s}i de alt gen precum \^{i}n stil de \c{s}ah distractiv. Cea mai mare parte din compozi\c{t}ii sunt de gen neortodoxal. C\^{a}teva zeci de probleme au primit men\c{t}iuni la concursuri, iar unele fiind premiate cu locul I, II sau III. Prefer s\u{a} fiu mai mult un analist \c{s}ahist dec\^{a}t un \c{s}ahist juc\u{a}tor.

\clearpage