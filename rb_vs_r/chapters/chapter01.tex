\section{Теоретические позиции.}%

Анализ соотношения сил ``Ладья и Слон против Ладьи'' имеет давнюю историю. Первый анализ данного окончания 
был опубликован Франсуа-Анре Филидором ещё в 1749 году \{1\}. Позже к анализу присоединились Жамбатиста 
Лолли (1763 год \{2\}), Кохрейн и другие. Уже в XXI веке, с появлением эндшпильных таблиц закрылись все 
вопросы и стали понятны все тонкости, потому что для любой конкретной позиции компьютер даст точный ответ 
на любой ход. Однако, до сих пор публикуются теоретические анализы данного соотношения сил. И связано это 
с фактом, что шахматисту нужно понимать стратегию выигрыша / ничьей, так как он не в состоянии запомнить 
все варианты.

По статистике, чуть менее 59\% позиций являются ничейными. Это значит, что данный материал, не смотря на 
ничейность, очень тяжело защищать. Поэтому важно знать основные теоретические позиции, чтобы уметь достигать 
нужного результата. Для выигрышных позиций это, разумеется, позиция Филидора (Диаграмма \thediagramcounter).

% new diagram
\def \fen {4k3/3r4/4K3/4B3/8/8/8/2R5 w - - 0 1}
\newchessgame[setfen=\fen, id=game\thediagramcounter]

\begin{center}
	\begin{tabular}{ c }
	\diagramnumber\hspace{5mm} Филидор Франсуа \\
	1749 \\
	\chessboard \\
	\bold{ход белых}
	\end{tabular}
\end{center}

Небольшой анализ данной позиции показывает, что любая попытка поставить мат отбивается перекрытием ладьёй:
\variation[invar, invar]{1. Rc8+  Rd8}, а промедление наталкивается на шахи \variation{1. Rh1  Re7+}. Первым делом необходимо выбить чёрную ладью с удобной горизонтали, что достигается в два хода: 

\resumechessgame[id=game\thediagramcounter]
\mainline[outvar, outvar]{1. Rc8+  Rd8 2. Rc7 Rd2}

Самое упорное. Игра чёрной ладьи по восьмой горизонтали быстро проигрывает: \variation[invar, invar]{2... Ra8 3. Rh7} с очевидной победой. В случае \newchessgame[setfen=\fen, id=game\thediagramcounter-1]\hidemoves{1. Rc8+  Rd8 2. Rc7} 
\mainline{2... Rd1} (Диаграмма \thediagramcounter{}a) победу гарантирует черное поле е1, так как в этом случае, в нужный момент, слон возбмёт его под контроль.

\begin{center}
	\begin{tabular}{ c }
	\diagramnumber{}a\hspace{5mm} после 2... Rd1\\
	\chessboard
	\end{tabular}
\end{center}

% new diagram
\stepcounter{diagramcounter}
\def \fen {6k1/5r2/6K1/6B1/8/8/8/4R3 w - - 0 1}
\newchessgame[setfen=\fen, id=game\thediagramcounter]
\begin{center}
	\begin{tabular}{ c }
	\diagramnumber\hspace{5mm} Лолли Жамбатиста \\ % author
	Osservazioni Teorico-Pratiche, 1763 \\ % additional
	\chessboard \\
	\bold{ход белых}
	\end{tabular}
\end{center}

%solution
\mainline{1. Re8+ Rf8 2. Re7 Rc8}
%(2... Rf3 $1 {<cook HC>}) (2... Rf2 $1 {<cook HC>}) (2... Rf1 $1 {<cook MF>} 3. Ra7 Rf3 4. Bf6 Rg3+ 5. Kf5 Rg2 6. Be5 Rc2 7. Rg7+ Kf8 8. Rd7 Kg8 9. Kf6 Rc6+ 10. Bd6 Rc1 $1 11. Rg7+ Kh8 12. Rg4 Rf1+ (12... Kh7 $2 13. Kf7 Kh6 14. Bf4+) 13. Bf4 Ra1 14. Be3 Rf1+ 15. Kg6 Kg8 16. Ra4 Rf7 17. Ra8+ Rf8 18. Ra2 Rf3 19. Bh6 Rg3+ 20. Bg5 Rf3 21. Bf6 Rg3+ 22. Kf5 Rg1 23. Rc2 Rb1 24. Bd4 Re1 $1) 
\mainline{3. Re6 Rf8 4. Bh6 Rc8 5. Bc1 Kf8 6. Ba3+ Kg8 7. Rb6 Kh8 8. Be7 Rg8+ 9. Kh6 Re8 10. Rc6 Kg8 11. Kg6 Rb8 12. Rc5 Re8 13. Bf6 Kf8 14. Rh5} 1-0

% new diagram
\stepcounter{diagramcounter}
\def \fen {7B/8/8/8/8/1K1R4/8/1k3r2 b - - 0 1}
\newchessgame[setfen=\fen, id=game\thediagramcounter, moveid=1b]
\begin{center}
	\begin{tabular}{ c }
	\diagramnumber\hspace{5mm} Chapais \\ % author
	Essais-analytiques sur les Echecs, 1780 \\ % additional
	\chessboard \\
	\bold{ход черных}
	\end{tabular}
\end{center}

%solution
\mainline{1... Rc1 2. Bb2 Rf1 3. Bf6} 

Или 3. Be5 

\mainline{3... Rc1 4. Bg5 Rf1 5. Rg3 Ka1 6. Bd2 Rb1+ 7. Ka3 Rd1 8. Rf3 Kb1 9. Kb3 Rg1 10. Rf2 Rg3+ 11. Bc3 Rg1 12. Ra2 $1}

% new diagram
\stepcounter{diagramcounter}
\def \fen {6k1/8/3B2K1/8/8/6R1/5r2/8 w - - 0 1}
\newchessgame[setfen=\fen, id=game\thediagramcounter]
\begin{center}
	\begin{tabular}{ c }
	\diagramnumber\hspace{5mm} Клинг Иозеф, Kuiper=J \\ % author
	Le Palamede, 1846 \\ % additional
	\chessboard \\
	\bold{ход белых}
	\end{tabular}
\end{center}

%solution
\mainline{1. Ra3 Rg2+ 2. Bg3 Kf8 3. Kf6 Kg8 4. Ra8+ Kh7 5. Ra7+ Kh6}

\variation{5... Kg8 6. Rg7+ Kh8 7. Rg4 Ra2 8. Be5 Kh7 9. Rg7+ Kh8 10. Rg1 Kh7 11. Rh1+ Kg8 12. Bd4 Ra6+ 13. Ke7 Rh6 14. Rg1+ Kh7 15. Rg7+ Kh8 16. Rf7+ Kg8 17. Rf8+ Kh7 18. Kf7}

\mainline{6. Rg7 Kh5 7. Rg5+ Kh6 8. Rg4 Kh7 9. Kf7 1-0}

% new diagram
\stepcounter{diagramcounter}
\def \fen {4k3/8/8/8/8/8/8/4K3 w - - 0 1}
\newchessgame[setfen=\fen, id=game\thediagramcounter]
\begin{center}
	\begin{tabular}{ c }
	\diagramnumber\hspace{5mm} 6k1/R7/8/5K2/8/2B5/8/2r5 w - - 0 1 \\ % author
	Le Palamede, 1846 \\ % additional
	\chessboard \\
	\bold{ход белых}
	\end{tabular}
\end{center}

%solution
\mainline{1. Rg7+ Kf8 2. Rc7 Rb1}

\variation{2... Rf1+ 3. Ke6 Kg8 4. Rg7+ Kf8 5. Rg4 Rb1 6. Bb4+}

\mainline{3. Bd4 Rd1 4. Be5 Rb1 5. Rd7 Kg8 6. Kf6 Rb6+ 7. Bd6 Rb8 8. Bc5 Rb3 9. Rg7+ Kh8 10. Rg1 Rf3+ 11. Kg6 Kg8 12. Rg4 Rc3 13. Bd6 Rc6 14. Kf6+ Kh7 15. Rg7+ Kh8 16. Rd7 Rc1 17. Be5 Rc6+ 18. Kf7+ Kh7 19. Rd8 Kh6 20. Bf6 Rc7+ 21. Be7 Kh7 22. Rd4 Rc6 23. Bd6 Kh6 24. Rd5 1-0}

% new diagram
\stepcounter{diagramcounter}
\def \fen {3k4/8/1r1BK3/8/2R5/8/8/8 w - - 0 1}
\newchessgame[setfen=\fen, id=game\thediagramcounter]
\begin{center}
	\begin{tabular}{ c }
	\diagramnumber\hspace{5mm} Kling=J Kuiper=J \\ % author
	Le Palamede, 1846 \\ % additional
	\chessboard \\
	\bold{ход белых}
	\end{tabular}
\end{center}

%solution
\mainline{1. Rc1}

Возможно также \italic{1. Rc5} или \italic{1. Rc3}.

\mainline{1... Ra6 2. Rh1 Kc8 3. Rb1 Ra8 4. Ke7 Ra7+ 5. Ke8 Ra8 6. Rb2 1-0}


% new diagram
\stepcounter{diagramcounter}
\def \fen {4k3/8/8/8/8/8/8/4K3 w - - 0 1}
\newchessgame[setfen=\fen, id=game\thediagramcounter]
\begin{center}
	\begin{tabular}{ c }
	\diagramnumber\hspace{5mm}  \\ % author
	 \\ % additional
	\chessboard \\
	\bold{ход белых}
	\end{tabular}
\end{center}

%solution

% new diagram
\stepcounter{diagramcounter}
\def \fen {4k3/8/8/8/8/8/8/4K3 w - - 0 1}
\newchessgame[setfen=\fen, id=game\thediagramcounter]
\begin{center}
	\begin{tabular}{ c }
	\diagramnumber\hspace{5mm}  \\ % author
	 \\ % additional
	\chessboard \\
	\bold{ход белых}
	\end{tabular}
\end{center}

%solution

% new diagram
\stepcounter{diagramcounter}
\def \fen {4k3/8/8/8/8/8/8/4K3 w - - 0 1}
\newchessgame[setfen=\fen, id=game\thediagramcounter]
\begin{center}
	\begin{tabular}{ c }
	\diagramnumber\hspace{5mm}  \\ % author
	 \\ % additional
	\chessboard \\
	\bold{ход белых}
	\end{tabular}
\end{center}

%solution

% new diagram
\stepcounter{diagramcounter}
\def \fen {4k3/8/8/8/8/8/8/4K3 w - - 0 1}
\newchessgame[setfen=\fen, id=game\thediagramcounter]
\begin{center}
	\begin{tabular}{ c }
	\diagramnumber\hspace{5mm}  \\ % author
	 \\ % additional
	\chessboard \\
	\bold{ход белых}
	\end{tabular}
\end{center}

%solution

% new diagram
\stepcounter{diagramcounter}
\def \fen {4k3/8/8/8/8/8/8/4K3 w - - 0 1}
\newchessgame[setfen=\fen, id=game\thediagramcounter]
\begin{center}
	\begin{tabular}{ c }
	\diagramnumber\hspace{5mm}  \\ % author
	 \\ % additional
	\chessboard \\
	\bold{ход белых}
	\end{tabular}
\end{center}

%solution

% new diagram
\stepcounter{diagramcounter}
\def \fen {4k3/8/8/8/8/8/8/4K3 w - - 0 1}
\newchessgame[setfen=\fen, id=game\thediagramcounter]
\begin{center}
	\begin{tabular}{ c }
	\diagramnumber\hspace{5mm}  \\ % author
	 \\ % additional
	\chessboard \\
	\bold{ход белых}
	\end{tabular}
\end{center}

%solution

% new diagram
\stepcounter{diagramcounter}
\def \fen {4k3/8/8/8/8/8/8/4K3 w - - 0 1}
\newchessgame[setfen=\fen, id=game\thediagramcounter]
\begin{center}
	\begin{tabular}{ c }
	\diagramnumber\hspace{5mm}  \\ % author
	 \\ % additional
	\chessboard \\
	\bold{ход белых}
	\end{tabular}
\end{center}

%solution

% new diagram
\stepcounter{diagramcounter}
\def \fen {4k3/8/8/8/8/8/8/4K3 w - - 0 1}
\newchessgame[setfen=\fen, id=game\thediagramcounter]
\begin{center}
	\begin{tabular}{ c }
	\diagramnumber\hspace{5mm}  \\ % author
	 \\ % additional
	\chessboard \\
	\bold{ход белых}
	\end{tabular}
\end{center}

%solution

% new diagram
\stepcounter{diagramcounter}
\def \fen {4k3/8/8/8/8/8/8/4K3 w - - 0 1}
\newchessgame[setfen=\fen, id=game\thediagramcounter]
\begin{center}
	\begin{tabular}{ c }
	\diagramnumber\hspace{5mm}  \\ % author
	 \\ % additional
	\chessboard \\
	\bold{ход белых}
	\end{tabular}
\end{center}

%solution

% new diagram
\stepcounter{diagramcounter}
\def \fen {4k3/8/8/8/8/8/8/4K3 w - - 0 1}
\newchessgame[setfen=\fen, id=game\thediagramcounter]
\begin{center}
	\begin{tabular}{ c }
	\diagramnumber\hspace{5mm}  \\ % author
	 \\ % additional
	\chessboard \\
	\bold{ход белых}
	\end{tabular}
\end{center}

%solution

% new diagram
\stepcounter{diagramcounter}
\def \fen {4k3/8/8/8/8/8/8/4K3 w - - 0 1}
\newchessgame[setfen=\fen, id=game\thediagramcounter]
\begin{center}
	\begin{tabular}{ c }
	\diagramnumber\hspace{5mm}  \\ % author
	 \\ % additional
	\chessboard \\
	\bold{ход белых}
	\end{tabular}
\end{center}

%solution

% new diagram
\stepcounter{diagramcounter}
\def \fen {4k3/8/8/8/8/8/8/4K3 w - - 0 1}
\newchessgame[setfen=\fen, id=game\thediagramcounter]
\begin{center}
	\begin{tabular}{ c }
	\diagramnumber\hspace{5mm}  \\ % author
	 \\ % additional
	\chessboard \\
	\bold{ход белых}
	\end{tabular}
\end{center}

%solution

% new diagram
\stepcounter{diagramcounter}
\def \fen {4k3/8/8/8/8/8/8/4K3 w - - 0 1}
\newchessgame[setfen=\fen, id=game\thediagramcounter]
\begin{center}
	\begin{tabular}{ c }
	\diagramnumber\hspace{5mm}  \\ % author
	 \\ % additional
	\chessboard \\
	\bold{ход белых}
	\end{tabular}
\end{center}

%solution

% new diagram
\stepcounter{diagramcounter}
\def \fen {4k3/8/8/8/8/8/8/4K3 w - - 0 1}
\newchessgame[setfen=\fen, id=game\thediagramcounter]
\begin{center}
	\begin{tabular}{ c }
	\diagramnumber\hspace{5mm}  \\ % author
	 \\ % additional
	\chessboard \\
	\bold{ход белых}
	\end{tabular}
\end{center}

%solution


\pagebreak